\section{Лекция 4}
\subsection{Структура линейного оператора}
ОСЛУ:
\begin{equation}
  \label{eq:04_1}
\begin{cases}
  (a_{11} - \lambda)x_1 + a_{12} x_2 + \ldots + a_{1n} x_n = 0 \\
  a_{21} x_1 + (a_{22} - \lambda)x_2 + \ldots + a_{2n}x_n = 0 \\
  \ldots \\
  a_{n 1} x_1 + a_{n 2} x_2 + \ldots + (a_{n n} - \lambda) x_{n} = 0
\end{cases}
\end{equation}
Характеристический многочлен
\[
\chi_A(\lambda) = \det(A_\phi - \lambda E) = 0
\]
\begin{statement}[О свойствах характеристического многочлена матрицы $A$]
\label{statement:04_1}
  \begin{itemize}
    \item [a) ] Корни характеристического многочлена $\chi_A(\lambda)$, принадлежащие $\mathbb{F}$, и только они являются собственными значениями лин. оператора $\phi$.
    \item [б) ] Характеристический многочлен лин. оператора $\phi$ не зависит от выбора базиса (хотя $A_\phi$ зависит).
  \end{itemize}
\end{statement}
\begin{proof}
  \begin{itemize}
    \item [а) ] Пусть $\chi_A(\lambda_0) = 0$. Тогда существует ненулевое решение $x_0$, такое что $\phi(x_0) = \lambda_0 x_0 \Rightarrow \lambda_0$ --- собственное значение оператора $\phi$. \\
      Пусть $\lambda_0$ --- собственное значение $\phi$. $\exists x_0 \neq 0$, т. ч. $\phi(x_0) = \lambda_0 x_0$ $\Rightarrow$ система $(\ref{eq:04_1})$ имеет при $\lambda = \lambda_0$ имеет ненулевое решение при $\lambda = \lambda_0$ $ \Rightarrow \chi_A(\lambda_0) = 0 \Rightarrow \lambda_0$ --- корень. \\

    \item [б) ]
      Пусть $e, f$ --- базисы в $V$.
      \[
      B = S^{-1}AS, S = S_{e \to f}
      \]
      \[
      \chi_B(\lambda) = \det(B - \lambda E) = \det(S^{-1}AS - S^{-1}(\lambda E)S) = 
      \]
      \[
       = \underbrace{\det S^{-1}}_{\frac{1}{y}} \cdot \det(A - \lambda E) \cdot \underbrace{\det S}_{y} = \det = \det(A - \lambda E) = \chi_A(\lambda)
      \]
  \end{itemize}
\end{proof}
\begin{symb}
  \[
  \chi_\phi(\lambda) := \chi_{A_\phi}(\lambda)
  \]
  \[
  \trace \phi := \trace A_\phi
  \]
  \[
  \det \phi := \det A_\phi
  \]
\end{symb}
\begin{consequence}
\label{consequence:04_1}
Если $V$ --- линейное пр-во над $\C$, $\dim V \geq 1$, то $\forall \phi \colon V \rightarrow V$ имеет хотя бы один вектор.
\end{consequence}
\begin{proof}
$\chi_\phi(\lambda)$ по ОТА имеет хотя бы один корень $\in \C$.
\end{proof}
\begin{consequence}
\label{consequence:04_2}
Если $V$ --- линейное пространство над $\C$, а также
\[
  \dim V = 2k + 1, k \in \N
\]
то $\forall \phi \colon V \rightarrow V$ имеет хотя бы один собственный вектор.
\end{consequence}
\begin{proof}
  $\chi_\phi(\lambda)$ имеет хотя бы один вещественный корень.
\end{proof}
\begin{note}
$\exists $ линейный оператор, не имеющий собственный векторов:
\[
  R(\phi) = \begin{pmatrix}\cos \phi & -\sin \phi \\ \sin \phi & \cos \phi \end{pmatrix}, \phi \neq \pi n, n \in \Z
\]
\[
\chi_{R(\phi)}(\lambda) = \lambda^{2} - 2\cos \phi \lambda + 1
\]
\[
  D = 4\cos^{2} \phi - 4 = -4\sin^{2}\phi < 0
\]
Над $\C$ два корня: $e^{-i\phi}, e^{i\phi}$
\[
  B = \begin{pmatrix}e^{-i\phi} & 0 \\ 0 & e^{i\phi} \end{pmatrix}
\]
\end{note}
\begin{definition}
  Линейный оператор $\phi \colon V \rightarrow V$, $V$ над $\mathbb{F}$ называется \textbf{диагонализируемым}, если в $V$ $\exists$ базис $e$, в котором $A_\phi$ диагональна:
  \[
    A_\phi = \begin{pmatrix} \lambda_1 & 0 & \ldots & 0 \\ 0 & \lambda_2 & \ldots & 0 \\ 0 & 0 & \ddots &  0 \\ 0 & 0 & \ldots & \lambda_n\end{pmatrix}
  \]
\end{definition}
\begin{theorem}[Критерий Диагонализируемости]
\label{theorem:04_1}
  $\phi \colon V \rightarrow V$ --- лин. оператор и пусть $\lambda_1, \ldots, \lambda_k$ --- все попарно различные собственные значения $\phi$. Тогда следующие условия эквивалентны:
  \begin{itemize}
    \item [а) ] $\phi$ диагонализируем
    \item [б) ] В $V$ $\exists$ базис $e$, состоящий из собственных векторов для $\phi$
    \item [в) ] $V = V_{\lambda_1} \oplus V_{\lambda_2} \oplus \ldots \oplus V_{\lambda_k}$
      \[
      V_{\lambda} = \ker(\phi - \lambda id)
      \]
  \end{itemize}
\end{theorem}
\begin{proof}
  \begin{itemize}
    \item a) $\Rightarrow $ б):
      \[
      \exists e \colon  A_\phi = \begin{pmatrix} \lambda_1 & 0 & \ldots & 0 \\ 0 & \lambda_2 & \ldots & 0 \\ 0 & 0 & \ddots &  0 \\ 0 & 0 & \ldots & \lambda_n\end{pmatrix} \iff \phi(e_i) = \lambda_i e_i
      \]
    \item б) $\Rightarrow$ в): разобъём базисные векторы по группам с собственным значениями:
      \[
      \lambda_1 \colon e_{11}, e_{12}, \ldots, e_{1 s_1}
      \]
      \[
      \vdots
      \]
      \[
      \lambda_k \colon e_{k 1}, e_{k 2}, \ldots, e_{k s_k}
      \]
      Тогда верно, что:
      \[
      Q_1 = <e_{11}, \ldots, e_{1 s_1}> \leq V_{\lambda 1}
      \]
      \[
      \vdots
      \]
      \[
      Q_k = <e_{k 1}, \ldots, e_{k s_k}> \leq V_{\lambda_k} 
      \]
      Поэтому:
      \[
      Q_1 \oplus Q_2 \oplus \ldots \oplus Q_k = V
      \]
      Следовательно:
      \[
      V_{\lambda_1} + V_{\lambda_2} + \ldots + V_{\lambda_k} = V
      \]
      А т. к. $\lambda_i$ попарно различны, то по теореме о характеризации прямой суммы, т. к. $V_{\lambda_i}$ --- ЛНЗ, то:
      \[
      V_{\lambda_1} \oplus V_{\lambda_2} \oplus \ldots \oplus V_{\lambda_k} = V
      \]
      А также:
      \[
      Q_i = V_{\lambda_i}
      \]
    \item в) $\Rightarrow$ а): пусть:
      \[
      V_{\lambda_1} \oplus V_{\lambda_2} \oplus \ldots \oplus V_{\lambda_k} = V
      \]
      Пусть $\begin{pmatrix}e_{i1} & \ldots & e_{i s_i} \end{pmatrix}$ --- базис в $V_{\lambda_i}$, а $e = \set{e_{ij}}$, тогда:
      \[
        A_\phi = \begin{pmatrix}\lambda_1 & \ldots & \ldots \\
        \ldots & \ddots & \ldots \\ \ldots & \ldots & \lambda_{1} & \ldots \\ \ldots & \ldots & \ldots & \lambda_{2} \\ \ldots & \ldots & \ldots & \ldots & \ddots & \ldots \\ \ldots & \ldots & \ldots \ldots & \ldots & \ldots & \lambda_{2} \\ \ldots & \ldots & \ldots & \ldots & \ldots & \ldots & \ddots\end{pmatrix}
      \]
  \end{itemize}
\end{proof}
\subsubsection{Алгебраическая и геометрическая кратности собственных значений}
Пусть $\phi \colon V \rightarrow V$ --- лин. оператор $V$ над $\mathbb{F}$:
\[
\chi_\phi(t) = \det(A - tE)
\]
Пусть $\lambda$ --- корень $\chi_\phi(t)$, т. е. $\lambda$ --- собственное значение оператора $\phi$.
\begin{definition}
Кратность $\lambda$, \textbf{как корня $\chi_\phi(t)$}, наз-ся \textbf{алгебраической кратностью} собсвтенного значения $\lambda$.
  \[
  alg(\lambda)
  \]
\end{definition}
\begin{definition}
  \textbf{Размерность собственного подпространства $V_\lambda$} называется \textbf{геометрической кратностью} собственного значения $\lambda$.
  \[
    \gem(\lambda)
  \]
\end{definition}
\begin{note}
Если $\lambda$ --- собственное значения оператора $\phi$, тогда:
\[
  \alg(\lambda) \geq 1
\]
\[
  \gem(\lambda) \geq 1
\]
\end{note}
\begin{statement}
\label{statement:04_2}
Пусть $\phi \colon V \rightarrow V$ --- лин. оператор. $U \leq V$ --- инвариантно отн-но $\phi$. $\psi = \phi|_{U} \in \mathcal{L}(U)$. Тогда:
\[
  \chi_\phi \vdots \chi_\psi
\]
\end{statement}
\begin{proof}
  Выберем базис в $V$, согласованный с инвариантным подпространством $U$:
  \[
    \underbrace{\underbrace{e_1, \ldots, e_k}_{\text{базис в $U$}}, e_{k + 1}, \ldots, e_n}_{\text{базис в $V$}} = e
  \]
  \[
    A_\phi = \begin{pmatrix} \frac{A_{\psi}}{O} & \frac{B}{C} \end{pmatrix}, k = \dim U
  \]
  \[
    \chi_\phi(t) = \begin{vmatrix} \frac{A_\psi - t E_k}{O} & \frac{B}{C - tE} \end{vmatrix} = \det(A_\psi - tE) \left|C - tE\right| = \chi_\psi(t) \cdot \chi_C(t)
  \]
  \[
  \Rightarrow \chi_\phi(t) \vdots \chi_\psi(t)
  \]
\end{proof}
\begin{consequence}
  \label{consequence:04_3}
  Пусть $\lambda$ --- произв. собственное значение оператора $\phi \colon V \rightarrow V$. Тогда $\gem(\lambda) \leq \alg(\lambda)$
\end{consequence}
\begin{proof}
  $V_\lambda$ --- инвариантно отн-но $\phi$. $\psi = \phi|V_\lambda$
  \[
  \chi_\phi \vdots \chi_\psi
  \]
  \[
  \chi_\psi = (\lambda - t)^{k}, k = \dim (V_\lambda)
  \]
  \[
  \Rightarrow \chi_\phi(t) \vdots (\lambda - t)^{k} \Rightarrow \alg(\lambda) \geq k = \gem(\lambda)
  \]
\end{proof}
\begin{note}
  Пусть $\phi$ --- диагонализируем, значит $\exists e = \begin{pmatrix}e_1 & \ldots & e_n \end{pmatrix}$, т. ч.
  \[
    A_\phi = \begin{pmatrix} \lambda_1 & \ldots & 0 \\ 0 & \ddots & 0 \\ 0 & \ldots & v_n \end{pmatrix}
  \]
  Тогда $\phi(e_i) = \lambda_i e_i, \forall i = \overline{1,n}$. Базис, в кот. $\phi$ диагональная --- это базис, состоящий, из собственных векторов, а числа на главной диагонали --- собственные значения.
  \[
    \trace \phi = \trace A = \sum_{i = 1}^{n} \lambda_i
  \]
  \[
    \det \phi = \det A = \prod_{i = 1}^{n} \lambda_i
  \]
  \[
    \chi_\phi(t) = \prod_{i = 1}^{n}(\lambda_i - t) \text{ над $\mathbb{F}$ --- линейно факторизуем}
  \]
  \textbf{Вывод:} всякий диагонализируемый оператор \textbf{линейно факторизуем}, т. е. его характеристический многочлен линейно факторизуем.
\end{note}
\begin{consequence}
  \label{consequence:04_4}
  Если $\phi$ не является линейно факторизуем, то он и не диагонализируем.
\end{consequence}
\begin{theorem}
\label{theorem:04_2}
  Линейный оператор $\phi \colon V \rightarrow V$ с собственными значениями $\lambda_1, \ldots, \lambda_k$ является диагонализируемым $\iff$
  \begin{itemize}
    \item [а) ] $\phi$ --- линейно факторизуем над $\mathbb{F}$ (т. е. $\chi_\phi(t)$ --- линейно факторизуем)
    \item [б) ] $\forall i = 1, \ldots, k \colon \alg(\lambda_i) = \gem(\lambda_i)$
  \end{itemize}
\end{theorem}
\begin{proof}
  \begin{itemize}
    \item [а) ] Необх.: пусть $\phi$ --- диагонализируем по Th $(\ref{theorem:04_1})$
      \[
      V = V_{\lambda_1} \oplus V_{\lambda_2} \oplus \ldots \oplus V_{\lambda_k}
      \]
      \[
      n = \sum_{i = 1}^{k} \dim (V_{\lambda_i}) = \sum_{i = 1}^{k} \gem(\lambda_i) = \deg \chi_\phi \geq \sum_{i = 1}^{k} \alg(\lambda_i)
      \]
      Но $\forall i = \overline{1,n} \colon \gem(\lambda_i) \leq \alg(\lambda_i)$
      \[
      \Rightarrow \forall i = \overline{1,n} \colon \gem(\lambda_i) = \alg(\lambda_i)
      \]
      \item [б) ] Дост.: пусть а) и б) выполнены:
        \[
       \dim(\underbrace{V_{\lambda_1} \oplus \ldots \oplus V_{\lambda_k}}_{\text{т. к. $\lambda_i$ попарно различны}})  = \sum_{i = 1}^{k} \dim V_{\lambda_i} = \sum_{i = 1}^{k} \gem{\lambda_i} = \sum_{i = 1}^{k} \alg(\lambda_i) = \deg \chi_\phi = \dim V
        \]
        \[
        V = V_{\lambda_1} \oplus \ldots \oplus V_{\lambda_k} \underbrace{\Rightarrow}_{\text{Th } \ref{theorem:04_1}} \phi \text{ --- диагонализируем}
        \]
  \end{itemize}
\end{proof}
\begin{example}[Одной лишь лин. факторизуемости $\phi$, даже в случае алг. замкнутого поля не достаточно, чтобы утверждать его диагонализируемост]
\[
  J_n(\lambda) = \begin{pmatrix} \lambda & 1 & 0 & \ldots & 0 \\ 0 & \lambda & 1 & \ldots & 0 \\ 0 & 0 & \ddots & \ldots & 0 \\ 0 & 0 & \ldots & \ldots & 1 \\ 0 & 0 & \ldots & \ldots & \lambda \end{pmatrix} \text{ --- Жорданова клетка порядка $n$, отвеч. $\lambda$}
\]
\[
  \chi_{J_n(\lambda)}(t) = \begin{vmatrix} \lambda - t & 1 \\ 0 & \lambda - t & 1 \\ & & \ddots \\ \ & & & 1 \\ & & & \lambda - t \end{vmatrix} = (\lambda - t)^{n}
\]
\[
\gem(\lambda) := \dim V_{\lambda} = \dim \ker(\phi - \lambda id) = \dim \ker(A - \lambda E) = \]
\[
  = n - \rk (A - \lambda E) = n - \rk (A_\lambda)
\]
\[
 \gem(\lambda) = n - \rk J_n(\lambda) = n - (n - 1) = 1 < n
\]
\end{example}
\subsection{Приведение линейно факторизуемого лин. оператора к верхнетреугольному виду}
\begin{statement}
  \label{statement:04_3}
  Пусть $\phi \colon V \rightarrow V$ --- лин. оператор:
  \[
  \phi_{\lambda} = \phi - \lambda id
  \]
  Тогда следующие условия эквивалентны:
  \begin{itemize}
    \item [а) ] Подпространство $U \leq V$ инвариантно отн-но $\phi$
    \item [б) ] $\exists \lambda \in \mathbb{F}\colon U$ --- инвариантно отн-но $\phi_{\lambda}$
    \item [в) ] $\forall \lambda \in \mathbb{F} \colon U$ --- инвариатно отн-но $\phi_{\lambda}$
  \end{itemize}
\end{statement}
\begin{proof}
  а) $\Rightarrow$ в) $\underbrace{\Rightarrow}_{\text{очев.}}$ б) $\Rightarrow$ a)
  \begin{itemize}
    \item а) $\Rightarrow$ в): $x \in U, \phi_{\lambda}(x) := (\phi - \lambda id)(x) = \underbrace{\phi(x)}_{\in U} - \underbrace{\lambda x}_{\in U} \in U$
    \item б) $\Rightarrow$ а): $\exists \lambda$, т. ч. $U$ --- инвариантно отн-но $\phi - \lambda id$. Покажем, что $U$ инвариантно относительно $\phi$.
      \[
      x \in U \colon \phi(x) = (\phi - \lambda id + \lambda id)(x) = \underbrace{(\phi - \lambda id)(x)}_{\in U} + \underbrace{(\lambda id)(x)}_{\in U} \in U
      \]
  \end{itemize}
\end{proof}
\begin{statement}
\label{statement:04_4}
Пусть $\phi \colon V \rightarrow V$ --- лин. оператор и $\chi_\phi(t)$ раскладывается на линейные множители (т. е. лин. факторизуем). $\dim_{\mathbb{F}} V = n$. Тогда у $\phi$ найдётся $(n - 1)$-мерное инвариантное подпространство.
\end{statement}
\begin{proof}
  \[
  \chi_\phi(t) = \prod_{i = 1}^{n} (\lambda_i - t) \Rightarrow \exists \lambda_n \in \mathbb{F}, \text{ кот. явл-ся собств. знач.}
  \]
  \[
  V_{\lambda_n} = \ker(\phi - \lambda_n id) \neq \set{0} \Rightarrow \dim \Image (\phi - \lambda_n id) \leq n - 1
  \]
  Пусть $U \leq V$, т. ч. $\Image(\phi - \lambda_n id) \leq U$, $\dim U = n - 1$
  \[
    (\phi - \lambda_n id)(U) \subseteq \Image(\phi - \lambda_n id) \subseteq U \Rightarrow U \text{ --- инв. отн-но $\phi - \lambda id$}
  \]
  \[
  \Rightarrow U \text{ --- инвариатно отн-но $\phi$}
  \]
\end{proof}
\begin{note}
  \textbf{Условие утв-я можно ослабить} (необходимо наличие хотя бы одного собств. знач-я у $\phi$)
\end{note}
\begin{theorem}
\label{theorem:04_3}
Пусть $\phi \colon V \rightarrow V$ --- лин. факторизуем над $\mathbb{F}$. Тогда $\exists e = \begin{pmatrix} e_1 & \ldots & e_n \end{pmatrix}$ в $V$, в котором:
\[
  A_\phi = \begin{pmatrix} \lambda_1 & * & * & \ldots & * \\ 0 & \lambda_2 & * & \ldots & * \\ 0 & 0 & \ddots & \ldots & * \\ 0 & 0 & 0 & \ddots & * \\ 0 & 0 & 0 & \ldots & \lambda_n \end{pmatrix}
\]
\end{theorem}
\begin{proof}
Покажем, что в $V$, $\exists$ цепочка вложенных подпространств, которые инв. отн-но $\phi$
\[
  \underbrace{\set{0} < U_1 < U_2 < \ldots < U_n = V, \dim U_i = i
}_{\text{флаг. подпространств}}
\]
Докажем $\exists$-ие флага индукцией по $\dim V$:
\begin{itemize}
  \item База: $\set{0} < V_1 = V$ --- флаг
  \item Переход: пусть для $\phi \colon W \rightarrow W$, $\dim W < n$, утв-е доказано. ($\phi$ линейно факторизуем). \\
    По утверждению $(\ref{statement:04_4})$ в $V$ найдётся $U_{n - 1}$, инвариантное отн-но $\phi$:
    \[
    \psi = \phi|U_{n - 1} \text{ --- линейно факторизуем (?)}
    \]
    \[
    \chi_\phi \vdots \chi_\psi
    \]
    \[
    \chi_\phi(t) = \prod_{i = 1}^{n} (\lambda_i - t) \Rightarrow \chi_\psi(t) = \prod_{i \neq j}^{} (\lambda_i - t)
    \]
    По предположению индукции $\exists$ флаг $\psi$ -- инв, поэтому:
    \[
    \underbrace{\set{0} < U_1 < U_2 < \ldots < U_{n - 1} < U_n = V}_{\phi \text{ --- инв.}}
    \]
    Тогда в базисе $e$, согласов. с флагом, $\begin{pmatrix}e_1 & \ldots & e_k \end{pmatrix}$ --- базис в $U_k$.
\end{itemize}
\end{proof}
\begin{consequence}
\label{consequence:04_5}
В условиях Th ($\ref{theorem:04_3}$), $\forall k = \overline{0, n-1}$ справедливо утв-е:
\[
  (\phi - \lambda_{k + 1}id)\ldots(\phi - \lambda_n id) V \subseteq U_k
\]
\end{consequence}
\begin{proof}
Индукцией по количеству скобок слева:
\[
  (\phi - \lambda_n id)V \overset{?}{\subseteq} U_{n- 1}
\]
\[
  A - \lambda_n E = \begin{pmatrix} \lambda_1 - \lambda_n & \ldots & * \\ \ldots & \ldots & \ldots \\ \ldots & \ldots & 0 \end{pmatrix}\begin{pmatrix}x_1 \\ \vdots \\ x_n \end{pmatrix} = \begin{pmatrix}y_1 \\ \vdots \\ y_{n - 1} \\ 0 \end{pmatrix} \in U_{n - 1} = <e_1, \ldots, e_{n - 1}>
\]
  Предполжение индукции: \[
    (\phi - \lambda_{k + 2}id) \ldots (\phi - \lambda_{n} id) V \subseteq U_{k + 1}
  \]
  \[
    (\phi - \lambda_{k + 1})U_{k + 1} \subseteq U_k
  \]
  \[
  \overbrace{\set{0} < U_1 < U_2 < \ldots < U_n}^{\text{инв. отн. $\phi$}} = V
  \]
  \[
  U_{k + 1} = U_k \oplus <e_{k + 1}>
  \]
  \[
    (\phi - \lambda_{k + 1} id) (U_k) \subseteq U_k \text{ --- т. к. $\phi$ --- инв.}
  \]
  \[
    (\phi - \lambda_{k + 1}id) (e_{k + 1}) = \phi(e_{k + 1}) - \lambda_{k + 1}e_{k + 1} = 
  \]
  \[
   = \sum_{i = 1}^{k} a_{i k + 1} e_i + \lambda_{k + 1} e_{k + 1} - \lambda_{k + 1}e_{k + 1} = \sum_{i = 1}^{k} e_{i k + 1}e_i \in U_{k}
  \]
\end{proof}
\begin{theorem}[Гамильтона, Кэли]
\label{theorem:04_4}
Пусть $\phi \colon V \rightarrow V$ --- лин. факторизуем над $\mathbb{F}$ (лин. оператор). $\chi_\phi(t)$ --- его характеристический многочлен. Тогда:
\[
  \chi_\phi(\phi) = O \text{ --- нулевой оператор}
\]
Эквив. формулировка в терминах матрицы: пусть $A \in {M_n}(\mathbb{F})$, $\chi_A(t)$ --- характ. многочлен матрицы $A$, и он лин. фактор. над $\mathbb{F}$, тогда:
\[
  \chi_A(A) = 0
\]
\end{theorem}
\begin{proof}
  \[
    k = 0 \Rightarrow (\phi - \lambda_1 id) \ldots (\phi - \lambda_n id) V \subset U_0 = \set{0}
  \]
  \[
  \forall x\colon \prod_{i = 1}^{n} (\phi - \lambda_i id) (x) = (-1)^{n} \chi_\phi(\phi)(x) = 0
  \]
  \[
  \Rightarrow \chi_\phi(\phi)(x) = 0
  \]
\end{proof}
\begin{note}
  Гамильтон и Кэли, независимо друг от друга, доказали это утв-е только для $\dim_{\C} V \leq 4$. Современное доказательство для общего случая принадлежит Фробениусу (1878 г.).
\end{note}
\begin{note}
  В теореме Гамильтона-Кэли можно отказаться от линейной факторизуемости. Пусть $\mathbb{F}$ не алгебраически замкнуто и $\chi_\phi(t)$ не раскладывается на линейные множители над $\mathbb{F}$.
  \[
    F \subset K
  \]
  \[
  \chi_\phi(\phi) = O, \phi \text{ --- в лин. пр-ве над $K$}
  \]
  \[
    \chi_\phi(t) \in F(t) \text{ --- в лин. пр-ве над $\mathbb{F}$}
  \]
\end{note}
