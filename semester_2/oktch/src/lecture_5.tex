\section{Лекция 5}
\subsection{Первообразные корни}
\[
  m \in \N, a \in \N, (a, m) = 1
\]
\begin{theorem}[Эйлера]
\label{theorem:04_4}
  \[
  a^{\phi(m)} \equiv 1 \pmod m
  \]
\end{theorem}
\begin{definition}
  Показатель числа $a \pmod m$ ($\delta(a)$) --- это
  \[
    \min k > 0 \colon a^{k} \equiv 1 \pmod m
  \]
\end{definition}
\begin{statement}
  \[
  \delta(a) | \phi(m)
  \]
\end{statement}
\begin{definition}
  $a$ --- \textbf{первообразный корень} $\mod m$, если
  \[
    \delta(a) = \phi(m)
  \]
  Обозначают как $g$.
\end{definition}
\begin{statement}
Если по $\mod m$, $\exists $ первообразный корень $g$, то:
\[
  1, g, g^{2}, \ldots, g^{\phi(m) - 1}
\]
Образуют \textbf{всю} привидённую систему вычетов $\mod m$.
\end{statement}
\begin{statement}
При $m = 2^{n}, n \geq 3$ первообразных корней $\mod m$ не существует.
\end{statement}
\begin{proof}
  \[
  \phi(m) = 2^{n - 1}
  \]
Покажем, что если
\[
  \begin{cases}
  a = 2t + 1 \\
  a^{2^{n - 1}}  \equiv 1 \pmod m
  \end{cases} \Rightarrow a^{2^{n - 2}} \equiv 1 \pmod m
\]
\[
a^{2} = 4t^{2} + 4t + 1 = 4t(t + 1) + 1 = 8t_1 + 1
\]
\[
a^{4} = (8t_1 + 1)^{2} = 64t_1^{2} + 16t_1 + 1 = 16t_2
\]
\[
a^{8} = 32t_3 + 1
\]
\[
\Rightarrow a^{2^{k}} = 2^{k + 2}t_k + 1
\]
Таким образом:
\[
a^{2^{n - 2}} = 2^{n}t_{n - 1} + 1 \equiv 1 \pmod 2^{n} 
\]
\end{proof}
\begin{theorem}
\label{theorem:05_1}
Первообразные корни $\mod m$ существуют, если и только если $m \in \set{2, 4, p ^{\alpha}, 2p ^{\alpha}}$, где $p$ --- нечётные простые.
\end{theorem}

\begin{definition}
  \textbf{Дискретный логарим}:
  \[
    x = g^{b} \iff \ind_{g} b = x \pmod p
  \]
\end{definition}

Докажем теорему для случая $m = p\colon$
\begin{proof}
  Положим $\tau = [\delta(1), \ldots, \delta(p - 1)]$. Тогда:
  \[
  x^{\tau} \equiv 1 \pmod p, \forall x \in \set{1, \ldots, p - 1} \Rightarrow \tau \geq p - 1
  \]
  С другой стороны:
  \[
  \tau = \prod_{i = 1}^{s} q_i^{\alpha_i}
  \]
  Тогда:
  \[
  \forall i = \overline{1,s}, \exists x_i \colon \delta(x_i) = a_i \cdot q_i ^{\alpha_i}
  \]
  \begin{statement}
    \[
      \delta(x_i^{\alpha}) = q_i^{\alpha_i}
    \]
  \end{statement}
  \begin{statement}
    \[
    \delta \left(\prod_{i = 1}^{s} x_i^{a_i}\right) = \prod_{i = 1}^{s} q_i^{\alpha_i} = \tau
    \]
  \end{statement}
  Отсюда следует, что
  \[
    \tau | (p - 1) \Rightarrow \tau \leq p - 1 \Rightarrow \tau = p - 1
  \]
  Ч. Т. Д.
\end{proof}

Теперь докажем для случая $p ^{\alpha}, \alpha > 1$
\begin{proof}
Пусть $g$ --- первообразый корень $\mod p$. Докажем:
\begin{lemma}
\[
\exists t \colon (g + pt)^{p - 1} \equiv 1 + pu, (u, p) = 1
\]
\end{lemma}
\begin{proof}
  \[
    (g + pt)^{p - 1} = g^{p - 1} + (p - 1) \cdot g^{p - 2} \cdot pt + p ^{2} \cdot a = 
  \]
  \[
   = 1 + pb + (p - 1) \cdot g^{p - 2} \cdot pt + p ^{2} \cdot a = 
  \]
  \[
   = 1 + p (b + (p - 1)g ^{p - 2} \cdot t + pa)
  \]
  Т. к. $(p - 1) g^{p - 2} \cdot t$ пробегает полную систему вычетов, то такое $t$ найдётся. 
\end{proof}

Пусть теперь $\delta = \delta(g + pt) \pmod p ^{\alpha}$. Хотим доказать, что:
\[
  \delta = p ^{\alpha - 1}(p - 1)
\]
\[
  (g + pt)^{\delta} \equiv 1 \pmod p \Rightarrow (g + pt)^{ \delta} \equiv 1 \pmod p
\]
\[
\Rightarrow \delta \vdots (p - 1)
\]
С другой стороны, $\delta | p ^{\alpha}(p - 1) \Rightarrow \delta = p ^{k}(p - 1)$
\[
  (g + pt)^{p - 1} = 1 + pu, (u, p) = 1
\]
\[
  (g + pt)^{p(p - 1)} = (1 + pu)^{p} = 1 + p ^{2} u + p ^{3} a = 
\]
\[
 = 1 + p ^{2}\underbrace{(u + pa)}_{u_1}, (u_1, p) = 1
\]
И т. д. получаем:
\[
  (g + pt)^{p ^{k}(p - 1)} = (1 + pu_{k - 1})^{p} = 1 + p ^{k + 1} u_k, (u_k, p) = 1
\]
Т. к.
\[
  \delta = \delta(g + pt) \Rightarrow 1 + p ^{k + 1}u_k \equiv 1 (\mod p^{\alpha})
\]
Отсюда следует, что $k + 1 = \alpha \Rightarrow k = \alpha - 1$
\end{proof}
Случай $2p^{\alpha}$ слишком тривиальный, чтобы его рассматривать.
\[
  \phi(2 p ^{\alpha}) = \phi(p ^{\alpha})
\]
Т. е. если $g + pt$ --- нечёт, то всё ок, иначе берём $g + pt + p ^{\alpha}$\\

Доказательство того, что по другим модулям нет первообразных корней остаётся студенту.

\begin{theorem}[Шевалле]
\label{theorem:05_2}
Пусть $F(x_1, \ldots, x_n)$ --- многочлен от $n$ переменных, $\deg F < n$. Пусть $p$ --- простое, тогда $N_p$ --- число решений сравнения:
\[
  F(x_1, \ldots, x_n) \equiv 0 \pmod p
\]
Тогда $N_p \equiv 0 \pmod p$
\end{theorem}
\begin{proof}
Заметим, что:
\[
N_p \equiv \sum_{x_1 = 1}^{p} \ldots \sum_{x_n = 1}^{p} \left(1 - F^{p - 1}(x_1, \ldots, x_n)\right) = 
\]
\[
 = p ^{n} - \sum_{x_1 = 1}^{p} \ldots \sum_{x_n = 1}^{p} F^{p - 1}(x_1 ,\ldots, x_n)
\]
\[
  F ^{p - 1}(x_1, \ldots, x_n) = \ldots + c x_1^{\alpha_1} \ldots x_n^{\alpha_n} + \ldots
\]
Достаточно доказать, что на $p$ делится любая сумма вида:
\[
\sum_{x_1 = 1}^{p} \ldots \sum_{x_n = 1}^{p} x_1^{\alpha_1} \ldots x_n^{\alpha_n} = 
\]
\[
 = \left(\sum_{x_1 = 1}^{p} x_1\right)\ldots\left(\sum_{x_n = 1}^{p} x_n\right)
\]
\begin{itemize}
  \item \underline{Случай 1:} Если среди $\alpha_i$ есть ноль, то соответствующая сумма $ = p \Rightarrow $ всё произведение делится на $p$
  \item \underline{Случай 2:} Пусть $p = 2$, тогда $\alpha_1 + \ldots + \alpha_n \leq n - 1 \Rightarrow $ выполняется \underline{случай 1}.
  \item \underline{Случай 3:} Пусть $p \geq 3$:
    \[
      \alpha_1 + \ldots + \alpha_n \leq (p - 1)(n - 1)
    \]
    Пусть $\forall i, \alpha_i \geq 1 \Rightarrow \exists i\colon 1 \leq \alpha_i \leq p - 2$
    \[
    S = \sum_{x_i = 1}^{p} x_i^{\alpha_i}
    \]
    Возьмём $g$ --- первообразный корнеь $\mod p$, тогда:
    \[
    g^{\alpha_i} S = \sum_{x_i = 1}^{p} (gx)^{\alpha_i} \equiv S \pmod p
    \]
    \[
    \Rightarrow S\underbrace{(g^{\alpha_i} - 1)}_{\neq 0} \equiv 0 \pmod p \Rightarrow S \equiv 0
    \]
\end{itemize}
\end{proof}
\subsubsection{Немного о шифровании}
Есть Алиса, Боб и Ева. Алиса и Боб хотят передевать сообщение, чтобы Ева их не смогла подслушать. Алиса выбирает число $a$, меньшее заданного $p$ (простое, порядка $10^{200}$, \underline{известно всем участникам}), и вычисляет:
\[
  g^{\alpha} \pmod p.
\]
Результат отправляет Бобу. Боб задумывает $b < p$, вычисляет $g^{b} \pmod p$, отправляет Алисе. У Алисы и Боба есть $g^{a}$ и $g^{b}$, из которых они оба получают $g^{ab}$. Это число будет ключом, который Алиса и Боб будут использовать при переписке. \\
Почему Ева не может узнать ключ? А потому, что задача \underline{дискретного логарифмирования}, т. е. решение ур-я (относитлельно $x$):
\[
  g^{x} \equiv c \pmod p
\]
это \underline{трудная задача} (в вычислительном плане).

