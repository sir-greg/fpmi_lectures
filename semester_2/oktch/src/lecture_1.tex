\section{Лекция 1}
\subsection{Квадратичные вычеты и невычеты}
Пусть $m \in \N$ --- наш модуль. Пусть $a \in \N \colon (a, m) = 1$. Рассмотрим сравнение:
\begin{equation}
  \label{eq:quadratic_residue_equation}
x^{2} \equiv a \pmod m
\end{equation}
Мы говорим, что $a$ является \textbf{квадратичным вычетом по модулю $m$}, если у сравнения (\ref{eq:quadratic_residue_equation}) \textbf{есть решение}.

Пусть $a$ --- квадратичный вычет. Будем всюду далее считать, что $m = p$ --- нечётное простое число. Тогда сравнение $(\ref{eq:quadratic_residue_equation})$ \textbf{имеет 2 решения по теореме Лагранжа}.

\begin{theorem}[Лагранжа]
\label{theorem:01_1}
Пусть:
\[
  f(x) = a_n x^{n} + \ldots + a_1 x + a_0, a_i \in \Z_p
\]
Тогда сравнение:
\[
  f(x) \equiv 0 \pmod p
\]
Имеет $ \leq n$ корней.
\end{theorem}
\begin{proof}
\underline{От противного}, пусть есть решения $x_1, \ldots, x_{n + 1}$. Представим $f(x)$ в виде:
\[
  f(x) = b_n(x - x_1)(x - x_2)\ldots(x - x_n) + 
\]
\[
     +  b_{n - 1}(x - x_1)\ldots(x - x_{n - 1}) +
\]
\[
        \vdots
\]
\[
    +  b_1(x - x_1) + 
\]
\[
      + b_0
\]
Рассм. $f(x_1) \equiv 0 \pmod p \Rightarrow f(x_1) \equiv b_0 \equiv 0 \pmod p$ 

\[
f(x_2) \equiv 0 \equiv b_1(x_2 - x_1) \Rightarrow b_1 \equiv 0 \pmod p
\]
Аналогичным образом, получаем, что $\forall i, b_i = 0$
\end{proof}
Таким образом, в нашем случае сравнение $(\ref{eq:quadratic_residue_equation})$ имеет ровно 2 корня:
\[
  x_1^{2} \equiv a \pmod p
\]
\[
  (-x_1)^{2} \equiv a \pmod p
\]
Выпишем все квадратичные вычеты по модулю $p$:
\[
1^{2}, 2^{2}, \ldots, \left(\frac{p - 1}{2}\right)^{2}
\]
Покажем, что это действительно все корни:
\[
1^{2} \equiv (\pm 1)^{2}
\]
\[
2^{2} \equiv (\pm 2)^{2}
\]
\[
\vdots
\]
\[
a^{2} \equiv (\pm a)^{2}
\]
Таким образом, все эти корни заполняют всю приведённую систему вычетов по модулю $p$. Поэтому мы имеем $\frac{p - 1}{2}$ \textbf{квадратичных вычетов} и $\frac{p - 1}{2}$ \textbf{квадратичных невычетов}.
\begin{definition}
\textbf{Символом Лежандра} числа $a$ по модулю $p$ (читается "$a$ по $p$"), называется число:
\[
  \left(\frac{a}{p}\right) = \begin{cases}
  0, a = 0 \\
  1, a \text{ --- квадратичный вычет} \\
  -1, a \text{ --- квадратичный невычет}
  \end{cases}
\]
\end{definition}
\begin{statement}
  \label{statement:01_1}
\[
\left(\frac{a}{p}\right) \equiv a^{\frac{p - 1}{2}} \pmod p
\]
\end{statement}
\begin{proof}
Рассм. МТФ:
\[
a^{p - 1} \equiv 1 \pmod p
\]
\[
a^{p - 1} - 1 \equiv 1 \pmod p
\]
\[
  (a^{\frac{p - 1}{2}} - 1)(a^{\frac{p - 1}{2}} + 1) \equiv 0 \pmod p
\]
Если $a$ квадратичный вычет, то:
\[
a \equiv x^{2} \pmod p \Rightarrow a^{\frac{p - 1}{2}} \equiv (x^{2})^{\frac{p - 1}{2}} \equiv x^{p - 1} \equiv 1 \pmod p
\]
У уравнения, $a^{\frac{p - 1}{2}} - 1 \equiv 0 \pmod p$ --- $\frac{p - 1}{2}$ корней, т. е. это все наши квадратичные вычеты. Таким образом:
\[
a^{\frac{p - 1}{2}} \equiv 1 \iff a \text{ --- квадратичный вычет}
\]
\[
a^{\frac{p - 1}{2}} \equiv -1 \iff a \text{ --- квадратичный невычет}
\]
Таким образом:
\[
\left(\frac{a}{p}\right) \equiv a^{\frac{p - 1}{2}} \pmod p
\]
\end{proof}
\begin{consequence}
  \label{cs:01_1}
\[
\left(\frac{ab}{p}\right) = \left(\frac{a}{p}\right)\left(\frac{b}{p}\right)
\]
\end{consequence}
\begin{consequence}
  \label{cs:01_2}
\[
\left(\frac{a^{2}}{p}\right) = 1
\]
\end{consequence}
\begin{consequence}
  \label{cs:01_3}
\[
\left(\frac{-1}{p}\right) = (-1)^{\frac{p - 1}{2}} = \begin{cases}
1, p = 4k + 1, k \in \Z \\
-1, p = 4k + 3, k \in \Z
\end{cases}
\]
\end{consequence}
\begin{consequence}
  \label{cs:01_4}
\[
\left(\frac{1}{p}\right) = 1
\]
\end{consequence}
\begin{statement}
  \label{statement:01_2}
\[
  a^{\frac{p - 1}{2}} \equiv (-1)^{\sum_{x = 1}^{p_1} \floor{\frac{2ax}{p}}} \pmod p
\]
\end{statement}
\begin{proof}
Рассм. $x = 1, 2, 3, \ldots, \frac{p - 1}{2}, p_1 := \frac{p - 1}{2}$
\[
ax = \varepsilon_x r_x \pmod p, \varepsilon_x \in \set{+1, -1}, r_x \in \set{1, \ldots, p_1}
\]
Перемножим все $ax$ и $\varepsilon_x r_x$, тогда т. к. $x$ и $r_x$ пробегают одни и те же числа, то:
\[
a^{\frac{p - 1}{2}} \equiv \varepsilon_1 \cdot \ldots \cdot \varepsilon_{p_1}
\]
\begin{statement}
  \label{statement:01_3}
\[
\varepsilon_x = (-1)^{\floor{\frac{2ax}{p}}}
\]
\end{statement}
\begin{proof}
Рассм. случаи принадлежности $ax$ к:
\begin{enumerate}
  \item $\set{1, \ldots, p_1}$
  \item $\set{p_1 + 1, p - 1}$
\end{enumerate}
\end{proof}
Тогда:
\[
  a^{\frac{p - 1}{2}} \equiv (-1)^{\sum_{x = 1}^{p_1} \floor{\frac{2ax}{p}}} \pmod p
\]
\end{proof}
\begin{statement}
  \label{statement:01_4}
Если $a$ --- \textbf{нечётное}, то:
\[
  \left(\frac{a}{p}\right) = (-1)^{\sum_{x = 1}^{p_1} \floor{\frac{ax}{p}}}
\]
\end{statement}
\begin{proof}
Рассм. $a$ --- нечёт. Рассм.
\[
  \left(\frac{2a}{p}\right) = \left(\frac{4(\sfrac{(a + 2)}{2})}{p}\right) = 
\]
\[
  = \left(\frac{4}{p}\right)\left(\frac{\frac{1}{2}(a + p)}{p}\right) = (-1)^{\sum_{x = 1}^{p_1} \floor{\frac{(a + p)x}{p}}} = (-1)^{\sum_{x = 1}^{p_1} \floor{\frac{ax}{p}} + \sum_{x = 1}^{p_1} x} = 
\]
\[
 = (-1)^{\sum_{x = 1}^{p_1} \floor{\frac{ax}{p}} + \frac{p_1(p_1 + 1)}{2}} = (-1)^{\sum_{x = 1}^{p_1} \floor{\frac{ax}{p}} + \frac{p ^{2} - 1}{8}}
\]
Подставим $a = 1$:
\[
  \left(\frac{2}{p}\right) = (-1)^{\frac{p ^{2} - 1}{8}}
\]
При этом в общем виде:
\[
  \left(\frac{2a}{p}\right) = \left(\frac{2}{p}\right) \left(\frac{a}{p}\right) = (-1)^{\frac{p ^{2} - 1}{8}}\left(\frac{a}{p}\right)
\]
Что равно тому, что получено выше. Сокращая одинаковые члены, получаем:
\[
  \left(\frac{a}{p}\right) = (-1)^{\sum_{x = 1}^{p_1} \floor{\frac{ax}{p}}}
\]
\end{proof}
\begin{consequence}
  \label{cs:01_5}
  \[
    \left(\frac{2}{p}\right) = (-1)^{\frac{p ^{2} - 1}{8}}
  \]
\end{consequence}
\textbf{Квадратичный закон взаимности}

Пусть $p$ и $q$ --- разные нечётные простые числа. Тогда:
\begin{equation}
  \label{eq:quadratic_law}
\left(\frac{p}{q}\right)\left(\frac{q}{p}\right) = (-1)^{p_1 \cdot q_1}
\end{equation}
\begin{proof}
\[
\left(\frac{p}{q}\right)\left(\frac{q}{p}\right) = (-1)^{\sum_{x = 1}^{q_1} \floor{\frac{px}{q}} + \sum_{y = 1}^{p_1} \floor{\frac{qy}{p}}}
\]
Положим:
\[
S = \set{(x, y)\colon x = 1,\ldots q_1; y = 1, \ldots, p_1}, \left|S\right| = p_1 \cdot q_1
\]
\[
S_1 = \set{(x, y) \in S \colon qy < px}
\]
\[
S_2 = \set{(x, y) \in S \colon qy > px}
\]
Тогда:
\[
\left|S\right| = \left|S_1\right| + \left|S_2\right|
\]
\[
qy < px \iff y < \frac{px}{q} \Rightarrow \left|S_1\right| = \sum_{x = 1}^{q_1} \floor{\frac{px}{q}}
\]
\[
qy > px \iff x < \frac{qy}{p} \Rightarrow \left|S_2\right| = \sum_{y = 1}^{p_1} \floor{\frac{qy}{p}}
\]
\[
\Rightarrow p_1q_1 = \left|S\right| = \left|S_1\right| + \left|S_2\right| = \sum_{x = 1}^{q_1} \floor{\frac{px}{q}} + \sum_{y = 1}^{p_1} \floor{\frac{qy}{p}}
\]
\end{proof}
