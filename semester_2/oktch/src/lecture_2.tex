\section{Лекция 2}
\subsection{Матрица Адамара}
\subsubsection{Определение}
\begin{definition}
  Матрица Адамара --- это квадратная матрица:
\[
  A_{n \times n} = \begin{pmatrix} a_{ij}\end{pmatrix}, a_{ij} \in \set{+1, -1}
\]
Такая, что любые две строки ортогональны (скалярное произведение в ОНБ $= 0$).
\end{definition}
Рассмотрим несколько случаев:
\begin{itemize}
  \item [$n = 1:$] \[
  \begin{pmatrix} 1 \end{pmatrix}
  \]
\item [$n = 2:$] \[
    \begin{pmatrix} 1 & 1 \\ 1 & -1 \end{pmatrix}
\]
\item [$n = 3:$] Невозможно
  \begin{note}
    Матриц Адамара нечётного размера не существует (кроме $n = 1$)
  \[
  \Rightarrow n \geq 2 \Rightarrow n = 2k
  \]
  \end{note}
\end{itemize}
\subsubsection{Необходимое условие существования}
\begin{theorem}
\label{theorem:02_1}
  $n \geq 2 \Rightarrow n = 4k, k \in \N$
\end{theorem}
\begin{proof}
\begin{task}
  Если у матрицы из $\pm 1$ попарно ортогональны строки, то у неё также попарно ортогональны и столбцы.
\end{task}
Б. О. О.:
\[
  H_n = \begin{pmatrix}1 & \ldots & 1 \\ 1 \\ \vdots & \pm 1 \\ 1 \end{pmatrix}
\]
Т. к. каждая строка ортогональна $1$-ой, то в каждой строке, кроме первой, поровну $1$ и $-1$

Б. О. О.:

Второя строка: $1, 1, 1, \ldots, 1, 1, 1, -1, -1, -1, \ldots, -1, -1, -1$

Третья строка: $1, \ldots, 1, -1, \ldots, -1, 1, \ldots, 1, -1, \ldots, -1$

Получаем 4 блока с одним скал. произведением: $x, \frac{n}{2} - x, \frac{n}{2} - x, x$:
\[
  x - \left(\frac{n}{2} - x\right) - \left(\frac{n}{2} - x\right) + x = 0
\]
\[
  \Rightarrow 4x - n = 0
\]
\[
  \Rightarrow n = 4x
\]
\end{proof}
\textbf{Гипотеза Адамара:} $n = 4k$ --- достаточное условие, для существования матрицы Адамара.

\subsubsection{Конструирование матриц Адамара}
 Алгоритм построения $H_{2^{n}}$ из $H_{2^{n - 1}}$:
 \[
   H_{2^{n}} = \begin{pmatrix}H_{2^{n - 1}}  & H_{2^{n - 1}} \\ H_{2^{n - 1}} & (-1) \cdot H_{2^{n - 1}}\end{pmatrix}
 \]
 \begin{definition}
  $A_n * B_m$ --- \textbf{кронекеровское умножение} квадратных матриц $A_n$ и $B_m$, задаваемое следующим образом:
  \[
    A_n * B_m = \begin{pmatrix}a_{11} \cdot B & \ldots & a_{1n} \cdot B \\ \vdots & \vdots & \vdots \\ a_{n 1} B & \ldots & a_{n n} \cdot B\end{pmatrix} = C_{mn}
  \]
 \end{definition}
 \begin{theorem}
 \label{theorem:02_2}
  Если $A, B$ --- матрицы Адамара, то $A * B$ --- тоже матрица Адамара.
 \end{theorem}
 \begin{theorem}[I конструкция Пэли]
 \label{theorem:02_3}
  Пусть $p = 4k + 3$ --- простое число. Тогда существует $\exists$ матрица Адамара порядка $p + 1$.
 \end{theorem}
 \begin{proof}
 Рассмотрим матрицу $Q = \begin{pmatrix}q_{ij} \end{pmatrix}$:
 \[
 q_{ij} = \left(\frac{i - j}{p}\right)
 \]
 Покажем, что скалярное произведение $\forall$ двух строк равно $-1$:
 \[
 \sum_{b = 1}^{p} \left(\frac{a - b}{p}\right) \left(\frac{a' - b}{p}\right) = \begin{bmatrix}c = a - b \\ a' - b = a' + a - b - a = c + a' - a \end{bmatrix} =
 \]
 \[
 = \sum_{c = 1}^{p - 1} \left(\frac{c}{p}\right)\left(\frac{c + a' - a}{p}\right) = \sum_{c = 1}^{p - 1} \left(\frac{c}{p}\right)\left(\frac{c(1 + c^{-1}(a' - a))}{p}\right) = 
 \]
 \[
 = \sum_{c = 1}^{p - 1} \left(\frac{1 + c^{-1}(a' - a)}{p}\right) = 0 - \left(\frac{1}{p}\right) = -1
 \]
 Тогда искомая матрица:
 \[
   H_{p + 1} = \begin{pmatrix}1 & \ldots & 1 \\ \vdots & Q'\\ 1\end{pmatrix}
 \]
  где $Q'$ матрица $Q$, где вместо $0$ стоят $-1$. Покажем, что это действительно матрица Адамара. Для двух строк $a$ и $a'$ скалярное произведение равно:
  \[
  -1 + 1 - \left(\frac{a - a'}{p}\right) - \left(\frac{a' - a}{p}\right) = 
  \]
  \[
    = -\left(\underbrace{\left(\frac{-1}{p}\right)}_{-1} + 1\right)\left(\frac{a' - a}{p}\right) = 0
  \]
  \[
  \left(\frac{-1}{p}\right) = (-1)^{\frac{p - 1}{2}} = (-1)^{\frac{4k + 2}{2}} = (-1)^{2k + 1} = -1
  \]
\end{proof}
\begin{theorem}[II конструкция Пэли]
\label{theorem:02_4}
Пусть $p = 4k + 1$ --- простое. Тогда $\exists$ матрица Адамара порядка $2(p + 1)$.
\end{theorem}
\begin{note}
В книжке Н. Холла "Комбинаторика" есть отдельная глава про матрицы Адамара (стоит прочитать).
\end{note}
\subsubsection{Плотность порядков матриц Адамара}
\begin{theorem}[$\sfrac{\text{б}}{\text{д}}$]
\label{theorem:02_5}
  \[
  \forall \varepsilon > 0, \exists n_0, \forall n \geq n_0 
  \]
 на отрезке $[n, (1 + \varepsilon)n]$ есть порядок матрицы Адамара.

 Переформулировка:
 \[
  \exists f \colon f(n) = o(n)
 \]
 на отрезке $[n, n + f(n)]$ есть порядок матрицы Адамара.
\end{theorem}
\subsubsection{Коды, исправляющие ошибки}

  Есть передатчик, приёмник и канал связи. По этому каналу связи передаются бинарные строки длины $n$. На канале есть помехи, т. е. произвольный бит может поменять значение. Пусть мы знаем, что кол-во ошибок $\leq k$. 

\textbf{Вопрос:} как организовать словарь кодовых слов (строк, которых мы передаём), что, несмотря на ошибки, приёмник сможет однозначно понять исходное слово по искажённому?

  Например, пусть наш словарь состоит из двух строк и $k = 1$:
  \[
    1110\ldots 0
  \]
  \[
    0111\ldots 0
  \]
  Эти два слова могут исказиться до $1111\ldots 0$, т. е. мы их не сможем различить. С другой стороны:
  \[
    1110\ldots 0
  \]
  \[
    0011\ldots 0
  \]
  Всегда можно различить, т. к. они не могут исказиться до одного и того же.

\begin{definition}
  \textbf{Расстояние Хэмминга} между двумя векторами --- это кол-во несовпадающих координат.
\end{definition}

  Основная задача кодирования: выбрать максимальное кол-во слов так (при заданных $n$ и $k$), чтобы \textbf{расстояние Хэминга между любыми двумя словами было $> 2k$}.

\subsubsection{$(n, M, d)$-код}
\begin{definition}
  $(n, M, d)$-код --- тройка объектов, в которой:
\begin{itemize}
  \item $n$ -- длина кодового слова;
  \item $M$ --- кол-во кодовых слов;
  \item $d$ --- минимальное Хэммингово расстояние.
\end{itemize}
\end{definition}
\begin{theorem}[Граница Плоткина]
\label{theorem:02_6}
  Пусть дан $(n, M, d)$-код, причём $2d > n$. Тогда $M \leq \frac{2d}{2d - n}$
\end{theorem}
\begin{proof}
  Будет доказана в следующий раз
\end{proof}
\begin{note}
  Матрицы Адамар дают неулучшаемую границу размера словаря.
  \[
    H = \begin{pmatrix}1 & \ldots & 1 \\ \vdots  & \pm 1 \\ 1  \end{pmatrix} \rightarrow \begin{pmatrix} \ldots & 1 \\ \pm 1 \\ \ldots\end{pmatrix}
  \]
  \[
  \Rightarrow \left(n - 1, n, \frac{n}{2}\right)\text{-код}
  \]
  Рассмотрим код из строк матрицы Адамара, заметим, что он достигает границы Плоткина.
\end{note}
