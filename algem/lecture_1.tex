\documentclass[12pt]{article}
\usepackage[T1, T2A]{fontenc}
\usepackage[utf8]{inputenc}
\usepackage[russian]{babel}
\usepackage{amsmath}
\usepackage{amsthm}
\usepackage{amssymb}
\usepackage{esvect}
\usepackage{listings}
\usepackage{xcolor}


% for large comments
\usepackage{blindtext, xcolor}
\usepackage{comment}

% for inkscape pictures
\usepackage{import}
\usepackage{xifthen}
\usepackage{pdfpages}
\usepackage{transparent}

\newcommand{\incfig}[1]{%
    \def\svgwidth{\columnwidth}
    \import{./figures/}{#1.pdf_tex}
}

\renewcommand{\C}{\mathbb{C}}
\newcommand{\R}{\mathbb{R}}
\newcommand{\Q}{\mathbb{Q}}
\newcommand{\Z}{\mathbb{Z}}
\newcommand{\N}{\mathbb{N}}

\newcommand{\floor}[1]{\left\lfloor #1 \right\rfloor}
\newcommand{\ceil}[1]{\left\lceil #1 \right\rceil}

% style of code listings
%\definecolor{codegreen}{rgb}{0,0.6,0}
%\definecolor{codegray}{rgb}{0.5,0.5,0.5}
%\definecolor{codepurple}{rgb}{0.58,0,0.82}
%\definecolor{backcolour}{rgb}{0.95,0.95,0.92}
%
%\lstdefinestyle{mystyle}{
%    backgroundcolor=\color{backcolour},
%    commentstyle=\color{codegreen},
%    keywordstyle=\color{magenta},
%    numberstyle=\tiny\color{codegray},
%    stringstyle=\color{codepurple},
%    basicstyle=\ttfamily\footnotesize,
%    breakatwhitespace=false,
%    breaklines=true,
%    captionpos=b,
%    keepspaces=true,
%    numbers=left,
%    numbersep=5pt,
%    showspaces=false,
%    showstringspaces=false,
%    showtabs=false,
%    tabsize=4
%}

\newtheorem{theorem}{\underline{Теорема}}[section]
\newtheorem{lemma}[theorem]{\underlind{Лемма}}
\newtheorem{statement}{\underline{Утверждение}}[section]
\newtheorem*{note}{\underline{Замечание}}
\newtheorem*{symb}{\underline{Обозначение}}
\newtheorem*{example}{\underline{Пример}}
\newtheorem*{consequence}{\underline{Следствие}}
\newtheorem*{solution}{\underline{Решение}}
\newtheorem*{axiom}{\underline{Аксиома}}

\theoremstyle{definition}
\newtheorem{definition}{\underline{Определение}}[section]

\theoremstyle{definition}
\newtheorem{task}{\underline{Задача}}[section]

\title{Алгем. \\ Лекция 1 \\ Алгебра матриц}
\author{Сергей Григорян}

\begin{document}
\maketitle
\newpage
\section{Инфа}
\textbf{Лектор:} Вадим Владимирович Штепин

\section{Матрицы}

\begin{definition}
\textbf{Матрица} - прямоугольная таблица чисел.
\end{definition}

\begin{symb}
    $A_{m*n} = \begin{pmatrix} a_{11} & a_{12} & a_{13} & \cdots & a_{1n} \\ 
    a_{21} & a_{22} & a_{23} & \cdots & a_{2n} \\ 
    \cdots \\ 
a_{m1} & a_{m2} & a_{m3} & \cdots & a_{mn}\end{pmatrix}$

    $a_{ij} \in \R$
\end{symb}

\begin{definition}
\textbf{Поле} - мн-во, на котором определены "+, -, *, /".
\end{definition}

\subsection{I. Сложение}

\begin{symb}
$M_{m * n} - $ мн-во всех матриц размера $m * n$

$A, B \in M_{m * n}, A + B \in M_{m * n}$
\end{symb}

\begin{definition}
    $[A + B]_{ij} = a_{ij} + b_{ij} = [A]_{ij} + [B]_{ij}$ - сложение матриц определено поэлементно.
\end{definition}

\subsection{II. Умножение матрицы на вещественное число $\lambda \in \R$}

\begin{definition}
    Умножение матрицы на число осущ. поэлементно:

$A \in M_{m * n}$ 

$\lambda \in \R$ 

$\lambda A \in M_{m * n}$

$[\lambda A]_{ij} = \lambda a_{ij} = \lambda [A]_{ij}$
\end{definition}

\begin{theorem}
Операции сложения матриц и  "$* \lambda$" удовл. след. св-вам $[A, B, C \in M_{m * n}]$:

\begin{enumerate}
    \item Коммутативность сложения: $ A + B = B + A$
    \item Ассоциативность сложения: $ (A + B) + C = A + (B + C)$
    \item Существование нулевой матрицы: $ \exists O \in M_{m * n}$, т. ч. $A + O = A, \forall A \in M_{m * n}$
    \item Св-во сущ. прот. матрицы: $\forall A \in M_{m * n} \exists (-A) \in M_{m * n}$, т. ч. A + (-A) = (-A) + A = 0
    \item Унитарность: $1 * A = A$;
    \item Ассоциативность отн-но скалярного мн-ва : $(\lambda * \mu) * A = \lambda * (\mu * A)$ ;
    \item Дистрибутивность $(\lambda + \mu) A = \lambda A + \mu A$ 
    \item Дистрибутивность $\lambda (A + B) = \lambda A + \lambda B$;
\end{enumerate}
\end{theorem}

\begin{proof}
\textbf{8) } $A, B \in M_{m * n}$ 
\[
[\lambda(A + B)]_{ij} = \lambda[A + B]_{ij} = \lambda (a_{ij} + b_{ij}) = \lambda * a_{ij} + \lambda * (b_{ij}) = [\lambda A]_{ij} + [\lambda B]_{ij} = [\lambda A + \lambda B]_{ij} 
.\] 
\end{proof}

\begin{definition}
\textbf{Линейное пр-во над $M_{m*n}$}:

Пусть $V$ - произв. мн-во, на кот. определены операции сложения эл-ов из $V$ и умн-я эл-ов из $V$ на эл-ты $\R$, и эти оп-ции удовл аксиомам (1-8). Тогда $V$ - действительное линейное (векторное) пр-во.
\end{definition}

\textbf{Вывод:} $M_{m*n}(\R)$ - действ. лин. пр-во.

\section{III. Транспонирование}
$A \in M_{m * n} \Rightarrow A^{T} \text{ или } A^{t} \in M_{n * m}$

\begin{definition}
 \[
     [A^{T}]_{ij} = [A]_{ji}
 .\] 
\end{definition}

\begin{example}
\[
    \begin{pmatrix} 1 & 9 & 9 & -1 \\ 3 & -7 & -2 & 4 \end{pmatrix}^{T} = \begin{pmatrix} 1 & 3 \\ 9 & -7 \\ 9 & -2 \\ -1 & 4 \end{pmatrix} 
.\] 
\end{example}

\section{IV. Умножение матриц}

\subsection{Частный случай}
$A \in M_{1*n}, B \in M_{n * 1}$ 
\[
    \begin{pmatrix} a_1 & a_2 & \cdots & a_n \end{pmatrix} * \begin{pmatrix} b_1 \\ b_2 \\ b_3 \\ \vdots \\ b_n \end{pmatrix} = a_1 b_1 + a_2 b_2 + \cdots + a_n b_n
.\] 

\subsection{Общий случай}

$A * B$ имеет смысл (опр.), если:

\[
    A \in M_{m * \underline{n}}, B \in M_{\underline{n} * k}
\]

Тогда:

\[
    C = A * B \in M_{m * k}
.\] 

\[
    [C]_{ij} = \sum_{s = 1}^{n} a_{is} b_{sj}
.\] 

\[
    \begin{pmatrix} \cdots & \cdots  & \cdots \\
a_{i1} & \cdots & a_{in} \\
\cdots & \cdots & \cdots \end{pmatrix} * 
    \begin{pmatrix} \cdots & b_{1j} & \cdots \\
        \vdots & \vdots & \vdots \\
        \cdots & b_{nj} & \cdots 
\end{pmatrix} = 
        \begin{pmatrix} \cdots & \cdots & \cdots \\
        \vdots & c_{ij} & \vdots \\
    \cdots & \cdots & \cdots \end{pmatrix}
.\] 

Чтобы получить эл-т $c_{ij}$ матрицы $C$, нужно умножить i-ую строку $A$ на j-ую строку $B$

\begin{example}
\[
    \begin{pmatrix} 1 & -1 & 2 \\ 4 & 5 & 3 \end{pmatrix} \begin{pmatrix} 1 & 4 & 2 \\ 2 & 5 & 7 \\ 3 & -1 & -3 \end{pmatrix} = \begin{pmatrix}  5 & -3 & -1 \\ 23 & 38 & 34  \end{pmatrix}
.\] 
\end{example}

\begin{statement}[О св-вах опер. транспонирования]
Операция транспонирования матрицы обладает св-вами.
\begin{enumerate}
    \item $(A^{T})^{T} = A$
    \item $(\lambda A) ^{T} = \lambda A^{T}$
    \item $(A + B)^{T} = A^{T} + B^{T}$
    \item Св-во транспон. произв-я:
        \[
            (A * B)^{T} = B^{T}A^{T}
        .\] 
\end{enumerate}
\end{statement}

\begin{proof}
\textbf{4)} Пусть матрица $A \in M_{m * n}, B \in M_{n * k}$. $AB \in M_{n * k} \Rightarrow (AB)^{T}\in M_{k * m}$ 

$B^{T} \in M_{k * n}, A^{T} \in M_{n * m} \Rightarrow B^{T}A^{T} \in M_{k * m}$ 

\[
    [(A * B)^{T}]_{ij} = [AB]_{ji} = \sum_{s = 1}^{n}a_{js}b{si} = \sum_{s = 1}^{n}b_{si}a_{js} =
\] 
\[
    = \sum_{s = 1}^{n}[B^{T}]_{is}[A^{T}]_{sj} = [B^{T}A^{T}]
.\] 
\end{proof}

\[
    (A * B * C)^{T} = C^{T} * B^{T} * A^{T}
.\] 

\begin{theorem} (О св-вах опер. "*" и "+")
\begin{enumerate}
    \item Ассоциативность умножения:
        \[
            (A * B) * C = A * (B * C)
        .\] 
    \item Левая дистрибутивность умножения отн-но сложение
        \[
        A * (B + C) = A * B + A * C
        .\] 
    \item Правая дистрибутивность умн. отн. слож:
        \[
            (A + B) * C = A * C + B * C
        .\] 
\end{enumerate}

\end{theorem}

\begin{proof}
    \textbf{1)}
    
    \[
        A \in M_{m * n}, B \in M_{n * k}, C \in M_{k * r}
    \]

Правая и левая часть, очев., имеют смысл.

\[
[(AB) * C]_{ij} = \sum_{s = 1}^{k} [AB]_{is}[C]_{sj} = \sum_{s = 1}^{k} (\sum_{t = 1}^{n} a_{it}b_{ts}) * c_{sj} = \sum_{s = 1}^{k} \sum_{t = 1}^{n}  a_{it} b_{ts} * c_{sj} = 
.\] 
\[
= \sum_{s = 1}^{k} a_{it} \sum_{t = 1}^{n} b_{ts} * c_{sj} = \sum_{s = 1}^{k} [A]_{it}[BC]_{tj} = [A(BC)]_{ij}
.\] 
\end{proof}

\begin{note}
Умножение матриц \textbf{некоммутативно}:
\[
AB \neq BA
.\] 
\begin{example}
\[
    \begin{pmatrix} 1 & -1 \\ 1 & -1 \end{pmatrix}
    \begin{pmatrix} 1 & 1 \\ 1 & 1 \end{pmatrix} =
    \begin{pmatrix} 0 & 0 \\ 0 & 0 \end{pmatrix}
.\] 
- это пример \textbf{делителя нуля}

\[
    \begin{pmatrix} 1 & 1 \\ 1 & 1 \end{pmatrix}
    \begin{pmatrix} 1 & -1 \\ 1 & -1 \end{pmatrix} = 
    \begin{pmatrix} 2 & -2 \\ 2 & -2 \end{pmatrix}
.\] 
\end{example}

\end{note}

\begin{definition}
Матрица $\triangle \in M_{n * n}$ наз-ся \textbf{диагональной}, если:

\[
    \triangle = \begin{pmatrix} \alpha_1 & \cdots & 0 \\
    0 & \alpha_i & 0 \\
0 & \cdots & \alpha_n\end{pmatrix}, ([\triangle]_{ij} = 0, i \neq j) 
\]
\end{definition}

\begin{statement}
    ~\newline
    
\begin{itemize}
    \item[a) ] Умножение матрицы $A$ слева на матрицу $\triangle$, если это возм.,
        \[
        [\triangle A]
        .\]
        равносильно умнож строк матрицы $A$ на числа $\alpha_1, \alpha_2, \cdots, \alpha_n$ соотв.
    \item[b) ] Умнож. $A$ справа на $\triangle$, если это возм.
        \[
        [A * \triangle ]
        .\] 
        равносильно умножению столбцов $A$ на числа $\alpha_1, \alpha_2, \cdots, \alpha_n$, соотв. 
\end{itemize}
\end{statement}




\end{document}
