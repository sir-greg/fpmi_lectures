\documentclass[12pt]{article}
\usepackage[T1, T2A]{fontenc}
\usepackage[utf8]{inputenc}
\usepackage[russian]{babel}
\usepackage{amsmath}
\usepackage{amsthm}
\usepackage{amssymb}
\usepackage{esvect}
\usepackage{listings}
\usepackage{xcolor}
\usepackage{mathrsfs}
\usepackage{braket}
\usepackage{physics}
\usepackage{xfrac}

% for large comments
\usepackage{blindtext, xcolor}
\usepackage{comment}

% for inkscape pictures
\usepackage{import}
\usepackage{pdfpages}
\usepackage{transparent}
\usepackage{xcolor}

\DeclareMathOperator{\inter}{int}
\DeclareMathOperator{\sign}{sign}

% for systems of equations
\newenvironment{system_and}%
{\left\lbrace\begin{array}{@{}l@{}}}%
{\end{array}\right.}
% for unions of equations
\newenvironment{system_or}%
{\left\lbrack\begin{array}{@{}l@{}}}%
{\end{array}\right.}

\newcommand{\incfig}[2][1]{%
    \def\svgwidth{#1\columnwidth}
    \import{./figures/}{#2.pdf_tex}
}

\pdfsuppresswarningpagegroup=1

\renewcommand{\C}{\mathbb{C}}
\newcommand{\R}{\mathbb{R}}
\newcommand{\Q}{\mathbb{Q}}
\newcommand{\Z}{\mathbb{Z}}
\newcommand{\N}{\mathbb{N}}

\newcommand{\floor}[1]{\left\lfloor #1 \right\rfloor}
\newcommand{\ceil}[1]{\left\lceil #1 \right\rceil}

% style of code listings
%\definecolor{codegreen}{rgb}{0,0.6,0}
%\definecolor{codegray}{rgb}{0.5,0.5,0.5}
%\definecolor{codepurple}{rgb}{0.58,0,0.82}
%\definecolor{backcolour}{rgb}{0.95,0.95,0.92}
%
%\lstdefinestyle{mystyle}{
%    backgroundcolor=\color{backcolour},
%    commentstyle=\color{codegreen},
%    keywordstyle=\color{magenta},
%    numberstyle=\tiny\color{codegray},
%    stringstyle=\color{codepurple},
%    basicstyle=\ttfamily,
%    breakatwhitespace=false,
%    breaklines=true,
%    captionpos=b,
%    keepspaces=true,
%    numbers=left,
%    numbersep=5pt
%    showspaces=false,
%    showstringspaces=false,
%    showtabs=false,
%    tabsize=4
%}

\newtheorem{theorem}{\underline{Теорема}}[section]
\newtheorem{lemma}[theorem]{\underline{Лемма}}
\newtheorem{statement}{\underline{Утверждение}}[section]
\newtheorem{axiom}{\underline{Аксиома}}[section]
\newtheorem*{note}{\underline{Замечание}}
\newtheorem*{symb}{\underline{Обозначение}}
\newtheorem*{example}{\underline{Пример}}
\newtheorem*{consequence}{\underline{Следствие}}
\newtheorem*{solution}{\underline{Решение}}

\theoremstyle{definition}
\newtheorem{definition}{\underline{Определение}}[section]

\theoremstyle{definition}
\newtheorem{task}{\underline{Задача}}[section]

\title{Матан}
\author{Сергей Григорян}
