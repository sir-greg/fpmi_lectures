\section{Лекция 5}

\begin{example}
\[
  a_n = \frac{1}{n^{2} + 1} + \frac{2}{n^{2} + 2} +\cdots + \frac{n}{n^{2} + n}, n \in \N
\] 
\[
  \frac{1 + 2 + \cdots + n}{n^{2} + n} \leq a_n \leq \frac{1 + 2 + \cdots + n}{n^{2} + 1}
\] 
\[
  \frac{n(n + 1)}{2(n^{2} + n)} \leq a_n \leq \frac{n(n + 1)}{2(n^{2} + 1)}
\] 
\[
  \frac{1}{2} \leq a_n \leq \frac{1 + \frac{1}{n}}{2 + \frac{2}{n^{2}}} (\frac{2}{n^{2}} \rightarrow 0, \frac{1}{n} \rightarrow 0)
\] 
\[
\Rightarrow \lim_{n\to\infty}a_n = \frac{1}{2}
\] 
\end{example}

\begin{definition}
Посл-ть $\{\alpha_n\}_{1}^{\infty}$ наз-ся \textbf{беск. малой}, если
\[
  \lim_{n\to\infty}\alpha_n = 0
\]
\end{definition}
\begin{note}
\[
\lim_{n\to\infty}a_n = a \iff a_n = a + \alpha_n, \text{где $\alpha_n$ - б. м.}
\]
\end{note}
\begin{example}
Пусть $\{\alpha_n\}_{1}^{\infty}$ - б. м., а $\{\beta_n\}_{1}^{\infty}$ - огранич. Тогда: $\{\alpha_n \beta_n\}_{1}^{\infty}$ - б. м.
\end{example}
\begin{proof}
Т. к. $\{\beta_n\}_{1}^{\infty}$ - огр., то $\exists C > 0 \colon \forall n (|\beta_n| \leq C)$
\[
-C |\alpha_n| \leq \alpha_n \beta_n \leq C |\alpha_n|
\] 
Крайние части $\rightarrow 0$ $\Rightarrow$ По. т. о двух полицейских $\lim_{n\to\infty} \alpha_n \beta_n = 0$
\end{proof}

\subsection{Монотонные п-ти}

\begin{definition}
П-ть $\{a_n\}_{1}^{\infty}$ наз-ся \textbf{нестрого возрастающей} (\textbf{строго возрастающей}), если \[
a_n \leq a_{n + 1} (a_n < a_{n + 1}), \forall n \in \N
\] 

П-ть $\{a_n\}_{1}^{\infty}$ наз-ся \textbf{нестрого убывающей} (\textbf{строго убыв.}), если:
\[
a_n \geq a_{n + 1} (a_n > a_{n + 1}), \forall n \in \N
\] 

Такие п-ти наз-ся \textbf{монотонными}.
\end{definition}

\begin{note}
Из опр-я следует, что $\{a_n\}_{1}^{\infty}$ нестрого возрастает $\iff$ $\{-a_n\}_{1}^{\infty}$ нестрого убывает.
\end{note}
\begin{note}
Если $a_n \leq a_{n + 1}, \forall n \in N \Rightarrow \forall n, m (n < m \rightarrow a_n \leq a_m)$
\end{note}
\begin{theorem}[Теорема о пределе монотонной п-ти]
ПУсть $\{a_n\}_{1}^{\infty}$ нестрого возрастает и огр. сверху, тогда $\{a_n\}_{1}^{\infty}$ сходиться и $\lim_{n\to\infty} a_n = \sup \{a_n\}_{1}^{\infty}$

Пусть $\{a_n\}_{1}^{\infty}$ нестрого убывает и огр снизу, тогда $\{a_n\}_{1}^{\infty}$ сходиться и $\lim_{n\to\infty} a_n = \inf \{a_n\}_{1}^{\infty}$
\end{theorem}
\begin{proof}
Док-ем первое утв. По условию $\exists c = \sup \{a_n\}_{1}^{\infty} \in \R$.

Зафикс. $\varepsilon > 0$. По опр. супремума $\forall n (a_n \leq c)$, также:
\[
\exists N (a_N > c - \varepsilon)
\] 
Поскольку $\{a_n\}_{1}^{\infty}$ нестрого возр., то $a_n \geq a_N, \forall n \geq N \Rightarrow$ при всех таких $n \geq N$ имеем:
\[
a_N \leq a_n
\] 
\[
c - \varepsilon < a_N \leq a_n \leq c < c + \varepsilon, \text{откуда}
\]
\[
|a_n - c| < \varepsilon
\] 
\[
  \Rightarrow \lim_{n\to\infty}a_n = c
\]
Второе утв. док-ся аналогично.
\end{proof}

\begin{lemma}[Нер-во Бернулли]
Если $n \in \N$ и $x \geq -1$, то:
\[
  (1 + x)^{n} \geq 1 + nx
\] 
\end{lemma}
\begin{proof}
\textbf{МММ}:
\begin{itemize}
  \item [n = 1] Верно.
  \item [$n \Rightarrow n + 1$] Пусть утв. верно для $n$. Тогда:
    \[
      (1 + x)^{n + 1} = (1 + x)(1 + x)^{n} \geq (1+x)(1 + nx) = 1 + (n + 1)x + n x^{2} \geq 1 + (n + 1)x
    \] 
\end{itemize}
\end{proof}

\begin{example}
Для $\forall x \in \R$ п-ть $a_n = (1 + \frac{x}{n})^{n}$ сходится.
\end{example}
\begin{proof}
Зафикс. $m \in \N$, что $m \geq |x|$. Тогда при:
\[
  n \geq m \colon a_n(x) > 0,
\]
а также:
\[
  \frac{a_{n + 1}(x)}{a_n(x)} = \frac{(1 + \frac{x}{n + 1})^{n + 1}}{(1 + \frac{x}{n})^{n}} = (1 + \frac{x}{n})\left(\frac{1 + \frac{x}{n} - \frac{x}{n} + \frac{x}{n + 1}}{1 + \frac{x}{n}}\right)^{n + 1} = (1 + \frac{x}{n})\left(1 - \frac{\frac{x}{n(n + 1)}}{1 + \frac{x}{n}}\right)^{n + 1}
\] 
Исследуем: $(- \frac{\frac{x}{n(n + 1)}}{1 + \frac{x}{n}})$. Она:
\begin{equation*}
\begin{system_and}
> 0, x < 0 \\
\geq -1, x \geq 0
\end{system_and}
\end{equation*}

По нер-ву Бернулли:
\[
(1 + \frac{x}{n})\left(1 - \frac{\frac{x}{n(n + 1)}}{1 + \frac{x}{n}}\right)^{n + 1} \geq (1 + \frac{x}{n})(1 - \frac{\frac{x}{n}}{1 + \frac{x}{n}}) = 1
\] 

Итак, $\{a_n\}_{1}^{\infty}(x)$ нестрого возр. при $n \geq m$. По доказанному $a_n(-x) \geq a_m(-x)$, при $n \geq m$.

Т. к.
\[
  a_n(x) a_n(-x) = \left(1 - \frac{x^{2}}{n^{2}}\right)^{n} \leq 1,
\]
то:
\[
  a_n(x) \leq \frac{1}{a_n(-x)} \leq \frac{1}{a_m(-x)}, \text{т. е.}
\]
$\{a_n\}_{1}^{\infty}$ огр. сверху.

Сл-но, по теореме о пределе монот. посл-ти. $\{a_n(x)\}_{1}^{\infty}$ сход-ся.
\end{proof}
\begin{definition}
\[
e = \lim_{n\to\infty} \left(1 + \frac{1}{n}\right)^{n}
\] 
\end{definition}
\begin{task}
Док-те, что $2 < e < 3$
\end{task}

\subsection{Последовательность вложенных отрезков}

\begin{definition}
П-ть отрезков $\{[a_n, b_n]\}_{1}^{\infty}$ наз-ся \textbf{вложенной}, если $\forall n \in \N ([a_{n + 1}, b_{n + 1}] \subset [a_n, b_n])$

Если к тому же, $b_n - a_n \rightarrow 0$, то п-ть $\{[a_n, b_n]\}_{1}^{\infty}$ наз-ся \textbf{стягивающейся}.

\end{definition}
\begin{theorem}[Кантор]
Всякая п-ть вложенных отрезков имеет общую точку. Если п-ть стягивающаяся, то такая точка единственная.
\end{theorem}

\begin{proof}
Пусть задана п-ть $\{[a_n, b_n]\}_{1}^{\infty}$ вложенных отр-ов. Тогда:
\[
\forall n \in \N \colon  a_1 \leq a_n \leq a_{n + 1} \leq b_{n + 1} \leq b_n \leq b_1
\] 

П-ть $\{a_n\}_{1}^{\infty}$ нестрого возр. и огр. сверху (числом $b_1$).

П-ть $\{b_n\}_{1}^{\infty}$ нестрого убыв. и огр. снизу (числом $a_1$)

$\Rightarrow$ $\{a_n\}_{1}^{\infty}, \{b_n\}_{1}^{\infty}$ сход., $a_n \rightarrow \alpha, b_n \rightarrow \beta$ и $\alpha \leq \beta$.

Итак $\forall n (a_n \leq \alpha \leq \beta \leq b_n)$, т. е.:
\[
[\alpha, \beta] \subset \bigcap_{n = 1}^{\infty} [a_n, b_n]
\] 

Если п-ть $\{[a_n, b_n]\}_{1}^{\infty}$ - стягив., то $b_n - a_n \rightarrow 0$

Пусть $x, y \in \bigcap_{n = 1}^{\infty} [a_n, b_n]$, тогда $|x - y| \leq b_n - a_n \Rightarrow x = y$

Т. е. $\bigcap_{n = 1}^{\infty} [a_n, b_n] = x$, где $x = \alpha = \beta$. 
\end{proof}

