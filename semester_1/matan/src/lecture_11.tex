\section{Лекция 11}
\subsection{Непрерывность ф-ции в точке}
\begin{definition}
Пусть $E \subset \R, a \in E$ и $f: E \rightarrow \R$. Ф-ция $f$ наз-ся \underline{непрерывной} в точке $a$, если:
\[
\forall \varepsilon > 0, \exists \delta > 0, \forall x \in E (\left|x - a\right| < \delta \Rightarrow \left|f(x) - f(a)\right| < \varepsilon)
\]
Иначе:
\[
x \in B_{\delta}(a) \Rightarrow f(x) \in B_{\varepsilon}(f(a))
\]
\end{definition}
\begin{note}
Из опр-я следует, что ф-ция не меняет значение \textbf{резко}
\end{note}
\textbf{Св-во (отделимость)}: если $f: E \rightarrow \R$ - непр-на в точке $a$ и $f(a) > 0 (< 0)$, то
\[
  \exists \delta > 0, \forall x \in B_{\delta}(a) \cap E (f(x) > \frac{f(a)}{2} (< \frac{f(a)}{2}))
\]
\begin{proof}
Пусть $f(a) > 0$. По непр-ти ф-ции в $a$, положим $\varepsilon = \frac{f(a)}{2}$. Тогда
\[
  \exists \delta > 0, \forall x \in B_{\delta}(a) \cap E (f(a) - \frac{f(a)}{2} < f(x) < f(a) + \frac{f(a)}{2}) \Rightarrow f(x) > \frac{f(a)}{2}
\]
\end{proof}
\begin{note}
В определении непр-ти ф-ции точка $a \in E \text{ - области определения}$, но \textbf{не обязана} быть предельной точкой $E$.
\end{note}
\begin{definition}
Точка, принадлежащая мн-ву, но не явл-ся его предельной точкой наз-ся \textbf{изолированной}.
\end{definition}
\begin{example}
\[
  E = (1, 2] \cup \set{5}
\]
Тогда точка $5$ - изолированная точка мн-ва $E$
\end{example}
\begin{theorem}
\label{th:num4}
Пусть $f: E \rightarrow \R, a \in E$. Следующие утв-я эквив-ны:
\begin{itemize}
  \item [1) ] $f$ непр-на в $a$
  \item [2) ] $\forall \set{x_n}, x_n \in E (x_n \rightarrow a \Rightarrow f(x_n) \rightarrow f(a))$
  \item [3) ] Либо $a$ - изолированная точка мн-ва $E$, либо $a$ - предельная точка мн-ва $E$ и $\lim_{x\to a} f(x) = f(a)$
\end{itemize}
\end{theorem}
\begin{proof}
  ~\newline
\begin{itemize}
  \item [1 $\Rightarrow$ 2)]  Рассм. $\set{x_n}, x_n \in E, x_n \rightarrow a$. Заф. $\varepsilon > 0$. По опр-ю непр-ти
    \[
      \exists \delta > 0, \forall x \in B_{\delta}(a) \cap E (\left|f(x) - f(a)\right| < \varepsilon)
    \]
    Т.к. $x_n \rightarrow a$, то $\exists N, \forall n \geq N (x_n \in B_{\delta}(a) \cap E)$, а значит,
    \[
      \left|f(x_n) - f(a)\right| < \varepsilon, \forall n \geq N
    \]
    Сл-но, $f(x_n) \rightarrow f(a)$
  \item [2 $\Rightarrow$ 3)] Если $a$ - предельная точка мн-ва $E$, то $\lim_{x\to a} f(x) = f(a)$, по опр-ю предела по Гейне. \\

    В противном случае, $a$ - изолированная точка области определения.
  \item [3 $\Rightarrow$ 1)] Если $a$ - изолированная точка мн-ва $E$, то $\exists \delta_0 > 0 \colon (B_{\delta_0}(a) \cap E = \set{a})$. Тогда опредение непр-ти выполняется для $\delta = \delta_0$. \\

    Если $a$ - предельная точка мн-ва $E$, то по опр-ю предела по Коши:
    \[
    \forall \varepsilon > 0, \exists \delta > 0, \forall x \in E (0 < \left|x - a\right| < \delta \Rightarrow \left|f(x) - f(a)\right| < \varepsilon)
    \]
    При $x = a$, следствие выше выпол-ся (очевидно). Это означает, что $f$ непр-на в $a$.
    
\end{itemize}
\end{proof}
\begin{consequence}
\label{cs:ar_op_nep}
Если $f, g \colon E \rightarrow \R$ - непр-ны в $a \in E$, то в этой точке непр-ны ф-ции:
\begin{itemize}
  \item [1) ] $f \pm g$
  \item [2) ] $f \cdot g$ \\
  \item [3) ] При доп. усл-ии $g \neq 0$: $\frac{f}{g}$
\end{itemize}
\end{consequence} 
\begin{proof}
Рассм. произвольную п-ть $\set{x_n}, x_n \in E, x_n \rightarrow a$. Т. к. $f, g$ - непр-ны в $a$, то $f(x_n) \rightarrow f(a)$ и $g(x_n) \rightarrow g(a)$. Тогда по св-вам предела п-ти имеем:
\begin{itemize}
  \item [1) ] $f(x_n) \pm g(x_n) \rightarrow f(a) \pm g(a)$
  \item [2) ] $f(x_n)g(x_n) \rightarrow f(a)g(a)$
  \item [3) ] $\frac{f(x_n)}{g(x_n)} \rightarrow \frac{f(a)}{g(a)}$
\end{itemize}
По Теореме $(\ref{th:num4})$, эти ф-ции непрерывны в $a$.
\end{proof}
\begin{example}
\[
P(x) = a_n x^{n} + a_{n - 1}x^{n - 1} + a_{n - 2}x^{n - 2} + \ldots + a_0, (a_i \in \R)
\]
Эта ф-ция непр-на в каждой точке $a \in \R$
\end{example}
\begin{proof}
\[
x \mapsto x
\]
\[
x \mapsto c, c \in \R
\]
Ф-ции выше непрерывны. Тогда по сл-ию $(\ref{cs:ar_op_nep})$ в $a$ непр-ны:
\[
  x \mapsto x^{k}, k \in \N
\]
А значит $P$ непр-на в $a$.
\end{proof}
\begin{theorem}[Непрерывность композиции]
  \label{th:nepr_comp}
Если ф-ция $f: E \rightarrow \R$ непр-на в $a$, $f(E) \subset D, g: D \rightarrow \R$ - непр-на в $b = f(a)$, то композиция $g \circ f \colon E \rightarrow \R$ непр-на в т. $a$.
\end{theorem}
\begin{proof}
  Рассм. произвольную п-ть $\set{x_n}, x_n \in E, x_n \rightarrow a$. Тогда: $f(x_n) \rightarrow f(a)$ по непр-ти $f$ в $a$. Кроме того:
  \[
  g(f(x_n)) \rightarrow g(f(a)) \text{ - по непр-ти $g$ в $f(a)$} \iff
  \]
  \[
    (g \circ f)(x_n) \rightarrow (g \circ f)(a)
  \]
  По Теореме $\ref{th:num4}$, ф-ция $g \circ f$ непр-на в т. $a$.
\end{proof}
\begin{definition}
Пусть $f: E \rightarrow \R$ и $a \in E$. Если $f|_{[a, +\infty)}$ непр-но в $a$, то говорят, что $f$ \underline{непр-на справа в т. $a$}. \\

Аналогично: $f|_{(-\infty, a]}$ непр-на в $a$, то $f$ \underline{непр-на слева в $a$.}
\end{definition}
\begin{note}
Если $a$ - предел. точка мн-ва $[a, +\infty) \cap E$, то $f$ непр. справа в т. $a \iff f(a + 0) = f(a)$
\end{note}
