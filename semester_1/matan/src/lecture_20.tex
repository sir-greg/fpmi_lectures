\section{Лекция 20}
\subsection{Формула Тейлора}
\begin{definition}
Пусть $n \in \N$ и ф-ция $f$ дифф-ма в $n$ раз в т. $a$. Тогда:
\[
f(x) = \sum_{k = 0}^{n} \frac{f^{(k)}(a)}{k!}(x - a)^{k} + r_n(x)
\]
Наз-ся \textbf{ф-лой Тейлора порядка $n$ ф-ции $f$ в т. $a$}. При этом
\[
  P_n(x) = P_{n, f, a}(x) = \sum_{k = 0}^{n} \frac{f^{(k)}(a)}{k!}(x - a)^{k}
\]
наз-ся \textbf{многочленом Тейлора}, а $r_n(x) = r_{n, f, a}(x)$ \textbf{--- остаточным членом.}
\end{definition}
\begin{example}
Если $P_n(x) = \sum_{k = 0}^{n} c_k (x - a)^{k}$ и $0 \leq m \leq n$, то:
\[
P_n^{(m)}(x) = \sum_{k = m}^{n} c_k\frac{k!}{(k - m)!} (x - a)^{k - m}
\]
поэтому:
\[
P_n^{(m)}(a) = m! c_m
\]
\[
P_n(x) = \sum_{k = 0}^{n} \frac{P_n^{(k)}(a)}{k!} (x - a)^{k}
\]
--- ф-ла Тейлора для мн-на $P_n$.
\end{example}
\begin{theorem}[Остаточный член в форме Пеано]
Пусть $n \in \N$ и ф-ция $f$ дифф-ма $n$ раз в $a$. Тогда
\[
  f(x) = \sum_{k = 0}^{n} \frac{f^{(k)}(a)}{k!}(x - a)^{k} + o((x - a)^{k}), x \rightarrow a
\]
\end{theorem}
\begin{proof}
Пусть $P_n(x) = \sum_{k = 0}^{n} \frac{f^{(k)}(a)}{k!}(x - a)^{k}$. Тогда
\[
  P_n^{(k)}(a) = f^{(k)}
\]
Поэтому для ост. члена $r_n(x) = f(x) - P_n(x)$. Выполнено:
\[
r_n(a) = r'_n(a) = \ldots = r_n^{(n)}(a) = 0
\]
По правилу Лопиталя:
\[
\lim_{x\to a} \frac{r_n(x)}{(x - a)^{n}} = \lim_{x\to a} \frac{r_n'(x)}{n(x - a)^{n - 1}} = \ldots = \lim_{x\to a} \frac{r_n^{(n - 1)}(x)}{n!(x - a)}
\]
Этот предел сущ-ет по опр-ю пр-й:
\[
\lim_{x\to a} \frac{r_n^{(n - 1)}(x)}{n!(x - a)} = \frac{1}{n!} \lim_{x\to a} \frac{r_n^{(n - 1)}(x) - r_n^{n - 1}(a)}{x - a} = \frac{1}{n!} r_n^{(n)}(a) = 0
\]
\end{proof}
\begin{consequence}
Пусть $n \in \N$, ф-ция $f$ дифф-ма $n$ раз в т. $a$ и $f'(a) = f''(a) = \ldots = f^{(n -1 )}(a) = 0$, но $f^{(n)}(a) != 0$. Тогда:
\begin{itemize}
  \item [1) ] Если $n$ чётно и $f^{(n)}(a) > 0, (f^{(n)}(a) < 0)$, то $a$ явл-ся точкой лок. минимума ф-ции $f$ (точкой лок. максимума).
  \item [2) ] Если $n$ нечётно, то $a$ не явл-ся точкой лок. экстремума ф-ции $f$.
\end{itemize}
\end{consequence}
\begin{proof}
По пред. теореме имеем:
\[
f(x) = f(a) + \ldots + \frac{f^{(n)}(a)}{n!}(x - a)^{n} + o((x - a)^{n}), x \rightarrow a
\]
или:
\[
f(x) - f(a) = \left(\frac{f^{(n)}(a)}{n!} + \alpha(x)\right)(x - a)^{n}
\]
Где,  $\alpha(x) \rightarrow 0$ при $x \rightarrow a$. Из стрем-я к $0$:
\[
\exists \delta > 0, \forall x \in \overset{\circ}{B_{\delta}}(a) \left(\left|\alpha(x)\right| < \left|\frac{f^{(n)}(a)}{n!}\right|\right)
\]
Тогда $\forall x \in \overset{\circ}{B_{\delta}}(a) \left(sign(\frac{f^{(n)}(a)}{n!} + \alpha(x)) = sign f^{(n)}(a)\right)$ 
\[
\Rightarrow sign(f(x) - f(a)) = sign(f^{(n)}(a)(x - a)^{n})
\]
Откуда следует, заявленное утв-е. 
\end{proof}
\begin{theorem}[Единственность]
  Пусть для ф-ции $f$ найдутся мн-ны $p_1(x)$ и $p_2(x)$ степени $\leq n$, т. ч.
  \[
  f(x) - p_1(x) = o((x - a)^{n})
  \]
  \[
  f(x) - p_2(x) = o((x - a)^{n}), x \rightarrow a
  \]
  Тогда $p_1(x) = p_2(x)$:
\end{theorem}
\begin{proof}
Положим $q(x) = p_1(x) - p_2(x)$. Тогда
\[
  q(x) = o((x - a)^{n}), x \rightarrow a
\]
Покажем, что $q(x) \equiv 0$. \\

Пусть $q(x) = c_0 + c_1(x - a) + \ldots + c_n(x - a)^{n}$. Предположим, что не все его коор-ты $ = 0$. Пусть $j --- min \colon c_j \neq 0$. Сделаем преобразования:
\[
  q(x) = o((x - a)^{n}), x \rightarrow a
\]
\[
  c_j(x - a)^{j} + \ldots + c_n(x - a)^{n} = ((x - a)^{n}), x \rightarrow a
\]
\[
  c_j + c_{j + 1}(x - a) + \ldots + c_n (x - a)^{n - j} = ((x - a)^{n - j}), x \rightarrow a
\]
Перейдя к пределу, получим $c_j = 0$!!! $\Rightarrow q(x) \equiv 0$
\end{proof}
\begin{theorem}[Единственность представления ф-лой Тейлора]
  \label{one_taylor}
Если ф-ция $f$ дифф-ма $n$ раз в $a$ и
\[
f(x) = \sum_{k = 0}^{n} c_k (x - a)^{k} + o((x - a)^{n}), x \rightarrow a
\]
Тогда $c_k = \frac{f^{(k)}(a)}{k!}, k = \overline{0, n}$
\end{theorem}
\begin{consequence}
Пусть $f$ дифф-ма $(n + 1)$ раз в т. $a$ и
\[
  f'(x) = \sum_{k = 0}^{n} c_k (x - a)^{k} + o((x - a)^{n}), x \rightarrow a
\]
Тогда:
\[
f(x) = 
\]
\end{consequence}
\begin{proof}
Из пред-я $f'$ имеем по т. $\ref{one_taylor}$:
\[
c_k = \frac{(f')^{(k)}(a)}{k!} \Rightarrow f^{(k + 1)}(a) = k! c_k, k = \overline{0, n}
\]
\end{proof}
\begin{definition}
Формула Тейлора в т. $a = 0$ --- \textbf{ф-ла Маклорена}.
\end{definition}
Основные разложения:
\begin{itemize}
  \item [I) ] Если $f(x) = e^{x}$, то $f^{k}(0) = e^{0} = 1$, для $k \in \N_0$, тогда:
    \[
    e^{x} = \sum_{k = 0}^{n} \frac{x^{k}}{k!} + o(x^{n}), x \rightarrow 0
    \]
  \item [II) ] Если $f(x) = \sin x$, то по инд-ции проверяется, что $f^{(m)}(x) = \sin(x + \frac{\pi}{2}m), m \in \N_0$. Значит, $f^{(2k)}(0) = 0, f^{(2k + 1)}(0) = (-1)^{k}, k \in \N_0$. След-но:
    \[
    \sin x = \sum_{k = 0}^{n} \frac{(-1)^{k}}{(2k + 1)!} x^{2k + 1} + o(x^{2n + 2}), x \rightarrow 0
    \]
  \item [III)] Если $f(x) = \cos x$, то по инд-ции уст-ся, что $f^{(n)}(x) = \cos(x + \frac{\pi}{2}m), m \in \N_0$. Поэтому $f^{(2k + 1)}(0) = 0, f^{(2k)} = (-1)^{k}, k \in \N_0$. Получаем:
    \[
    \cos x = \sum_{k = 0}^{n} \frac{(-1)^{k}}{(2k)!} x^{2k} + o(x^{2n + 1}), x \rightarrow 0
    \]
  \item [IV)] Если $f(x) = (1 + x)^{\alpha}, \alpha \in \R$, то
    \[
      f^{(k)}(x) = \alpha(\alpha - 1)(\alpha - 2)\ldots(\alpha - k + 1)(1 + x)^{\alpha - k}, k \in \N_0
    \]
    Положим:
    \[
    C_{\alpha}^{0} = 1, C_{\alpha}^{k} = \frac{\alpha(\alpha - 1)\ldots(\alpha - k + 1)}{k!}
    \]
    \[
      (1 + x)^{\alpha} = \sum_{k = 0}^{n} C_{\alpha}^{k} x^{k} + o(x^{n}), x \rightarrow 0
    \]
    В част-ти, $\frac{1}{1 + x} = \sum_{k = 0}^{n} (-1)^{k}x^{k} + o(x^{n}), x \rightarrow 0$
  \item [V)] Если $f(x) = \ln(1 + x)$, то $f(0) = 0$:
    \[
    f^{(k)}(x) = \frac{(-1)^{k - 1} (k - 1)!}{(1 + x)^{k}}, k \in \N
    \]
    Получаем:
    \[
    \ln(1 + x) = \sum_{k = 1}^{n} \frac{(-1)^{k - 1}}{k} x^{k} + o(x^{n}), x \rightarrow 0
    \]
\end{itemize}
\begin{task}
Представить ф-лой Маклорена ф-цию $f(x) = \arctg x$ до $o(x^{2n + 2})$
\end{task}
\begin{task}
  До $o(x^{3})$, представить ф-лой Маклорена:
\[
f(x) = \ln(1 + \sin x)
\]
\[
  \ln(1 + w) = w - \frac{w^{2}}{2} + \frac{w^{3}}{3} + o(w^{3})
\]
\[
  \lim_{w\to 0} \frac{\ln(1 + w) - p(w)}{w^{3}} = 0, w = \sin x
\]
\[
  \ln(1 + \sin x) = \sin x - \frac{\sin^{2} x}{2} + \frac{\sin^{3} x}{3} + o(\sin^{3} x)
\]
\[
  \sin x \sim x \Rightarrow o(\sin^{3} x) = o(x^{3})
\]
\[
  \sin x = x - \frac{x^{3}}{6} + o(x^{3}), x \rightarrow 0
\]
\[
  \sin^{2} = x^{2} + 2x o(x^{2}) + o(x^{4}) = x^{2} + o(x^{3})
\]
  \[
  \sin^{3} = (x + o(x))^{3} = x^{3} + o(x^{3})
  \]
  \[
  \ln(1 + \sin x) = (x - \frac{x^{3}}{6} + o(x^{3})) - \frac{1}{2}(x^{2} + o(x^{3})) + \frac{1}{3} (x^{3} + o(x^{3})) + o(x^{3}) = x - \frac{1}{2}x^{2} + \frac{1}{6}x^{3} + o(x^{3})
  \]
\end{task}
