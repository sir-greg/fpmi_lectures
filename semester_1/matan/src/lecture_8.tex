\section{Лекция 8}

\subsection{\textsection 3: Топология $\R$}
Пусть $a \in \R$ и $\varepsilon > 0$.
\begin{symb}
  \begin{itemize}
    \item
$B_{\varepsilon}(a) = (a - \varepsilon, a + \varepsilon)$ - $\varepsilon$-окрестность $a$
  \item
    $\overset{\circ}{B_{\varepsilon}}(a) = B_{\varepsilon}(a) \backslash \set{a} = (a - \varepsilon, a) \cup (a, a + \varepsilon)$ - проколотая $\varepsilon$-окр-ть т. a
  \end{itemize}
\end{symb}
\begin{definition}
Пусть $E \subset \R$ и $x \in \R$
\begin{itemize}
  \item [1) ] Точка $x$ наз-ся \underline{внутренней точкой} мн-ва $E$, если $\exists \varepsilon > 0 (B_{\varepsilon}(x) \subset E)$ \\

    $(int) E$ - мн-во всех внут. точек $E$
  \item [2) ] Точка $x$ наз-ся \underline{внешней точкой} мн. $E$, если $\exists \varepsilon > 0 (B_{\varepsilon}(x) \subset \R \backslash E)$
    $(ext) E$ - мн-во внешних точек $E$
  \item [3) ] Точка $x$ наз-ся \underline{граничной точкой} мн-ва $E$, если
    \[
      \forall \varepsilon > 0 (B_{\varepsilon}(x) \cap E \neq \emptyset \land B_{\varepsilon}(x) \cap (\R \backslash E) \neq \emptyset)
    \]
    $\sigma E$ - мн-во всех граничных точек $E$
\end{itemize}
\end{definition}
\begin{note}
Из опр-я следует:
\[
\R = (int) E \sqcup (ext) E \sqcup \sigma E
\]
\end{note}
\begin{example}
  \[
    E = (0, 1], (int) E = (0; 1), (ext) E = (-\infty; 0) \cup (1; +\infty), \sigma E = \set{0, 1}
  \]
\end{example}
\begin{definition}
Мн-во $G \subset \R$ наз-ся \textbf{открытым}, если все его точки яв-ся внутренними (т. е. $G = (int) G$)
\end{definition}
\begin{definition}
Мн-во $F \subset \R$ наз-ся \textbf{замкнутым}, если $\R \backslash F$ - открыто.
\end{definition}
\begin{example}
\begin{itemize}
  \item [1) ] $(a, b)$ - открытое.
  \item [2) ] $[a, b]$ - замкнутое.
\end{itemize}
\end{example}
\begin{lemma}
  \begin{itemize}
    \item [a) ]
Объединение любого семейства открытых мн-в открыто.
    \item [b) ] Пересечение конечного сем-ва открытых мн-в открыто.
    \item [c) ] $\R, \emptyset$ - открыты
  \end{itemize}
\end{lemma}
\begin{proof}
  \begin{itemize}
    \item [a) ] Пусть $\set{G_\lambda}_{\lambda \in \Lambda}$ - семейство открытых мн-в.
      \[
        G = \bigcup_{\lambda \in \Lambda}^{}G_\lambda \text{ и } x \in G
      \]
      По опр-ю:
      \[
        \exists \lambda_0 \in \Lambda (x \in G_\lambda\text{ - открыто}) \iff \exists \varepsilon > 0 \colon B_{\varepsilon}(x) \subset G_\lambda \subset G
      \]
      \\
      Сл-но, $B_{\varepsilon}(x) \subset G$, т. е. $x$ - внут. точка $G$
    \item [b) ] ПУсть $\set{G_k}_{k = 1}^{m}$ - семейство открытых мн-в, $G = \bigcap_{k = 1}^{m} G_k, x \in G$. По опр. пересечения:
      \[
     \forall k, x \in G_k \Rightarrow \forall k, \exists \varepsilon_k > 0 \colon B_{\varepsilon_k}(x) \subset G_k
      \]
      \[
      \varepsilon = \underset{1 \leq k \leq m}{min} \set{\varepsilon_k}
      \]
      Тогда $\varepsilon > 0$ и $B_{\varepsilon}(x) \subset B_{\varepsilon_k}(x) \subset G_k, \forall k \Rightarrow B_{\varepsilon}(x) \subset G = \bigcap_{k = 1}^{m}$, т. е. $x$ - внут. точка $G$
    \item [c) ] Открытость $\R, \emptyset$ следует из опр-я.
  \end{itemize}
\end{proof}
\begin{lemma}
\begin{itemize}
  \item [a) ] Объединение конечного семейства замкнутых мн-в замкнуто
  \item [b) ] Пересечение любого семейства замкнутых мн-в замкнуто
  \item [c) ] $\R, \emptyset$ - замкнуты
\end{itemize}
\end{lemma}
\begin{proof}
\begin{itemize}
  \item [a, b) ] \[
  \R \backslash (\bigcap_{\lambda \in \Lambda}^{}F_\lambda) = \bigcup_{\lambda \in \Lambda}^{} (\R \backslash F_\lambda).
  \]
  \[
  \R \backslash (\bigcup_{k = 1}^{m} F_k) = \bigcap_{k = 1}^{m} (\R \backslash F_k)
  \]
\item [c)] Очев.
\end{itemize}
\end{proof}
\begin{definition}
Пусть $E \subset \R$ и $x \in \R$. Точка $x$ наз-ся предельной точкой мн-ва $E$, если:
\[
\forall \varepsilon > 0 (\overset{\circ}{B_{\varepsilon}}(x) \cap E \neq \emptyset)
\]
\end{definition}
\begin{lemma}
Точка $x$ - предельная точка $\iff$
\[
  \exists \set{x_n}_{x_n \neq x} \subset E \colon (x_n \rightarrow x)
\]
\end{lemma}
\begin{proof}
\begin{itemize}
  \item [=>)]
    \[
      x_n \in \overset{\circ}{B_{\frac{1}{n}}}(x) \cap E, \forall n \Rightarrow x_n \neq x \text{ и } x_n \in E \Rightarrow x - \frac{1}{n} < x_n < x + \frac{1}{n} \Rightarrow x_n \rightarrow x
    \]
  \item [<=)] Зафикс. $\varepsilon > 0$. Тогда $\exists N \colon \forall n \geq N (x_n \in (x -\varepsilon, x + \varepsilon))$ \\
    Сл-но, $x_N \in \overset{\circ}{B_{\varepsilon}}(x) \cap E$
\end{itemize}
\end{proof}

\begin{theorem}[Критерий замкнутости]
Следующие утв. эквивалентны:
\begin{itemize}
  \item [1) ] $E$ - замкнуто;
  \item[2) ] $E$ содержит все свои граничные точки;
  \item [3) ] $E$ содержит все свои предельные точки;
  \item [4) ] Если п-ть $\set{x_n}$ точек из $E$ сходится к $x$, то $x \in E$
\end{itemize}
\label{th:zamk_krit}
\end{theorem}
\begin{proof}
  ~\newline
\begin{itemize}
  \item [1 => 2)] Пусть $x \in \R \backslash E \text{ (открытое) } \Rightarrow \exists B_{\varepsilon}(x) \subset \R \backslash E$, т. е. $x$ - внешняя точка $E$. Тогда $\sigma E \subset E$
  \item [2 => 3)] $E$ содержит все свои граничные точки. Рассм. 2 случая:
    \begin{itemize}
      \item [a) ] $x$ - внутренняя точка $\Rightarrow x \in E$
      \item [b) ] $x$ - граничная точка $E \Rightarrow x \in E$ - по усл. 2)
    \end{itemize}
  \item [3 => 4)] Пусть $\set{x_n} \subset E, x_n \rightarrow x$ \\

    Предположим, что $x \not\in E \overset{\text{Л2}}{\Rightarrow}$ $x$ - предельная точка $E$
  \item [4 => 1)] $x \in R \backslash E$. Предположим, что $x$ - не внутренняя точка $E$. Тогда:
    \[
    \forall n \colon B_{\frac{1}{n}}(x) \cap E \neq \emptyset
    \]
    Пусть $x_n \in B_{\frac{1}{n}}(x)$. Имеем $\set{x_n} \subset E \Rightarrow x_n \rightarrow x \in E !!!!!!$
\end{itemize}
\end{proof}
\begin{example}
Пусть $L$ - мн-во част. пределов числовой п-ть $\set{a_n}$. Покажем, что $L$ - замкнуто.
\end{example}
\begin{proof}
Пусть $\set{x_n} \subset L$, $x_n \rightarrow x$ \\

По опр-ю част. предела, найдётся строго возрастающая п-ть номеров $\set{n_k}$, что $\left|a_{n_k} - x_k\right| < \frac{1}{k}$ \\

Сл-но:
\[
\left|a_{n_k} - x\right| \leq \left|a_{n_k} - x_k\right| + \left|x_k - x\right| < \frac{1}{k} + \left|x_k - x\right|
\]
Т. е. $x \in L$, по эквив. п.1 и п. 4 (Теоремы \ref{th:zamk_krit}) заключаем, что $L$ - замкнуто.
\end{proof}
\begin{definition}
$\overline{E} = E \cup \sigma E $ - замыкание мн-ва $E$
\end{definition}
\begin{lemma}
Мн-во $\overline{E}$ является замкнутым. \\
Кроме того, $\overline{E} = E \cup \set{x \colon x \text{ - предельная точка $E$}}$
\end{lemma}
\begin{proof}
Пусть $x \in \R \backslash \overline{E} \Rightarrow x$ - внешн. точка $E$, т. е.
\[
  \exists B_{\varepsilon}(x) \subset \R \backslash E
\]
Если $B_{\varepsilon}(x) \cap \sigma E \neq \emptyset$, то $B_{\varepsilon}(x) \cap E \neq \emptyset !!!$ \\

Сл-но, $B_{\varepsilon}(x) \subset \R \backslash \overline{E}$, т. е. $x$ - внут. точка $\R \backslash \overline{E}$ \\

Вторая часть следует из наблюдений: \\

(1) любая предельная точка $E$ либо внутренняя, любо граничная. \\

(2) граничная точка $E$, не принадлежащая $E$, является предельной.
\end{proof}
\begin{task}
  \begin{itemize}
    \item [1) ] $x \in \overline{E} \iff \exists \set{x_n} \subset E (x_n \rightarrow x)$
    \item [2) ] $\overline{E} = \bigcap_{}^{} \set{F\colon F \text{ - замкнуто и $F \supset E$}}$
  \end{itemize}
\end{task}
