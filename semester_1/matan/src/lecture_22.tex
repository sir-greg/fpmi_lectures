\section{Лекция 22}
\begin{consequence}
Пусть $f$ непр-на на пром. $I$ и дважды дифф-ма на $\inter I$.
\begin{itemize}
  \item [1) ] Ф-ция $f$ выпукла вниз на $I$ $\iff f''(x) \geq 0, x \in \inter I$
  \item[2) ] Если $f''(x) > 0$ на $\inter I$, то $f$ строго выпукла вниз на $I$.
\end{itemize}
\end{consequence}
\begin{example}
\begin{itemize}
  \item [1) ] $y = a^{x}, a \neq 1$, строго выпукла вниз на $\R$, т. к.:
    \[
      (a^{x})'' = a^{x}\ln^{2} a > 0
    \]
  \item [2) ] $y = \ln x$, строго вогнута (выпукла вверх) на $(0, +\infty)$, т. к.:
    \[
      (\ln x)'' = -\frac{1}{x^{2}} < 0
    \]
  \item [3) ] $y = x^{p}$ на $(0, +\infty), p \in (-\infty, 0) \cup (1, +\infty)$
    \[
      (x^{p})'' = p(p - 1)x^{p - 2} > 0 \Rightarrow \text{ выпукла вниз}
    \]
    При $p \in (0, 1)$ --- вогнута.
  \item [4) ] $\ln(1 + x) < x$, при $x > -1, x \neq 0$:
    \[
    y = \ln(1 + x) \text{ --- строго выпукла вверх (вогнута) на $(-1, +\infty)$}
    \]
    \[
    y = x \text{ --- касат. к $x \mapsto \ln(1 + x)$ в точке $x = 0$}
    \]
    По т. $\ref{16s}$ получаем заявленное нер-во.
\end{itemize}
\end{example}
\begin{definition}
Пусть $f$ опр-на на пром-ке $I$ и $a \in \inter I$. Если:
\begin{itemize}
  \item [1) ] ф-ция $f$ имеет различный характер выпуклости на $(a - \delta, a], [a, a + \delta)$ для некот. $\delta > 0$
  \item [2) ] $\exists f'(a) \in \overline{\R}$
  \item [3) ] $f$ --- непр-на в $a$.
\end{itemize}
Тогда точка $a$ наз-ся \textbf{точкой перегиба} ф-ции $f$.
\end{definition}
\begin{consequence}
Если ф-ция $f$ дважды дифф-ма на $\inter I$ и $a \in \inter I$ --- точка перегиба $f$, то $f''(a) = 0$:
\end{consequence}
\begin{proof}
  Пусть для опр-ти $f$ выпукла вниз на $(a - \delta, a]$ и выпукла вверх (вогнута) на $[a, a + \delta), \delta > 0$. Тогда $f'$ нестрого возрастает на $(a - \delta, a]$ и $f'$ нестрого убывает на $[a, a + \delta)$. Следовательно $a$ --- точка локального максимума ф-ции $f'$. По Т. Ферма $f''(a) = 0$
\end{proof}
Выпуклость ф-ции гарантирует её "хорошее поведение" на $(a, b)$ \\
\textbf{Ключом является следующий факт:}
\begin{lemma}[Лемма о 3-ёх хордах]
  \label{lm:3hord}
  Пусть ф-ция $f$ выпукла вниз на $(a, b)$ и $a < x_1 < x < x_2 < b$. Тогда:
  \begin{equation}
    \label{*}
    \frac{f(x) - f(x_1)}{x - x_1} \leq \frac{f(x_2) - f(x_1)}{x_2 - x_1} \leq \frac{f(x_2) - f(x)}{x_2 - x}
  \end{equation}
\end{lemma}
\begin{proof}
Рассм. $\lambda(t) = \frac{f(x_2) - f(x_1)}{x_2 - x_1}(t - x_1) + f(x_1)$. Тогда $f(x_1) = \lambda(x_1), f(x_2) = \lambda(x_2)$ и ввиду выпуклости вниз $f(t) \leq \lambda(t)$. Поэтому:
\[
  \frac{f(x) - f(x_1)}{x - x_1} \leq \frac{\lambda(x) - \lambda(x_1)}{x - x_1}, \frac{\lambda(x_2) - \lambda(x)}{x_2 - x} \leq \frac{f(x_2) - f(x)}{x_2 - x}
\]
Однако обе дроби с $\lambda$ равны:
\[
  \frac{f(x_2) - f(x_1)}{x_2 - x_1}
\]
\end{proof}
\begin{task}
Д-те, что каждое из этих нер-в эквив-но вып-ти вниз.
\end{task}
\begin{consequence}
  \label{monot:vip}
Для любой точки $x \in (a, b)$ ф-ция $\nu \colon (a, b) \backslash \set{x} \rightarrow \R$,
\[
\nu(y) = \frac{f(y) - f(x)}{y - x}
\]
нестрого возрастает на $(a, b) \backslash \set{x}$
\end{consequence}
\begin{proof}
$y, z \in (a, b) \backslash \set{x}, y < z$
\end{proof}
\begin{theorem}
\label{17s}
Если ф-ция $f$ выпукла вниз на $(a, b)$, то $f$ непр-на на $(a, b)$ и дифф-ма там, за исключением не более чем счётного мн-ва.
\end{theorem}
\begin{proof}
Зафикс. $x \in (a, b), \nu(y) = \frac{f(y) - f(x)}{y - x}$. По следствию $\ref{monot:vip}$, $\nu(y)$ нестрого возрастает. Тогда по теореме о пределах монотонной функции, сущ-ют конечные $\nu(x - 0) \leq \nu(x + 0)$, т. е. $\exists$ конечные левая и правая производная  $f$ в точке $x\colon f_-'(x) \leq f_+'(x) \Rightarrow f$ непр-на слева и справа в точке $x$, а значит непр-на в точке $x$. \\

Перейдём к пределу в левом нер-ве $\ref{*}$ при $x \rightarrow x_1 + 0$, а также в правом нер-ве $\ref{*}$ при $x \rightarrow x_2 - 0$. Получаем:
\[
f_+'(x_1) \leq \frac{f(x_2) - f(x_1)}{x_2 - x_1} \leq f_-'(x_2)
\]
Учитывая, что $f_-'(x_1) \leq f_+'(x_1)$, отсюда следует, что $g = f_-'$ нестрого возр-ет на $(a, b)$ \\

По т. о разрывах монотонной ф-ции $g$ может иметь на $(a, b)$ разрывы только $I$ рода и их не более чем счётно. Покажем, что в точках непр-ти $g$ ф-ция $f$ дифф-ма. В самом деле, выберем $x_0 < x$, тогда $f_-'(x_0) \leq f_+'(x_0) \leq f_-'(x)$, откуда:
\[
0 \leq f_+'(x_0) - f_-'(x_0) \leq f_+'(x) - f_-'(x_0)
\]
$\Rightarrow f_+'(x_0) = f_-'(x_0) \Rightarrow $ $f$ дифф-ма в т. $x_0$
\end{proof}
\begin{theorem}[Нер-во Йенсена]
\label{18}
Пусть ф-ция $f$ выпукла (вогнута) на $I$. $x_1, \ldots, x_n \in I$ и $\lambda_1, \ldots, \lambda_n \geq 0$, т. ч. $\sum_{i = 1}^{n} \lambda_i = 1$. Тогда справедливо:
\[
f\left(\sum_{i = 1}^{n} \lambda_i x_i\right) \leq \sum_{i = 1}^{n} \lambda_i f(x_i), (\geq)
\]
\end{theorem}
\begin{proof}
Пусть ф-ция $f$ выпукла на $I$.\\
ММИ:
\begin{itemize}
  \item $n = 2$ --- в точности опр-е выпуклости
\item Пусть нер-во верно для $n$. Установим справедливость для $n + 1$. Т. к. случай $\lambda_{n + 1} = 1$ --- очев., считаем, что $\lambda_{n + 1} < 1$. Положим:
  \[
  y = \frac{\lambda_1}{1 - \lambda_{n + 1}} x_1 + \ldots + \frac{\lambda_n}{1 - \lambda_{n + 1}} x_n
  \]
  Т. к. $\sum_{i = 1}^{n} \frac{\lambda_i}{1 - \lambda_{n + 1}} = 1$ :
  \[
    \underset{k}{\min} x_k \leq y \leq \underset{k}{\max} x_k
  \]
  \[
  \Rightarrow f\left(\sum_{i = 1}^{n + 1} \lambda_i x_i\right) = f\left((1 - \lambda_{n + 1})y + \lambda_{n + 1}x_{n + 1}\right) \leq (1 - \lambda_{n + 1})f(y) + \lambda_{n + 1}f(x_{n + 1})
  \]
  По предположению инд-ции:
  \[
  f(y) \leq \sum_{i = 1}^{n} \frac{\lambda_1}{1 - \lambda_{n + 1}} f(x_i)
  \]
  Подставляя $f(y)$ в пред-ее нер-во, получаем то, что нужно.
\end{itemize}
\end{proof}
\begin{example}
Пусть $x_1, \ldots x_n \geq 0$, тогда:
\[
  \frac{\sum_{i = 1}^{n} x_i}{n} \geq \sqrt[n]{\prod_{i = 1}^{n} x_i}
\]
\end{example}
\begin{proof}
$y = \ln x, \lambda_1 = \ldots = \lambda_n = \frac{1}{n}$ \\
По нер-ву Йенсена:
\[
\ln\left(\sum_{i = 1}^{n} \frac{1}{n}x_i\right) \geq \frac{1}{n}\sum_{i = 1}^{n} \ln x_i = \ln\left(\prod_{i = 1}^{n} x_i\right)^{\frac{1}{n}}
\]
\[
\Rightarrow \frac{\sum_{i = 1}^{n} x_i}{n} \geq \sqrt[n]{\prod_{i = 1}^{n} x_i}
\]
\end{proof}
\begin{example}[Нер-во Гельдера]
  Пусть $a_1, \ldots, a_n, b_1, \ldots, b_n \geq 0$ и
  \[
    p, q > 1, \frac{1}{p} + \frac{1}{q} = 1
  \]
  Тогда справ-во нер-во:
  \[
  \sum_{k = 1}^{n} a_k b_k \leq \left(\sum_{k = 1}^{n} a_k^{p}\right)^{\frac{1}{p}} \left(\sum_{k = 1}^{n} b_k^{q}\right)^{\frac{1}{q}}
  \]

\end{example}
\begin{proof}
\[
f = x^{p} \text{ --- выпукла вниз }
\]
\[
x_k = \frac{a_k}{b_k^{\frac{q}{p}}}, \lambda_k = \frac{b_k^{q}}{\sum_{i = 1}^{n} b_i^{q}}
\]
\[
  \lambda_k x_k = \frac{a_k b_k^{q\left(1 - \frac{1}{p}\right)}}{\sum_{i = 1}^{n} b_i^{q}} = \frac{a_k b_k}{\sum_{i = 1}^{n} b_i^{q}}
\]
\[
  \lambda_k f(x_k) = \frac{b_k^{q}}{\sum_{i = 1}^{n} b_i^{q}} \frac{a_k^{p}}{b_k^{q}} = \frac{a_k^{p}}{\sum_{i = 1}^{n} b_i^{q}}
\]
\[
  \left(\frac{\sum_{i = 1}^{n} a_i b_i}{\sum_{i = 1}^{n} b_i^{q}}\right)^{p} \leq \frac{\sum_{i = 1}^{n} a_i^{p}}{\sum_{i = 1}^{n}b_i^{q}} \iff \sum_{i = 1}^{n} a_i b_i \leq \left(\sum_{i = 1}^{n} a_i^{p}\right)^{\frac{1}{p}} \left(\sum_{i = 1}^{n} b_i^{q}\right)^{\frac{1}{q}}
\]
\end{proof}
\begin{example}[Нер-во Минковского]
Пусть $a_1, \ldots, a_n, b_1, \ldots, b_n \geq 0$ и $p \geq 1$. Тогда:
\[
\left(\sum_{i = 1}^{n} (a_i + b_i)^{p}\right)^{\frac{1}{p}} \leq \left(\sum_{k = 1}^{n} a_k^{p}\right)^{\frac{1}{p}} + \left(\sum_{k = 1}^{n} b_k^{p}\right)^{\frac{1}{p}}
\]
\end{example}
\begin{proof}
$p = 1$ --- верно. $p > 1, q = \frac{p}{p- 1}$:
\[
  (a_1 + b_1)^{p} + \ldots + (a_n + b_n)^{p} = (a_1 + b_1)(a_1 + b_1)^{p - 1} + \ldots + (a_n + b_n)(a_n + b_n)^{p - 1} = 
\]
\[
 = a_1 (a_1 + b_1)^{p - 1} + \ldots + a_n(a_n + b_n)^{p - 1} + b_1(a_1 + b_1)^{p - 1} + \ldots + b_n(a_n + b_n)^{p -1} \underset{\text{нер-во Гельдера}}{\leq} \ldots
\]
\[
  \left(\sum_{i = 1}^{n} a_i^{p}\right)^{\frac{1}{p}} \cdot \left(\sum_{i = 1}^{n} (a_i + b_i)^{p}\right)^{\frac{1}{q}} + \left(\sum_{i = 1}^{n} b_1^{p}\right)^{\frac{1}{p}}\left(\sum_{i = 1}^{n} (a_i + b_i)^{p}\right)^{\frac{1}{q}}
\]
Поделим LHS и RHS на $\left(\sum_{i = 1}^{n} (a_i + b_i)^{p}\right)^{\frac{1}{q}}$ и получим желаемый результат.
\end{proof}
