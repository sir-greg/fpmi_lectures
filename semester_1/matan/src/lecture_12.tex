\section{Лекция 12}
\begin{lemma}
  \label{lm:bp}
Если $f$ - непр-на на $[a, b]$ и $f(a)f(b) < 0$, то
\[
  \exists c \in [a, b] \colon f(c) = 0
\]
\end{lemma}
\begin{proof}
Можно считать, что $f(a) < 0 < f(b)$. В противном случае заменим $f$ на $(-f)$. \\
Построим п-ть отр-ов $\set{[a_n, b_n]}$ по индукции:
\[
[a_1, b_1] \colon= [a, b] \text{ и если } [a_k, b_k] \text{ - построен, положим } 
\]
\[
[a_{k + 1}, b_{k + 1}] = \begin{cases}
[a_k, \frac{a_k + b_k}{2}], \text{ если } f(\frac{a_k + b_k}{2}) \geq 0 \\
[\frac{a_k + b_k}{2}, b_k], \text{ если } f(\frac{a_k + b_k}{2}) < 0
\end{cases}
\]
По индукции будет построена стягивающаяся п-ть вложенных отр-ов $\set{[a_n, b_n]}$, т. ч.:
\[
  f(a_n) \leq 0 \\
  f(b_n) > 0
\]
По т. Кантора о вложенных отр-ах, сущ-ет $c \in \bigcap_{n = 1}^{\infty} [a_n, b_n]$, причём $a_n \rightarrow c$ и $b_n \rightarrow c$. По непр-ти в точке $c$, переходя в нер-ве к пределу:
\[
f(a_n) \leq 0 < f(b_n) \Rightarrow f(c) \leq 0 \leq f(c) \Rightarrow f(c) = 0
\]
\end{proof}
\begin{definition}
Будем говорить, что число $s$ лежит строго между числа $\alpha$ и $\beta$, если $max(a, b) > s > min(a, b)$.
\end{definition}
\begin{theorem}[Больцано-Коши о промежуточных значениях]
  \label{th:bc_mid_values}
Если ф-ция $f$ непр-на на $[a, b]$ и число $s$ лежит строго между $f(a)$ и $f(b)$, то:
\[
\exists c \in (a, b) \colon f(c) = s
\]
\end{theorem}
\begin{proof}
Рассм. $g = f - s$. Тогда $g$ непр-на на $[a, b]$. Сл-но, $g(a)g(b) < 0$. Тогда по лемме ($\ref{lm:bp}$)
\[
  \exists c \in (a, b) \colon g(c) = 0 \iff f(c) = s
\]
\end{proof}
\begin{task}
Приведите пример разрывной ф-ции $f: [0, 1] \rightarrow \R$, т. ч. $\forall [a, b] \subset [0, 1]$, $f$ принимает все значения между $f(a)$ и $f(b)$
\end{task}
Напомним, что $I \subset \R$ - промежуток $\iff$ 
\[
\forall x, y \in I ([x, y] \subset I)
\]
\begin{consequence}
  \label{cs:cs_th7}
Если ф-ция $f$ непр-на на промеж. $I$, то $f(I)$ - промежуток.
\end{consequence}
\begin{proof}
Выберем $y_1, y_2 \in f(I)$ ($y_1 < y_2$) $\Rightarrow$
\[
  \exists x_1, x_2 \in I \colon (f(x_1) = y_1, f(x_2) = y_2)
\]
Если $y_1 < y < y_2$, то, по теореме ($\ref{th:bc_mid_values}$) $ \exists x \in (x_1, x_2) \colon f(x) = y$. Т. к. $I$ - промежуток, $x_1, x_2 \in I$, то $x \in I$, а значит $y \in f(I)$, т. е. $f(I)$ - промежуток.
\end{proof}
\begin{task}
Док-те, что если $f$ - непр-на на $[a, b]$, то $f([a, b])$ - отрезок
\end{task}
\begin{lemma}
  Пусть $f$ монотонна на пром-ке $I$. Если $f(I)$ - это пром-к, то $f$ - непр-на на $I$.
\end{lemma}
\begin{proof}
~\newline

Пусть $f$ нестрого возрастает на $I$. Если $f$ разрывна в точке $c \in I$. То $f(c - 0) \leq f(c) \leq f(c + 0)$ и хотя бы один из интервалов $(f(c - 0), f(c))$ или $(f(c), f(c + 0))$ непуст.\\

(Если $c$ - концевая точка $I$, то сущ-ет только один из пределов, для кот. и проводим рассуждение.) \\
Пусть $Y = (f(c), f(c + 0)) \neq \emptyset$. Тогда
\[
  \forall t \in I, t \leq c (f(t) \leq f(c))
\]
Также
\[
  \forall t \in I, t > c (f(t) > \underset{(c, \sup I)}{\inf} f(x) \geq f(c + 0))
\]
Сл-но, $f(I)$ не явл. пром-ом.
\end{proof}
\begin{theorem}[об обратной ф-ции]
  \label{th:rev_func_th}
  Пусть $f$ непр-на и строго монотонна на пром. $I$, тогда:
  \begin{itemize}
    \item [1) ] $f(I)$ - пром-ок
    \item [2) ] $f: I \rightarrow f(I)$ - биекция
    \item [3) ] $f^{-1}: f(I) \rightarrow I$ - непр-на и строго монотонна на $f(I)$
\end{itemize}
\end{theorem}
\begin{proof}
По следствию $(\ref{cs:cs_th7})$,  $Y = f(I)$ явл-ся пром-ом. Ф-ция $f$ инъективна в силу строгой монотонности. \\

Сл-но, $f: I \rightarrow Y$ - биекция, и сущ-ют $f^{-1}: Y \rightarrow I$ \\

Б. О. О. пусть $f$ строго возрастает на $I$ \\

Пусть $y_1, y_2 \in Y, y_1 < y_2 \Rightarrow \exists x_1, x_2 \in I \colon f(x_1) = y_1, f(x_2) = y_2$ \\

Если $x_1 \geq x_2 \Rightarrow f(x_1) \geq f(x_2)$ - в силу возрастания $f$ $\Rightarrow y_1 \geq y_2!!!$ \\

Таким образом, если $y_1, y_2 \in Y (y_1 < y_2 \Rightarrow f^{-1}(y_1) < f^{-1}(y_2))$ - т. е. $f^{-1}$ строго возрастает на $Y$. \\

$f^{-1}(Y) = I$ - пром-к $\Rightarrow f^{-1}$ - непр-на на $Y$
\end{proof}
\begin{example}
Для $\forall x \geq 0, n \in \N \exists! y \geq 0 \colon y^{n} = x$. Пишут, что:
\[
  y = \sqrt[n]{x}
\]
Кроме того, $f(x)\colon [0; +\infty) \rightarrow \R, f(x) = \sqrt[n]{x}$ - непр-на и строго монотонна. 
\end{example}
\begin{proof}
Рассм. ф-цию $g: [0; +\infty) \rightarrow \R, g(y) = y^{n}$ \\

Ф-ция $g$ - непр-на и строго возрастает на $[0; +\infty)$, причём:
\[
  g(0) = 0, \lim_{y\to +\infty} g(y) = +\infty
\]
По теореме ($\ref{th:rev_func_th}$) $\exists f = g^{-1} \colon [0; +\infty) \rightarrow [0; +\infty)$:
\[
  f(x) = \sqrt[n]{x}
\]
\end{proof}
\subsection{Счётные и несчётные мн-ва}
\begin{definition}
Мн-во $A$ наз-ся \underline{счётным} если $\exists f: \N \rightarrow A$ - биекция.
\end{definition}
\begin{note}
\[
A = \set{a_1, a_2, \ldots}
\]
\[
  \forall i, j (i \neq j \Rightarrow a_i \neq a_j)
\]
\end{note}
\begin{example}
\[
\Z \text{ - счётно}
\]
\[
  \ldots \ldots -2, -1, 0, 1, 2, \ldots
\]
\[
h(n) = \begin{cases}
  \frac{n - 1}{2}, \text{ n - нечётно} \\
  -\frac{n}{2}, \text{ n - чётно}
\end{cases}
\]
\end{example}
\begin{lemma}
Всякое бесконечное мн-во $A \subset \N$ - счётно.
\end{lemma}
\begin{proof}
Пусть $n_1 = min(A)$. Если $n_1\ldots n_k$ - определена, то по инд-ции определим:
\[
n_{k + 1} = min(A \backslash \set{n_1\ldots n_k})
\]
Поскольку при переходе к подмн-ву минимум не уменьшается и $n_{k+ 1} \not\in \set{n_1, \ldots n_k}$, то $n_{k + 1} > n_k$ \\

Предположим, что $\exists m \in A$ и $m \neq n_k, \forall k$. Тогда по инд-ции
\[
  n_k < m, \forall k \Rightarrow m > n_m \geq m!!!
\]
Сл-но, $\sigma: \N \rightarrow A, \sigma(k) = n_k$ - строго возр. биекция.
\end{proof}
\begin{definition}
Мн-во \underline{не более чем счётно}, если оно конечно или счётно.
\end{definition}
\begin{consequence}
Всякое подмн-во счётного мн-ва не более чем счётно.
\end{consequence}
\begin{proof}
Рассм. конечное подмн-ва $A$ счётного мн-ва $X$. $g: X \rightarrow \N$ - биекция $\Rightarrow$
\[
  g\colon A \rightarrow g(A) \text{ - биекция мн-ва $A$ и $g(A) \subset \N$}
\]
\end{proof}
\begin{theorem}
$\N \times \N$ - счётно
\end{theorem}
\begin{proof}
Идея: Сделаем таблицу и рассматриваем её подиагонально, затем нумеруем эл-ты в диагоналях. \\

\begin{center}
\begin{tabular}{ |c|c|c|c|c| } 
  $(k, m)$ & 1& 2& 3& \\
 \hline
    1       & $(1, 1)$ & $(1, 2)$ & $(1, 3)$ & \ldots \\
 \hline
    2       & $(2, 1)$ & $(2, 2)$ & $(2, 3)$ & \ldots \\
 \hline
    3       & $(3, 1)$ & $(3, 2)$ & $(3, 3)$ & \ldots \\
 \hline
    4       & $(4, 1)$ & $(4, 2)$ & $(4, 3)$ & \ldots \\
 \hline
\end{tabular}
\end{center}
\[
p \in \N
\]
\[
M_p = \set{(k, m) \colon 1 \leq m \leq p, k = p + 1 - m}
\]
\[
g(p) = 1 + 2 + \ldots + p - 1 = \frac{p(p - 1)}{2}
\]
\[
N_p = \set{n \colon g(p) + 1 \leq n \leq g(p) + p = g(p + 1)}
\]
\[
f: \N \times \N \rightarrow \N
\]
\[
f(k, m) = g(k + m - 1) + m
\]
\end{proof}
\begin{consequence}
Мн-во $\Q$ - счётно.
\end{consequence}
\begin{proof}
Любое рац. число можно записать в виде несокр. дроби $\frac{p}{q}$, т. е.:
\[
f_1 \colon r \rightarrow (p, q) \text{ - инъекция}
\]
\[
\Q \rightarrow \underset{(p, q)}{\Z \times \N} \rightarrow \N \times \N \rightarrow \N
\]
\[
F: \Q \rightarrow \N, F = f_3 \circ f_2 \circ f_1 \text{ - инъекция}
\]
\[
\Rightarrow F(\Q) \subset \N \Rightarrow \Q \text{ - не более чем счётно и беск $\Rightarrow$ счётно}
\]
\end{proof}
\begin{theorem}
$\R$ несчётно
\end{theorem}
\begin{proof}
Пред-м, что $\R = \set{x_n | n \in \N}$ \\
Рассм. $[a, b] \subset \R$. Разобъём $[a, b]$ на три отр-ка и обозн.  $[a_1, b_1]$ тот из ни, который не сод-т $x_1$. По инд-ции построим п-ть влож. отр-ов $\set{[a_k, b_k]}$, не содержащую $x_k, \forall k$. Однако сущ-ет точка, общая для всех отр-ов $\Rightarrow \forall n x \in [a, b], x_n \not\in x_n \Rightarrow \forall n \colon x_n \neq x$
\end{proof}
