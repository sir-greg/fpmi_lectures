\section{Лекция 27}
\subsection{Ф-ла Ньютона Лейбница}
\begin{theorem}[Ф-ла Ньютона-Лейбница]
\label{th:27-Newton-Leibniz}
Если $f \in R[a, b]$ и имеет (обобщ.) первообразную на $[a, b]$, то:
\[
\int_{a}^{b} f \, dx = F(b) - F(a)
\]
\end{theorem}
\begin{proof}
Пусть $T = \set{x_i}_{i = 0}^{n}$ --- разбиение $[a, b]$. Если $F$ --- первообразная, то по т. Лагранжа:
\[
\exists c_i \in (x_{i - 1}, x_i) \colon F(x_i) - F(x_{i - 1}) = f(c_i) \triangle x_i
\]
А значит ($m_i = \underset{[x_{i - 1}, x_i]}{\inf} f, M_i = \underset{[x_{i - 1}, x_i]}{\sup} f$)
\[
m_i \triangle x_i \leq F(x_i) - F(x_{i - 1}) \leq M_i \triangle x_i
\]
\begin{note}
Для обобщённой первообразной это нер-во вып-ся по следствию т. $10'$.
\end{note}
Просуммируем получ. нер-ва, то $i = 1, \ldots, n$, получим:
\[
s_T(f) \leq F(b) - F(a) \leq S_T(x_i)
\]
Перейдём к $\sup$ слева и к $\inf$ справа по всем разбиениям $T$:
\[
\underline{\int_{a}^{b}} f \, dx \leq F(b) - F(a) \leq \overline{\int_{a}^{b}} f \, dx
\]
Т. к. $f \in R[a, b]$, то "$=$", а значит:
\[
\int_{a}^{b} f(x) \, dx = F(b) - F(a)
\]
\end{proof}
\subsection{Интеграл с переменным пределом}
\begin{definition}
Пусть $I$ --- невырожд. пром-к и $a \in I$. Пусть $f$ опр-на на $I$ и $f \in R[\alpha, \beta], \forall [\alpha, \beta] \subset I$. Тогда:
\[
F\colon I \rightarrow R
\]
\[
F(x) = \int_{a}^{x} f(t) \, dt
\]
наз-ся \underline{интегралом с переменным верх. пределом}.
\end{definition}
\begin{theorem}[Основаная теорема интегрального исчисления]
\label{th:27-Basic-Integral-Theorem(8)}
Пусть $f$ опр-на на пром. $I, a \in I$, пусть $f \in R[\alpha, \beta], \forall [\alpha, \beta] \subset I$. Тогда $F$ непр-на на $I$. Кроме того, если $f$ непр. в т. $x$, то $F$ дифф-ма в т. $x$ и $F'(x) = f(x)$
\end{theorem}
\begin{proof}
Пусть $x \in I$. Выберем $\sigma > 0$ так, что:
\[
[\alpha, \beta] = [x - \sigma, x + \sigma] \cap I \text{---  невырожд. отрезок}
\]
По усл-ию $f \in R[\alpha, \beta]$, в част-ти $f$ огр-на на $[\alpha, \beta], \left|f\right| \leq \left|M\right|$, на $[\alpha, \beta]$ \\
Тогда для $\forall y \in [\alpha, \beta]$:
\[
\left|F(y) - F(x)\right| = \left|\int_{a}^{y} f(t) \, dt - \int_{a}^{x} f(t) \, dt\right| = 
\]
\[
 = \left|\int_{x}^{y} f(t) \, dt\right| \leq \left|\int_{x}^{y} \left|f(t)\right| \, dt\right| \leq M|y - x|
\]
Сл-но, $\lim_{y\to x} F(y) = F(x) \iff F$ непр-на в произвольной т. $x$ $\iff F$ непр-на на $I$ \\
Зафикс. $\varepsilon > 0$. По опр-ю непр-ти $f$ в т. $x$
\[
  \exists \delta > 0, \forall t \in \overset{\circ}{B_{\delta}}(x) \cap I, \left|f(t) - f(x)\right| < \varepsilon
\]
Тогда для любого $y \in \overset{\circ}{B_{\delta}}(x) \cap I$:
\[
\left|\frac{F(y) - F(x)}{y - x} - f(x)\right| = \left|\frac{1}{y - x}\int_{x}^{y}  f(t) \, dt - \frac{1}{y - x} \int_{x}^{y} f(x) \, dt\right| = 
\]
\[
=  \frac{1}{\left|y -x \right|}\left|\int_{x}^{y} (f(t) - f(x)) \, dt\right| = \frac{1}{\left|y - x\right|} \left|\int_{x}^{y} \left|f(t) - f(x)\right| \, dt\right| \leq
\]
\[
 \leq  \varepsilon \cdot \frac{1}{\left|y -x \right|} \left| \int_{x}^{y}\, dt\right| = \varepsilon
\]
Ч. Т. Д.
\end{proof}
\begin{consequence}
Всякая непр-ная на пром. $I$ ф-ция имеет первообразную. Всякая монотонная на $I$ ф-ция имеет обобщённую первообразную.
\end{consequence}
\begin{note}
Если $f \in R[a, b]$ и имеет (обобщ.) первообразную по $F$ на $[a, b]$, то $F$ с точностью до константы совпадает с интегралом с переменным пределом.
\end{note}
\subsection{Приёмы интегрирования}
\begin{theorem}
\label{th:ways-of-integration(9)}
Пусть $f$ непр-на на $I$, $\phi \colon [\alpha, \beta] \rightarrow I$ дифф-ма на $[\alpha, \beta], \phi \in R[\alpha, \beta]$, тогда:
\[
\int_{a}^{b} f(x) \, dx = \int_{\alpha}^{\beta} (f \circ \phi) \phi'(t) \, dt
\]
Где $a = \phi(\alpha), b = \phi(\beta)$
\end{theorem}
\begin{proof}
Т. к. $f$ непр-на, то $f \circ \phi$ непр-на на $[\alpha, \beta]$, а значит, $(f \circ \phi) \cdot \phi' \in R[\alpha, \beta]$. Пусть $F$ --- первообраз. $f$ на $I$. Т. к.:
\[
  (F \circ \phi)' = F'(\phi) \cdot \phi' = (f \circ \phi) \phi'
\]
на $[\alpha, \beta]$ \\
По ф-ле Н-Л:
\[
  \int_{\alpha}^{\beta} (f \circ \phi) \phi'(t) \, dt = F \circ \phi \vert_{\alpha}^{\beta} = F(\phi(\beta)) - F(\phi(\alpha)) = F(b) - F(a) = \int_{a}^{b} f(x) \, dx
\]
\end{proof}
\begin{theorem}
\label{th:integration-by-parts(10)}
Пусть $f, g$ дифф-мы на $[a, b]$ и $f', g' \in R[a, b]$. Тогда:
\[
\int_{a}^{b} f'(x)g(x) \, dx = f(x)g(x)|_{a}^{b} - \int_{a}^{b} f(x)g'(x) \, dx
\]
(Ф-ла интегрирования по частям)
\end{theorem}
\begin{proof}
Обозначим:
\[
h = f'g + fg'
\]
Т. к.:
\[
h = (fg)'
\]
то $fg$ --- первообразная $h$ на $[a, b]$. По ф-ле Ньютона-Лейбница получаем:
\[
\int_{a}^{b} f'g \, dx + \int_{a}^{b} fg' \, dx = (fg)|_{a}^{b} = f(b)g(b) - f(a)g(a)
\]
\end{proof}
\begin{example}
  \[
    Y_m = \int_{0}^{\frac{\pi}{2}} \sin^{m}x \, dx, m \in \N_0
  \]
\end{example}
\begin{proof}
Для $m \geq 2$ по ф-ле инт-я по частям имеем:
\[
Y_m = \int_{a}^{b} \sin^{m - 1}x (-\cos x)' \, d = \underbrace{-\sin^{m - 1}x \cos x |_{0}^{\frac{\pi}{2}}}_{=0} + (m - 1)\int_{0}^{\frac{\pi}{2}} \sin^{m - 2}\cos^{2}x \, dx = 
\]
\[
 = (m - 1)(Y_{m - 2} - Y_m) \Rightarrow Y_M = \frac{m - 1}{m} Y_{m - 2}
\]
\[
 \Rightarrow Y_M = 
 \begin{cases}
 \frac{(m - 1)!!}{m!!} \frac{\pi}{2}, m \text{ --- чёт.} \\
 \frac{(m - 1)!!}{m!!}, m \text{ --- нечет.}
 \end{cases}
\]
\end{proof}
\begin{example}[Ф-ла Валлиса]
  \[
  \pi = \lim_{n\to \infty} \frac{1}{n} \left(\frac{(2n)!!}{(2n - 1)!!}\right)^{2}
  \]
\end{example}
\begin{proof}
На $[0, \frac{\pi}{2}]$
\[
  \sin^{2n + 1}x \leq \sin^{2n} x \leq \sin^{2n - 1} x
\]
Из предыдущего примера:
\[
  \frac{(2n)!!}{(2n + 1)!!} \leq \frac{(2n - 1)!!}{(2n)!!}\frac{\pi}{2} \leq \frac{(2n - 2)!!}{(2n - 1)!!}
\]
Обозначим $x_n = \frac{1}{n}\left(\frac{(2n)!!}{(2n - 1)!!}\right)^{2}$
\[
\vdots
\]
\end{proof}
\begin{theorem}
\label{th:integr-mid-theorem(11)}
Пусть $f, g \in R[a, b], m \leq f \leq M$ на $[a, b], g$ не меняет знака на $[a, b]$. Тогда $\exists \lambda \in [m, M]$:
\[
  \int_{a}^{b} f(x)g(x) \, dx = \lambda \int_{a}^{b} g(x) \, dx
\]

\end{theorem}
\begin{proof}
Пусть $g \geq 0$:
Тогда:
\[
  mg \leq fg \leq Mg
\]
\[
  m \int_{a}^{b} g(x) \, dx \leq \int_{a}^{b} f(x)g(x) \, dx \leq M\int_{a}^{b} g(x) \, dx
\]
Если $\int_{a}^{b} g(x) \, dx = 0 \Rightarrow \int_{a}^{b} f(x)g(x) \, dx = 0$ \\
Если $\int_{a}^{b} g(x) \, dx \neq 0 (> 0)$, то:
\[
\lambda = \frac{\int_{a}^{b} f(x)g(x) \, dx}{\int_{a}^{b} g(x) \, dx} \in [m; M]
\]
Случай $g \leq 0$ св-ся аналогично.
\end{proof}
\begin{task}
Док-ть, что $\lambda$ может быть выбрана на $(m; M)$
\end{task}
