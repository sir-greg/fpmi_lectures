\section{Лекция 23}
\subsection{Неопределённый интеграл}
\begin{definition}
Пусть ф-ция $f$ опр-на на пром-ке $I$. 
\begin{itemize}
  \item [1) ] Ф-ция $F \colon I \rightarrow \R$ наз-ся \textbf{первообразной на $I$}, если $F'(x) = f(x), \forall x \in I$
  \item [2) ] Ф-ция $F \colon I \rightarrow \R$ наз-ся обобщённой первообразной на $I$, если $F$ непр-на на $I$ и $F'(x) = f(x), \forall x \in I \backslash A$, причём $A$ не более чем счётно.
\end{itemize}
\end{definition}
\begin{example}
\[
f = \sign x, I = [-1, 1]
\]
По т. Дарбу, всякая производная дифференцируемой ф-ции принимает все промежуточные значиния $\Rightarrow$ $f$ не имеет первообразной на отрезке $[-1, 1]$ \\

Её \textbf{обобщённая} первообразная: $F(x) = \left|x\right|$
\end{example}
\begin{theorem}[Описание класса первообразных]
\label{th:19}
Если $F$ --- первообразная (обобщённая) $f$ на $I$ и $c \in \R$, то $F + c$, тоже обощённая первообразная $f$ на $I$. \\
Если $F_1, F_2$ --- первообразные (обобщённые) $f$ на $I$, то их разность постонна на $I$.
\end{theorem}
\begin{proof}
$(F_1 - F_2)' = F_1' - F_2' = f - f = 0 \overset{\text{условие постоянства}}{\Rightarrow} F_1 - F_2 = c \in \R$ \\
Для обобщённых первообразных следует из дополнения к теореме $10$ ($10'$)
\end{proof}
\begin{definition}
Произвольная первообразная ф-ции $f$ на $I$ наз-ся неопределённым интегралом ф-ции $f$ на $I$ и обозначается:
\[
\int f(x) dx \text{ или } \int f dx
\]
\end{definition}
\begin{note}
Операция перехода от ф-ции к её неопр. интегралу наз-ся \textbf{интегрированием}.
\end{note}
\begin{note}
  Формально $dx$ в обозначении не несёт смысловой нагрузки, однако его использование \textbf{бывает весьма полезным}, если трактовать $f dx$ как дифференциал. ($f' dx = df$)
\end{note} 
\begin{note}
Из неудобств отметим, что в обозначении никак не фигурирует пром-к $I$.
\end{note}
\begin{statement}
Неопределённый интеграл имеет следующие св-ва:
\begin{itemize}
  \item [1) ] Если $\exists \int f dx$ на $I$, то $\left(\int f dx\right)' = f$ на $I$
  \item [2) ] Если $\exists \int f dx, \int g dx$ на $I$, а $\lambda, \mu \in \R$, то на $I$ сущ-ет:
    \[
    \int (\lambda f + \mu g) dx = \lambda \int f dx + \mu \int g dx + C, C \in \R
    \]
    для некоторого $C \in \R$ \\
  \item [3) ] Если $u, v$ - дифф-мые ф-ции на $I$ и $\exists \int vu' dx$ на $I$, то на $I$ сущ-ет:
    \[
      \int v'u dx
    \]
    А также верна ф-ла (\textbf{интегрирование по частям}):
    \[
      \int vu' dx = vu - \int v'u dx + C, C \in \R
    \]
    для некоторого $C \in \R$ \\
    Или:
    \[
    \int u dv = vu - \int v du + C, C \in \R
    \]
    для некоторого $C \in \R$ \\
  \item [4) ] Если $F$ --- первообразная $f$ на $I$, $\phi$ дифф. на пром-ке $Y, \phi(Y) \subset I$, то сущ-ет:
    \[
    \int f(\phi(t)) \phi'(t) dt = F(\phi(t)) + C, C \in \R
    \]
    для некоторого $C \in \R$ (\textbf{ф-ла подстановки}) \\
\end{itemize} 
\end{statement}
\begin{note}
Если дополнительно $\phi$ строго монотонна на $Y$, то на $\phi(Y)$
\[
t = \phi^{-1}(x)
\]
\[
    \int f(\phi(t)) \phi'(t) dt = \int f(x) dx + C 
\]
\end{note}
\begin{theorem}[Таблица неопределённых интегралов]
  \label{th:integral_table}
  Смотри талблицу в книжке.
\end{theorem}
\begin{task}
Пусть $f$ дифф-ма на $I$ с $f' \neq 0$ на $I$. Пусть $F$ --- первообразная $f$ на $I$. Запишите:
\[
\int f^{-1}(y) dy
\]
через $f$.
\end{task}
\begin{note}
 В отличии от операции дифференцирования, операция интегрирования выводит за пределы элементарных ф-ций, наприме:
 \[
  \int e^{-x^{2}} dx
 \]
\end{note}
\begin{note}
Все св-ва переносятся на обобщ. интеграл.
\end{note}
