\section{Лекция 9}

\begin{definition}
\textbf{Семейство } $\set{G_\lambda}$ наз-ся \underline{покрытием} мн-ва $E$, если $E \subset \bigcup_{\lambda \in \Lambda}^{} G_\lambda$. Покрытие наз-ся \underline{открытым}, если все $G_\lambda$ открыты.
\end{definition}
\begin{example}
$\set{\left(\frac{1}{n}, 1\right)}_{n \in \N}$ - открытое покр-е $(0, 1)$
\[
\bigcup_{n = 1}^{\infty} \left(\frac{1}{n}, 1\right)
\]
\end{example}
\begin{theorem}[Лемма Гейне-Бореля]
Если $\set{G_\lambda}_{\lambda \in \Lambda}$ образует открытое покр-е отрезка $[a, b]$, то:
\[
\exists \lambda_1, \ldots, \lambda_n \in \Lambda \colon ([a, b] \subset G_{\lambda_1} \cup \ldots \cup G_{\lambda_n})
\]
\end{theorem}
\begin{proof}
Предположим, что из открытого покр-я $\set{G_\lambda}_{\lambda \in \Lambda}$ отрезка $[a, b]$ нельзя выбрать конечное подпокрытие. \\

Разделим $[a, b]$ пополам и обозначим за $[a_1, b_1]$ ту его половину, кот. не покрыв-ся конечным набором $G_\lambda$. \\

Разделим $[a_1, b_1]$ пополам и обозначим за $[a_2, b_2]$ ту его половину, кот. не покр-ся конечным набором $G_\lambda$ \\

\ldots \\

По индукции будет построена стягивающаяся п-ть отрезков, каждый из кот. не покрыв-ся конечным набором $G_\lambda$ \\

По т. Кантора о вложенных отрезках, найдётся т. $c \in \bigcap_{i = 1}^{n} [a_i, b_i]$. Т. к.
\[
  c \in [a, b] \subset \bigcup_{\lambda \in \Lambda}^{} G_\lambda \Rightarrow \exists \lambda_0 \in \Lambda (c \in G_\lambda \text{ - открытое})
\]
\[
  \Rightarrow \exists B_{\varepsilon}(c) \subset G_{\lambda_0}
\]
Выберем $k$ так, что $b_k - a_k = \frac{b - a}{2^{k}} < \varepsilon$ \\

Сл-но, $c - a_k < \varepsilon$ и $b_k - c < \varepsilon$. Откуда:
\[
  [a_k, b_k] \subset B_{\varepsilon}(c) \subset G_{\lambda_0} !!! (\text{с построением п-ти $\set{[a_n, b_n]}$})
\]
\end{proof}
\begin{consequence}
Если $F$ - замкнутое огр. мн-во в $\R$ и $\set{G_\lambda}_{\lambda \in \Lambda}$ - откр. покр-е $F$, то:
\[
\exists \lambda_1, \ldots, \lambda_n \in \Lambda \colon (F \subset G_{\lambda_1} \cup \ldots \cup G_{\lambda_n})
\]
\end{consequence}
\begin{proof}
Т. к. $F$ - огр., то $\exists[a, b] \colon F \subset [a, b]$. Сем-во $\set{G_\lambda}_{\lambda \in \Lambda} \cup \set{\R \backslash F}$ отк-е покр-е $[a, b]$, т. к. $\bigcup_{\lambda \in \Lambda}^{} G_{\lambda} \cup (\R \backslash F) = \R$ \\

По т. Гейне-Бореля $\exists \lambda_1, \ldots, \lambda_n \in \Lambda$:
\[
[a, b] \subset G_{\lambda_1} \cup G_{\lambda_2} \cup \ldots \cup G_{\lambda_n} \cup (\R \backslash F)
\]
\[
  \Rightarrow F \subset G_{\lambda_1} \cup \ldots \cup G_{\lambda_n}
\]
\end{proof}
Введм следующее обозначение:
\begin{symb}
\[
B_{\varepsilon}(+\infty) = (\frac{1}{\varepsilon}; +\infty) \cup \set{+\infty} \text{ - $\varepsilon$-окр-ть $+\infty$}
\]
\[
\overset{\circ}{B_{\varepsilon}}(+\infty) = (\frac{1}{\varepsilon}, +\infty) \text{ - проколотая $\varepsilon$-окр-ть $+\infty$}
\]
\[
  B_{\varepsilon}(-\infty) = (-\infty, -\frac{1}{\varepsilon}) \cup \set{-\infty}
\]
\[
\overset{\circ}{B_{\varepsilon}}(-\infty) = (-\infty, -\frac{1}{\varepsilon})
\]
\end{symb}
Поскольку все определения этого параграфа давались на языке окр-тей, то всё это верно и для $\overline{\R}$ \\

$E \subset \overline{\R}$. В част-ти $+\infty(-\infty)$ - предел. точка мн-ва $E \subset \overline{\R} \iff E \backslash \set{+\infty}$ неогр. сверху ($E \backslash \set{-\infty}$ - неогр. снизу).

На языке окр-ти можно дать общее определение предела:
\begin{definition}
Точка в $b \in \overline{\R}$ наз-ся \textbf{пределом} числовой п-ти $\set{a_n}$, если:
\[
\forall \varepsilon > 0, \exists N \colon \forall n \geq N (a_n \in B_{\varepsilon}(b))
\]
\end{definition}

\subsection{\textsection 4: Непрерывные ф-ции}
\subsubsection{Предел ф-ции в точке}
Пусть $\exists \in \R$, задана ф-ция $f: E \rightarrow \R$. \\
Пусть $a, b \in \overline{\R}$
\begin{definition}[по Коши]
Точка $b$ наз-ся пределом ф-ции $f$ в т. $a$, если $a$ - предельная точка мн-ва $E$ и:
\[
\forall \varepsilon > 0, \exists \delta > 0, \forall x \in E (x \in \overset{\circ}{B_{\delta}}(a) \rightarrow f(x) \in B_{\varepsilon}(b))
\]
Пишут $b = \lim_{x\to a} f(x)$ или $f(x) \rightarrow b$ при $x \rightarrow a$
\[
  (f(\overset{\circ}{B_{\delta}}(a) \cap E) \subset B_{\varepsilon}(b))
\]
\end{definition}
\begin{note}
Если для ф-ции $f: \N \rightarrow \R$ - положить $a = +\infty$: дост-но положить $N = \floor{\frac{1}{\delta}} + 1$
\end{note}
\begin{definition}
Число $b$ наз-ся пределом ф-ции $f$ в точке $a \in \R$, если $a$ - предельная точка мн-ва $E$ и:
\[
\forall \varepsilon > 0, \exists \delta > 0, \forall x \in E (0 < \left|x - a\right| < \varepsilon \Rightarrow \left|f(x) - b\right| < \varepsilon)
\]
\end{definition}
\begin{definition}[по Гейне]
  Точка $b$ наз-ся пределом ф-ции $f$ в точке $a$, если $a$ - предельная точка мн-ва $E$ и:
  \[
  \forall \set{x_n}, x_n \in E\backslash\set{a} (x_n \rightarrow a \Rightarrow f(x_n) \rightarrow b)
  \]
\end{definition}
\begin{note}
  Поскольку $a$ - предельная точка мн-ва $E$, то
  \[
    \forall \delta > 0 \colon \overset{\circ}{B_{\delta}}(a) \cap E \neq \emptyset
  \]
  и сущ-ет $\set{x_n} \subset E \backslash \set{a}, x_n \rightarrow a$
\end{note}
\begin{example}
\[
f: \R \rightarrow \R, f(x) = x^{2}
\]
\[
a \in \R \text{ - предельная точка $\R$}
\]
Покажем, что:
\[
\lim_{x\to a} x^{2} = a^{2}
\]
Зафикс. $\varepsilon > 0$ и пусть $\delta \leq 1$:
\[
0 < \left|x - a\right| < \delta \leq 1
\]
\[
\left| x + a\right| = \left|x - a + 2a\right| < \left|x - a\right| + 2\left|a\right| \leq 1 + 2\left|a\right|
\]
Возьмем $\delta = min\set{1, \frac{\varepsilon}{2\left|a\right| + 1}}$:
\[
0 < \left|x - a\right| < \delta \iff 0 < \left|x - a\right| < \frac{\varepsilon}{2\left|a\right| + 1} \iff \left|x^{2} - a^{2}\right| < \left|x - a\right|(2\left|a\right| + 1) < \varepsilon
\]
Рассм. по Гейне:
\[
x_n \neq a, x_n \rightarrow a \Rightarrow x_n^{2} \rightarrow a^{2} \iff f(x_n) \rightarrow f(a)
\]
\end{example}
\begin{theorem}
Определения по Коши и по Гейне \textbf{равносильны}.
\end{theorem}
\begin{proof}
Пусть $f: E \rightarrow \R$ и $a$ - предельная точка мн-ва $E$.
\begin{itemize}
  \item [\textbf{Опр. 1 $\Rightarrow$ Опр. 2})] Пусть $b = \lim_{x\to a} f(x)$ по Коши \\ 

    Рассм. произвольную п-ть $\set{x_n}, x_n \in E \backslash \set{a}$ и $x_n \rightarrow a$. Заф. $\varepsilon > 0$. По опр-ю предела ф-ции $\exists \delta > 0, \forall x \in \overset{\circ}{B_{\delta}}(a) \cap E \colon (f(x) \in B_{\varepsilon}(b))$ \\

    Т. к. $x_n \rightarrow a$, то $\exists N, \forall n \geq N (x_n \in B_{\delta}(a))$. Имеем $x_n \in \overset{\circ}{B_{\delta}}(a) \cap E$ при всех $n \geq N$, а значит, $f(x_n) \in B_{\varepsilon}(b)$ при всех $n \geq N$. Сл-но, $f(x_n) \rightarrow b$. Опр. 2 выполн-ся.

  \item [\textbf{Опр. 2 $\Rightarrow$ Опр. 1})] Пусть $b = \lim_{x\to a} f(x)$ по Гейне. Предположим, что Опр. 1 не выполняется:
    \[
    \exists\varepsilon > 0, \forall \delta > 0, \exists x \in E (x \in \overset{\circ}{B_{\delta}}(a) \land f(x) \not\in B_{\varepsilon}(b))
    \]
    Положим $\delta = \frac{1}{n}, n \in \N$ и соотв. знач-е $x$ обозначим через $x_n$. По индукции будет построена посл-ть $\set{x_n}$, т. ч. $x_n \in \overset{\circ}{B_{\frac{1}{n}}}(a) \cap E$. \\

    Имеем $\set{x_n} \subset E \backslash \set{a}$ и по т. о зажатой п-ти $x_n \rightarrow a$, а значит $f(x_n) \rightarrow b$ \\

    По опр-ю предела посл-ти $\exists N, \forall n \geq N(f(x_n) \in B_{\varepsilon}(b))!!!$ \\

    Сл-но опр. 2 не выполняется !!!
\end{itemize}
\end{proof}
\begin{note}
  Опр-е предела по Гейне можно ослабить, считая, что $\set{x_n}$ монотонна. (Задача !)
\end{note}
\textbf{Св-ва предела ф-ции:} \\

Пусть $f, g, h: E \rightarrow \R$ и $a$ - предел. точка $E$:
\begin{itemize}
  \item [C1: ] \textbf{(Единственность предела)} Если $\lim_{x\to a} f(x) = b$ и $\lim_{x\to a} f(x) = c$, то $b = c$
    \begin{proof}
    Рассм. произвольую п-ть $\set{x_n}, x_n \in E \backslash \set{a}$ и $x_n \rightarrow a$ \\ 

    По опр-ю Гейне $f(x_n) \rightarrow b$ и $f(x_n) \rightarrow c$. Т. к. предел посл-ти единственнен, то $b = c$
    \end{proof} 
  \item [C2: ] \textbf{(Предел по подмн-ву)} Если $a$ - предел. точка мн-ва $D \subset E$ и
    \[
      \lim_{x\to a} f(x) = b
    \]
    Тогда $\lim_{x\to a} (f|_{D}) = b$
    \begin{proof}
    Рассм. произв. $\set{x_n}, x_n \in D \backslash \set{a}, x_n \rightarrow a$. Тогда:
    \[
      (f|_{D})(x_n) = f(x_n) \rightarrow b
    \]
    По опр-ю Гейне $ b = \lim_{x\to a}(f|_D)(x)$
    \end{proof}
  \item [C3: ] \textbf{(Предел зажатой ф-ции)} Если:
    \[
      \exists \delta > 0, \forall x \in \overset{\circ}{B_{\delta}}(a) \cap E \colon (f(x) \leq h(x) \leq g(x))
    \]
    и $\lim_{x\to a}f(x) = \lim_{x\to a} g(x) = b$, то сущ-ет $\lim_{x\to a} h(x) = b$

    \begin{proof}
    Рассм. произв. $\set{x_n}, x_n \in E \backslash \set{a}, x_n \rightarrow a$. Тогда $\exists N, \forall n \geq N (x_n \in \overset{\circ}{B_{\delta}}(a) \cap E)$, а значит:
    \[
    f(x_n) \leq h(x_n) \leq g(x_n), \forall n \geq N
    \]
    По т. о зажатой п-ти: 
    \[
    \begin{cases}
    f(x_n) \rightarrow b \\
    g(x_n) \rightarrow b
    \end{cases} \Rightarrow h(x_n) \rightarrow b
    \]
    По опр-ю Гейне $b = \lim_{x\to a}h(x)$
    \end{proof}
  \item [C4: ] \textbf{(Свойство локализации)} Если $f$ и $g$ совпадают на $\overset{\circ}{B_{\delta}}(a) \cap E$ и $\lim_{x\to a} f(x) = b$, то сущ-ет $\lim_{x \to a}g(x) = b$
\end{itemize}
