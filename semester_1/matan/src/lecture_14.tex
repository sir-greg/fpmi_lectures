\section{Лекция 14}
\begin{definition}
Натуральным логарифмом наз-ся ф-ция $\ln \colon (0, +\infty)$, обратная к $\exp$
\end{definition}
\begin{note}
По т. об обратной ф-ции и св-в экспоненты, можно получить св-ва нат. логарифма:
\begin{itemize}
  \item $\ln$ непр-на на обл-ти определения.
  \item $\ln$ строго возр.
  \item $\ln$ отображет $(0, +\infty)$ на $\R$, при этом, если $x_1, x_2 > 0 \Rightarrow$
    \[
    \ln(x_1 \cdot x_2) = \ln (x_1) + \ln (x_2)
    \]
\end{itemize}
\end{note}
\begin{definition}
Пусть $a > 0, a \neq 1$. \underline{Показательной ф-цией с основанием $a$} наз-ся ф-ция: $x \mapsto a^{x} = \exp(x\ln a), x \in \R$
\end{definition}
\begin{note}
Показательная ф-ция непр-на, строго монотонна (при $a > 1$ строго возрастает, иначе - строго убывает), а также отображает $\R$ на $(0, +\infty)$
\end{note}
\begin{note}
Пусть $n \in \N, n > 1$. Тогда:
\[
  \left(a^{\frac{1}{n}}\right)^{n} = \exp(\frac{1}{n}\ln a) \cdot \exp(\frac{1}{n}\ln a) = \exp(\ln a) = a
\]
Сл-но, $a^{\frac{1}{n}} = \sqrt[n]{a}$
\end{note}
\begin{definition}
Пусть $a > 0, a \neq 1$. \underline{Логарифмической ф-цией с основанием $a$} наз-ся ф-ция $\log_a \colon (0; +\infty) \rightarrow \R$. Обратная к показательной ф-ции $x \mapsto a^{x}, x \in \R$
\end{definition}
\begin{note}
Логарифмическая ф-ция непр-на, строго монотонна и отображает $(0, +\infty)$ на $\R$. Кроме того:
\[
x = a^{y} \iff x = \exp(y\ln a) \iff \ln(x) = y\ln a \Rightarrow \log_a(x) = \frac{\ln x}{\ln a}
\]
\end{note}
\begin{definition}
Пусть $a \in \R$. \underline{Степенной ф-цией с показателем $\alpha$} наз-ся ф-ция $x \mapsto x^{\alpha}, x \in E$, где:
\begin{itemize}
  \item [1) ] $\alpha \in \N \cup \set{0} \Rightarrow E = \R$, при этом $x^{0} = 1, x^{\alpha} = \underset{\alpha \text{ раз}}{x \cdot \ldots \cdot x}$
  \item [2) ] $\alpha \in -\N \Rightarrow E = \R \backslash \set{0}$, при этом $x^{\alpha} = \frac{1}{x^{-\alpha}}$
  \item [3) ] $\alpha \in \R \backslash \Z \Rightarrow E = (0, +\infty)$, при этом $x^{\alpha} = \exp(\alpha\ln x)$
\end{itemize}
\end{definition}
\begin{note}
Если в последнем случае $\alpha > 0$, то полагаем $0^{\alpha} = 0$ (т. е. $0$ включаем в $E$), это согласуется с тем, что:
\[
  \lim_{x \to +0} \exp(\alpha \ln x) = 0
\]
\end{note}
\begin{note}
Из св-в $\exp$ и $\ln$ получаем, что степенная ф-ция непр-на на $E$, на $(0, +\infty)$ строго возрастает на  при $\alpha > 0$ и строго убывает при $\alpha < 0$
\end{note}
\begin{lemma}[Замечательные пределы]
\[
\lim_{x\to 0} \frac{e^{x} - 1}{x} = 1
\]
Кроме того:
\[
\lim_{x\to 0} \frac{\ln(x + 1)}{x} = 1
\]
\[
\lim_{x\to 0} (1 + x)^{\frac{1}{x}} = e
\]
\end{lemma}
\begin{proof}
По пред. лемме при $x < 1$:
\[
1 + x \leq e^{x} \leq \frac{1}{1 - x} \iff x \leq e^{x} - 1 \leq \frac{x}{1 - x} \iff
\]
\[
  \begin{cases}
1 \leq \frac{e^{x} - 1}{x} \leq \frac{1}{1 - x}, x > 0 \\
\frac{1}{1 - x} \leq \frac{e^{x} - 1}{x} \leq 1, x < 0
  \end{cases} \Rightarrow \lim_{x\to 0} \frac{e^{x} - 1}{x} = 1
\]
Ф-ция $g(y) = \begin{cases}
\frac{y}{e^{y} - 1}, y \neq 0 \\
1, y = 0
\end{cases}$ - непр. в 0. Также $f(x) = \ln(x + 1)$ непр-на в 0. Тогда композиция $g \circ f$ непр-на в 0
\[
h(x) = g \circ f(x) \Rightarrow h(x) = \begin{cases}
  \frac{\ln(x + 1)}{x}, x > 0 \\
  1, x = 0
\end{cases}
\]
\[
  \Rightarrow \lim_{x\to 0} h(x) = \lim_{x\to 0} \frac{\ln (x + 1)}{x} = g(0) = 1 
\]
Тогда и $\exp \circ h(x) = (1 + x)^{\frac{1}{x}}$ непр-на в 0 $\Rightarrow \lim_{x\to 0} (1 + x)^{\frac{1}{x}} = e^{1} = e$
\end{proof}
\begin{task}
Док-те, что $\lim_{x\to 0} \frac{(1 + x)^{\alpha} - 1}{x} = \alpha$
\end{task}
\begin{example}
\[
e^{\pi} \lor \pi^{e}
\]
\end{example}
\begin{proof}
\[
e^{x} > 1 + x
\]
\[
x = \frac{\pi}{e} + 1
\]
\[
e^{\frac{\pi}{e} - 1} > \frac{\pi}{e} \iff e^{\frac{\pi}{e}} > \pi \Rightarrow e^{\pi} > \pi^{e}
\]
\end{proof}
\subsubsection{Ликбез по тригономе}
\begin{lemma}
Для всех $x \in (0, \frac{\pi}{2})$ верно:
\[
\sin x < x < \tg x
\]
\end{lemma}
\begin{proof}
Picture:
\[
S_{\triangle AOB} < S_{\text{сек. } AOB} < S_{\triangle AOC}
\]
\[
\Rightarrow \frac{1}{2}\sin x < \frac{1}{2} x < \frac{1}{2}\tg x
\]
\end{proof}
\begin{consequence}
Для всех $x \in \R$. Верно $\left|\sin x\right| < \left| x\right|$, причём рав-во имеет место только при $x = 0$
\end{consequence}
\begin{proof}
Если $x \in (0, \frac{\pi}{2})$, то нер-во следует по лемме. \\
Если $x \geq \frac{\pi}{2}$, то $\left|\sin x\right| \leq 1 < \frac{\pi}{2} \leq x$ \\
Если $x < 0$, то $\left|\sin x\right| = \left|\sin (-x)\right| < \left| (-x)\right| = \left| x\right|$
\end{proof}
\begin{consequence}
Ф-ции $\sin$ и $\cos$ непр-ны на $\R$.
\end{consequence}
\begin{proof}
Пусть $a \in \R$, тогда:
\[
\left|\sin x - \sin a\right| = 2\left|\sin \frac{x - a}{2}\right|\left|\cos \frac{x + a}{2}\right| \leq 2\frac{\left|x - a\right|}{2} = \left|x - a\right| \rightarrow 0
\]
Сл-но, $\sin x$ в точке $a$ равен $\sin a \Rightarrow \sin x$ - непр-на. Аналогично доказывается непр-ть $\cos x$ или из ф-л тригонометрии:
\[
  \cos x = \sin(\frac{\pi}{2} - x) \Rightarrow
\]
$\cos x$ непр-н как композиция непр. ф-ций.
\end{proof}
\begin{consequence}
\[
\lim_{x\to 0} \frac{\sin x}{x} = 1
\]
\end{consequence}
\begin{proof}
$x \in (0, \frac{\pi}{2}) \Rightarrow \frac{\sin x}{x} < 1$ и $\cos x < \frac{\sin x}{x} < 1$ (из леммы)
В силу чётности, $\lim_{x\to_-0} \frac{\sin x}{x} = 1$ $\Rightarrow$ предел = 1. 
\end{proof}
\begin{definition}
Обратные тригонометрические ф-ции:
\begin{itemize}
  \item [1) ] $\arcsin$:
    \[
    \arcsin = (\sin|_{[-\frac{\pi}{2}, \frac{\pi}{2}]})^{-1}
    \]
  \item [2) ] $\arccos \colon [-1, 1] \rightarrow [0, \pi]$
    \[
    \arccos = (\cos|_{[0, \pi]})^{-1}
    \]
  \item[3) ] $\arctg \colon \R \rightarrow (-\frac{\pi}{2}, \frac{\pi}{2})$
    \[
    \arctg = (\tg|_{(-\frac{\pi}{2}, \frac{\pi}{2})})^{-1}
    \]
  \item [4) ] $\arcctg \colon \R \rightarrow (0, \pi)$
    \[
    \arcctg = (\ctg|_{(0, \pi)})^{-1}
    \]
\end{itemize}
\end{definition}
\begin{definition}
\underline{Основными элементарными ф-циями} наз-ся:
\begin{itemize}
  \item $x \mapsto c, c \in \R$
  \item $x \mapsto a^{x}, a > 0, a \neq 1$
  \item $x \mapsto \log_a x, a > 0, a \neq 1$
  \item $x \mapsto x^{\alpha}$
  \item $\sin, \cos, \tg, \ctg$
  \item $\arcsin, \arccos, \arctg, \arcctg$
\end{itemize}
\end{definition}
\begin{definition}
\underline{Элементарной ф-цией} наз-ся любая ф-ция, полученная конечным числом арифметических операций или взятием их композиции.
\end{definition}
\begin{example}
\[
\sh \colon \R \rightarrow \R, \sh x = \frac{e^{x} - e^{-x}}{2}
\]
\[
\ch \colon \R \rightarrow \R, \ch x = \frac{e^{x} + e^{-x}}{2}
\]
\[
  \th \colon \R \rightarrow \R, \th = \frac{\sh x}{\ch x}
\]
\[
\cth \colon \R \backslash \set{0} \rightarrow \R, \cth x = \frac{\ch x}{\sh x}
\]
\end{example}
\begin{theorem}
Всякая элементарная ф-ция непр-на на своей области определения.
\end{theorem}
\subsubsection{Сравнение ф-ций}
\begin{definition}
Пусть $f, g \colon E \rightarrow \R$, $a$ - предельная точка $E$ и сущ-ет $\alpha \colon E \rightarrow \R$ и $\delta > 0$, такие, что
\[
  f(x) = \alpha(x)g(x), \forall x \in \overset{\circ}{B_{\delta}}(a) \cap E
\]
Тогда:
\begin{itemize}
\item [1) ] Если $\alpha(x) \rightarrow 1$ при $x \rightarrow a$, то говорят, что ф-ции $f$ и $g$ \underline{эквивалентны} \underline{(асимптотически равны)} при $x \rightarrow a$. Пишут $f(x) \sim g(x)$ при $x \rightarrow a$
  \item [2) ]  Если $\alpha(x) \rightarrow 0$ при $x \rightarrow a$, то говорят, что ф-ция $f$ \underline{беск. мала по сравн. с ф-цией} $g$ при $x \rightarrow a$, пишут $f(x) = o(g(x))$, при $x \rightarrow a$
  \item [3) ] Если $\alpha$ - огр-на, то говорят, что ф-ция $f$ \underline{ограничена по сравнению} с $g$ при $x \rightarrow a$. Пишут, что $f(x) = O(g(x)), x \rightarrow a$
\end{itemize}
\end{definition}
\begin{note}
Если $g(x) \neq 0$ в некот. проколот. окр-ти $a$, с учётом обл. опр-я, то:
\begin{itemize}
  \item [1) ] \[
  f(x) \sim g(x), x \rightarrow a \iff \lim_{x\to a} \frac{f(x)}{g(x)} = \lim_{x\to a}\alpha(x) = 1
  \]
\item [2) ] \[
  f(x) = o(g(x)), x \rightarrow a \iff \lim_{x\to a} \frac{f(x)}{g(x)} = 0
\]
\item [3) ] \[
  f(x) = O(g(x)), x \rightarrow a \iff \exists M > 0, \exists \delta > 0, \forall x \in \overset{\circ}{B_{\delta}}(a) \cap E \left(\left|\frac{f(x)}{g(x)}\right| \leq M\right)
\]
\end{itemize}
\begin{proof}
  \begin{itemize}
    \item [$\Rightarrow$)]
Следует из опр-я.
  \item [$\Leftarrow$)] Положим:
    \[
      \alpha \colon E \rightarrow \R, \alpha(x) = \begin{cases}
        \frac{f(x)}{g(x)}, x \in \overset{\circ}{B_{\delta}}(a) \cap E \\
        \text{что угодно}, \text{ иначе}
      \end{cases}
    \]
  \end{itemize}
\end{proof}
\end{note}
\begin{task}
Доказать, что $\sim$ - отн. эквив-ти.
\end{task}
\begin{example}
\begin{itemize}
  \item [1) ] \[
  x^{n} = o(x^{m}), x \rightarrow 0 \iff n > m
  \]
  \[
  x^{n} = x^{n - m} x^{m}
  \]
\item [2) ] \[
  x^{n} = o(x^{m}), x \rightarrow \pm \infty \Rightarrow m > n
\]
\item [3) ] \[
  x = O(\sin x), x \rightarrow 0
\]
  \[
  x + \cos x = O(x), x \rightarrow \pm \infty
  \]
\item [4) ] \[
  x \sim \sin x \sim e^{x} - 1 \sim \ln(1 + x), x \rightarrow 0
\]
\end{itemize}
\end{example}
\begin{note}
  Читаем $f(x) = O(g(x)), f(x) = o(g(x))$ - слева направо!
\end{note}
\begin{lemma}
\[
x \rightarrow a
\]
Тогда справедливо:
\begin{itemize}
  \item [1) ] \[
  o(f) \pm o(f) = o(f), O(f) \pm O(f) = O(f)
  \]
\item [2) ] \[
  o(f) = O(f)
\]
\item [3) ] \[
  o(O(f)) = o(f), O(o(f)) = o(f)
\]
\item [4) ] \[
  o(f)O(g) = o(fg)
\]
\end{itemize}
\end{lemma}
\begin{proof}
\begin{itemize}
  \item [3) ]\[
      \begin{cases}
  u = o(v), x \rightarrow a \\
  v = O(f), x \rightarrow a
      \end{cases} \Rightarrow 
      \begin{cases}
       u(x) = \alpha(x)v(x)  \\
       v(x) = \beta(x)f(x)
      \end{cases} \forall x \in \overset{\circ}{B_{\delta}}(a) \cap E
  \]
  \[
  \Rightarrow u(x) = \alpha(x)\beta(x)f(x) = \gamma(x)f(x), \gamma \rightarrow 0, x \rightarrow a
  \]
  \[
  \Rightarrow u(x) = o(f), x \rightarrow a
  \]
\end{itemize}
\end{proof}
\begin{lemma}
\begin{itemize}
  \item [1) ] \[
  f(x) \sim g(x), x \rightarrow a \iff f(x) = g(x) + o(g(x)), x \rightarrow a
  \]
\item [2) ] \[
  f_1(x) \sim f_2(x), g_1(x) \sim g_2(x), x\rightarrow a \Rightarrow f_1(x)g_1(x) \sim f_2(x)g_2(x)
\]
Кроме того, если $g_{1, 2}(x) \neq 0$ в некот. прок. окр-ти $a$, то:
\[
  \frac{f_1(x)}{g_1(x)} \sim \frac{f_2(x)}{g_2(x)}, x \rightarrow a
\]
\item [3) ] \[
    f(x) \sim g(x), x \rightarrow a
\]
То пределы $f(x), g(x)$ при $x \rightarrow a$ сущ-ют одновременно (в $\overline{\R}$), и если сущ-ют, то равны.
\end{itemize}
\end{lemma}
\begin{proof}
\begin{itemize}
  \item [1) ] \[
  f(x) \sim g(x), x \rightarrow a \iff f(x) = \alpha(x) g(x), x \in \overset{\circ}{B_{\delta}}(a), \alpha(x) \rightarrow 1
  \]
  \[
  f(x) = \alpha(x) g(x) =  g(x) + g(x)\underset{\rightarrow 0}{(\alpha(x) - 1)} \iff f(x) = g(x) + o(g(x))
  \]
\item [2) ] Пусть $g_1(x) \neq 0, \forall x \in \overset{\circ}{B_{\delta}}(a) \cap E$
  \[
  g_2(x) = \alpha(x) g_1(x), \forall x \in \overset{\circ}{B_{\delta}}(a) \cap E
  \]
  \[
  \alpha(x) \rightarrow 1, x \rightarrow a
  \]
  Т. к. $\alpha(x)$, то $\exists \delta_1 > 0, \forall \in \overset{\circ}{B_{\delta_1}}(a) \cap E (\alpha(x) \in (\frac{1}{2}, \frac{3}{2}))$ \\
  $\delta_0 = min(\delta, \delta_1)$. Тогда $g_2(x) \neq 0$ на $\overset{\circ}{B_{\delta_2}}(a) \cap E$ \\
  Рассм. $\frac{1}{g_1(x)}, \frac{1}{g_2(x)}, x \in \overset{\circ}{B_{\delta}}(a) \cap E \Rightarrow \forall x\in \overset{\circ}{B_{\delta}}(a) \cap E (\frac{1}{g_1(x)} = \frac{1}{\alpha(x)g_2(x)}) \Rightarrow \frac{1}{g_1(x)} \sim \frac{1}{g_2(x)}, x \rightarrow a$
\end{itemize}
\end{proof}
\begin{example}
\[
  \lim_{x\to 0} \frac{\tg x - \sin x}{(\sqrt {x + 4} - 2)(2^{x} - 1)^{2}}
\]
\end{example}
\begin{solution}
\[
\lim_{x\to 0} \frac{\tg x}{x} = \lim_{x\to 0} \frac{\sin x}{x} \cdot \frac{1}{\cos x} = 1 \Rightarrow \tg x \sim x, x \rightarrow 0
\]
\[
\tg x - \sin x = \frac{\sin x}{\cos x}(1 - \cos x) = \frac{2\sin x \sin^{2}\frac{x}{2}}{\cos x} \sim 2 \cdot x \cdot (\frac{x}{2})^{2} \sim \frac{x^{3}}{2}, x \rightarrow 0
\]
\[
  \sqrt{x + 4} - 2 = \frac{x}{\sqrt{x + 4} + 2} \sim \frac{x}{4}, x \rightarrow 0
\]
\[
  e^{x} - 1 \sim x, x \rightarrow 0
\]
\[
  \lim_{x\to 0} \frac{\tg x - \sin x}{(\sqrt {x + 4} - 2)(2^{x} - 1)^{2}} = \lim_{x\to 0} \frac{\frac{1}{2}x^{3}}{\frac{x}{4} \cdot x^{2}} = 2
\]
\[
\tg x \sim x, x \rightarrow 0 \iff \tg x = x + o(x), x \rightarrow 0
\]
\[
\sin x \sim x, x \rightarrow 0 \iff \sin x = x + o(x), x \rightarrow 0
\]
\[
\tg x - \sin x = o(x), x \rightarrow
\]
\end{solution}
