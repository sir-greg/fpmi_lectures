\section{Лекция 7}
\subsection{Критерий Коши}
\begin{definition}
Посл-ть $\{a_n\}_{1}^{\infty}$ наз-ся \textbf{фундаментальной}, если:
\[
\forall \varepsilon > 0, \exists N \colon \forall n, m \geq N (\left|a_n - a_m\right| < \varepsilon)
\]
\end{definition}
\begin{lemma}
Всякая фундаментальная п-ть огр-на
\end{lemma}
\begin{proof}
Пусть $\{a_n\}_{1}^{\infty}$ - фундаментальна. По опр-ю:
\[
  \exists N \colon \forall n, m \geq N (\left|a_n - a_m\right| < 1)
\]
В част-ти:
\[
a_N - 1 < a_n < a_N + 1
\]
для всех $n \geq N$ ($m = N$)

Положим
\[
  \alpha = min(a_1, \ldots, a_{N - 1}, a_N - 1)
\]
\[
  \beta = max(a_1, \ldots, a_{N - 1}, a_{N} + 1)
\]
. Тогда:
\[
\alpha \leq a_n \leq \beta
\]
при всех $n \in \N$
\end{proof}
\begin{theorem}[Коши]
П-ть $\{a_n\}_{1}^{\infty}$ - сходится $\iff$ $\{a_n\}_{1}^{\infty}$ - фундаментальна.
\end{theorem}
\begin{proof}
\begin{itemize}
  \item[$\Rightarrow$)] Пусть $\lim_{n\to\infty} a_n = a$. Зафикс. $\varepsilon > 0$. По опр-ю предела:
    \[
    \exists N, \forall n \in \N (\left|a_n - a\right| < \frac{\varepsilon}{2})
    \]
    Тогда при всех $n, m \geq N$:
    \[
    \left|a_n - a_m\right| \leq \left|a_n - a\right| + \left|a_m - a\right| < \frac{\varepsilon}{2} * 2 = \varepsilon
    \]
  \item [$\Leftarrow$)] По предыдущей лемме, п-ть $\{a_n\}_{1}^{\infty}$ - ограничена $\Rightarrow$ по т. Больцано-Вейерштрасса (Б-В) $\{a_n\}_{1}^{\infty}$ имеет сход. подпосл-ть $\{a_{n_k}\}_{1}^{\infty} \rightarrow a$

    Покажем, что $a = \lim_{n\to\infty}$. Зафикс. $\varepsilon > 0$. По опр-ю фундаментальности:
    \[
    \exists N, \forall n, m \geq N (\left|a_n - a_m\right| < \frac{\varepsilon}{2})
    \]
    Т. к. $\{a_{n_k}\} \rightarrow a \Rightarrow$
    \[
    \exists K \colon \forall k \geq K (\left|a_{n_k} - a\right| < \frac{\varepsilon}{2})
    \]
    Положим $M = max(N, K)$. Тогда $n_M \geq M \geq N; n_M \geq M \geq K$

    Поэтому при всех $n \geq N$:
    \[
    \left|a_n - a\right| \leq \left|a_n - a_{n_M}\right| + \left|a_{n_M} - a\right| < \frac{\varepsilon}{2} + \frac{\varepsilon}{2} = \varepsilon
    \]

\end{itemize}
\end{proof}
\begin{note}
Критерий Коши позволяет доказывать существование предела, \textbf{без явного нахождения его значения}

Кроме того, критерий позволяет \textbf{оценить скорость сходимости к пределу} (перейдём к пределу по $m$ в определении фунд-ти):
\[
\left|a_n - a\right| \leq \varepsilon, \text{при всех $n \geq N$}
\]
\end{note}
\begin{task}
Покажите, что если всякая фундаментальная посл-ть сх-ся (сходится), то выполняется аксиома непрерывности. А именно:

Пусть $\mathbb{F}$ - упоряд. поле, на котором выполняется аксиома Архимеда
\end{task}

\subsection{Частичные пределы}
\begin{definition}
  Точка $a \in \overline{\R}$ наз-ся частичным пределом числовой посл-ти $\{a_n\}_{1}^{\infty}$, если $\exists \set{a_{n_k}}$ - подпосл-ть $\set{a_n} \colon \lim_{k\to\infty} a_{n_k} = a$
  \[
 L\set{a_n} - \text{мн-во частичных пределов $\set{a_n}$} 
  \]
\end{definition}
\begin{example}
$\pm 1$ - частичные пределы $a_n = (-1)^{n}$
\[
a_{2k} \rightarrow 1, a_{2k - 1} \rightarrow -1
\]
\end{example}
Пусть задана числовая посл-ть $\set{a_n}$

Положим
\[
  M_n = \underset{k \geq n}{\sup} \set{a_k}
\]
\[
  m_n = \underset{k \geq n}{\inf} \set{a_k}
\]
Пусть $\set{a_n}$ огр. сверху. Тогда все $M_n \in \R$ \\
Поскольку при переходе к подмн-ву $\sup$ не увеличивается, то $\set{M_n}$ нестрого убывает \\
$\Rightarrow$ $\exists \lim_{n\to\infty} M_n$ \\

Пусть $\set{a_n}$ не огр. сверху. Тогда все $M_n = +\inf$ \\
Положим $\lim_{n\to\infty} M_n = +\infty$ \\

Аналогично для $\set{m_n}$ (Огр./Неогр. снизу). \\

Итак, посл-ти $\set{m_n}$ и $\set{M_n}$ имеют предел в $\overline{\R}$

\begin{definition}
Величина $\lim_{n\to\infty} \underset{k \geq n}{\sup} \set{a_k}$ - \textbf{верхний предел} $\set{a_n}$ и об-ся $\overline{\lim_{n\to\infty}}a_n$ \\

Величина $\lim_{n\to\infty} \underset{k \geq n}{\inf} \set{a_k}$ - \textbf{нижний предел} $\set{a_n}$ и об-ся $\underline{\lim_{n\to\infty}}a_n$
\end{definition}
\begin{note}
Т. к. $m_n \leq M_n, \forall n \in \N$, тогда:
\[
\underline{\lim_{n\to\infty}}a_n \leq \overline{\lim_{n\to\infty}}a_n
\]
\end{note}
\begin{task}
  \label{prot_pred}
\[
\overline{\lim_{n\to\infty}}(-a_n) = -\underline{\lim_{n\to\infty}}a_n
\]
\end{task}
\begin{theorem}
Верхний (нижний) предел - наибольший (наименьший) из част. пределов посл-ти.
\end{theorem}
\begin{proof}
\[
M = \overline{\lim_{n\to\infty}}a_n, m = \underline{\lim_{n\to\infty}}a_n
\]
Нужно показать, что $M, m$ - это ч. п. $\set{a_n}$  и любой ч. п. лежит между ними.
\begin{itemize}
  \item [1) ]  Покажем, что есть подп-ть $\set{a_n}$, сх-ся к $M$:
    \begin{itemize}
      \item [I. ] $M \in \R$. Имеем
        \[
          M = \inf \set{M_n}
        \]
        По опр-ю $\sup, \exists n_1 \colon (M - 1 < a_{n_1})$
        \[
        M_{n_1 + 1} = \underset{k \geq n_1 + 1}{\sup} \set{a_k} \Rightarrow \exists n_2 > n_1 \colon (M - \frac{1}{2} < a_{n_2}) 
        \] и т. д. \\
        Таким образом, по индукции, будет построена подп-ть $\set{a_{n_k}}$, т. ч.
        \[
          M - \frac{1}{k} < a_{n_k}
        \]
        Имеем:
        \[
        M - \frac{1}{k} < a_{n_k} \leq M_{n_k}
        \]
        Края нер-ва сх-ся к $M \Rightarrow $ по т. о зажатой посл-ти, $a_{n_k} \rightarrow M$
      \item [II. ] $M = +\infty$, тогда $\set{a_n}$ неогр. сверху $\Rightarrow$ (по Теореме 8') она имеет под-пть, сх-ся к $+\infty$
      \item [III. ] $M = -\infty$. Т. к. $a_n \leq M_n, \forall n \Rightarrow \lim_{n\to\infty} a_n = -\infty$
    \end{itemize}
  \item [2) ] Для $m$ - док-во аналогично, или сводиться к $M$ по задаче prot-pred
  \item [3) ] Пусть $\set{a_{n_k}}, a_{n_k} \rightarrow a$. Тогда:
    \[
    m_{n_k} \leq a_{n_k} \leq M_{n_k}, \forall k \Rightarrow m \leq a \leq M \text{(част. пределы)}
    \]
\end{itemize}
\end{proof}
\begin{consequence}
$\exists \lim_{n\to\infty}a_n$ (в $\overline{\R}$) $\iff \overline{\lim_{n\to\infty}}a_n = \underline{\lim_{n\to\infty}} a_n$ \\
В этом случае все три предела равны.
\end{consequence}
\begin{proof}
  \begin{itemize}
    \item [$\Rightarrow$)]
По лемме 4, любая подпосл-ть имеет предел $a \Rightarrow \overline{\lim_{n\to\infty}}a_n = \lim_{n\to\infty}a_n = \underline{\lim_{n\to\infty}}a_n$ 
    \item [$\Leftarrow$)] \[
    m_n \leq a_n \leq M_n
    \]
    для всех $n \Rightarrow a_n \rightarrow a$ (Края $\rightarrow a$)
  \end{itemize}
\end{proof}
\begin{lemma}
Для $c \in \R$ верно:
\begin{equation*}
c = \overline{\lim_{n\to\infty}}a_n \iff 
\begin{cases}
\forall \varepsilon > 0, \exists N, \forall n \geq N (a_n < c + \varepsilon) \text{ (1)}\\
\forall \varepsilon > 0, \forall N, \exists n \geq N (a_n > c-\varepsilon) \text{ (2)}
\end{cases}
\end{equation*}
\begin{equation*}
c = \underline{\lim_{n\to\infty}}a_n \iff
\begin{cases}
\forall \varepsilon > 0, \forall N, \exists n \geq N (a_n < c + \varepsilon) \\
\forall \varepsilon > 0, \exists N, \forall n \geq N (a_n > c - \varepsilon)
\end{cases}
\end{equation*}
\end{lemma}
\begin{proof}
Докажем, для верх предела:
\[
  \overline{\lim_{n\to\infty}}a_n = \lim_{n\to\infty} M_n, M_n = \underset{k \geq n}{\sup} \set{a_k}
\]
\[
\text{(1) } \iff \forall \varepsilon > 0, \exists N (M_N < c + \varepsilon)
\]
\[
\text{(2) } \iff \forall \varepsilon > 0, \forall N (M_n > c - \varepsilon)
\]
Напомним, что $\set{M_n}$ - нестрого убыв. \\

Тогда $(1) \land (2) \iff c = \lim_{n\to\infty} M_n \text{ ($ = \inf \set{M_n}$)}$
\end{proof}
