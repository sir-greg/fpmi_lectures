\section{Лекция 15}
\begin{definition}
Пусть $f$ опр-на в некот. окр-ти $+\infty$. Пряая $y = kx + b$ наз-ся наклонной асимптотой $f$ при $x \rightarrow +\infty$, если:
\[
f(x) = kx + b + o(1), x \rightarrow +\infty
\]
Аналогично опр-ся накл. асимптота при $x \rightarrow -\infty$
\end{definition}
\begin{theorem}
Пусть $f$ опр-на в некот. окр-ти $+\infty, k, b \in \R$. Прямая $y = kx + b$ - наклонная асимптота $f$ при $x \rightarrow +\infty \iff$
\[
k = \lim_{x\to +\infty} \frac{f(x)}{x} \text{ и } b = \lim_{x\to +\infty}(f(x) - kx)
\]
\end{theorem}
\begin{proof}
\begin{itemize}
  \item [$\Rightarrow$) ] Пусть $y = kx + b$, накл. асимпт. $f$ при $x \rightarrow +\infty$, тогда $f(x) = kx + b + o(1), x \rightarrow +\infty \Rightarrow$
    \[
      \frac{f(x)}{x} = k + \frac{1}{x}(b + o(1)) \Rightarrow \lim_{x\to +\infty} \frac{f(x)}{x} = k
    \]
    \[
      f(x) - kx = b + o(1) \Rightarrow \lim_{x\to +\infty} (f(x) - kx) = b
    \]
  \item [$\Leftarrow$)] Рассм. $\alpha(x) = f(x) - kx - b$, где $k, b$ - пределы из усл-я:
    \[
    \lim_{x\to +\infty}  \alpha(x) = 0
    \]
    Сл-но, $f(x) = kx + b + \alpha(x), \alpha(x) \rightarrow 0, x \rightarrow +\infty$
\end{itemize}
\end{proof}
\begin{note}
Справедливо аналогичное утв-е при $x \rightarrow -\infty$
\end{note}
\begin{definition}
Пусть $a \in \R$. Ф-ция $f$ опр-на на $(\alpha, a)$ или $(a, \beta)$. \\
Прямая $x = a$ наз-ся вертикальной асимптотой ф-ции $f$, если хотя бы один из $f(a + 0)$ или $f(a - 0)$ равен $+\infty(-\infty)$
\end{definition}
\subsection{Дифференцируемые ф-ции}
Пусть $I$ - невырожд. пром-к в $\R$ (содержит более 1 точки).
\begin{definition}
  Пусть $f: I \rightarrow \R, a \in I$:
  \[
  f'(a) = \lim_{x\to a} \frac{f(x) - f(a)}{x - a} \text{ - производная ф-ции $f$ в точке $a$}
  \]
  Если предел конечен, то $f$ наз-ся \underline{дифференцируемой} в $a$.
\end{definition}
\begin{example}
\begin{itemize}
  \item [1) ] $f: \R \rightarrow \R, f(x) = kx + b$
    \[
    f'(a) = \lim_{x\to a} \frac{k(x - a)}{x - a} = k
    \]
  \item [2) ] $f: \R \rightarrow \R, f(x) = sign(x)$ 
    \[
    f'(0) = \lim_{x\to 0} \frac{sign(x) - 0}{x - 0} = \lim_{x\to 0} \frac{1}{\left|x\right|} = +\infty
    \]
\end{itemize}
\end{example}
\underline{Геометрический смысл производной}: Пусть $f$ дифф. в $a$:
\[
l \colon y = \frac{f(t) - f(a)}{t - a}(x - a) + f(a) \text{ - прямая, проход. через $(a, f(a)), (t, f(t))$}
\]
Тогда: 
\[
  K_{\text{сек.}} = \frac{f(t) - f(a)}{t - a} \rightarrow f'(a) = K_{\text{кас.}}, t \rightarrow a
\]
