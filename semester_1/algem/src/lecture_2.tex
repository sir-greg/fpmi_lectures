%\documentclass[12pt]{article}
%\usepackage[T1, T2A]{fontenc}
%\usepackage[utf8]{inputenc}
%\usepackage[russian]{babel}
%\usepackage{amsmath}
%\usepackage{amsthm}
%\usepackage{amssymb}
%\usepackage{esvect}
%\usepackage{listings}
%\usepackage{xcolor}
%\usepackage{mathrsfs}
%
%% for large comments
%\usepackage{blindtext, xcolor}
%\usepackage{comment}
%
%% for inkscape pictures
%\usepackage{import}
%\usepackage{xifthen}
%\usepackage{pdfpages}
%\usepackage{transparent}
%
%\newcommand{\incfig}[1]{%
%    \def\svgwidth{\columnwidth}
%    \import{./figures/}{#1.pdf_tex}
%}
%
%\renewcommand{\C}{\mathbb{C}}
%\newcommand{\R}{\mathbb{R}}
%\newcommand{\Q}{\mathbb{Q}}
%\newcommand{\Z}{\mathbb{Z}}
%\newcommand{\N}{\mathbb{N}}
%
%\newcommand{\floor}[1]{\left\lfloor #1 \right\rfloor}
%\newcommand{\ceil}[1]{\left\lceil #1 \right\rceil}
%
%% style of code listings
%%\definecolor{codegreen}{rgb}{0,0.6,0}
%%\definecolor{codegray}{rgb}{0.5,0.5,0.5}
%%\definecolor{codepurple}{rgb}{0.58,0,0.82}
%%\definecolor{backcolour}{rgb}{0.95,0.95,0.92}
%%
%%\lstdefinestyle{mystyle}{
%%    backgroundcolor=\color{backcolour},
%%    commentstyle=\color{codegreen},
%%    keywordstyle=\color{magenta},
%%    numberstyle=\tiny\color{codegray},
%%    stringstyle=\color{codepurple},
%%    basicstyle=\ttfamily,
%%    breakatwhitespace=false,
%%    breaklines=true,
%%    captionpos=b,
%%    keepspaces=true,
%%    numbers=left,
%%    numbersep=5pt,
%%    showspaces=false,
%%    showstringspaces=false,
%%    showtabs=false,
%%    tabsize=4
%%}
%
%\newtheorem{theorem}{\underline{Теорема}}[section]
%\newtheorem{lemma}[theorem]{\underlind{Лемма}}
%\newtheorem{statement}{\underline{Утверждение}}[section]
%\newtheorem{axiom}{\underline{Аксиома}}[section]
%\newtheorem{character}{\underline{Свойство}}[section]
%\newtheorem*{note}{\underline{Замечание}}
%\newtheorem*{symb}{\underline{Обозначение}}
%\newtheorem*{example}{\underline{Пример}}
%\newtheorem*{consequence}{\underline{Следствие}}
%\newtheorem*{solution}{\underline{Решение}}
%
%\theoremstyle{definition}
%\newtheorem{definition}{\underline{Определение}}[section]
%
%\theoremstyle{definition}
%\newtheorem{task}{\underline{Задача}}[section]
%
%\title{Алгем. \\ Лекция 2}
%\author{Сергей Григорян}

\section{Упражняемся}
$A \in M_{m*n}$
 Произвольную i-ую строку будем записывать в виде:
 \[
     A_{i*} = \begin{pmatrix} a_{i1} & a_{i2} & \cdots & a_{in}\end{pmatrix}
 .\] 

 \begin{definition}
     \textbf{Линейная комбинация (ЛК)} строк $A_{1*}, \cdots , A_{m*}$ наз-ся форм. алг. выр-е:
 \[
 \alpha_1 A_{1*} + \alpha_2 A_{2*} + \cdots + \alpha_m A_{m*}  \in M_{1n}
 .\] 
 \end{definition}

 \begin{statement}
     \begin{itemize}
         \item [a) ]  Пусть $A  \in M_{m*n}, B \in M_{n*k}$. Тогда строки матрицы $AB$ явл \textbf{ЛК} строк матрицы $B$ с коэф. из соотв. строки матрицы $A$
         \item [b) ] Столбцы матрицы $AB$ явл. ЛК столбцов матрицы $A$ с коэф. из соотв. столбцов матрицы $B$. 
     \end{itemize}

 \end{statement}
 \begin{proof}
 \textbf{b) } Пусть $C = AB  \in M_{m*k}$

  \[
      C_{*j} = \begin{pmatrix}c_{1j} \\ c_{2j} \\ \vdots \\ c_{mj}\end{pmatrix} = \begin{pmatrix} \sum_{s = 1}^{n} a_{1s}b_{sj} \\ \sum_{s = 1}^{n}  a_{2s} b_{sj} \\ \vdots \\ \sum_{s = 1}^{n} a_{ms}b_{sj}  \end{pmatrix} = \sum_{s = 1}^{n} b_{sj} \begin{pmatrix}a_{1s} \\ a_{2s} \\ \vdots \\ a_{ms} \end{pmatrix} = \sum_{s = 1}^{n}  b_{sj} A_{*s}
  .\] 
 \end{proof}
 
 \section{Векторная алгебра}
 
$V_i$ - линейное пространство  i-ого измерения. ($i = 1, 2, 3$)

\begin{definition}
Две точки $X, Y \in V_i$ определяют направленный отрезок, если известно, какая из этих точек первая, какая вторая.
\[
\overline{XY} \text{ - направленный отрезок}
.\] 
$|\overline{XY}| = XY$ - длина напр. отр.
\end{definition}
\begin{symb}
\[
\overline{0} \text{ - нулевой напр. отр.}
.\] 
\end{symb}

\begin{definition}
$\overline{XY} = \overline{X'Y'} \iff $
\begin{itemize}
    \item [a) ] $XY = X'Y'$
    \item [b) ] $\overline{XY}$ и $\overline{X'Y'}$ - коллинеарны ($\exists $ прямая, || им обоим)
    \item [c) ] $\overline{XY} \text{ и } \overline{X'Y'}$ - сонаправлены.
\end{itemize}
\end{definition}

\begin{definition}
Вектор - это класс направленных отрезков, кот. равны некоторому фиксированному напр. отр.
\end{definition}
\begin{symb}
    $\overline{a}, \overline{b}, \overline{c}$
\end{symb}

\begin{statement}
Два напр. отр. $\overline{XY}$ и $\overline{X'Y'}$ определяют (порождают) один и тот же вектор т. и т. т., когда они равны.
\end{statement}
\begin{proof}
~\newline

\textbf{a) Необходимое: } Пусть $\overline{XY}$ и $\overline{X'Y'}$ опр. один и тот же вектор $ \Rightarrow \overline{XY} = \overline{X'Y'} = \overline{a}$
 
\textbf{b) Достаточное: } Пусть $\overline{XY} = \overline{X'Y'} \Rightarrow $ они содерж. в одном классе $\overline{a} \Rightarrow $  они опред. один и тот же вектор.
\end{proof}
\begin{definition}
$\overline{XY} = \overline{a} \iff $ он порождает вектор $a$
\end{definition}

\section{Операции с векторами}
\subsection{I. Сложение}
\begin{note}
При данном векторе $\overline{a}$ и фикс. точке $X$, то найдётся напр. отр. $\overline{XY} = \overline{a}$
\end{note}

\begin{definition}
Пусть напр. отр. $\overline{XY}$ опр. $\overline{a}$, $\overline{YZ}$ опр. $\overline{b}$ :

\textbf{Сумма векторов:} вектором $\overline{a} + \overline{b}$  назыв. вектор, порожд. $\overline{XZ}$
\end{definition}
\begin{note}
Данное опр. \textbf{корректно}, и не зависит от начальной точки $X$
\end{note}
\begin{proof}
***Рисунок***
\end{proof}

\subsection{Умножение вектора на $\lambda \in \R$}
Рассм. напр. отр. $\overline{a} = \overline{XY}$ и $\overline{XZ} \colon $
\begin{itemize}
    \item [a) ] $XZ = |\lambda| * XY$ 
    \item [b) ] $\overline{XZ}$ - коллинеарен $\overline{XY}$ 
    \item [c) ] $\overline{XZ}$ сонаправлен $\overline{XY}$, при $\lambda > 0$ 

        $\overline{XZ}$ прот. направлен. $\overline{XY}$ при $\lambda < 0$ :
\end{itemize}

Вектор, определяемы напр. отр. $\overline{XZ}$, наз-ся вектором $\lambda \overline{a}$ 

\begin{proof}
to do by yourself
\end{proof}

\begin{theorem}
Операции "+" и "*$\lambda$" удовл. след. св-вам:
\begin{enumerate}
    \item Коммутативность сложения (Вытекает из св-в параллелограмма):
        \[
        \overline{a} + \overline{b} = \overline{b} + \overline{a}
        .\] 
    \item Ассоциативность сложения:
        \[
            (\overline{a} + \overline{b}) + \overline{c} = \overline{a} + (\overline{b} + \overline{c})
        .\] 
    \item $\exists  \overline{o} \colon  \overline{o} + \overline{a} = \overline{a} + \overline{o} = \overline{a}, \forall \overline{a}  \in V_i$
    \item $\forall \overline{a}  \in V_i \text{ }\exists (-\overline{a})  \in V_i \colon \overline{a} + (-\overline{a}) = (\overline{-a}) + \overline{a} = \overline{o}$
    \item Унитарность:
        \[
        1 * \overline{a} = \overline{a}, \forall \overline{a}  \in V_i
        .\] 
    \item \[
            (\lambda * \mu) * \overline{a} = \lambda * (\mu * \overline{a})
    .\] 
\item \[
        (\lambda + \mu) * \overline{a} = \lambda \overline{a} + \mu * \overline{a}
.\] 
\item \[
\lambda(\overline{a} + \overline{b}) = \lambda \overline{a} + \lambda \overline{b}
.\]  
\end{enumerate}
\end{theorem}
\begin{note}
Мн-во векторов является действительным линейным пространством отн-но мн-ва $\R$.
\end{note}

\section{Системы векторов в пр-ве $V_i$}
$ V_i, i = 1, 2, 3$ 

$\overline{v_1}, \overline{v_2}, \cdots, \overline{v_n}  \in V_i$ 

\begin{symb}
    \[
    \sum_{i = 1}^{n}  \alpha_i \overline{v_i} \text{ - наз-ся ЛК векторов}
    .\] 

    Если $\alpha_i = 0, \forall i=1\cdots n$, то такая ЛК наз-ся \textbf{тривиальной}.

    Если $\exists i \colon  \alpha_i \neq 0$, то ЛК \textbf{нетривиальная}.
\end{symb}

\begin{definition}[ЛЗ система векторов]
Система векторов $\overline{v_1}, \overline{v_2}, \cdots, \overline{v_n}$ наз-ся \textbf{линейно зависимой (ЛЗ)}, если $\exists $ \textbf{нетривиальная ЛК} этих векторов, равная $\overline{o}$
\end{definition}

\begin{definition} [ЛНЗ сис. вект.]
Система векторов $\overline{v_1}, \overline{v_2}, \cdots , \overline{v_n}$ наз-ся \textbf{линейно независимой (ЛНЗ)}, если \textbf{$ \not\exists $ нетривиальной ЛК} этих векторов, равной $\overline{o}$
\end{definition}
\begin{example}
\[
\overline{a} = \begin{pmatrix}1 \\ 0 \\ 0 \end{pmatrix}, \overline{b} = \begin{pmatrix}0 \\ 1 \\ 0 \end{pmatrix}, \overline{c} = \begin{pmatrix}0 \\ 0 \\ 1 \end{pmatrix}, \text{ - ЛНЗ сист. вект.}
.\] 
Док-во ЛНЗ: предствить, что есть коэф-ты, дающие ЛК = $\overline{o}$, и показать, что она тривиальная.
\end{example}
\begin{statement}
Система векторов $\overline{v_1}, \overline{v_2}, \cdots , \overline{v_n} \text{ - ЛЗ } \iff $ хотя бы один из них представим в виде ЛК остальных.
\end{statement}
\begin{proof}
\begin{itemize}
    \item [a) ] \textbf{Необх:} пусть $\begin{pmatrix} \overline{v_1} & \overline{v_2} & \cdots & \overline{v_n} \end{pmatrix} $ - ЛЗ:

        \[
        \Rightarrow \exists \text{ нетрив. ЛК } \colon \alpha_1 \overline{v_1} + \alpha_2 \overline{v_2} + \cdots + \alpha_n \overline{v_n} = \overline{o}
        .\] 
        Пусть $\alpha_i \neq 0 \colon $ 

        \[
        \frac{\alpha_1}{\alpha_i} \overline{v_1} + \cdots + \overline{v_i} + \cdots + \frac{\alpha_n}{\alpha_i}  \overline{v_n} = \overline{o}
        .\] 
    \[
    \overline{v_i} = -\frac{\alpha_1}{\alpha_i} \overline{v_1} - \cdots - \frac{\alpha_n}{\alpha_i} \overline{v_n}
    .\] 
\item [b) ] \textbf{Дост.:} Пусть $\overline{v_i} = \lambda_1 \overline{v_1} + \cdots +  \lambda_n \overline{v_n}$ 
\[
\Rightarrow \lambda_1 \overline{v_1} + \cdots + \lambda_n \overline{v_n} - \overline{v_i} = \overline{o}
.\] 
\end{itemize}
\end{proof}
\begin{note}
\textbf{НЕВЕРНО }было бы сформ. утв. вот так: каждый из вектор выразим в виде ЛК остальных.
\begin{example}
\[
\overline{a}, \overline{b} \text{ - неколлин.}
.\] 
\[
    \Rightarrow \text{Для } \begin{pmatrix} \overline{a} & \overline{a} & \overline{b} \end{pmatrix} \text{ - это неверно, т. к. $b$ не выразим через $a$}
.\] 
\[
\text{Но } 1 * \overline{a} + (-1) * \overline{a} + 0 * \overline{b} = \overline{o} \text{ - нетривиальная ЛК}.
\] 
\end{example}

\end{note}
\begin{statement}
    \begin{itemize}
        \item [a) ] Если система $\overline{v_1}, \overline{v_2}, \cdots , \overline{v_n}$ - ЛЗ $\Rightarrow$ всякая её \textbf{надсистема} тоже ЛЗ
        \item[b) ] Если система $\overline{v_1}, \overline{v_2}, \cdots, \overline{v_n}$ - ЛНЗ $\Rightarrow$, то всякая её подсистема ЛНЗ. 
    \end{itemize}
\end{statement}
\begin{proof}
\begin{itemize}
    \item [a) ] $\exists \alpha_1, \cdots , \alpha_n, \text{- не все равны $\overline{o}$}$, тогда $\sum_{i = 1}^{n} \alpha_i \overline{v_i} = \overline{o}$
         $\Rightarrow \sum_{i = 1}^{n}  \alpha_i \overline{v_i} + \sum_{i = n + 1}^{n + k} 0 * \overline{v_j} = \overline{o}$
     \item [b) ] Пусть подсистема $\begin{pmatrix} \overline{v_1} & \overline{v_2} & \cdots & \overline{v_k} \end{pmatrix}$ - ЛЗ (от прот.), тогда по a), $\begin{pmatrix} \overline{v_1} & \cdots  && \overline{v_n} \end{pmatrix}$ - ЛНЗ \textbf{$ \Rightarrow $ Противоречие }
\end{itemize}
\end{proof}

\begin{statement}
    Пусть $\begin{pmatrix}\overline{v_1} & \overline{v_2} & \cdots & \overline{v_n} \end{pmatrix}$ - ЛНЗ сист. векторов в $V_i$. Тогда каждый вектор $\overline{w} \in  V_i$ выражется через $\begin{pmatrix}\overline{v_1} & \overline{v_2} & \cdots & \overline{v_n} \end{pmatrix}$ не более чем одним способом.
\end{statement}

\begin{proof}
    \[
        \overline{w} = \begin{pmatrix}\overline{v_1} & \overline{v_2} & \cdots & \overline{v_n} \end{pmatrix} \begin{pmatrix} \alpha_1 \\ \alpha_2 \\ \vdots \\ \alpha_n \end{pmatrix} = \overline{V} \alpha = \overline{V} \beta
    \]
    \[
    \Rightarrow \overline{o} = \overline{V} (\alpha - \beta)
    .\] 
\end{proof}

