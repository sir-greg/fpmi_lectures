\section{Лекция 18}
$F$ - поле
\begin{definition}
Система линейных ур-ий (СЛУ) - система ур-ий, сост. из ур-ий первой степени:
\begin{equation}
  \label{eq:general_lse}
\begin{cases}
a_{11} x_1 + a_{12} x_2 + \ldots + a_{1n} x_n = b_1 \\
a_{21} x_1 + a_{22} x_2 + \ldots + a_{2n} x_n = b_2 \\
\vdots \\
a_{m1} x_1 + a_{m 2} x_2 + \ldots + a_{mn} x_n = b_m
\end{cases}
\end{equation}
 Причём, $a_{ij}, b_i \in F$ \\
\end{definition}
\begin{symb}
  \[
  A \in M_{m \times n}(F)
  \]
  \[
  X = \begin{pmatrix}x_1 \\ x_2 \\ \vdots \\ x_n \end{pmatrix} \in F^{n}
  \]
  \[
  B \in F^{m}
  \]
  Тогда система записывается в формате:
  \[
  AX = B
  \]
  Расширенной матрицей $A$ наз-ся:
  \[
  \widetilde{A} = (A | B) \in M_{m \times (n + 1)}(F)
  \]
\end{symb}
\begin{definition}
СЛУ наз-ся \textbf{совместной}, если она имеет хотя одно решение. Если она не имеет решений, то она \textbf{несовместна}.
\end{definition}
\begin{definition}
Совместная СЛУ наз-ся \textbf{определённой}, если она имеет \textbf{единтсвенное решение}, и \textbf{неопределённой} ---  иначе.
\end{definition}
\begin{statement}
Всякое решение $X$ системы ($\ref{eq:general_lse}$) - это набор коэф., с кот. столбец $B$ свобоных членов, представляется в виде ЛК столбцов матрицы $A$.
\end{statement}
\begin{proof}
Стобцы матрицы $AX$ - это ЛК столбцов $A$ с коэф. из $X$
\end{proof}
\begin{consequence}
Если столбцы $A$ - ЛНЗ, то система ($\ref{eq:general_lse}$) имеет не более чем одно решение.
\end{consequence}
\begin{proof}
Если $A$ - несовместна, то следствие верно. Иначе: \\
Пусть $X_1 \neq X_2$ --- два решения. 
\[
AX_1 = b
\]
\[
AX_2 = b
\]
\[
\Rightarrow AX_1 - AX_2 = A(X_1 - X_2) = 0, \text{ причём } X_1 - X_2 \neq 0
\]
Получили, что есть нетрив. ЛК столбоцов матрицы $A$, дающая 0, что противоречит ЛНЗ столбцов $A$.
\end{proof}
\begin{definition}
Системе:
\[
AX = B
\]
Соотв. \textbf{однородная} система:
\[
AX = 0
\]
\end{definition}
\begin{statement}
Мн-во $V_0$ решений однородной СЛУ явл-ся подпр-ом в $F^{n}$ ($V_0 \leq F^{n}$)
\end{statement}
\begin{proof}
\[
X_1, X_2 \in V_0
\]
\[
AX_1 + AX_2 = A(X_1 + X_2) = 0 \Rightarrow (X_1 + X_2) \in V_0
\]
\[
AX_1 = 0 \Rightarrow \lambda A X_1 = 0, \lambda \in F 
\]
\[
X = 0 \in V_0
\]
\[
\Rightarrow V_0 \leq F^{n}
\]
\end{proof}
\begin{statement}
Пусть даны: неоднородн система $AX = B$ и $V_b$ --- её мн-во решений. Пусть также $X_0$ --- частное решение этой СЛУ. \\
Пусть $AX = 0$ соотв. однородн. СЛУ и $V_0$ - её решения. Тогда:
\[
V_b = X_0 + V_0
\]
\end{statement}
\begin{proof}
\begin{itemize}
  \item [$\supseteq$)] \[
  X_0 + V_0 = \set{X_0 + u | u \in V_0}
  \]
  \[
  A(X_0 + u) = AX_0 + Au = AX_0 = B \Rightarrow X_0 + u \in V_b
  \]
\item [$\subseteq$)] \[
\forall X \in V_b
\]
\[
AX = B = AX_0 \Rightarrow A(X - X_0) = B \Rightarrow X - X_0 \in V_0 \Rightarrow X \in V_0 + X_0
\]
\end{itemize}
\end{proof}
\subsection{Элементарные преобразования строк матрицы}
\begin{definition}
Элементарные преобразования (ЭП) строк матрицы $M_{m\times n}(F)$ --- это преобразования 3-ех типов:
\begin{itemize}
  \item [I тип:] $(i \neq j) \colon $ К $i$-ой строке $M$ прибавляем $j$-ую строку, умноженную на $\lambda \in F$:
    \[
    \overline{a_i} \mapsto \overline{a_i} + \lambda \overline{a_j}
    \]
  \item [II тип: ] $(i \neq j)\colon $ перемена местами $i$-ой и $j$-ой строки:
    \[
    \overline{a_i} \leftrightarrow \overline{a_j}
    \]
  \item [III тип: ] $i$-ая строка умножается на $\lambda \neq 0$.
\end{itemize}
\end{definition}
\begin{statement}
ЭП строк $M$ $\iff$ умножению $M$ слева на одну из элементарных матриц.
  \[
  E_{ij} \text{ - матрица с 1 в $(i; j)$ и 0 в других местах}
  \]
\begin{itemize}
  \item [I тип:] \[
      D_{ij} = E + \lambda E_{ij}
  \]
\item [II тип:]
      \[
      P_{ij} = E - E_{ii} - E_{jj} + E_{ij} + E_{ji}
      \]
  \item [III тип:]
    \[
      Z_i = E + E_{ii} \cdot \lambda
    \]
\end{itemize}
\end{statement}
\begin{statement}
Все матрицы ЭП обратимы.
\end{statement}
\begin{proof}
\[
D_{ij}^{-1}(\lambda) = D_{ij}(-\lambda)
\]
\[
P^{-1}_{ij} = P_{ij}
\]
\[
Z_i^{-1}(\lambda) = Z_i(\lambda^{-1})
\]
\end{proof}
\begin{task}
Показать, что если совершать умножение матрицы $M$ на матрицы ЭП нужно размера \textbf{справа}, то получатся ЭП столбцов.
\end{task}
\begin{definition}
  Для строки $\begin{pmatrix} a_1 & a_2 & \ldots & a_n \end{pmatrix}$, первый ненулевой её эл-т наз-ся \textbf{лидером}. (или ведущим элементом)
\end{definition}
\begin{example}
\[
  \begin{pmatrix}0 & 0 & 0 & \underline{7} & 4 & 0 & 0 \end{pmatrix}
\]
\end{example}
\begin{definition}
  Матрица $A_{m \times n}$ наз-ся \textbf{ступенчатой}, если выполняются два условия:
  \begin{itemize}
    \item [a) ] Если $a_{ij}$ и  $a_{i + 1, k}$ --- лидеры 2-х соседний строк, то $j < k$
    \item [b) ] Ниже нулевой строки $A$ могут расп-ся только нулвые строки $A$.
  \end{itemize}
\end{definition}
\begin{theorem}
  Всякую матрицу можно привести к ступенчатому виду с помощью конечного числа ЭП строк.
\end{theorem}
\begin{proof}[Док-во: \textbf{Прямой ход метода Гаусса}]
$A_{m \times n}$. Доказывать будем индукцией по $m$ (числу строк). 
\begin{itemize}
  \item [База: ] $m = 1$ - очев., т. к. одна строка --- это уже ступеначатая матрица.
  \item [Предп. инд.: ] Пусть дана матрица размер $(m - 1) \times n$ - утв. справедливо. Д-ем для матр. $m \times n$. \\
    
    Найдём в матрице $A$ лидера строки с наименьшим номером столбца. При необходимости, передвинем его на 1-ую строку $A$. Пусть теперь $a_{1k}$ - лидер первой строки. Используя ЭП $I$ типа, обнулим $k$-ые члены строк ниже. Мысленно уберём $1$-ую строку и применим предп. инд-ции к оставшейся матрице. Получили матрицу ступ. вида.
\end{itemize}
\end{proof}
\begin{definition}
Ступенчатая матрица $A$ наз-ся упрощённой, если вып-ся два усл-ия:
\begin{itemize}
  \item [a)] Лидеры всех строк равны $1$. \\
  \item [b) ] Столбцы, содерж. лидеров строк, содержат только нулевые эл-ты, за искл. лидера, кот. равен $1$
\end{itemize}
\end{definition}
\begin{theorem}
Всякую ненулевую матрицу, можно привести к упрощ. виду, с помощью конечного числа ЭП строк.
\end{theorem}
\begin{proof}[Док-во: \textbf{Обратный ход метода Гаусса}]
Приведём $A$ к ступенч. виду. Пусть $a_{1 k_1}, a_{2 k_2}, \ldots, a_{r k_r}$ --- лидеры строк ступ. матрицы $A'$. \\
Для каждого $i=\overline{1,r}$ умножим $i$-ую строку на $\frac{1}{a_{i k_i}}$. Тогда лидеры станут равны 1. \\
Затем, будем идти от $r$-ой строки к $1$-ой. Для $i$-ой строки, обнулим эл-ты $a_{j k_i}$ над ней ЭП $I$-ого типа. Получили нужный вид.
\end{proof}
\begin{theorem}
Если от СЛУ $(A | B)$ перейти к СЛУ $(A' | B')$ с помощью конечного числа ЭП строк, то эти системы эквив-ны.
\end{theorem}
\begin{proof}
Дост-но док-ть для одно ЭП:
\[
\exists \text{ ЭМ } Q \colon (A' | B') = (QA | QB)
\]
$V$ - мн-во решений СЛУ $(A | B)$. $V'$ - мн-во решений СЛУ $(A' | B')$. \\
\[
X_0 \in V \Rightarrow AX_0 = B \Rightarrow QAX_0 = QB \Rightarrow A'X_0 = B' \Rightarrow X_0 \in V'
\]
\[
X_0' \in V' \Rightarrow A'X_0' = B' \Rightarrow Q^{-1}A'X_0' = Q^{-1}B' \Rightarrow AX_0' = B \Rightarrow X_0' \in V
\]
\end{proof}
\subsection{Метод Гаусса}
\[
AX = B
\]
\[
  \widetilde{A} = (A | B) \text{ - расширенная матрицы}
\]
\begin{itemize}
  \item [I шаг:] Приведём $\widetilde{A}$ к ступ. виду $\widetilde{A}_\text{ступ.}$
  \begin{itemize}
    \item [I случай: ] В $\widetilde{A}_\text{ступ.}$ есть лидер в столбце своб. членов $\Rightarrow $ СЛУ несовм.
    \item [II случай: ] В $\widetilde{A}_\text{ступ.}$ такого лидера нет. Покажем, что СЛУ совместна. \\
      Пусть лидеры в $\widetilde{A}_\text{ступ.}$: $a_{1 k_1}, a_{2 k_2}, \ldots a_{r k_r}$
      \begin{definition}
      Назовём $x_{k_1}, x_{k_2}, \ldots, x_{k_r}$ --- \textbf{главными} (базисными), а остальные --- \textbf{свободными} (параметрические).
      \[
      1 \leq k_1 < \ldots < k_r \leq n
      \]
      \end{definition}
      \begin{itemize}
        \item [II, a)] Все неизв. --- главные (свободных нет). Тогда $r = n$:
          \[
            \begin{cases}
            a_{11} x_1 + a_{12} x_2 + \ldots + a_{1n} x_n = b_1 \\
            a_{21} x_1 + a_{22} x_2 + \ldots + a_{2n} x_n = b_2 \\
            \vdots \\
            a_{m 1} x_1 + a_{m 2} x_2 + \ldots + a_{m n} x_n = b_m
            \end{cases}
          \]
          Тогда $x_i = b_i$
      \end{itemize}
  \end{itemize}
\end{itemize}
