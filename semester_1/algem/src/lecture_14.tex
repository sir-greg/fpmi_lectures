\section{Лекция 14}
\subsection{Алгебраические структуры}
\begin{definition}
\textbf{Группой} наз-ся мн-во $G$ с опред. на нём бинарной алг. операцией. (Обозначим как $*\colon G \times G \rightarrow G$ - отображение) \\
Кроме того, * удовл. след. св-вам:
\begin{itemize}
  \item [I) ] Ассоциативность: $(a*b)*c = a*(b*c)$ 
  \item [II) ] $\exists$ нейтрального эл-та $e$ отн-но $*$:
    \[
    a * e = e * a = a
    \]
  \item [III) ] $\exists$ обратный эл-т $a^{-1}$:
    \[
    a * a^{-1} = a^{-1} * a = e
    \]
\end{itemize}
\end{definition}
\begin{example}
  \begin{itemize}
    \item [1) ]
$(\Z, +), (\Q, +), (\R, +)$ - $0$ нейтр. эл-т, $\forall a \rightarrow -a $ - противоположный (обратный) эл-т.
\item [2) ] $(\R \backslash \set{0}, *), (\Q \backslash \set{0}, *)$ 
\item [3) ] $(\R, *)$ - не группа, нарушается III для 0
\item [4) ] Пусть $X$ - произв. мн-во, $S(X)$ - мн-во всех вз. однозн. отобр. $X \rightarrow X$:
  \[
    \phi, \psi \text{ - вз. одн. отобр.}
  \]
  \[
    (\phi \cdot \psi)(x) = \phi(\psi(x))
  \]
  Тогда:
  \[
    (S(X), \circ) \text{ - группа}
  \]
  \[
  e(x) = x
  \]
\item [5) ] Пусть $X = \set{1, 2, \ldots n}$
  \[
  \phi \colon \set{1, 2, \ldots n} \rightarrow \set{1, 2, \ldots n} \text{ - подстановка}
  \]
  \[
  S(\set{1, 2, \ldots n}) = S_n \text{ - симметрич. группа степени $n$.}
  \]
\end{itemize}
\end{example}
\begin{statement}
Во всякой группе $G$ нейтральный эл-т единственный.
\end{statement}
\begin{proof}
  \[
  e = e * e' = e'
\]
\end{proof}
\begin{definition}
Пусть $G$ группа. Эл-т $b$ наз-ся \textbf{левым обратным} к $a$, если $b * a = e$ \\

Эл-т $c$ наз-ся правым обратным к $a$, если $c * a = e$
\end{definition}
\begin{statement}
$\forall a \in G$ левый обратный к нему совпад. с правым обратным к нему и совпад. с $a^{-1}$
\end{statement}
\begin{proof}
  \[
  b * a = e, a * c = e
  \]
\[
  c = e * c = (b * a) * c = b * (a * c) = b * e = b
\]
\[
  \Rightarrow b * a = a * b = e \Rightarrow b = a^{-1}
\]
В част-ти, для каждого эл-та $a$ обратный эл-т единственный.
\end{proof}
\begin{definition}
Мн-во $R$ с опред. на нём бинарной алг. операциями $"+"$ и $"*"$ наз-ся \textbf{кольцом}, если эти операции удовл. св-вам:
\item [a) ] $(R, +)$ - абелева группа (т. е. группа с комутативностью).
\item [b) ] Ассоц. $*$
\item [c) ] Левая и правая дистрибутивность $*$ отн-но $+$:
  \[
    (a + b) * c = a * c + b * c
  \]
  \[
    a * (b + c) = a * b + a * c
  \]
\end{definition}
\begin{example}
\begin{itemize}
  \item [1) ] $(\Z, +, *), (\Q, +, *), (\R, +, *)$ - 0 - нейтр. эл-т $+$
  \item [2) ] $(M_n(\R), +, *)$
\end{itemize}
\end{example}
\begin{definition}
Если в $R$ $\exists 1 \in R$, т. ч.:
\[
1 * a = a * 1 = a, \forall \in R
\]
то 1 наз-ся единицей кольца.
\end{definition}
\subsection{Сравнения и вычеты}
\begin{definition}
Назовём $a, b \in Z$ сравнимыми по модулю $n$ $(n \in \N, n > 1)$, если $a$ и $b$ имеют равные остатки при делении на $n$.
\end{definition}
\begin{symb}
  \[
  a \equiv b \pmod n \iff a - b = qn, q \in \Z
  \]
  \[
  2 \equiv 17 \pmod 5
  \]
  \[
  3 \equiv 0 \pmod 3
  \]
\end{symb}
\begin{note}
Сравнения по одному и другому $\mod$ можно складывать и умножать:
\[
  \begin{cases}
a_1 \equiv b_1 \pmod n \\
a_2 \equiv b_2 \pmod n  
  \end{cases} \Rightarrow \begin{cases}
  a_1 \pm a_2 \equiv b_1 \pm b_2 \pmod n \\
  a_1 \cdot a_2 \equiv b_1 \cdot b_2 \pmod n
  \end{cases}
\]
\end{note}
\begin{proof}
\[
  (a_1 \pm a_2) - (b_1 \pm b_2) = (a_1 - b_1) \pm (a_2 - b_2) = q_1n \pm q_2 n = n(q_1 \pm q_2) \vdots n
\]
\[
  a_1 a_2 = (b_1 + q_1n)(b_2 + q_2n) = (b_1b_2 + (q_2b_1 + q_1b_2 + q_1q_2n)n) \vdots n
\]
\[
 \Rightarrow a_1a_2 - b_1b_2 \vdots n
\]
\end{proof}
\begin{symb}
\[
a \in \Z 
\]
\[
\set{a + n\cdot q} \Rightarrow \overline{a} \text{ - класс вычетов $a$ по модулю $n$}
\]
Классы вычетов по модулю $n$ $\rightarrow \Z_n$:
\[
\overline{0}, \overline{1}, \overline{2}, \ldots, \overline{n - 1}
\]
\end{symb}
\begin{note}
\[
\overline{a} + \overline{b} = \overline{a + b}
\]
\[
\overline{a} \cdot \overline{b} = \overline{a \cdot \overline{b}}
\]
\end{note}
Проверка коректности:
\[
  \begin{cases}
a \equiv a_1 \pmod n \\ 
b \equiv b_1 \pmod n  
  \end{cases} \Rightarrow \overline{a} + \overline{b} \overset{?}{=} \overline{a_1} + \overline{b_1}
\]
\[
a + b \equiv a_1 + b_1
\]
\[
\overline{a} + \overline{b} \equiv \overline{a + b} \equiv \overline{a_1 + b_1} \equiv \overline{a_1} + \overline{b_1}
\]
\begin{statement}
Мн-во $Z_n$ классов вычетов по мод. $n$ явл-ся кольцом с операциями $"+", "*"$
\end{statement}
\begin{proof}
Опер-ция опр-на и кореектна:
\[
  (\Z_n, +) \text{ - абелева группа}
\]
  \[
  \overline{0} \text{ - нейтральный эл-т}
  \]
\end{proof}
\begin{definition}
Пусть $R$ - кольцо с 1. \\
Эл-т $a \in R$ - обратимый $\iff \exists b \in R \colon a * b = b * a = 1$
\end{definition}
\begin{definition}
$R^{*}$ - мн-во всех обратимых эл-ов кольца $R$ с 1.
\end{definition}
\begin{statement}
$R^{*}$ - группа с операцией умножения.
\end{statement}
\begin{proof}
Покажем, что если $a$ обратим, то обратный к нему эл-т $b$ тоже обратим:
\[
a * b = b * a = 1 \Rightarrow \text{ по опр-ю это верно }
\]
\[
\Rightarrow a \in R^{*} \Rightarrow b \in R^{*}
\]
Покажем теперь, что если $a, b \in R^{*} \Rightarrow a * b \in R^{*}$:
\[
a, b \in R^{*} \Rightarrow a^{-1}, b^{-1} \in R^{*}
\]
\[
  (ab)^{-1} = b^{-1}a^{-1}
\]
\[
  abb^{-1}a^{-1} = a * 1 * a^{-1} = 1
\]
\[
\Rightarrow a \cdot b \in R^{*}
\]
\end{proof}
\begin{task}
$Z_n^{*}$ - мн-во всех классов вычетов, взаимно простых с $n$.
\end{task}
\begin{statement}
  В любом кольце $R$:
  \[
  0 * a = a * 0 = 0, \forall a \in R
  \]
\end{statement}
\begin{proof}
\[
 0 * a + 0 * a = (0 + 0) * a = 0 * a
\]
\[
0 * a = 0
\]
\end{proof}
\begin{consequence}
Если $R$ - ненулевое кольцо с 1. То $0 \neq 1$:
\end{consequence}
\begin{proof}
  От прот. пусть $0 = 1$: \\
$\forall a \in R\colon a = a * 1 = a * 0 = 0 \Rightarrow R$ - нулевое. Противоречие!!!
\end{proof}
\begin{consequence}
Если $R$ ненулевое кольцо с 1, то $0 \not\in R^{*}$
\end{consequence}
\begin{definition}
Мн-во $F$ с опред. на нём бинарными алг. операциями $+, *$ наз-ся \textbf{полем}, если:
\begin{itemize}
  \item [1) ] $(F, +)$ - абелева группа с нейтр. эл-ом 0.
  \item [2) ] $(F \backslash \set{0}, *)$ - абелева группа с нейтр. эл-ом 1.
  \item [3) ]  $(a + b)c = ac + bc$ - дистрибутивность.
\end{itemize}
\end{definition}
\begin{note}
В любом поле содерж. 0 и 1. $\Rightarrow \left|F\right| \geq 2$
\end{note}
\begin{note}
\[
F^{*} = F \backslash \set{0}\text{ - мультипликативная группа поля}
\]
\end{note}
\begin{definition}
Поле - это коммутативное колько с 1, у кот. каждый ненулевой эл-т обратим.
\end{definition}
\begin{example}
\begin{itemize}
  \item [1) ] $(\Q, +, *)$ - поле рац. чисел.
  \item [2) ] $(\R, +, *)$ - поле действ. чисел.
  \item [3) ] $(\C, +, *)$ - поле комплексных чисел.
  \item [4) ] $(Boolean)$
\end{itemize}
\end{example}
\begin{statement}
В поле нет делителей нуля.
\end{statement}
\begin{proof}
 Пусть $a \cdot b = 0, a \neq 0, b \neq 0$:
 \[
 a \cdot b = 0 \Rightarrow a = 0 \cdot b ^{-1} = 0 !!!
 \]
\end{proof}
\begin{theorem}
Кольцо классов вычетов $\Z_n$ явл-ся полем $\iff$ $n$ - простое.
\end{theorem}
\begin{proof}
\begin{itemize}
  \item [a) ] Необходимость. Пусть $n$ - сост. $\Rightarrow \exists p, q > 1 \colon n = pq$
    \[
    \overline{p} \cdot \overline{q} = \overline{p\cdot q} = \overline{n} = \overline{0} \Rightarrow \overline{p}, \overline{q} \text{ - делители 0 - противоречие с тем, что $\Z_n$ - поле!!!}
    \]
  \item [b) ] Дост. Пусть $n$ - простое, покажем, что $(\Z_n \backslash \set{0}, \cdot)$ - абелева группа. \\
    Нетривиальная часть: покажем, что $\forall \overline{a} \neq \overline{0}, \exists$ обратимый. \\
    Для этого покажем, что:
    \[
    \overline{0} \cdot \overline{a}, \overline{1} \cdot \overline{a}, \ldots, \overline{(n - 1)}\overline{a} \text{ - попарно различны.}
    \]
   Пусть $\overline{k} \overline{a} = \overline{l} \overline{a}$, б. о. о. $0 \leq k < l \leq n - 1$ .
   \[
     \overline{(l - k)a} = \overline{o} \iff n | (l - k)a
   \]
   Однако $n \not{|} a, \Rightarrow n | (l - k) \Rightarrow l = k!!! \Rightarrow \exists b \colon \overline{b} \overline{a} = \overline{a} \overline{b} = \overline{1} \text{ и } \overline{b} \neq \overline{0}$
\end{itemize}
\end{proof}
