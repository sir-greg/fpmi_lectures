\section{Лекция 26}
\subsection{Определители произовльного порядка}
\[
\sigma \colon M \rightarrow M \text{ --- подстановка}
\]
\begin{statement}
  Следующие три определения эквив-ны: подстановка наз-ся \textbf{чётной}, если:
\begin{itemize}
  \item [1) ] Чётности верхней и нижней строк совпадают
  \item [2) ] $n_1 + n_2 \text{ --- чётно}$, где $n_i$ - число инверсий в $i$-ой строке ($i = 1, 2$)
  \item [3) ] Она раскладывается в произведение чётного числа транспозиций
\end{itemize}
\end{statement}
\begin{proof}
  ~\newline
\begin{itemize}
  \item [$1 \iff 2$) ] $n_1 + n_2 \in 2\Z \iff \begin{pmatrix}n_1 \\ n_2 \end{pmatrix} = \begin{pmatrix}\text{ч} \\ \text{ч}\end{pmatrix}\lor \begin{pmatrix}n_1 \\ n_2 \end{pmatrix} = \begin{pmatrix}\text{нч} \\ \text{нч} \end{pmatrix} \iff $чётность строк совпадет
  \item [$1 \iff 3$)] Т. к. каждая транспозиция меняет чётность кол-ва инверсий, то число множителей в произведении чётно.
\end{itemize}
\end{proof}
\begin{symb}
\textbf{Знак подстановки}:
\[
\varepsilon \colon S_h \rightarrow \set{\pm 1}
\]
\[
\varepsilon(\sigma) = \begin{cases}
1, \text{ если } \sigma \text{ --- чётно} \\
-1, \text{если } \sigma \text{ --- нечётно}
\end{cases} = (-1)^{\inv(\sigma)} = (-1)^{\tau(\sigma)}
\]
\[
\inv(\sigma) \text{ --- суммарное число инверсий ($n_1 + n_2$)}
\]
\[
\tau(\sigma) \text{ --- размер минимального по кол-ву транспозиций разложения $\sigma$}
\]
\end{symb}
\begin{statement}
Знак перестановки явл-ся гомоморфизмом мультипликативных групп:
\[
\varepsilon(\sigma \cdot p) = \varepsilon(\sigma) \cdot \varepsilon(p)
\]
\end{statement}
\begin{proof}
\[
\sigma = \tau_1 \ldots \tau_k
\]
\[
p = \tau_1' \ldots \tau_s'
\]
\[
\Rightarrow \sigma \cdot p = \tau_1 \ldots \tau_k \cdot \tau_1' \ldots \tau_s'
\]
\[
\varepsilon(\sigma \cdot p) = (-1)^{k + s} = (-1)^{k} (-1)^{s} = \varepsilon(\sigma) \cdot \varepsilon(p)
\]
\end{proof}
Вспомним определитель 3-его порядка:
\[
  \begin{vmatrix}a_{11} & a_{12} & a_{13} \\ a_{21} & a_{22} & a_{23} \\ a_{31} & a_{32} & a_{33} \end{vmatrix} = a_{11}a_{22}a_{33} + a_{12}a_{23}a_{31} + a_{21}a_{32}a_{13} - 
\]
\[
  - a_{13}a_{22}a_{31} - a_{11}a_{23}a_{32} - a_{21}a_{12}a_{33}
\]
Сделаем сопоставление:
\[
  a_{i_1 j_1} a_{i_2 j_2} a_{i_3 j_3} \mapsto \begin{pmatrix}i_1 & i_2 & i_3 \\ j_1 & j_2 & j_3 \end{pmatrix}
\]
\begin{center}
\begin{tabular}{ |c|c|c| } 
 \hline
  Слагаемое & Подстановка $\sigma$ & $\varepsilon(\sigma)$ \\
 \hline
  $a_{11} a_{22} a_{33}$ & $id$ & $+1$ \\
  $a_{12}a_{23}a_{31}$ & $\begin{pmatrix}1 & 2 & 3 \\ 2 & 3 & 1\end{pmatrix}$ & $+1$ \\
  $a_{13}a_{21}a_{32}$ & $\begin{pmatrix}1 & 2 & 3 \\ 3 & 1 & 2 \end{pmatrix}$ & $+1$ \\
  \vdots & \vdots & \vdots \\
 \hline
\end{tabular}
\end{center}
На основании этой таблицы строиться формула общего вида (и соотв. определение):
\[
  \det A = \left|A\right| = \sum_{\sigma \in S_n}^{} \varepsilon(\sigma) a_{1 \sigma(1)} a_{2 \sigma(2)} \cdot \ldots \cdot a_{n \sigma(n)}
\]
\begin{statement}
При транспонировании матрицы $A$ её определитель не меняется. Пусть $a_{1 \sigma(1)}\ldots a_{n \sigma(n)}$ входит в состав $\det A$ (и соотв. в состав $\det (A^{T})$)
\[
  \det A \rightarrow \varepsilon\begin{pmatrix}1 & 2 & \ldots & n \\ \sigma(1) & \sigma(2) & \ldots & \sigma(n) \end{pmatrix}
\]
\[
  \det A^{T} \rightarrow \varepsilon\begin{pmatrix}\sigma(1) & \sigma(2) & \ldots & \sigma(n) \\ 1 & 2 & \ldots & n \end{pmatrix}
\]
\end{statement}
\subsection{Полилинейные кососимметрические ф-ции} 
\begin{definition}
$f \colon V^{n} \rightarrow F$ наз-ся \textbf{полилинейной}, если она линейна по каждому из своих арг-ов:
\end{definition}
\[
V^{n} = \set{(a_1, \ldots, a_n), a_i \in V}
\]
\[
f(a_1, \ldots, a_n) \in F
\]
Линейность подразумевает:
\begin{itemize}
  \item [1) ] Аддитивность:
    \[
    f(\ldots, a_i + a_i', \ldots) = f(\ldots, a_i, \ldots) + f(\ldots, a_i', \ldots)
    \]
  \item [2) ] Однородность:
    \[
    \forall \lambda \in F \colon f(\ldots \lambda a_i \ldots) = \lambda f(\ldots a_i \ldots)
    \]
\end{itemize}
Пусть $\charac F \neq 2$
\begin{definition}
Полилин. ф-ция $f \colon V^{n} \rightarrow F$ наз-ся кососимметрическим, если:
\begin{itemize}
  \item [a)] $f(\ldots a_i \ldots a_j \ldots) = -f(\ldots a_j \ldots a_i \ldots), i \neq j$
  \item [b)] $f(\ldots a \ldots a \ldots) = 0$
\end{itemize}
\end{definition}
\begin{proof}
  ~\newline
\begin{itemize}
  \item [a) $\Rightarrow$ b)] \[
  f(\ldots a \ldots a \ldots) = -f(\ldots a \ldots a \ldots) 
  \]
  Если $\charac F = 2$, то опр-е не пол-ся. Иначе всё ок.
\item [b) $\Leftarrow$ a)]
  \[
  0 = f(\ldots a_i + a_j \ldots a_i + a_j) = f(\ldots a_i \ldots a_j \ldots) + f(\ldots a_i \ldots a_i \ldots) + 
  \]
  \[
   + f(\ldots a_j \ldots a_j \ldots) + f(\ldots a_j \ldots a_i \ldots) = 
  \]
  \[
    = f(\ldots a_i \ldots a_j \ldots) + f(\ldots a_j \ldots a_i \ldots)
  \]
\end{itemize}
\end{proof}
\begin{note}
В случае $\charac F = 2$, п. $b) $ выбирается в кач-ве опр-я.
\end{note}
\begin{statement}
Пусть $f \colon V^{n} \rightarrow F$ --- полилин. кососим., тогда $\forall \sigma \in S_n$
\[
f(a_{\sigma(1)} \ldots a_{\sigma(n)}) = \varepsilon(\sigma) f(a_1\ldots a_n)
\]
\end{statement}
\begin{proof}
  Индукция по $\tau(\sigma)$:
  \begin{itemize}
    \item [База: ] $\tau(\sigma) = 1 \Rightarrow \sigma$ --- очев.
    \item [Переход:] Пусть для $\sigma$, т. ч. $\tau(\sigma) < k$, утв-е вып-ся. Тогда для $\tau(\sigma) = k$, утв-е верно, (делаем ещё один $swap$, чётность мен-ся, и утв-е верно)
  \end{itemize}
\end{proof}
\begin{theorem}[О характеризации определителя его св-вам]
\label{th:polilin_and_kos_1}
\[
A \in M_n(F)
\]
Тогда:
\begin{itemize}
  \item [a) ] $\det A$ --- полилин., кососим. ф-ция от строк (или столбцов) матрицы $A$
  \item [b) ] Пусть $f \colon M_n(F) \rightarrow F$ --- полилин. кососим. ф-ция от строк (или столбцов) матрицы. Тогда:
    \[
    f(A) = f(E) \cdot \det A, \text{ где $E$ --- единич. матрица}
    \]
\end{itemize}
\begin{proof}
  \begin{itemize}
    \item [a) ] Зафикс. все элем-ты, матрицы $A$ : $a_{ij}, i > 1$:
      \[
      \det A = \sum_{\sigma \in S_n}^{} \varepsilon(\sigma) \cdot a_{1 \sigma(1)} \ldots a_{n \sigma(n)} = \sum_{j}^{} \alpha_j a_{1j}
      \]
      --- лин. форма коорд. $I$ строки. \\
      \begin{itemize}
        \item [1)]
    $\charac F \neq 2$. Проверим, что $\det A$ --- кососим ф-ция от строк $A$:
    \[
    \det (\overline{a_1} \ldots \overline{a_i} \ldots \overline{a_j} \ldots \overline{a_n}) \overset{?}{=} -\det (\overline{a_1} \ldots \overline{a_j} \ldots \overline{a_i} \ldots \overline{a_n})
    \]
    \begin{itemize}
      \item [I. ] $a_{1 \sigma(1)} \ldots a_{i \sigma(i)} \ldots a_{j \sigma(j)} \ldots a_{n \sigma(n)} \mapsto \det A$
      \item [II. ] $a_{1 \sigma(1)} \ldots a_{j \sigma(j)} \ldots a_{i \sigma(i)} \ldots a_{n \sigma(n)} \mapsto \det A'$
    \end{itemize}
     $\Rightarrow \varepsilon(I) = -\varepsilon(II) \Rightarrow \det A' = -\det A$
   \item [2) ] $\charac F = 2$
     \[
     \det (\overline{a_1} \ldots \overline{a_i} \ldots \overline{a_j} \ldots \overline{a_n}) \overset{?}{=} 0, (\overline{a_i} = \overline{a_j})
     \]
      \end{itemize}
    \item [b) ] \[
        e_1 = \begin{pmatrix}1 & 0 & \ldots & 0 \end{pmatrix}
    \]
    \[
      e_2 = \begin{pmatrix}0 & 1 & \ldots & 0 \end{pmatrix}
    \]
    \[
      \vdots
    \]
    \[
      e_n = \begin{pmatrix}0 & 0 & \ldots & 1 \end{pmatrix}
    \]
      \[
    f(a_1, \ldots, a_n) = f\left(\sum_{j_1}^{} a_{1 j_1}e_{j_1}, \sum_{j_2}^{} a_{2, j_2} e_{j_2} \ldots \sum_{j_n}^{} a_{n j_n} e_{j_n}\right) = 
    \]
    \[
     = \sum_{j_1}^{} \sum_{j_2}^{} \ldots \sum_{j_n}^{} \varepsilon(j) a_1 {j_1} \ldots a_{n j_n} f(e_1, \ldots, e_n) = \det A f(E)
    \]
  \end{itemize}
\end{proof}
\end{theorem}
\begin{statement}
\begin{itemize}
  \item [a) ] Если над матрицей $A$ совершить ЭП строк $I$ типа ($a_i \mapsto a_i + \lambda a_j$), то опр-тель не меняется.
  \item [b) ] При преобразованиях второго типа ($a_i \leftrightarrow a_j$) опр-тель изменит свой знак.
  \item [c) ] При преобразованиях третьего типа $(a_i \mapsto \lambda a_i)$ опр-тель умножается на $\lambda$
\end{itemize}
\end{statement}
\begin{proof}
\begin{itemize}
  \item [a) ] \[
      \det (\overline{a_1} \ldots \overline{a_i} + \lambda\overline{a_j} \ldots \overline{a_n}) = 
  \]
  \[
   = \det(\overline{a_1} \ldots \overline{a_i} \ldots \overline{a_n}) + \lambda\det (\overline{a_1} \ldots \overline{a_j} \ldots \overline{a_j} \ldots \overline{a_n}) = 
  \]
  \[
   = \det(\overline{a_1} \ldots \overline{a_i} \ldots \overline{a_n})
  \]
\item [b)] Из кососим.
\item [c) ] Следсвие однородности.
\end{itemize}
\end{proof}
\begin{definition}
Матрица $A \in M_n(F)$ наз-ся верхнетреугольной (нижнетреугольной), если $a_{ij} = 0, i > j (i < j)$
\end{definition}
\begin{statement}
Определитель верхнетреугольной (нижнетреугольной) матрицы равен произведению эл-ов на главной диагонали.
\end{statement}
\begin{proof}
  \[
  \varepsilon(\sigma) a_{1 \sigma(1)} a_{2 \sigma(2)} \ldots a_{n \sigma(n)} \rightarrow \det A
  \]
  Если $\sigma \neq e$, то $\exists i$, т. ч. $\sigma(i) < i \Rightarrow a_{i \sigma(i)} = 0$ (Легко док-ть от прот.) $\Rightarrow$ единственное ненулевое произведение --- произведение эл-ов главной диагонали.
\end{proof}
\begin{definition}
Минором $k$-ого порядка матрицы $A$ наз-ся $\det M_{j_1 j_2 \ldots j_k}^{i_1 i_2 \ldots i_k}$
\end{definition}
\begin{definition}
Ранг матрицы по минорам наз-ся порядок её наибольшего ненулевого минора. ($\det M \neq 0$)
\[
\rk_M A
\]
\end{definition}
\begin{theorem}[Фробениус, 1873-75 гг.]
\label{th:minor_rk_is_ok}
Все 3 понятия ранга матрицы эквив-ны, т. е.:
\[
\rk_r A = \rk_c A = \rk_M A
\]
\end{theorem}
