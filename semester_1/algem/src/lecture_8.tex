\section{Лекция 8}

\subsection{Пучок прямых на пл-ти}
$V_2$, с фикс. ДСК
\begin{definition}
\textbf{Пучком пересекающихся прямых} на пл-ти наз-ся мн-во всех прямых на пл-ти, проходящих через. фикс. точку.
\end{definition}
\begin{definition}
\textbf{Пучком параллельных прямых} на пл-ти наз-ся мн-во всех прямых пл-ти, параллельных некоторой фикс. прямой.
\end{definition}
\begin{theorem}
Пусть даны две \underline{различные} прямые $l_1$, $l_2$ на пл-ти:
\[
l_1 \colon A_1x + B_1y + C_1 = f_1(x, y) = 0
\]
\[
l_2 \colon A_2x + B_2 y + C_2 = f_2(x, y) = 0
\]
Тогда пучок прямых, задаваемый (порождаемый) прямыми $l_1$ и $l_2$ состоит из тех и только тех прямых, коор-ты точек которых удовл. ур-ю:
\begin{equation}
  \label{eq:puchok}
\alpha f_1(x, y) + \beta f_2(x, y) = 0
\end{equation}
\[
\alpha, \beta \in \R, \alpha^{2} + \beta^{2} \neq 0
\]
Выр-е $\alpha f_1(x, y) + \beta f_2(x, y) \not\equiv 0$
\end{theorem}
\begin{proof}
  \begin{itemize}
    \item [a) ] Пусть $l_1 \cap l_2 = \set{x_0}$ \\

      Покажем, что всякая прямая $l$, коор-ты точек кот. удовл. ур-ю (\ref{eq:puchok}), принадлежит пучку, порождаемому $l_1, l_2$ \\

      По усл.:
      \[
      f_1(x_0) = f_2(x_0) = 0
      \]
      \[
      \Rightarrow \alpha f_1(x, y) + \beta f_2(x, y) = 0 
      \] 
      $\Rightarrow$ $l$ проходит через $x_0$ \\

      Пусть $l$ такова, что она принадлежит пучку, порожд. $l_1, l_2$. Покажем, что $\exists \alpha, \beta \in \R, \alpha^{2} + \beta^{2} \neq 0$, такие, что $l$ задаётся ур-ем (\ref{eq:puchok})

      Пусть $X \in l, X \neq x_0$:
      \[
      \alpha = f_2(X), \beta = -f_1(X)
      \]
      \begin{equation}
      f_2(X)f_1(x, y) - f_1(X)f_2(x, y) = 0
      \label{eq:proof_puchok}
      \end{equation}
      Если подставим $X$:
      \[
      f_2(X) f_1(X) - f_1(X) f_2(X) = 0
      \]
      прямая (\ref{eq:proof_puchok}) проходит через $X_0$ и $X$ $\Rightarrow$ это прямая $l$.
    \item [b) ] Пусть $l_1 || l_2 \iff \overline{n_1} || \overline{n_2}$ \\

      Пусть прямая $l$ такова, что её коорд. удовл. усл. ($\ref{eq:puchok}$) $\Rightarrow$
      \[
      \overline{n_l} = \alpha\overline{n_1} + \beta\overline{n_2}, \overline{n_l} || n_1, n_2
      \]
      Обратно: пусть $l$ принадлежит пучку параллельных прямых, порожд. $l_1$ и $l_2$ \\
      
      \[
      \alpha = f_2(X), \beta = -f_1(X)
      \]
      \[
      \alpha f_1(x, y) + \beta f_2(x, y) = 0
      \]
      \[
      f_2(X) f_1(x, y) - f_1(X) f_2(x, y) = 0, \text{ (при $X$ равно 0)}
      \]
      \[
      \Rightarrow \overline{n} || \overline{n_1}, \overline{n_2}
      \]
  \end{itemize}
  \begin{definition}
  Ур-е (\ref{eq:puchok}) наз-ся ур-ем пучка прямых, порожд. $l_1$ и $l_2$
  \end{definition}
\end{proof}

\subsection{Приложения в планиметрии}

\subsubsection{Расстояние от точки до прямой}

\begin{symb}
Расстояние от точки до прямой ($p(X, l)$)
\end{symb}
\begin{statement}
Пусть прямая $l$ в ПДСК задана общим ур-ем $Ax + By + C = 0$. Пусть $X \underset{(O, G)}{\longleftrightarrow} \begin{pmatrix}x \\ y \end{pmatrix}$. Тогда:
\[
p(X, l) = \frac{\left|Ax + By + C\right|}{\sqrt{A^{2} + B^{2}}}
\]
\end{statement}
\begin{proof}
Пусть $\overline{a} \underset{}{\longleftrightarrow} \begin{pmatrix}-B \\ A \end{pmatrix}$ \\

(Picture 2.1)
\[
p(X, l) = h = \frac{S_{\overline{a} \times \overline{X_0X}}}{\left|\overline{a}\right|}
\]
\[
  S_{\overline{a} \times \overline{X_0X}} = S = \left|\begin{vmatrix}x - x_0 & -B \\ y - y_0 & A \end{vmatrix}\right| = \left|A(x - x_0) + B(y - y_0)\right| = \left|Ax + By - (Ax_0 + By_0)\right| = \left|Ax + By + C\right|
\]
\[
\Rightarrow p(X, l) = \frac{S}{\left|\overline{a}\right|} = \frac{\left|Ax + By + C\right|}{A^{2} + B^{2}}
\]
\end{proof}

\subsubsection{Вычисление угла между пересекающимися прямыми}

\begin{definition}
\textbf{Углом между двумя пересекающимися прямыми} наз-ся наименьший из двух смежных углов, порождённый пересечением прямых.
\end{definition}
Picture(2.2)
\textbf{Вычисление}:
\[
l_i \colon A_ix + B_i y + C_i = 0, i = 1, 2
\]
\[
\overline{n_1} \begin{pmatrix}A_1 \\ B_1 \end{pmatrix}, \overline{n_2} \begin{pmatrix}A_2 \\ B_2 \end{pmatrix}
\]
\[
\phi = \angle(\overline{n_1}, \overline{n_2})
\]
\begin{equation*}
\phi = \begin{cases}
\psi, \text{ если } \psi \leq \frac{\pi}{2} \\
\pi - \psi, \text{ если } \psi > \frac{\pi}{2}
\end{cases}
\end{equation*}
\[
\cos \phi = \left|\cos\phi\right| = \left|\cos\psi\right| = 
\]
\begin{equation*}
0 \leq \phi \leq \frac{\pi}{2} \colon \left|\cos (\pi - \phi)\right| = \left|-\cos \psi\right| = \left|\cos \psi\right| = \frac{\left|(\overline{n_1}, \overline{n_2})\right|}{\left|\overline{n_1}\right|\left|\overline{n_2}\right|} =
\end{equation*}
\[
  \frac{\left|A_1A_2 + B_1B_2\right|}{\sqrt{A_1^{2} + B_1^{2}}\sqrt{A_2^{2} + B_2^{2}}}
\]

\begin{statement}
Косинус угла между двумя пересек. прямыми:
\[
l_i \colon A_i x + B_i y + C_i = 0
\]
Может быть вычислен по формуле:
\[
\cos\phi = \frac{\left|A_1A_2 + B_1B_2\right|}{\sqrt{A_1^{2} + B_1^{2}}\sqrt{A_2^{2} + B_2^{2}}}
\]
\end{statement}
\subsection{Пл-ть в пр-ве}
$V_3$, с фикс. ДСК (O, $G$)
\begin{definition}
Направляющий векторы пл-ти - это пара неколлинеарных векторов, задающих эту пл-ть.
\end{definition}
Picture (2.3) \\

Пусть $\alpha$ - наша пл-ть, $X_0 \in \alpha$, $\overline{a}, \overline{b}$ - напр. векторы $\alpha$ \\
\[
\overline{X_0X}, \overline{a}, \overline{b} \text{ - коллинеарны} \iff X \in \alpha
\]
\[
\Rightarrow \overline{X_0X} = s \overline{a} + t \overline{b}; s, t \in \R
\]
\[
\overline{r} - \overline{r_0} = s\overline{a} + t \overline{b}
\]
\begin{equation}
\overline{r} = \overline{r_0} + s\overline{a} + t \overline{b}
\label{eq:vector-plane-eq}
\end{equation}
 - векторное ур-е прямой

\[
\overline{a} \underset{(O, G)}{\longleftrightarrow} \begin{pmatrix}a_1 \\ a_2 \\ a_3 \end{pmatrix}
\]
\[
\overline{b} \underset{(O, G)}{\longleftrightarrow} \begin{pmatrix}b_1 \\ b_2 \\ b_3\end{pmatrix}
\]
\begin{equation}
\begin{cases}
x = x_0 + sa_1 + tb_1 \\
y = y_0 + sa_2 + tb_2 \\
z = z_0 + sa_3 + tb_3
\end{cases}
\label{eq:coord-plane}
\end{equation}
- координатное ур-е прямой
\begin{equation}
  (\overline{r} - \overline{r_0}, \overline{a}, \overline{b}) = 0
  \label{eq:mixedprod-plane-eq}
\end{equation}
\begin{equation}
  \iff \begin{vmatrix}x - x_0 & a_1 & b_1 \\ y - y_0 & a_2 & b_2 \\ z - z_0 & a_3 & b_3\end{vmatrix} = 0
  \label{eq:det-plane-eq}
\end{equation}
Если раскроем определитель по соотв. формуле:
\begin{equation}
Ax + By + Cz + D = 0, A^{2} + B^{2} + C^{2} \neq 0 \text{ (т. к. иначе $\overline{a} || \overline{b}$)}
\label{eq:general_plane_eq}
\end{equation}
- общее ур-е пл-ти
\begin{statement}
Пусть $X_0 \begin{pmatrix}x_0 \\ y_0 \\ z_0 \end{pmatrix}$ принадлежит пл-ти, заданной общ. ур-ем (\ref{eq:general_plane_eq}). Тогда т. $X_1\begin{pmatrix}x_1 \\ y_1 \\ z_1 \end{pmatrix}$ принадлежит пл-ти $\pi$
\[
  \iff A(x_0 - x_1) + B(y_0 - y_1) + C(z_0 - z_1) = 0
\]
\end{statement}
\begin{proof}
Аналогично прямой
\end{proof}
\begin{consequence}
Вектор $\overline{c} \underset{(O, G)}{\longleftrightarrow} \begin{pmatrix} \alpha \\ \beta \\ \gamma \end{pmatrix}$ - направляющий вектор пл-ти $\pi \iff A\alpha + B\beta + C\gamma = 0$
\end{consequence}
\begin{proof}
Т. пл-ти $X_0$ - начало $\overline{c}$, а $X_1$ - конец $\overline{c}$
\[
\overline{c} || \pi \iff X_1 \in \pi \iff A(x_1 - x_0) + B(y_1 - y_0) + C(z_1 - z_0) = 0 \iff
\]
\[
A\alpha + B\beta + C\gamma = 0
\]
\end{proof}
\begin{consequence}
Пусть, для определённости, $\pi \colon Ax + By + Cz + D = 0$. $A \neq 0$ (Б. О. О.) Тогда векторы:
\[
\begin{pmatrix} -B \\ A \\ 0 \end{pmatrix} \text{ и } \begin{pmatrix}-C \\ 0 \\ A \end{pmatrix}
\]
Ненулевые, неколлинеарны и параллельны пл-ти $\pi$, т. е. они могут быть выбраны в кач-ве напр. векторов $\pi$ \\

В кач-ве нач. точки $X_0$ можно взять точки с коор-т $\begin{pmatrix}-\frac{D}{A} \\ 0 \\ 0\end{pmatrix}$
\end{consequence}
\begin{consequence}
Все указанные методы задания пл-ти эквивалентны 
\end{consequence}
Пусть теперь $(O, G)$ - прямоугольная (ПДСК)
\[
  A\alpha + B\beta + C\gamma = 0 \iff \begin{pmatrix} \alpha & \beta & \gamma \end{pmatrix}^{T} \begin{pmatrix} A \\ B \\ C \end{pmatrix} = (\overline{c}, \overline{n}) = 0
\]
Где $\overline{n} \underset{(O, G)}{\longleftrightarrow} \begin{pmatrix}A \\ B \\ C \end{pmatrix}$

\begin{definition}
  $\overline{n}$ наз-ся вектором нормали к пл-ти $\pi$
\end{definition}
\begin{statement}
В ПДСК вектор нормали $\overline{n}$ к $\pi$ ортогонале любому вектору, параллельному $\pi$
\end{statement}

Пусть теперь $(O, G)$ - произвольная ДСК. \\
\begin{definition}
Тогда вектор, сопоставленный пл-ти $Ax + By + Cz + D = 0$, с коор-тами $\begin{pmatrix} A \\ B \\ C \end{pmatrix}$ наз-ся \textbf{сопутствующим вектором пл-ти}.
\end{definition}
\begin{statement}
  Плоскости $A_ix + B_iy + C_iz + D_i = 0, i = 1,2$, с сопутствующими векторами, соотв. $\overline{n_1}, \overline{n_2}$ параллельны тогда и только тогда, когда:
  \[
  \overline{n_1} || \overline{n_2}
  \]
 Пл-ти $\pi_1, \pi_2$ совпадают тогда и только тогда, когда их уравнения пропорциональны
\end{statement}
\begin{proof}
\item [a) ] Усли ур-я пропорциональны $\Rightarrow \pi_1 = \pi_2$ 
\item [b) ] Пусть $\overline{n_1} || \overline{n_2}$ и при этом ур-я не пропорциональны, покажем, что $\pi_1 || \pi_2$ и $\pi_1 \not\equiv \pi_2$
  \begin{proof}
  \[
  \frac{A_1}{A_2} = \frac{B_1}{B_2} = \frac{C_1}{C_2} = \lambda, \text{ но } \frac{D_1}{D_2} \neq \lambda \iff D_1 - \lambda D_2 \neq 0
  \]
  \[
  \lambda(A_2x + B_2y + C_2z + D_2) = (A_1x + B_1y + C_1z + D_1) + (\lambda D_2 - D_1)
  \]
  $\Rightarrow \not\exists X_0 \colon X_0 \in \pi_1 \cap \pi_2 \Rightarrow \pi_1 || \pi_2$
  \end{proof}
\item [c) ] Пусть $\overline{n_1} \not{||} \overline{n_2}$ \\

  Покажем, что пл-ти пересекаются:
  \begin{equation*}
    \begin{cases}
  A_1x + B_1y + C_1z + D_1 = 0 \\
  A_2x + B_2y + C_2z + D_2 = 0 \\
  z = z_0
    \end{cases} \iff
    \begin{cases}
    A_1x + B_1y = -D_1 - C_1z_0 \\
    A_2x + B_2y = -D_2 - C_2z_0
    \end{cases}
  \end{equation*}
  \[
   \begin{vmatrix} A_1 & B_1 \\ A_2 & B_2 \end{vmatrix} \neq 0 \Rightarrow \exists! X_0 \begin{pmatrix}x_0 \\ y_0 \\ z_0 \end{pmatrix} \text{ удовл. системе}
  \]
  При этом пл-ти $\pi_1$ и $\pi_2$ не совпадают:
  \begin{proof}
    Пусть $\pi_1 \equiv \pi_2$:
    \begin{equation*}
    \begin{cases}
    z = 0 \\
    A_1x + B_1y + D_1 = 0 \\
    A_2x + B_2y + D_2 = 0
    \end{cases} \Rightarrow \begin{pmatrix} A_1 \\ B_1 \end{pmatrix} || \begin{pmatrix} A_2 \\ B_2 \end{pmatrix} \Rightarrow \text{ПРОТИВОРЕЧИЕ}
    \end{equation*}
  \end{proof}
\end{proof}
