\section{Лекция 16}
\subsection{Линейные пр-ва}
Пусть $F$ - поле. \\
\begin{definition}
\textbf{ЛП (линейным пр-вом)} над полем $F$ наз-ся мн-во $V$, на кот. опр-ны оп-ции:
\begin{itemize}
  \item [a) ] Сложение эл-ов из
    \[
      V \colon \forall a, b \in V \hookrightarrow a + b \in V
    \]
  \item [b) ] Умножение эл-ов $V$ на число из $F$:
    \[
      \forall \lambda \in F, a \in V, \lambda a \in V
    \]
  \item [c) ] $(V_1, +)$ - абелева группа.
  \item [d) ] Унитарность:
    \[
      1 * a = a, \forall a \in V
    \]
  \item [e) ] Ассоциативность отн-но скалярного множителя:
    \[
      (\lambda \cdot \mu) a = \lambda \cdot (\mu a), \forall \lambda, \mu \in F, a \in V
    \]
  \item [f)] Дистрибутивность:
    \[
      (\lambda + \mu) a = \lambda a + \mu a
    \]
  \item [g)] \[
    \lambda(a + b) = \lambda a + \lambda b
  \]
\end{itemize}
Эл-ты ЛП принято называть \textbf{векторами}. $\overline{0}$ - нулевой вектор.
\end{definition}
\begin{example}
\begin{itemize}
  \item [0) ] Нулевое пр-во $\set{\overline{0}}$:
    \[
    \overline{0} + \overline{0} = \overline{0}
    \]
    \[
    \lambda \overline{0} = \overline{0}
    \]
  \item [1) ] $M_{m \times n}(F)$ - лин. пр-во отн-но естественных операций.
    \[
    M_{m \times 1}(F) = \Set{\begin{pmatrix}a_1 \\ a_2 \\ \vdots \\ a_m \end{pmatrix}} = F^{m} \text{ - арифметическое пр-во над $F$ раз-ти $m$}
    \]
  \item [2) ] $V_i, i = 1, 2, 3$. $F = \R$
  \item [3) ] $F[x]$ - пр-во мн-нов с коэфф-ми из поля $F$
    \[
    F_n[x] = \set{f(x) \in F[x] | deg(f) \leq n}
    \]
\end{itemize}
\end{example}
\subsubsection{Подпр-во ЛП}
Пусть $V$ - ЛП на поле $F$.
\begin{definition}
Непустое подмн-во $W \subset V$, наз-ся \textbf{подпр-вом} в $V$, если оно само явл-ся ЛП отн-но операций, опред. в $V$.
\end{definition}
\begin{symb}
$W \leq V$ - $W$ подпр-во $V$
\end{symb}
\begin{statement}
Если $W \leq V$, то $0_W = 0_V$, и если для $w \in W, -w $ - ему прот. вектор в $W$, то он же явл-ся прот. вектором в $V$.
\end{statement}
\begin{proof}
Было доказано в терминах подгрупп.
\end{proof}
\begin{statement}[Критерий подпр-ва]
Непустое подмн-во $W \subset V$ над $F$ - подпр-во в $V$ $\iff$
\begin{itemize}
  \item [a) ] $W$ замкнуто от-но сложения, т. е.:
    \[
    \forall a, b \in W \hookrightarrow a + b \in W
    \]
  \item [b) ] $W$ замкнуто от-но умножения на скаляр, т. е.:
    \[
    \forall \lambda \in F, \forall a \in W \hookrightarrow \lambda a \in W
    \]
\end{itemize}
\end{statement}
\begin{proof}
\begin{itemize}
  \item [$\Rightarrow$)] Очевидно.
  \item [$\Leftarrow$)] Пусть усл-ия $a$ и $b$ вып-ся. Верно ли:
\[
  W \overset{?}{\leq} V
\]
\[
  a \in W \colon (-1)a \in W. \text{ Покажем, что $(-1)a = -a$}
\] 
\[
  (-1)a + a = (-1)a + 1 \cdot a = (-1 + 1) a = 0 a = \overline{0}
\]
\[
  a + (-a) = \overline{0} \Rightarrow \overline{0} \in W
\]
Из этих следствий следует верность критерия подпр-ва.
\end{itemize}
\end{proof}
\begin{consequence}
Пересечение любого числа подпр-в ЛП $V$ само явл-ся подпр-вом.
\end{consequence}
\begin{proof}
$W_i \leq V \Rightarrow \bigcap_{i}^{} W_i \leq V$
\end{proof}
\subsubsection{Подполе лин. объектов системы векторов}
Пусть $S$ - произв. сист. векторов из $V$ (возм. бесконечное)
\begin{definition}
Линейная оболочка системы $S$ наз-ся наименьшая по включению подпр-во в $V$, содерж. $S$
\end{definition}
\begin{symb}
\[
<S> = \bigcap_{W \leq V, S \leq W}^{} W
\]
\end{symb}
\begin{statement}
$<S> = \Set{\sum_{i = 1}^{n} \alpha_i s_i | s_i \in S, \alpha_i \in F, n \in \Z_{+}}$
\end{statement}
\begin{note}
Если $n = 0$, то рассм. $\overline{0}$
\end{note}
\begin{proof}
\[
L = \Set{\sum_{i = 1}^{n} \alpha_i s_i | s_i \in S, \alpha_i \in F, n \in \Z_{+}}
\]
\[
s_i \in S \Rightarrow 1 \cdot s_i \in L \Rightarrow \forall s \in S, s \in L
\]
Покажем, что $L \leq V \land S \subset L$:
\[
\sum_{i}^{} \alpha_i s_i \in L, \sum_{i}^{} \beta_i s_i \in L \Rightarrow \sum_{i}^{} (\alpha_i + \beta_i) s_i \in L
\]
\[
\lambda(\sum_{}^{} \alpha_i s_i) = \sum_{i}^{} (\lambda \alpha_i) s_i \Rightarrow L \leq V
\]
По опред. $\Rightarrow <S> \subset L$. Теперь покажем $L \subset <S>$:
\[
s_i \in S, \forall i \Rightarrow s_i \in <S>
\]
Т. к. $<S>$ - подпр-во $V$
\[
  \Rightarrow \alpha \cdot s_i \in <S>, \forall \alpha \in F\Rightarrow \sum_{i}^{} \alpha_i s_i \in <S> \Rightarrow L \subset <S>
\]
\end{proof}
\begin{definition}
Если $<S> = V$, то говорят, что $V$ порождено $S$.
\end{definition}
\begin{definition}
ЛП $V$ наз-ся \textbf{конечно-порождённым}, если оно имеет конечное порождающее мн-во
\end{definition}
\subsubsection{Базис}
\begin{definition}
  Пусть $V$ - ЛП над $F$. Базисом в $V$ наз-ся уп. система векторов $G = \begin{pmatrix}e_1 & e_2 & e_3 & \ldots & e_n \end{pmatrix}$, если вып-ны усл-ия:
  \begin{itemize}
    \item [a) ] $G$ - ЛНЗ над $F$ (т. е. $\sum_{i}^{}\alpha_i e_i = \overline{0} \iff \alpha_i = 0 \in F, \forall i$).
    \item [b) ] Каждый вектор пр-ва $V$ представим в виде ЛК векторов $G$. Это усл-ие равносильно следующему:
      \[
      <\set{e_1, \ldots, e_n}> = V
      \]
  \end{itemize}
\end{definition}
\begin{example}
\begin{itemize}
  \item [1) ] $F^{n}$ базис:
    \[
      e_1 = \begin{pmatrix}1 \\ 0 \\ 0 \\ \vdots \\ 0 \end{pmatrix}, e_2 = \begin{pmatrix} 0 \\ 1 \\ 0 \\ \vdots \\ 0 \end{pmatrix}, \ldots e_n = \begin{pmatrix} 0 \\ 0 \\ 0 \\ \vdots \\ 1 \end{pmatrix}
    \]
    \[
    \begin{pmatrix}\alpha_1 \\ \vdots \\ \alpha_n \end{pmatrix} = \sum_{i = 1}^{n} \alpha_i e_i
    \]
  \item [2) ] $F_n[x]$ базис:
    \[
    1, x, x^{2}, \ldots, x^{n}
    \]
\end{itemize}
\end{example}
\begin{statement}
Всякое конечнопорождённое ЛП $V$ имеет базис.
\end{statement}
\begin{proof}
Среди все конечных мн-во, порождающих $V$, выберем наименьшее по мощности. (мощность конечного мн-ва - это число его эл-ов). $\Rightarrow S_0$. Явл-ся ли $S_0$ базисом? \\

Если $S_0$ ЛЗ, то $\exists s_0 \in S_0$, представимый как ЛК остальных эл-ов мн-ва $\Rightarrow S_0 \subset <S_0 \backslash \set{s_0}> \Rightarrow <S_0 \backslash \set{s_0}> = V$. Но это противоречие с тем, что $S_0$ - наименьшее по мощности. $\Rightarrow S_0$ - ЛНЗ.
\end{proof}
\begin{statement}[Основная лемма теории ЛП]
  $V$ - ЛП над $F$. $V = \begin{pmatrix} u_1 & \ldots & u_n \end{pmatrix}$ и $W = \begin{pmatrix}w_1 & \ldots & w_m \end{pmatrix}$. Известно, что $\forall w_i \in W$ - представим как ЛК векторв $V$. Тогда, если $m > n$, то сист. $W$ - ЛЗ 
\end{statement}
\begin{proof}
Индукция по $n$:
\begin{itemize}
  \item База: n = 1
\[
V = \begin{pmatrix} u \end{pmatrix}
\]
По усл-ию:
\[
w_1 = \lambda_1 u, w_2 = \lambda_2 u, \ldots w_m = \lambda_m u
\]
Если $\exists \lambda_i = 0$, то $W$ - ЛЗ. Иначе возьмём ЛК:
\[
\lambda_2 w_1 - \lambda_1 w_2 + 0w_3 + 0w_4 + \ldots + 0w_m = 0 \Rightarrow W \text{ - ЛЗ}
\]
  \item Переход: пусть утв. справедливо, для $V$, т. ч. $|V| = n - 1$. Докажем, для $|V| = n$:
    \[
    w_1 = \sum_{i = 1}^{n} \lambda_{1i} u_i
    \]
    \[
    \vdots
    \]
    \[
    w_j = \sum_{i = 1}^{n} \lambda_{ji} u_i
    \]
    Для каждого $i = 2, m$, отнимем от $w_i$ $w_1 \cdot \frac{\lambda_{1i}}{\lambda_{11}}$. Таким образом перейдем к системам:
    \[
      \overline{V} = \begin{pmatrix}u_2 & \ldots & u_n \end{pmatrix}, \overline{W} = \begin{pmatrix}w_2 - w_1 \cdot \frac{\lambda_{1i}}{\lambda_{11}} & \ldots \end{pmatrix}
    \]
    По предположению индукции: $\overline{W}$ - ЛЗ $\Rightarrow W$ - ЛЗ.
\end{itemize}
\end{proof}
