\section{Лекция 11}
\subsection{Ортогональная классификация алг. кривых II порядка}
\begin{definition}
Алг. кривой (п-тью) наз-ся мн-во всех точек $V_2$ ($V_3$), удовл. ур-ию:
\[
P(x, y) = 0 (P(x, y, z) = 0)
\]
\end{definition}
\begin{definition}
Наименьшая из степеней мн-ов, задающих кривую (пов-ть) наз-ся её \textbf{порядком}.
\end{definition}
\begin{statement}
Порядок алг. кривой (п-ти) не зависит от выбора ДСК.
\end{statement}
\begin{proof}
$(O, \overline{e_1}, \overline{e_2})$ и $(O', \overline{e_1'}, \overline{e_2'})$
\[
S = S_{G\to G'} \iff G' = GS
\]
\[
\begin{pmatrix}x \\ y \end{pmatrix} = S\begin{pmatrix}x' \\ y' \end{pmatrix} + \begin{pmatrix} \alpha \\ \beta \end{pmatrix}, O' \underset{(O, G)}{\longleftrightarrow}\begin{pmatrix}\alpha \\ \beta \end{pmatrix}
\]
\[
\Gamma \colon P(x, y) = 0 \mapsto Q(x', y') = 0
\]
пор. $P(x, y) \geq $ пор. $Q(x', y')$
\begin{note}
Каждый моном мн-на $Q$ получается из соотв. монома мн-на $P$
\end{note}
В то же время $\exists$ обратная замена коор-т:
\[
\alpha = S\alpha' + \gamma
\]
\[
\alpha - \gamma = S\alpha'
\]
\[
 \alpha' = S^{-1}(\alpha - \gamma) = S^{-1}\alpha - S^{-1}\gamma
\]
пор. $Q(x', y') \geq $ пор. $P(x, y)$
\end{proof}
\begin{equation}
Ax^{2} + 2Bxy + Cy^{2} + 2Dx + 2Ey + F = 0, (A^{2} + B^{2} + C^{2} \neq 0)
\end{equation}
Б. О. О. система коор-т - ПДСК. Будем упрощать, переходя к новой ПДСК $(O', G')$
\begin{itemize}
  \item [I)] Можно избавиться от слагаемого $2Bxy$ подходящим поворотом системы коор-т.
    \[
      G' = GS = G \begin{pmatrix}\cos \phi & -\sin\phi \\ \sin\phi & \cos\phi \end{pmatrix} \iff \begin{pmatrix}x \\ y \end{pmatrix} = \begin{pmatrix}\cos\phi & -\sin\phi \\ \sin\phi & \cos\phi\end{pmatrix}\begin{pmatrix}x' \\ y' \end{pmatrix}
    \]
    После перехода в новую ПДСК:
    \[
    A(x'\cos\phi - y'\sin\phi)^{2} + 2B(x'\cos\phi - y'\sin\phi)(x'\sin\phi + y'\cos\phi) + C(x'\sin\phi + y'\cos\phi)^{2} + \ldots = 
    \]
    \[
  = A'x'^{2} + 2B'x'y' + C'y'^{2}
    \]
    \[
    2B' = -2A\sin\phi\cos\phi + 2B(\cos^{2}\phi - \sin^{2}\phi) + 2C\sin\phi\cos\phi = (C - A)\sin 2\phi + 2B\cos 2\phi
    \]
    \begin{itemize}
      \item [a)] Пусть $A \neq C$:
        \[
        B' = 0 \iff (C - A)\tg 2\phi + 2B = 0 \iff \tg 2\phi = \frac{2B}{A - C}
        \]
        \[
        \tg 2\phi = \frac{2\tg \phi}{1 - \tg^{2}\phi}
        \]
      \item [b) ] Пусть $A = C$:
        \[
        B' = 0 \iff \cos 2\phi = 0
        \]
        Можем взять $\phi = \frac{\pi}{4}$
    \end{itemize}
  \begin{equation}
  A'x'^{2} + C'y'^{2} + 2D'x' + 2E'y' + F' = 0
  \end{equation}
\item [II)]
  \begin{lemma}
    Если хотя бы один из коэф-ов при квадратах коор-т (т. е. $A'$ или $C'$) - ненулевой, то подходящим параллельным переносом начала коор-т вдоль соотв. оси можно избавиться от линейного члена по соотв. коор-те. [Если $A' \neq 0$, то можно избавиться от члена $2D'x'$; $C' \neq 0 \rightarrow 2E'y'$]
  \end{lemma}
  \begin{proof}
  Пусть $A' \neq 0$:
  \[
  A'(x'^{2} + \frac{2D'}{A'}x' + \frac{D'}{A'}) + C'y'^{2} + 2E'y' + F' - \frac{(D')^{2}}{A'} = 0
  \]
  Замена коор-т:
  \[
  \begin{cases}
  x'' = x' + \frac{D'}{A'} \\
  y'' = y'
  \end{cases}
  \]
  \[
  \Rightarrow A'x''^{2} + C'y'^{2} + 2E'y' + F'' = 0
  \]
  Применение леммы никак не меняет коэф-ти квадратичной части ур-я.
  \end{proof}
\end{itemize}

Получили:
\begin{equation}
Ax^{2} + Cy^{2} + 2Dx + 2Ey + F = 0
\end{equation}
\textbf{3 случая:}
\begin{itemize}
  \item [1) ] $AC > 0$ - эллиптический тип кривых (E)
  \item [2) ] $AC < 0$ - гиперболический тип (H)
  \item [3) ] $A = 0$ или $C = 0$ - параболлический тип (P)
\end{itemize}
\begin{itemize}
  \item [E) ] Эллиптический случай: $AC > 0$: применяя лемму приведём ур-е к виду:
    \begin{equation}
      \label{eq:etype}
    Ax^{2} + Cy^{2} = -F
    \end{equation}
    При необходимости домножаем ур-е $(\ref{eq:etype})$ на $(-1)$, далее считаем, что $A > 0, C > 0$ \\
    Пусть $F' \neq 0$:
    \[
      \frac{x^{2}}{a^{2}} + \frac{y^{2}}{b^{2}} = \begin{cases}
      1, \text{ (1)} \\
      -1, \text{ (2)}\\
      0, \text{ (3)}
      \end{cases}
    \]
    Где $a^{2} = \frac{F}{A}, b^{2} = -\frac{F}{C}$. В ур-ях (1)-(3) м. считать, что $a \geq b > 0$. Иначе, если $a < b$, то применим $R(\frac{\pi}{2})$:
    \[
      \begin{pmatrix}x' \\ y' \end{pmatrix} = \begin{pmatrix} 0 & -1 \\ 1 & 0 \end{pmatrix} \begin{pmatrix}X \\ Y \end{pmatrix}
    \]
    \[
      \frac{x'^{2}}{a^{2}} + \frac{y'^{2}}{b^{2}} = \frac{X^{2}}{b^{2}} + \frac{Y^{2}}{a^{2}}, b > a
    \]
    \begin{definition}
    Ур-ие (1) наз-ся каноническим ур-ем эллипса, а соотв. кривая - эллипс.
    \end{definition}
    \begin{definition}
    Ур-ие (2) наз-ся каноническим ур-ем мнимого эллипса, а кривая - мнимый эллипс.
    \end{definition}
    \begin{definition}
    Ур-ие (3) наз-ся канонический ур-ем пары пересекающихся мнимых прямых, а кривая - пара пересекающихся мнимых прямых.
    \end{definition}
  \item [H)] Гиперболический случай: $AC < 0$: \\
    Применяя лемму, получим:
    \[
    A'x^{2} + C'y^{2} = -F''
    \]
    \begin{itemize}
      \item [1) ] $F'' \neq 0$. При необходимости , применяя поворот на $R(\frac{\pi}{2})$, получим:
        \[
        \text{знак}(A') = \text{знак}(-F'')
        \]
        \[
          \text{знак}(C') = -\text{знак}(F'') = \text{знак}(F'')
        \]
        Разделим на $(-F'')$:
        \begin{equation}
          \label{eq:num4}
          \frac{x''^{2}}{a^{2}} - \frac{y''^{2}}{b^{2}} = 1
        \end{equation}
        \[
        a^{2} = -\frac{F''}{A'} > 0, b^{2} = \frac{F''}{C'} > 0
        \]
        $a \lor b$ не имеет значения.
        \begin{definition}
        Ур-ие $\ref{eq:num4}$ наз-ся канон. ур-ем гиперболы, а кривая - гипербола.
        \end{definition}
      \item [2) ] $F'' = Q \colon $
        \begin{equation}
          \label{eq:num5}
          \frac{x''^{2}}{a^{2}} - \frac{y''^{2}}{b^{2}} = 0
        \end{equation}
        \begin{definition}
        Ур-ие $\ref{eq:num5}$ наз-ся ур-ем пары действительных прямых, а кривая - пара действительных прямых. 
        \end{definition}
      \item [P) ] $A' = 0$ либо $C' = 0$. При необходимости применяя $R(\frac{\pi}{2})$, можно считать, что $A' = 0$. Применяя лемму, аннулируем линейную часть:
        \[
        C'y'^{2} + 2D'x' + F' = 0
        \]
      \item [1) ]
        Пусть $D' \neq 0$:
        \[
        C'y'^{2} + 2D'(x' + \frac{F'}{2D'}) = 0
        \]
        Замена:
        \[
          \begin{cases}
          x' + \frac{F''}{2D''} = x* \\
          y* = y'
          \end{cases}
        \]
        \[
        C'y*^{2} + 2D'x* = 0
        \]
        \[
        y*^{2} = 2px^{*}, p = -\frac{D'}{C'}
        \]
        \begin{note}
        Параметр $p$ м. считать полож. $p > 0$. Иначе применим поворот $R(\pi)$:
        \[
          \begin{cases}
        X* = -X \\
        Y* = -y
          \end{cases}
        \]
        \begin{equation}
          \label{eq:num6}
        y*^{2} = 2px 
        \end{equation}
        \end{note}
        \begin{definition}
        Ур-е $(\ref{eq:num6})$ наз-ся канон. ур-ем параболы.
        \end{definition}
    \end{itemize}
  \item [2) ] Пусть $D = 0$:
    \[
    y'^{2} = \begin{cases}
      a^{2}, \text{ (7)} \\
      -a^{2}, \text{ (8)} \\
      0, \text{ (9)}
    \end{cases}
    \]
    \begin{definition}
    Ур-ем (7) наз-ся кан. ур-е пары действ. параллельных прямых
    \end{definition}
    \begin{definition}
    Ур-ем (8) наз-ся кан. ур-е пары мнимых. параллельных прямых
    \end{definition}
    \begin{definition}
    \ldots пара действ. совпад. прямых.
    \end{definition}
\end{itemize}
\begin{note}
ПДСК $ \rightarrow $ ПДСК - и все указ. преобразования сохр. ориент. пл-ти.
\end{note}
\begin{theorem}
Всякую кривую второго порядка можно привести к одному из 9 канон. видов.
\end{theorem}
\subsection{Инвариант кривых II пор.}
\begin{equation*}
Ax^{2} + 2Bxy + Cy^{2} + 2Dx + 2Ey + F = 0
\end{equation*}
\begin{definition}
Инвариантом ур-я кривой, наз-ся непостоянная ф-ция от её коэф-ов, которая не меняется при переходе от ПДСК к ПДСК.
\end{definition}
\begin{theorem}
Следующие 3 ф-ции являются инвариантами кривой: 
\[
  \triangle = \begin{vmatrix}A & B & D \\ B & C & E \\ D & E & F \end{vmatrix}, \delta = \begin{vmatrix}A & B \\ B & C \end{vmatrix}, I = A + C
\]
\end{theorem}
\begin{proof}
Запишем ур-е в матричном виде:
\[
  \begin{pmatrix}x & y & z \end{pmatrix} \begin{pmatrix}A & B & D \\ B & C & E \\ D & E & F \end{pmatrix} \begin{pmatrix}x \\ y \\ z \end{pmatrix}  = 0, (z = 1)
\]
\[
  \begin{pmatrix}x \\ y \end{pmatrix} = \begin{pmatrix}\cos\phi & -\sin\phi \\ \sin\phi & \cos\phi \end{pmatrix}\begin{pmatrix}x' \\ y' \end{pmatrix} + \begin{pmatrix}\alpha \\ \beta \end{pmatrix} \iff \begin{pmatrix}x \\ y \\ z \end{pmatrix} = \begin{pmatrix}\cos\phi & -\sin\phi & \alpha \\ \sin\phi & \cos\phi & \beta \\
0 & 0 & 1\end{pmatrix}\begin{pmatrix}x' \\ y' \\ z' \end{pmatrix}
\]
В новой ПДСК:
\[
  \begin{pmatrix}x' & y' & z' \end{pmatrix} \begin{pmatrix}\cos\phi & \sin\phi & 0 \\ -\sin\phi & \cos\phi & 0 \\ \alpha & \beta & 1 \end{pmatrix}\begin{pmatrix}A & B & D \\ B & C & E \\ D & E & F \end{pmatrix}\begin{pmatrix}\cos\phi & -\sin\phi & \alpha \\ \sin\phi & \cos\phi & \beta \\ 0 & 0 & 1 \end{pmatrix}\begin{pmatrix}x' \\ y' \\ z' \end{pmatrix} |_{z' = 1} = 0
\]
\[
  A' = \begin{pmatrix}\cos\phi & \sin\phi & 0 \\ -\sin\phi & \cos\phi & 0 \\ \alpha & \beta & 1\end{pmatrix} \begin{pmatrix} A & B & D \\ B & C & E \\ D & E & F \end{pmatrix} \begin{pmatrix}\cos\phi & -\sin\phi & \alpha \\ \sin\phi & \cos\beta & \beta \\ 0 & 0 & 1 \end{pmatrix}
\]
\[
  \triangle' = det\begin{pmatrix}\cos\phi & \sin\phi & 0 \\
    -\sin\phi & \cos\phi & 0 \\ \alpha & \beta & 1\end{pmatrix} \triangle \cdot det\begin{pmatrix}\cos\phi & -\sin\phi & \alpha \\ \sin\phi & \cos\phi & \beta \\ 0 & 0 & 1 \end{pmatrix} = \triangle
\]
\[
  \delta' = det R(-\phi) \delta \cdot det R(\phi) \Rightarrow \delta' = \delta
\]
\begin{definition}
\textbf{След квадратной матрицы} $tr A$ - сумма чисел, стоящих на главной диагонали.
\end{definition}
\begin{note}
 \[
 tr (AB) = tr(BA)
 \]
\end{note}
$I' = tr(R(-\phi) \delta R(\phi)) = tr(\delta \cdot R(\phi) R(-\phi)) = tr(\delta) = I$
\end{proof}

