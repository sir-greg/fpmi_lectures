\section{Лекция 7}
\begin{definition}
Пусть $\overline{a}, \overline{b}, \overline{c} \in V_3$ \\
Двойным векторным произв. наз-ся выр-е: $[\overline{a}, [\overline{b}, \overline{c}]]$
\end{definition}
\begin{theorem}[Тождество БАЦ-ЦАБ]
\[
[\overline{a}, [\overline{b}, \overline{c}]] = \overline{b}(\overline{a}, \overline{c}) - \overline{c}(\overline{a}, \overline{b})
\]
\end{theorem}
\begin{proof}
Выделим правый ОНБ след. образом:
\[
  \overline{e_1} || \overline{a}
\]
\[
\overline{e_2}, \text{т. ч. } (\overline{a}, \overline{b}, \overline{e_2}) - \text{компланарная сист.}
\]
\[
\overline{e_3} = [\overline{e_1}, \overline{e_2}]
\]
Тогда:
\[
\overline{a} \underset{G}{\longleftrightarrow} \begin{pmatrix}\alpha \\ 0 \\ 0 \end{pmatrix}
\]
\[
\overline{b} \underset{G}{\longleftrightarrow} \begin{pmatrix} \beta_1 \\ \beta_2 \\ 0 \end{pmatrix}
\]
\[
\overline{c} \underset{G}{\longleftrightarrow} \begin{pmatrix}\gamma_1 \\ \gamma_2 \\ \gamma_3 \end{pmatrix}
\]
\[
  [\overline{b}, \overline{c}] = \begin{vmatrix} \overline{e_1} & \overline{e_2} & \overline{e_3} \\ \beta_1 & \beta_2 & 0 \\ \gamma_1 & \gamma_2 & \gamma_3\end{vmatrix} =  \beta_2\gamma_3\overline{e_1} - \overline{e_2} \beta_1\gamma_3 + \overline{e_3}(\beta_1\gamma_2 - \beta_2\gamma_1) \underset{G}{\longleftrightarrow} \begin{pmatrix}\beta_2\gamma_3 \\ -\beta_1\gamma_3 \\ \beta_1\gamma_2 - \beta_2\gamma_1 \end{pmatrix}
\]
\[
  [\overline{a}, [\overline{b}, \overline{c}]] = \begin{vmatrix}\overline{e_1} & \overline{e_2} & \overline{e_3} \\ \alpha & 0 & 0 \\ \beta_2\gamma_3 & -\beta_1\gamma_3 & \beta_1\gamma_2 - \beta_2\gamma_1 \end{vmatrix} \underset{G}{\longleftrightarrow} \begin{pmatrix} 0 \\ -\alpha(\beta_1\gamma_2 - \beta_2\gamma_1) \\ -\alpha\beta_1\gamma_3 \end{pmatrix}
\]
\[
  \overline{b}(\overline{a}, \overline{c}) - \overline{c}(\overline{a}, \overline{b}) = \begin{pmatrix}\alpha\beta_1\gamma_1 \\ \alpha\beta_2\gamma_1 \\ 0 \end{pmatrix} - \begin{pmatrix}\alpha\beta_1\gamma_1 \\ \alpha\beta_1\gamma_2 \\ \alpha\beta_1\gamma_3 \end{pmatrix} = [\overline{a}, [\overline{b}, \overline{c}]]
\]
\end{proof}
\begin{consequence}[Тождество Якоби]
\[
[\overline{a}, [\overline{b}, \overline{c}]] + [\overline{b}, [\overline{c}, \overline{a}]] + [\overline{c}, [\overline{a}, \overline{b}]] = \overline{o}, \forall \overline{a}, \overline{b}, \overline{c}
\]
\end{consequence}

\subsection{Понятие ур-я мн-ва. Задание прямой на пл-ти}
$V_2$ или $V_3$ с фикс. ДСК.
\begin{definition}
Ур-ем мн-ва $M \subset V_i$ наз-ся высказывание, верное $\forall x \in M$ и неверное $\forall x \in V_i \backslash M$
\end{definition}
$V_2$ с фикс. ДСК
\begin{definition}
Ненулевой вектор $\overline{a}$, кот. $||$ данной прямой $l$ наз-ся её \textbf{направляющим вектором}.
\end{definition}
Picture(2) \\
\underline{Векторное параметрическое ур-е прямой:}
\begin{equation}
\overline{r} = \overline{r_0} + t \overline{a}, t \in \R \\
\end{equation}
\[
\overline{a} \underset{G}{\longleftrightarrow} \begin{pmatrix}\alpha_1 \\ \alpha_2 \end{pmatrix}
\]
\[
\overline{r} \underset{G}{\longleftrightarrow} \begin{pmatrix}x \\ y \end{pmatrix}
\]
\[
\overline{r_0} \underset{G}{\longleftrightarrow} \begin{pmatrix}x_0 \\ y_0 \end{pmatrix}
\]
\underline{Коорд. параметрическое ур-е прямой:}
\begin{equation}
\begin{cases}
x = x_0 + \alpha_1 t \\
y = y_0 + \alpha_2 t
\end{cases}, t \in \R
\end{equation}
\underline{Каноническое ур-е прямой:}
\begin{equation}
t = \frac{x - x_0}{\alpha_1} = \frac{y - y_0}{\alpha_2} = \frac{z - z_0}{\alpha_3}
\end{equation}
Если $\alpha_2 = 0$: \\
$l\colon y - y_0 = 0$
\begin{note}
Если одна из коор-т напр. вектора равна 0, то соотв. коор-т можно приравнять к начальной
\end{note}
\[
\alpha_2 (x - x_0) - \alpha_1(y - y_0) = 0
\]
При $A = \alpha_2, B = -\alpha_1$, имеем \underline{общее ур-е прямой на пл-ти:}
\begin{equation}
Ax + By + C = 0
\end{equation}
\[
\overline{a} = \begin{pmatrix}-B \\ A \end{pmatrix}
\]
\begin{statement}
Пусть $l$ задана общ. ур-ем (4), $X_0 \begin{pmatrix}x_0 \\ y_0 \end{pmatrix} \in l$ \\
Тогда $X_1 \begin{pmatrix}x_1 \\ y_1 \end{pmatrix} \in l \iff A(x_1 - x_0) + B(y_1 - y_0) = 0$
\end{statement}
\begin{proof}
 \begin{itemize}
   \item [a)] Необходимое:
     \begin{equation*}
       \begin{cases}
     Ax_0 + By_0 + C = 0 \\
     Ax_1 + By_1 + C = 0 
       \end{cases} \Rightarrow A(x_1 - x_0) + B(y_1 - y_0) = 0
     \end{equation*}
   \item [b) ] Достаточное:
     \[
     X_0 \in l \text{ и } A(x_1 - x_0) + B(y_1 - y_0) = 0
     \]
     \[
     \Rightarrow Ax_1 + By_1 + C = 0
     \]
 \end{itemize}
\end{proof}
\begin{consequence}
Вектор $\overline{b} \begin{pmatrix}p_1 \\ p_2 \end{pmatrix}$ явл. направляющим вектором $l\colon$
\[
Ax + By + C = 0 \iff Ap_1 + Bp_2 = 0
\]
\end{consequence}
\begin{proof}
Picture(1)
\end{proof}
\begin{theorem}
Пусть $l\colon Ax + By + C = 0$. \\
Тогда любой напр. вектор этой прямой коллинеарен вектору $\begin{pmatrix}-B \\ A \end{pmatrix}$, а в кач-ве начальной точки $X_0$ этой прямой можно взять:
\[
X \begin{pmatrix}-\frac{AC}{A^{2} + B^{2}} \\ -\frac{BC}{A^{2} + B^{2}} \end{pmatrix}
\]
\end{theorem}
\begin{proof}
  \begin{itemize}
    \item[a) ]
\[
\lambda\begin{pmatrix}-B \\ A \end{pmatrix} - \text{вектор}
\]
\[
A(-\lambda B) + B(-\lambda A) = 0 \Rightarrow \text{напр.}
\]
\item [b) ] \[
-\frac{A^{2}C}{A^{2} + B^{2}} - \frac{B^{2}C}{A^{2} + B^{2}} + C = -C + C = 0
\]
  \end{itemize}
\end{proof}
\begin{consequence}
  Все рассм. выше способы задания прямой $l$ - эквивалентны.
\end{consequence}
\subsubsection{Случай ПДСК}
$(O, G)$ - ПДСК
\begin{statement}
$l\colon Ax + By + C = 0$ \\
Тогда вектор $\overline{n}\begin{pmatrix} A \\ B \end{pmatrix} \perp l$
\end{statement}
\begin{proof}
\[
\overline{a}\begin{pmatrix}-B \\ A \end{pmatrix} \Rightarrow (\overline{n}, \overline{a}) = A(-B) + B(-A) = 0 \Rightarrow \overline{n} \perp \overline{a}
\]
\end{proof}
\begin{definition}
Вектор $\overline{n}$ наз-ся вектором нормали к прямой $l$
\end{definition}
\underline{Ур-е прямой с угловым коэффициентом} (ПДСК):
Picture (3) and (4)
\begin{note}
Если $B \neq 0$ (и только в этом случае), то ур-е $l\colon Ax + By + C = 0$ можно записать в виде:
\[
 y = -\frac{A}{B}x - \frac{A}{B}
\]
\end{note}
\subsubsection{Признаки параллельности/перпендикулярности прямых на плоскости}
\begin{statement}
\begin{itemize}
  \item [a)]  Прямые:
    \begin{equation*}
      \begin{cases}
    y = k_1 x + b_1 \\
    y = k_2 x + b_2
      \end{cases} \iff k_1 = k_2
    \end{equation*} 
  \item [b) ] Прямые: $i = \overline{1,2} \colon A_i x + B_i y + C_i = 0 $ параллельны $\iff \begin{vmatrix}A_1 & B_1 \\ A_2 & B_2 \end{vmatrix} = 0$ 
\end{itemize}
\end{statement}
\begin{proof}
\begin{itemize}
  \item[a) ] Picture(5)\[
  k_1 = \tg \phi = k_2
  \]
\item [b) ] \[
    l_1 || l_2 \iff \overline{n_1} || \overline{n_2} \iff S(\overline{n_1}, \overline{n_2}) = 0 \iff \begin{vmatrix} A_1 & A_2 \\ B_1 & B_2 \end{vmatrix}
\]
\end{itemize}
\end{proof}
\begin{statement}
\begin{itemize}
  \item[a) ] Прямые:
    \begin{equation*}
    \begin{cases}
    y = k_1x + b_1 \\
    y = k_2x + b_2
    \end{cases}
    \end{equation*}
    перпендикулярны, при $k_1k_2 = -1$
  \item [b) ] \[
  l_1 \perp l_2 \iff A_1A_2 + B_1B_2 = 0
  \]
\end{itemize}
\end{statement}
\begin{proof}
\begin{itemize}
  \item [a) ] \[
  \phi_1 = \phi_2 + \frac{\pi}{2} \iff \tg\phi_1 = \tg(\phi_2 + \frac{\pi}{2}) = -\ctg\phi_2 = -\frac{1}{\tg\phi_2} \iff 
  \]
  \[
  \iff \tg\phi_1 \tg\phi_2 = -1 \iff k_1 k_2 = -1
  \]
\item [b) ] \[
    l_1 \perp l_2 \iff (\overline{n_1}, \overline{n_2}) = 0 \iff A_1A_2 + B_1B_2 = 0
\]
\end{itemize}
\end{proof}
\begin{statement}
Прямые $l_1\colon A_1x + B_1y + C_1 = 0$, $l_2\colon A_2x + B_2y + C_2 = 0$
\begin{itemize}
  \item [a) ] Пересекаются по одной точке $\iff \begin{vmatrix}A_1 & B_1 \\ A_2 & B_2 \end{vmatrix} \neq 0$ 
  \item [b) ] Параллельны (включая совпадение) $\iff \begin{vmatrix} A_1 & B_1 \\ A_2 & B_2 \end{vmatrix} = 0$
  \item [c) ] Совпадают $\iff$ ур-я пропорциональны
\end{itemize}
\end{statement}
\begin{proof}
\begin{equation*}
\begin{cases}
A_1x + B_1y + C_1 = 0 \\
A_2x + B_2y + C_2 = 0
\end{cases}
\end{equation*}
\begin{itemize}
  \item [a) ] Единственное решение при $\begin{vmatrix}A_1 & B_1 \\ A_2 & B_2 \end{vmatrix} \neq 0$
  \item [b) ] Противоположность $a) \colon \begin{vmatrix} A_1 & B_1 \\ A_2 & B_2 \end{vmatrix} = 0$
  \item [c) ] Пусть $l_1 = l_2 \Rightarrow$
    \begin{equation*}
    A_1B_2 = B_1A_2 \Rightarrow \frac{A_1}{A_2} = \frac{B_1}{B_2} = \lambda \Rightarrow
    \begin{cases}
   A_1 = \lambda A_2 \\
   B_1 = \lambda B_2
    \end{cases} \text{(где $\lambda \neq 0, \lambda \neq \infty$)}
    \end{equation*}
\end{itemize}
\end{proof}
\begin{definition}
Полупл-тью, определяемой прямой $l$ и вектором нормали $\overline{n}$, наз-ся мн-во всех точек $x$ пл-ти, т. ч. вектор $\overline{X_0X}$ составляет с вектором $\overline{n}$ угол $\leq \frac{\pi}{2}$
\[
\cos\phi \geq 0 \iff (\overline{X_0X_1}, \overline{n}) \geq 0
\]
\[
A(x - x_0) + B(y - y_0) \geq 0
\]
\[
Ax_0 + By_0 = 0 \Rightarrow Ax + By + C \geq 0
\]
Picture (7)
\end{definition}
