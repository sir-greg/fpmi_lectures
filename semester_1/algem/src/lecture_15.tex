\section{Лекция 15}
\subsection{Характеристика поля}
$F$ - поле.
\[
\exists 0, 1 \in F, 0 \neq 1 \\
\]
\[
\underset{n}{1 + 1 + 1 + \ldots + 1} = n_F
\]
Положим: \\ 
\[
  0_F = 0
\]

\[
  (-n_F) = -(n_F), n \in \N
\]
\begin{lemma}
\[
  (n + m)_F = n_F + m_F
\]
\[
  (nm)_F = n_F \cdot m_F
\]
\end{lemma}
\begin{proof}
  $n > 0, m > 0$:
  \[
    (\underset{n}{1 + 1 + \ldots + 1})(\underset{m}{1 + 1 + \ldots + 1}) = \underset{n \cdot m}{1 + 1 + \ldots + 1}
  \]
\end{proof}
\begin{definition}
\textbf{Хар-кой поля} $F$ наз-ся наим. \underline{натур.} число $n \in \N$, т. ч.:
\[
n_F = 0
\]
Если $\forall n \in \N, n_F \neq 0$, то говорят, что хар-ка равна $0$.
\end{definition}
\begin{example}
$\Z_p \colon \underset{p}{\overline{1} + \overline{1} + \ldots + \overline{1}} = \overline{0} = \overline{p}$ \\
Поля: $\Q, \R, \C$ имеют хар-ку 0.
\end{example}
\begin{symb}
$char(F)$ - хар-ка поля $F$
\end{symb}
\begin{statement}
 Если поле $F$ имеет ненулевую хар-ку ($char(F) \neq 0$), то $char(F) = p$, где $p$ - простое число.
\end{statement}
\begin{proof}
От прот., пусть $char(F) = n$, $n$ - составное:
\[
n = p \cdot q, 1 < p, q < n
\]
\[
n_F = p_F \cdot q_F = 0!!! \text{(Прот-е, т. к. в поле нет делителей нуля.)}
\]
\[
\Rightarrow char(F) \text{ - простое.}
\]
\end{proof}
\begin{definition}
Пусть $G$ - группа/кольцо/поле. Непустое подмн-во $H \subset G$ наз-ся \textbf{подгруппой/подкольцом/подполем}, если оно само является группой/кольцом/полем, отн-но операции, опр-ой на $G$.
\end{definition}
\begin{statement}
Если $H$ - подгруппа в группе $G$, то $e_G = e_H$.
\end{statement}
\begin{proof}
\[
e_H \cdot e_H = e_H
\]
В $G$ для $e_H$ есть обратный $e_H^{-1}$:
\[
e_H = e_H \cdot e_G = e_G
\]
\end{proof}
\begin{consequence}
У подкольца $0$ совпадает с $0$ кольца, а у всякого подполя 0 и 1 совпадают с 0 и 1 поля.
\end{consequence}
\[
  (F, +) \text{ - аб. гр. с нейтр. эл-ом 0}
\]
\[
  (F, *) \text{ - аб. гр. с нейтр. эл-ом 1}
\]
\begin{statement}[Критерий подгруппы]
Непустое подмн-во $H$ в группе $G$ явл. подгруппой в ней $\iff$
\begin{itemize}
  \item [a) ] $H$ замкнуто отн-но групповой оп-ции в $G$ (*)
    \[
    \forall a, b \in H (a * b \in H)
    \]
  \item [b) ] $H$ замкнуто отн-но взятия обратного эл-та, т. е.:
    \[
    \forall a \in H (a^{-1} \in H)
    \]
\end{itemize}
\end{statement}
\begin{proof}
\begin{itemize}
  \item [1) ] \textbf{Необх.} Пусть $H$ - подгруппа в $G$ [$H \leq G$] - очев., по опр-ю подгруппы.
  \item [2) ] \textbf{Дост.} $H \neq \emptyset$ и выполн-ся усл-я $a), b)$
    \[
    a) \iff \text{"*" опр-на в $H$} 
    \]
    \begin{itemize}
      \item Ассоц-ть вып-ся в $H$, т. к. вып-ся в $G$
      \item $\forall a \in H, \exists a^{-1} \in H$
      \item $\forall a \in H \Rightarrow \exists a^{-1} \in H \Rightarrow a * a^{-1} = e \in H$
    \end{itemize}
\end{itemize}
\end{proof}
\begin{statement}
Пусть $G$ - группа/кольцо/поле. Пусть $G_i$ - подгруппа/подкольцо/подполе $G$. Тогда:
\[
\bigcap_{i}^{} G_i \text{ - подгруппа/подкольцо/подполе}
\]
\end{statement}
\begin{proof}
Докажем для поля $F$:
\[
\forall i, F_i \leq F
\]
\[
  (F_i, +) \text{ - аб. группа} \Rightarrow 
\]
\[
  \forall i \colon 
\begin{cases}
\forall a, b \in F_i \Rightarrow a + b \in F_i \\
\forall a \in F_i \Rightarrow -a \in F_i
\end{cases} \rightarrow \bigcup_{i}^{} (F_i, +) \text{ - аб. группа.}
\]
\[
\forall i \colon (F_i^{*}, *) \text{ - аб. группа} \Rightarrow \forall a, b \in F_i^{*} \Rightarrow a * b \in F_i, a^{-1} \in F_i \Rightarrow (\bigcap_{i}^{}F_i^{*}) \text{ - аб. группа.}
\]
\end{proof}
\subsection{Гомоморфизим и изоморфизм групп.}
Пусть $(G_1, *)$, $(G_2, *)$ - группы.
\begin{definition}
Отображение $\phi: G_1 \rightarrow G_2$ наз-ся гомоморфизмом, если $\phi$ сохраняет в этих группах операции.
\[
\forall a, b \in G_i \hookrightarrow \phi(a \circ b) = \phi(a) * \phi(b)
\]
\end{definition}
\begin{definition}
  Отобр. $\phi: X \rightarrow Y$ наз-ся инъективным, если:
  \[
  \forall a, b \in X \colon a \neq b \hookrightarrow \phi(a) \neq \phi(b)
  \]
\end{definition}
\begin{definition}
  Отобр. $\phi: X \rightarrow Y$ наз-ся сюрьективным, если:
  \[
  \phi(X) = Y, (\forall y \in Y, \exists x \in X \colon \phi(x) = y)
  \]
\end{definition}
\begin{definition}
  Отобр. $\phi \colon X \rightarrow Y$  наз-ся биективным, если оно С + И.
\end{definition}
\begin{definition}
  \textbf{Изоморфизм} - биективный гомоморфизм.
\end{definition}
\begin{note}
Всё перечисленное для групп переносится на кольца и поля.
\end{note}
\begin{statement}
При гомоморфизме групп $f: G_1 \rightarrow G_2$:
\begin{itemize}
  \item [a) ] Нейтральный эл-т переходит в нейтральный:
    \[
    f(e_{G_1}) = e_{G_2}
    \]
  \item [b) ] $\phi$ - коммутирует со взятием обратно эл-та:
    \[
    \phi(a^{-1}) = \phi^{-1}(a)
    \]
\end{itemize}
\end{statement}
\begin{proof}
\begin{itemize}
  \item [a) ] $*$ - умножение:
    \[
    e_1 * e_1 = e_1 \Rightarrow \phi(e_1) \cdot \phi(e_1) = \phi(e_1) = \phi^{-1}(e_1)
    \]
    \[
    \phi(e_1) = \phi(e_1) \cdot e_2 = e_2
    \]
  \item [b) ] \[
  a \cdot a^{-1} = a^{-1} \cdot a = e_1
  \]
  \[
  \phi(a) \phi(a^{-1}) = \phi(a^{-1}) \phi(a) = e_2
  \]
  \[
  \phi(a^{-1}) = \phi^{-1}(a)
  \]
\end{itemize}
\end{proof}
\begin{consequence}
При гомоморфизме полей 0 и 1 первого поля переходят в 0 и 1 второго.
\end{consequence}
\subsection{Простое подполе}
\begin{definition}
Поле $F$ наз-ся \textbf{простым}, если оно не имеет подполей, отличных от него самого.
\end{definition}
\begin{example}
Поле $\Q$ и $\Z_p$ - простые поля.
\end{example}
\begin{proof}
  Пусть $M \subset \Q$ - простое.
  \[
    0, 1 \in M
  \]
  \[
  \underset{n}{1 + 1 + \ldots + 1} = n \in M \Rightarrow \frac{1}{n} \in M \Rightarrow \frac{m}{n} \in M \Rightarrow \Q \subset M
  \]
  \[
  \Rightarrow M = \Q
  \]
  Аналогично, пусть $N \subset \Z_p$:
  \[
  \overline{0}, \overline{1} \in N \Rightarrow k * \overline{1} = \underset{k}{\overline{1} + \overline{1} + \ldots + \overline{1}} \in N \Rightarrow \Z_p \subset N \Rightarrow \Z_p = N 
  \]
\end{proof}
\begin{theorem}
Всякое поле содержит пустое подполе, и притом только 1.
\end{theorem}
\begin{proof}
$F$ содержит подполя $F_i$ ($F_i \subset F$). Положим:
\[
D = \bigcap_{F_i \leq F}^{} F_i \Rightarrow D \leq F, \text{ причём $D$ в любом другом подполе поля $F$}
\]
Почему $D$ простое подполе? \\
От прот., пусть $M \leq D \leq F \Rightarrow M \leq F \land D \not\subset M!!$, т. е. есть подполе $F$, в кот. нет $D$ - противоречие. \\
Почему оно единственно? \\
От прот., пусть $D$ и $D'$ - 2 простых подполя $\Rightarrow D \cap D'$ - подполе поля $F$. \\
\[
D \cap D' \subset D, D' \Rightarrow D \cap D' = D, D' \Rightarrow D = D'
\]
\end{proof}
\begin{theorem}[Об описании простых подполей]
  \begin{itemize}
    \item [a)] Если $char(F) = 0$, то его простое подполе $D$ изоморфно $\Q$
    \item [b) ] Если $char(F) = p, p$ - простое, то его простое подполе $D$ изоморфно $\Z_p$
  \end{itemize}
\end{theorem}
\begin{proof}
\begin{itemize}
  \item [a) ] $0, 1 \in D$. Если $n_F = 0 \Rightarrow n = 0$
    \[
    \Rightarrow \underset{n}{1 + 1 + \ldots + 1} = n_F \in D \Rightarrow \exists \text{ вложение } \Z \text{ в } F \colon n \vdash n_F
    \]
    Это гомоморфизм, т. к.:
    \[
      (n + m) = n_F + m_F
    \]
    \[
      (n \cdot m)_F = n_F \cdot m_F
    \]
    Пусть $n_F = m_F \Rightarrow (n \cdot m)_F = 0 \Rightarrow n - m = 0 \Rightarrow n = m$
\end{itemize}
\end{proof}
