\section{Лекция 24(вроде)}
\begin{statement}
  \label{lo:inj}
ЛО $\phi: U \rightarrow V$ --- инъективно $\iff$ $\ker \phi = \set{\overline{0}}$
\end{statement}
\begin{proof}
\begin{itemize}
  \item Необх.: пусть $\phi$ - инъективно $\Rightarrow$ $\forall x \neq \overline{0} \hookrightarrow$
    \[
    \phi(x) \neq \phi(\overline{0}) = \overline{0} \Rightarrow \ker \phi = \set{\overline{0}}
    \]
  \item Дост.: пусть $\ker \phi = \set{\overline{0}}$. Покажем, что $\phi$ - инъективно. Пусть $\exists x_1, x_2 \in V \colon \phi(x_1) = \phi(x_2)$
    \[
    \hookrightarrow \phi(x_1) - \phi(x_2) = \phi(x_1 - x_2) = \overline{0} \Rightarrow x_1 = x_2
    \]
\end{itemize}
\end{proof}
\begin{consequence}
Пусть $\phi \colon V \rightarrow W$ --- ЛО, кот. удовл. одному из двух условий экв-ных условий утв-я $(\ref{lo:inj})$. Тогда $\phi$ переводит ЛНЗ в ЛНЗ.
\end{consequence}
\begin{proof}
Пусть система $x_1, \ldots, x_n$ --- ЛНЗ. От прот., пусть:
\[
  \phi(x_1), \ldots, \phi(x_n) \text{ --- ЛЗ}
\]
Тогда $\exists$ нетрив. ЛК:
\[
\lambda_1 \phi(x_1) + \lambda_2 \phi(x_2) + \ldots + \lambda_n \phi(x_n) = \overline{0}
\]
\[
  \phi(\lambda_1 x_1 + \ldots + \lambda_n x_n) = \overline{0} \Rightarrow \sum_{i = 1}^{n} \lambda_i x_i = \overline{0} \in \ker \phi, \exists i \colon \lambda_i \neq 0
\]
Прот. с тем, что $x_1, \ldots, x_n$ --- ЛНЗ.
\end{proof}
\begin{theorem}[Теорема о гомоморфизмах ЛП]
\label{th:1:hom_lp}
Пусть $\phi \colon V \rightarrow W$ - ЛО. Пусть $V = \ker \phi \oplus U$. Тогда $\exists$ канонический изоморфизм пр-в $U$ на $\Im \phi$. Более того, если:
\[
\phi|_U \colon U \rightarrow \Im \phi \text{ --- изоморфизм}
\]
\end{theorem}
\begin{proof}
$\phi(U) \subseteq \Im \phi$. 
...???
\end{proof}
\begin{theorem}[О ядре и образе ЛО]
\label{th:2:ker_proj_lo}
$\phi \colon V \rightarrow W$ --- ЛО. Тогда справ-во:
\[
\dim \ker \phi + \dim \Im \phi = \dim V
\]
\end{theorem}
\begin{proof}
Пусть, как в теореме $(\ref{th:1:hom_lp})$, $V = \ker \phi \oplus U$:
\[
\phi|_U \colon U \rightarrow \Im \phi \Rightarrow \dim V = \dim \Im \phi
\]
По т. $(\ref{th:1:hom_lp})$:
\[
\dim V = \dim \ker V + \dim V = \dim \ker \phi + \dim \Im \phi
\]
\end{proof}
\begin{note}
  Верно ли, что если $\phi \colon V \rightarrow V$, то $\hookrightarrow V = \ker \phi \oplus \Im \phi$. Нет, это не так.
\end{note}

\begin{definition}[Матрицы ЛО]
  $\phi \colon V \rightarrow W$. Пусть
  \[
    G = \begin{pmatrix}e_1 & \ldots & e_n \end{pmatrix} \text{ --- базис $V$ }
  \]
  \[
    G' = \begin{pmatrix}f_1 & \ldots & f_m \end{pmatrix} \text{ --- базис $W$}
  \]
  \[
  W \ni \phi(e_1) = a_{11} f_1 + \ldots a_{m 1} f_m
  \]
  \[
    \vdots
  \]
  \[
   \phi(e_n) = a_{n 1} f_1 + \ldots + a_{n m} f_m
  \]
  \[
    A_{\phi} = \begin{pmatrix}a_{11} & a_{12} & \ldots & a_{1n} \\ \vdots & \vdots & \ldots & \vdots \\ a_{m 1} & a_{m 2} & \ldots & a_{mn}\end{pmatrix} \in M_{m \times n}(F)
  \]
  Можно записать иначе:
  \[
  \begin{pmatrix}\phi(e_1) \\ \vdots \\ \phi(e_n) \end{pmatrix} = A_{\phi}^{T} \begin{pmatrix}f_1 \\ \vdots \\ f_n \end{pmatrix}
  \]
  \[
    \begin{pmatrix}\phi(e_1) & \ldots & \phi(e_n) \end{pmatrix} = \begin{pmatrix}f_1 & \ldots & f_n \end{pmatrix} A_{\phi}
  \]
  \[
  \phi(G) = f \cdot A_{\phi}
  \]
\end{definition}
\begin{definition}
Построенная матрица $A_\phi$ наз-ся матрицей ЛО $\phi$ отн-но базисов $G$ и $G'$
\[
\phi \underset{(G, G')}{\longleftrightarrow} A_\phi
\]
\end{definition}
\begin{statement}
\[
\phi \colon V \rightarrow W
\]
Пусть $G$ --- базис в $V$, $f$ --- базис в $W$
\[
\phi \underset{(G, G')}{\longleftrightarrow} A_\phi, V \ni x\underset{G}{\longleftrightarrow} \alpha, \phi(x) \underset{f}{\longleftrightarrow} \beta
\]
Тогда $\beta = A_\phi \alpha$
\end{statement}
\begin{proof}
\[
x = G \alpha, \phi(x) = f \beta
\]
\[
\phi(x) = \phi(G \alpha) = \phi(G) \cdot \alpha = f \cdot A_\phi \cdot \alpha \Rightarrow \beta = A_\phi \alpha
\]
\end{proof}
\subsection{Операции над ЛО}
Пусть $\mathcal{L}(V, W)$ --- мн-во всех ЛО из $V$ в $W$. (или $\hom(V, W)$)
\[
\phi, \psi \in \mathcal{L}(V, W)
\]
\[
 \Rightarrow (\phi + \psi)(x) = \phi(x) + \psi(x)
\]
\[
   (\lambda \phi)(x) = \lambda \phi(x)
\]
Покажем аддитивность:
\[
  (\phi + \psi)(x + y) = \phi(x + y) + \psi(x + y) = \phi(x) + \phi(y) + \psi(x) + \psi(y) = 
\]
\[
 = (\phi + \psi)(x) + (\phi + \psi)(y)
\]
\begin{note}
Легко проверить выполнение аксиом ЛП для $\mathcal{V, W}$, причём в кач-ве нулевого вектора выступет нулевое отображение.
\end{note}

\begin{statement}
  Соответствие:
\[
\phi \underset{(G, G')}{\longleftrightarrow} A_\phi
\]
явл-ся изоморфизмом пр-ва $\mathcal{L}(V, W)$ на пр-во $M_{m \times n}(F)$
\end{statement}
\begin{proof}
\begin{itemize}
  \item [a) ] Сохранение $"+"$?
    \[
      (\phi + \psi)(G) = \begin{pmatrix} (\phi + \psi)(e_1) & \ldots & (\phi + \psi)(e_n)) \end{pmatrix} = 
    \]
    \[
    = \begin{pmatrix}\phi(e_1) + \psi(e_1) & \ldots & \phi(e_n) + \psi(e_n) \end{pmatrix}
    \]
    \[
      = \begin{pmatrix}\phi(e_1) & \ldots & \phi(e_n) \end{pmatrix} + \begin{pmatrix}\psi(e_1) & \ldots & \psi(e_n) \end{pmatrix} = f \cdot A_\phi + f \cdot A_\psi = 
    \]
    \[
     = f (A_\phi + A_\psi)
    \]
  \item [b) ] Биективность? Инъективность возникает из того, что только $0$ имеет нулевую матрицу. \\
    Сюрьективность? $\forall A \in M_{m \times n} (F) \exists!$ ЛО, со столбцами вида $\phi(e_1), \ldots, \phi(e_n)$
\end{itemize}
\end{proof}
\begin{consequence}
\[
\dim \mathcal{L}(V, W) = \dim M_{m \times n} = m \cdot n = \dim W \cdot \dim V
\]
\end{consequence}
\subsection{Ранг лин. отображения}
\begin{definition}
  $\phi \colon V \rightarrow W$ --- ЛО. Ранг $\phi (\rk \phi)$ наз-ся размерностью пр-во $\Im \phi$
\end{definition}
\begin{theorem}[О ранге лин. отображения]
\label{th:3_rank_lo}
$\phi: V \rightarrow W$ --- ЛО. Тогда $\rk \phi$ равен $\rk A_{\phi}$ не зависимо от выбора базисов в $V$ и $W$.
\end{theorem}
\begin{proof}
Вспомним рав-во:
\[
\Im \phi = <\phi(e_1), \ldots, \phi(e_n)>, \text{ где $E$ --- базис $V$}
\]
\[
\forall i \colon \phi(e_i) \in \Im \phi \Rightarrow \supseteq
\] 
$\subseteq ?$ Пусть $y \in \Im \phi$. Тогда $\exists x \in V \colon y = \phi(x) = \phi(\sum_{ i = 1}^{n} x_i e_i) = $
\[
= \sum_{ i = 1}^{n} x_i \phi(e_i) \in <\phi(e_1), \ldots, \phi(e_n)>
\]
\[
\rk \phi = \dim \Im \phi = \dim <\phi(e_1), \ldots, \phi(e_n)> = \rk A_\phi
\]
\end{proof}
\subsection{Изменение матр. ЛО при замене базисов}
\begin{theorem}
\label{th:4}
Пусть $\phi: V \rightarrow W$ --- ЛО. Пусть $G, G'$ --- базисы $V$, $G' = GS$, (т. е. $S = S_{G\to G'}$) \\
Пусть $F, F'$ - базисы $W$, $F' = FT$ \\
Пусть $\phi \underset{(G, F)}{\longleftrightarrow} A_{\phi}$ и $\phi \underset{(G', F')}{\longleftrightarrow} A_{\phi}'$ \\
Тогда:
\[
A'_\phi = T^{-1}\cdot A_\phi \cdot S
\]
\end{theorem}
\begin{proof}
\[
\phi(G) = F \cdot A_\phi \text{ и } \phi(G') = F' \cdot A'_\phi
\]
\[
F = F' \cdot T^{-1}
\]
Имеем:
\[
\phi(G') = \phi(GS) = \phi(G) \cdot S = F \cdot A_\phi \cdot S = F' \cdot T^{-1} \cdot A_{\phi} \cdot S
\]
\[
 \Rightarrow A'_\phi = T^{-1} \cdot A_\phi \cdot S
\]
\end{proof}
\begin{consequence}
Пусть $T$ и $S$ невырожденные матрицы, т. ч.:
\[
T^{-1} \cdot A \cdot S \text{ --- имеет смысл}
\]
Тогда:
\[
  \rk (T^{-1} \cdot A \cdot S) = \rk A
\]
\end{consequence}
\begin{proof}
  \[
    A = A_\phi, \phi \colon V \rightarrow W
  \]
\[
  \rk(A_\phi) = \rk (T^{-1} \cdot A_\phi \cdot S)
\]
Т. к. ранг не зависит от выбранных базисов.
\end{proof}
К какому наиболее простому виду можно привести матрицу отображения подоходящей заменой базиса? Ответ: к единичному диагональному виду.
\begin{theorem}
\label{th:5}
Пусть $\phi \colon V \rightarrow W$ --- ЛО. Тогда в $V$ и $W$ $\exists G, F$ --- базисы, т. ч.:
\[
  \phi \underset{(G, F)}{\longleftrightarrow} \begin{pmatrix}E_r & 0 \\ 0 & 0 \end{pmatrix}, r = \rk \phi
\]
\begin{proof}
Пусть $V = U \oplus \ker \phi$, $U$ --- прямое дополнение к $\ker \phi$
\[
\phi|_U \colon U \rightarrow \Im \phi
\]
Выберем в пр-ве $V$ базис, согласованный с разл. в $\oplus$, т. е.:
\[
e_1, \ldots e_r \text{ --- базис в $U$}, e_{r + 1}, \ldots, e_n \text{ --- базис в $\ker \phi$}
\]
\[
f_1 = \phi(e_1), \ldots, f_r = \phi(e_r) \text{ --- базис в $\Im \phi$}
\]
И дополним его до базиса в $W$
\[
f_{r + 1}, \ldots, f_n
\]
Покажем, что пара базисов $(E, F)$ --- искомая пара базисов.
\[
\phi(e_1) = f_1 = 1 \cdot f_1 + 0 \cdot f_2 + \ldots + \cdot 0 \cdot f_n
\]
\[
\phi(e_2) = f_2 = 0 \cdot f_1 + 1 \cdot f_2 + 0 \cdot f_3 + \ldots + 0 \cdot f_n
\]
\[
\vdots
\]
\[
\phi(e_r) = 0 \cdot f_1 + 0 \cdot f_2 + \ldots + 0 \cdot f_{r - 1} + 1 \cdot f_r + 0 \cdot f_{r + 1} + \ldots + 0 \cdot f_n
\]
\[
\phi(e_{r + i}) = \overline{0}, \forall i > 0
\]
\end{proof}
\end{theorem}
