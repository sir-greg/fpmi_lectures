\section{Лекция 11 (Хэш таблицы)}
Хотим сделать хэш-таблицу.
\begin{definition}
Ключи $x \neq y$  образуют коллизию отн-но $h$, если $h(x) = h(y)$
\end{definition}
\begin{note}
Пусть $h$ --- случайная, а $x \neq y$ --- ключи. Тогда:
\[
P_h(h(x) = h(y)) = \frac{1}{k}
\]
\end{note}
\begin{proof}
$P_h(h(x) = h(y)) = \sum_{i = 1}^{k} \underbrace{P(h(x) = i, h(y) = i)}_{\frac{1}{k^{2}}} = \frac{1}{k}$ 
\end{proof}
Версии:
\begin{itemize}
  \item Алгоритмы (хэш-таблица цепочками)
    \[
    h \colon U \rightarrow \set{0, 1, \ldots, m - 1}
    \]
    Заводим $m$ односвязных списков $l_0, \ldots, l_{m - 1}$
    \begin{itemize}
      \item insert x: вычисл. $i = h(x)$ добавл. $x$ в список $l_i$ (добавляем в начало)
      \item erase x: вычисл. $i = h(x)$, удалить $x$ из $h(i)$
      \item find x: вычисл. $i = h(x)$, проверяем, есть ли $x$ в $l_i$
    \end{itemize}
\end{itemize}
\begin{note}
Если всего добавлено $n$ ключей, то вообщем размеры всех списков $\approx \frac{n}{m}$. Более того, если $x$ --- левый ключ, то мат. ожид. дляны списка $l_{h(x)}$ есть $\frac{n}{m}$
\end{note}
\[
E\left|l_{h(x)}\right| = \sum_{i = 1}^{n} P_{h}(h(y_i) = h(x)) = \frac{n}{m}
\]
\begin{definition}
Пусть $H = \set{h_s \colon U \rightarrow \set{0, 1, \ldots, m - 1}}$
\end{definition}
Такое сем-во ф-ций наз-ся универ. сем-вом хэш. ф-ций, если $\forall x \neq y\in U$
\[
P_s(h_s(x) = h_s(y)) \leq \frac{1}{m}
\]
Мораль: вместо идеальной хэш ф-ции можно брать $h_s \in H$
\begin{statement}
Пусть $U = \Z_p = \set{0, 1, \ldots, p - 1}$, $p$ --- простое, тогда при любом $m < p$ след. сем-во будет универс. $H = \set{h_{a, b}(x) = ((a x + b) \% p) \% m}$, $a, b \in \Z_p$
\end{statement}
\begin{proof}
Пусть $x \neq y \in \Z_p$. Хотим оценить $P_{a, b}(h_{a, b}(x) = h_{a, b}(y))$
\[
  \underbrace{(ax + b) \% p}_{u} \% m \text{ и } \underbrace{(ay + b) \% p}_{v} \% m
\]
\begin{itemize}
  \item [1) ] $u = v \Rightarrow (ax + b) \% p = (ay + b) \% p \Rightarrow a(x - y) \% p = 0$
    \[
    \Rightarrow x - y \% p = 0 
    \]
    $P(u = v) = \frac{1}{p}$
  \item [2) ] $u \neq v$
    \[
      \begin{pmatrix}x & 1 \\ y & 1 \end{pmatrix}\begin{pmatrix}a \\ b \end{pmatrix} = \begin{pmatrix}u \\ v \end{pmatrix} \text{ в $\Z_p$}
    \]
    Невыр. $\Rightarrow$ $\exists$ реш. $(a, b)$
    \[
    P_{a, b}(h(x) = h(y)) \leq \frac{1}{p} + P_{u \neq v}(u \% m = v \% m)
    \]
    Т. к. все пары $(u \neq v)$ равновероятно получаются:
    \[
      (x, y) \rightarrow ((ax + b) \% p, (ay + b) \% p)
    \]
    Кол-во коллизий есть $v$ есть $\approx \frac{1}{p}$
\end{itemize}
\end{proof}
\textbf{Новый алгоритм}: \\
Заведем пустую хэш-таблицу с $m$ списками, пусть $n$ кол-во добавленных эл-ов. \\
$\alpha = \frac{n}{m}$ --- Load factor \\
insert, erase, find --- делаются как раньше. \\
rehash --- когда $\alpha$ превышает некий порог (обычно $0.75$), то увеличиваем $m$ вдвое, выбираем новое $h$, перекладываем всё в новую hash table.
\begin{theorem}
\label{}
В среднем такая структура работает за $O^{*}(1)$ в среднем.
\end{theorem}
\subsection{Совершенное хэширование}
\begin{theorem}[(б/д) Парадокс дней рождения]
Пусть $x_1, \ldots, x_n$ --- случайные числа в $\Z_m$:
\begin{itemize}
  \item [1) ] Если $m \geq n^{2}$, то $P(\text{есть коллизии}) \leq \frac{1}{2}$
  \item [2) ] Если $m \leq \frac{1}{5}n^{2}$, то $P(\text{есть коллизии}) > \frac{1}{2}$
  \item [3) ] Более того, если $H$ - сем-во универс. хэш ф-ций, то пункт $1)$ берём даже, если вместо $x_1, \ldots, x_n$ брать $h(x_1), \ldots, h(x_n)$
\end{itemize}
\end{theorem}
В структуре лежат только $x_1, \ldots, x_n$:
insert x $\in \set{x_1, \ldots, x_n}$ \\
erase x $\in \set{x_1, \ldots, x_n}$ \\
find x (x --- любой)
