\section{Лекция 4}
\subsection{Кучи}
\begin{definition}
Куча - СД, умеющая:
\begin{itemize}
  \item Хранить мультимн-во эл-ов $S$
  \item $insert(x)$ - добавить $x$ в $S$
  \item $getMin()$ - вернуть $min(S)$
  \item $extractMin()$ - найти $min(S)$ и удалить его
  \item $decreaseKey(ptr, \triangle), \triangle > 0$ - уменьшить число по адресу $ptr$ на $\triangle$
\end{itemize}
\textbf{Применения}: алгоритмы Дейкстры, Прима
\end{definition}
\subsubsection{Бинарная (двоичная) куча}
Храним $S$ в массиве $a_1, a_2, \ldots, a_n$

Picture(1)

\textbf{Требование кучи}: эл-т, записанный в каждой вершине, $\leq$ всех эл-ов своего поддерева

Тогда $getMin()\colon$ return $a_1$;

\lstset{style=mystyle}
\begin{lstlisting}[language=C++, caption=siftUp and siftDown]
siftUp(u):
  if (v == 1) return
  p = (v / 2)
  if (a[p] > a[v]) {
    swap(a[p], a[v])
    siftUp(p)
  }

siftDwon(v)
  if (2 * v > n) return
  u = 2v
  if (2 * v + 1 <= n && a[2 * v + 1] < a[u]): u = 2 * v + 1
  if (a[u] < a[v]) {
    swap(a[u], a[v])
    siftDown(u)
  }

\end{lstlisting}
Остальные методы в heap.cpp

Асимптотика всего: $O(\log n)$

\textbf{Корректность}
\begin{lemma}
Пусть $a_1, \ldots, a_n$ - корректная куча \\
Пусть $a_v \leftarrow x$
\begin{itemize}
  \item [1) ] Если $a_v$ уменьшилось, то после $siftUp(v)$ куча вновь станет корректной
  \item [2) ] Если $a_v$ увеличилось, то после $siftDown(v)$ куча вновь станет корректной
\end{itemize}
\end{lemma}
\begin{proof}
\begin{itemize}
  \item [1) ] Индукция по $v$:
    \begin{itemize}
      \item [База:] $v = 1$: куча остаётся корректной, $siftUp$ при уменьшении корня ничего не делает
      \item [Переход: ] $a_v \leftarrow x$
        \begin{itemize}
          \item [a) ] $x \geq a_p$ - родитель; нер-во сохраняется, $siftUp$ ничего не делает, куча остаётся корректной
          \item [b) ] $x < a_p$. Тогда сделаем $swap(a_p, a_v)$, тогда нер-во снова сохр., и, по предположению индукции, после $siftUp(p)$ - куча становится корректной. \\
            Picture(2)
        \end{itemize}
    \end{itemize}
  \item [2) ] Индукция от листьев к корню
    \begin{itemize}
      \item [База:] $v$ - лист, куча корректна, $siftDown$ ничего не делает 
      \item [Переход:] Пусть $a_u$ - наименьший из детей $v$ \\
        Picture(3)
    \end{itemize}
\end{itemize}
\end{proof}
\subsubsection{HeapSort}
Алгоритм:
\[
a_1, a_2, \ldots, a_n
\]
\begin{itemize}
  \item [1) ] $insert a_1, \ldots, a_n$ 
  \item [2) ] $extractMin \text{ n раз}$ \end{itemize}
Асимптотика: $O(n\log n)$
\begin{note}
Heapsort основан на сравнениях
\end{note}
\begin{consequence}
$\neg\exists$ реализации кучи осн. на сравнениях, в кот. $insert$ и $extractMin$ работают за $O(n)$
\end{consequence}
Процедура $heapify$: строит корректную кучу по $n$ эл-ам без доп. памяти за $O(n)$
\lstset{style=mystyle}
\begin{lstlisting}[language=Python, caption=Heapify]
for i = n...1:
  siftDown(i)
\end{lstlisting}
Корректность? Индукцией по $i$: после вызова $siftDown(i)$, поддерево с корнем $i$ станет корректной кучей.
\begin{proof}
\begin{itemize}
  \item [База:] $i$ - лист
  \item [Переход:] Picture(4)
\end{itemize}
\end{proof}
Асимптотика: $O(n)$ \\
Время работы:
\begin{itemize}
  \item $\frac{n}{2}$ вершин обраб. за 1 оп.
  \item $\frac{n}{4}$ вершин обраб. за 2 оп.\\
    \vdots
\end{itemize}
\[
Sum = \frac{n}{2} \cdot 1 + \frac{n}{4} \cdot 2 + \ldots = \frac{n}{2} + \frac{n}{4} + \ldots + (\frac{n}{4} \cdot 1 + \frac{n}{8} \cdot 2 + \frac{n}{16} \cdot 3 + \ldots) =
\]
\[
= n + \frac{n}{2} + (\frac{n}{8} + \frac{n}{16} \cdot 2 + \frac{n}{32} \cdot 3 + \ldots) \leq 2 \cdot n
\]
\subsubsection{Удаление из кучи}
$erase x$:
\begin{itemize}
  \item [a)] По указателю на $x$
    \begin{itemize}
      \item [1) ] $a_v \leftarrow -\infty$
      \item [2) ] $siftUp(v)$
      \item [3) ] $extractMin()$
    \end{itemize}
  \item [b)] По значению $x$ \\
    У нас нет способа найти $x$ в куче, поэтому:
    \begin{itemize}
      \item [1) ] Заведём кучи $A$ - то, что добавили, $D$ - то, что хотим удалить. При запросе удаления $x$, добавляем его в $D$
      \item [2)] Если при запросе $getMin(), A.getMin() == D.getMin()$, то удаляем $min$ в обоих кучах и смотрим далее.
    \end{itemize}
\end{itemize}
Итого: $n$ запросов $= O(n\log n)$
