\section{Лекция 10}
Храним мн-во эл-ов $S$:
\begin{itemize}
  \item $find(x) \colon$ сообщить, есть ли $x$ в $S$
  \item $insert(x) \colon $ добавить $x$ в $S$ (дубликаты игнорируются)
  \item $erase(x) \colon$ удалить $x$ из $S$
  \item [optional: ] перечислить эл-ты $S$ в порядке возрастания
  \item [optional: ] $merge(S_1, S_2) = S_3 = S_1 \cup S_2$
  \item [optional: ] $split(S, k) = \set{S_1, S_2} \colon $
    \[
    S_1 = \set{x \in S | x \leq k}
    \]
    \[
    S_2 = \set{x \in S | x > k}
    \]
\end{itemize}
\begin{definition}
\textbf{Бинарное дерево поиска} --- это дерево, со следующими св-вами:
\begin{itemize}
  \item [1) ] У каждой вершины не более 2-ух сыновей
  \item [2) ] Если в вершине записано число $x$, то во всех вершинах левого его поддерева значение в вершине $< x$, а в правом поддереве --- $> x$.
\end{itemize}
\end{definition}
Реализация операций:
\begin{itemize}
  \item [1) ] $find(x)$. Идём с корня в ту сторону, в кот. имеет смысл идти, пока не найдем $x$
  \item [2) ] $insert(x)$. Идём вдоль того же пути. Если пытаемся пойти в пустоту, то вставляем туда же наш $x$.
  \item [3) ] $erase(x)$. Идём по дереву. Находим $x$. Если имеется только одно поддерево, то очевидно. Иначе, делаем swap() с минимум правого поддерева (или максимум левого поддерева), и удаляем вершину, с кот. свапнули.
\end{itemize}
Всё работает за $O(h)$, где $h$ - высота дерева. Хотим: $h = O(\log n)$. Начнём с АВЛ.
\begin{definition}
AVL-дерево --- бинарное дерево поиска, т. ч. $\forall v \colon \left|h(L_v) - h(R_v)\right| \leq 1$, где $L_v, R_v$ - левое и правое поддерево $v$.
\end{definition}
\begin{statement}
AVL-дерево на $n$ вершинах имеет глубину $O(\log n)$
\end{statement}
\begin{proof}
Пусть $S(h)$ - мин. число вершин в AVL-дереве глубины $h$. Тогда:
\[
S(1) = 1
\]
\[
S(2) = 2
\]
\[
S(3) = 4
\]
\[
S(4) = 7
\]
Заметим, что это:
\[
S(h) = 1 + S(h - 1) + S(h - 2)
\]
Покажем, что:
\[
S(h) = F_{h + 2} - 1
\]
База, очев. Переход:
\[
S(h) = 1 + S(h -1 ) + S(h - 2) = 1 + F_{h + 1} - 1 + F_{h} - 1 = F_{h + 2} - 1
\]
\[
\Rightarrow S(h) = \Theta(\phi^{h}), \phi = \frac{1 + \sqrt{5}}{2}
\]
\end{proof}
Значит, если в AVL-дереве $n$ вершин и глубина $h$, то:
\[
n \geq S(h) = \Theta(\phi^{h}) \Rightarrow h = O(\log_{\phi}(n)) = O(\log n)
\]
Как же сохранять такую структуру? Введём для $v$:
\[
  \triangle(v) = h(L) - h(R)
\]
Определим малый поворот (и большой поворот) ... \\

\begin{definition}
  Splay-дерево - бинарное дерево поиска, т. ч. при каждом обращении к эл-ту, он поднимается в корень дерева.
\end{definition}
 Как именно эл-т поднимается в корень?
 \begin{itemize}
   \item [1) ] $zigzag(x)$.
   \item [2) ] $zigzig(x)$
   \item [3) ] $zig(x)$
 \end{itemize}
 \subsubsection{Метод потенциалов}
 $\Phi, \Phi \geq 0$:
 \[
 \Phi_{start} = 0
 \]
 \[
 \Phi_i \text{ - числовое состояние системы на $i$-ом шаге}
 \]
 \[
 a_i = t_i + \Phi_{i} - \Phi_{i - 1} \text{ --- уч. стоимость}
 \]
 \[
 t_i \text{ --- реальное время работы}
 \]
 \subsubsection{Возвращаемся к $splay$-дерево}
 Обсудим основные операции:
 \begin{itemize}
   \item [1) ] $splay(x)$ - комбинация преобразований $zig, zigzag, zigzig$, такая, что в конце, $x$ - корень.
     \[
     S(v) --- \text{ кол-во вершин в поддереве $v$}
     \]
     \[
     r(v) = \log_2 S(v)
     \]
     \[
     \Phi = \sum_{v \in V}^{} r(v)
     \]
     \begin{statement}
     \[
     a(splay(x)) \leq 1 + 3(r'(x) - r(x))
     \]
     \begin{proof}
     \[
     a(zig(x)) = 1 + r'(x) + r'(p) - r(x) - r(p) \leq 1 + r'(x) - r(x) \leq 1 + 3(r'(x) - r(x))
     \]
     \end{proof}
     где $r(x)$ и $r'(x)$ - старый и новый ранги $x$.
     \end{statement}
 \end{itemize}

