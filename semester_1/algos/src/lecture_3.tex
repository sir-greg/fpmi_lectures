\section{Лекция 3}

\begin{note}
$QuickSelect$ можно реализовать с привлечением $O(1)$ доп. памяти. (время $O(n)$ в среднем)
\end{note}
\begin{solution}
Поддерживаем указатели \ldots
\end{solution}

\subsection{Дерандомизация}

\begin{definition}
$a_1, a_2, \ldots , a_n$ массив. Его \textbf{медианой} наз-вом $a_{\ceil{\frac{n}{2}} }$ ($a$ - отсортирован)
\end{definition}

\begin{algo}[Медиана медиан]
$DQS(a_1, \ldots , a_n, k)$ (Derandomised Quick Select)

Разделим $a$ на блоки по 5 эл-ов. Пусть $b_i$ - медиана $i$-ого блока. Пусть $x = DQS(b_1, b_2, \ldots , b_{\floor{\frac{n}{5}}}, \frac{n}{10})$ - медиана массива $b$. Тогда $x$ - наш pivot.
\end{algo}
\lstset{style=mystyle}
\begin{lstlisting}[language=Python, caption=Updated DQS]
DQS(A, k):
  x = DQS(b_1, b_2, ..., b_(n/5), n/10)
  Parition(A, X)
  if (k <= l) => return DQS(B, k)
  if (k <= l + m) => return x
  return DQS(D, k - l - m)
\end{lstlisting}

\begin{statement}
Пусть $T(n)$ - время работы алгоритма $DQS$ на массива длины $n$ $\Rightarrow$
\[
T(n) = O(n)
\] 
\end{statement}
\begin{proof}
  $x$ - явл. порядковой статистикой массива $A$ с номером $\in [\frac{3}{10}n, \frac{7}{10}n]$
  
  Почему? Представим $b$ как табл. так, что $Mediana(b_i) < Mediana(b_j) (i < j)$
\[
  \begin{pmatrix}b_{11} & b_{12} & b_{13} & b_{14} & b_{15} \\
  b_{21} & b_{22} & b_{23} & b_{24} & b_{25} \\
\vdots & \vdots & \vdots & \vdots & \vdots \\
b_{\frac{n}{10}1} &b_{\frac{n}{10}2} & b_{\frac{n}{10}3} & b_{\frac{n}{10}4} & b_{\frac{n}{10}5}\end{pmatrix}
\] 
Тогда $x$ в центре табл., Ч. Т. Д.
\[
T(n) = O(n) (\text{поиск $b_1, \ldots , b_n$}) + T(\frac{n}{5})(\text{Нахождение $x$}) + O(n)(\text{Partition}) + T(\frac{7n}{10})
\] 
\[
T(n) \leq T(\frac{n}{5}) + T(\frac{7n}{10}) + Cn
\] 
Докажем, что $T(n) \leq 10Cn, \forall n$

МММ:
\begin{itemize}
  \item [База: ] $n \leq 5$ очев.
  \item [Переход: ] $T(n) \leq T(\frac{n}{5}) + T(\frac{7n}{10}) + Cn \leq \frac{10Cn}{5} + 7Cn + Cn == 10Cn $
\end{itemize}
\end{proof} 

\subsubsection{Derandomized Quick Sort}

Используя $DQS(A, \frac{n}{2})$ в качестве pivot, получаем новую асимптотику:
\[
T(n) \leq 2T(\frac{n}{2}) + O(n)
\] 
\[
\Rightarrow T(n) = O(n\log n)
\] 
\subsection{Сортировка чисел}

\begin{task}
Пусть есть $a_1, \ldots, a_n$ - массив чисел $a_i \in \set{0, 1,\ldots, k} $. Отсортировать $a$.
\end{task}

\begin{solution}
~\newline
  \lstset{style=mystyle}
  \begin{lstlisting}[language=Python, caption=Counting sort]
    cnt[k + 1] = {0, ..., 0}
    for i=1..n
      ++cnt[a[i]]
    for x=0..k:
      for i=1..cnt[x]
        print(x)
  \end{lstlisting}
  Асимптотика: $O(n + k)$
\end{solution}
\begin{definition}
  \textbf{Стабильная сортировка:}

$a_1, a_2, \ldots , a_n \Rightarrow a_{\sigma(1)}, a_{\sigma(2)}, \ldots, a_{\sigma(n)}$, при усл. что равные эл-ты сохраняют свой отн. порядок, т. е., если $a_{\sigma(i)} = a_{\sigma(j)}$ и $i < j$, то $\sigma(i) < \sigma(j)$
\end{definition}

\textbf{Стабильная сортировка подсчётом}:
\lstset{style=mystyle}
\begin{lstlisting}[language=Python, caption=stable counting sort]
cnt[k + 1] = {0, .., 0};
for i=1..n
  ++cnt[a[i]]
for x = 1..k:
  cnt[x] += cnt[x - 1]
b = {-1, .., -1}
for i=1..k:
 b[cnt[a[i]]] = a[i] // sigma[cnt[a[i]]] = i
 --cnt[a[i]]
\end{lstlisting}
$cnt[x]$ - кол-во эл-ов $\leq$ x

Асимптотика: $O(n + k)$

\begin{task}
Дан массив пар чисел:
\[
  (a_1, b_1), (a_2, b_2), \ldots , (a_k, b_k)
\] 
\[
a_i, b_i \in \set{0, 1, \ldots, k}
\] 
Отсортировать!
\end{task}
\begin{solution}
\begin{itemize}
  \item [1) ] Отсортируем массив пар по $b$ 
  \item [2) ] Результат \underline{стабильно}  отсортируем по $a$
\end{itemize}
Почему это корректно?

Пусть $(a_i, b_i) < (a_j, b_j)$ $\iff$
\begin{equation*}
  \begin{system_or}
    a_i < a_j \\
    a_i = a_j, b_i < b_j (\text{стабильность!})
  \end{system_or}
\end{equation*}
\end{solution}

\subsubsection{Least significant digit sort}

\begin{task}
Отсортировать массив чисел, $a_i \in \set{0, 1, \ldots, k}$, $k$ очень большое
\end{task}
\begin{solution}
Сортируем от младшего разряда к старшему \underline{стабильно}

Асимптотика: $O(\log k (n + 10))$ (десятич. система)

Вместо 10 - СС можно использовать СС с основанием $2^{b}$:
\[
  x \mod 2^{b} = x \land ((1 << b) - 1)
\] 
\end{solution}
