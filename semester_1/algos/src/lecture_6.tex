\section{Лекция 6}
\begin{statement}[Метод бух. учёта]
  Пусть к структуре поступает $n$ запросов, $t_i$ - реальное время выполнения $i$-ого запроса. \\
  Пусть $d_i$ - число монет, положенных на счёт; $w_i$ число снятых монет. Тогда, если баланс на счёте всегда $\geq 0$, то:
  \[
  a_i = t_i + d_i - w_i
  \]
  учётная стоимость $i$-ого запроса.
\end{statement}
\begin{proof}
$\set{a_i}$ - явл-ся уч. стоимостями, если реальное время = $O(\sum_{i = 1}^{n}a_i)$
\[
\sum_{i = 1}^{n} a_i = \sum_{i = 1}^{n} t_i + d_i - w_i = \sum_{i = 1}^{n} t_i+ \sum_{i = 1}^{n} (d_i - w_i), (d_i - w_i \geq 0)
\]
\[
\Rightarrow \sum_{i = 1}^{n} a_i \geq \sum_{i = 1}^{n} t_i
\]
\end{proof}
Рассм. динам. массив:
\begin{itemize}
  \item  Лёгкий push-back/pop-back:
    \begin{itemize}
      \item $t_i = O(1)$
      \item $d_i \leq 20$
      \item $w_i = 0$
      \item $a_i = O(1)$
    \end{itemize}
  \item Тяжёлый push-back/pop-back:
    \begin{itemize}
      \item $t_i \leq 4C$
      \item $w_i = t_i$
      \item $d_i = 0$
      \item $a_i = 0$
    \end{itemize}
  $O^{*}(1)$ - учётное время работы всех операций
\end{itemize}
\subsection{Связные списки}
\begin{itemize}
  \item [1) ] За линейное от размера списка время можно просмотреть все его эл-ты
  \item [2) ] В списке можно выполнять удаление по указателю за $O(1)$
\end{itemize}
\begin{note}
Для удобства реализации, в начало и конец списка можно положить некий фиктивный эл-т.
\end{note}
\subsection{Куча Фиббоначи}
Умеет:
\begin{itemize}
  \item [1)] getMin - $O(1)$
  \item [2) ] insert - $O(1)$ 
  \item [3) ] merge - $O(1)$
  \item [4) ] extractMin - $O^{*}(\log n)$
  \item [5) ] decreaseKey - $O^{*}(1)$
  \item []
\end{itemize}
Куча - \textbf{связный список} деревьев, каждое из кот. удовл. требованиям кучи. \\

Что храним в вершине?
\begin{itemize}
  \item [1) ] Указатель на левого и правого брата.
  \item [2) ] element x $\in X$
  \item [3) ] \textbf{Связный список} детей (А именно, указатель на самого левого сына) 
  \item [4) ] Степень вершины (Кол-во детей) (int degree)
  \item [5) ] bool mark: вырезался ли 1 из сыновей.
  \item [6) ] Указатель на родителя.
\end{itemize}
Разбираем операции:
\begin{itemize}
  \item \underline{getMin:} Вместе со списком корней будем хранить указатель на минимальный корень.
  \item \underline{Merge:} склеиваем два списка корней и пересчитываем min-root
  \item \underline{insert x:} Создаём кучу из одного эл-та + merge.
  \item \underline{extractMin}: Удалим вершину min-root, а всех детей merge-ым со старой кучей.
  \item \underline{decreaseKey ptr $\triangle$}: у корней можно удалять произв. кол-во сыновей., а у всех остальных не больше одного.
\end{itemize}
\subsubsection{Consolidate (Операция причёсывания кучи)}
Будем проходиться по всем корням и объединять деревья одного ранга. (ранг = degree). Объединение: \\

   Из двух деревье одного ранга ($H_1$, $H_2$), пусть $H_1$ - с меньшим числом в корне. Тогда подвесим $H_2$ к $H_1$. Теперь ранг $H_1$ увеличился на 1. \\

   Пусть $D(n)$ - max возможный ранг вершины в куче из $n$ эл-ов. (Позже покажем, что $D(n) = O(\log n)$). Тогда реальное время работы:
   \[
     D(n) + \#(\text{Объединений деревьев})
   \]

\subsubsection{Анализ времени работы}
Метод бух. учёта: \\

На каждом корне лежит по 1 монетке, на каждой вершине с mark = true лежит по 2 монетки. \underline{Тогда}:
\begin{itemize}
  \item [1) ] decreaseKey работает за O(1)
  \item [2) ] extractMin работает за $O^{*}(D(n))$
\end{itemize}
Осталось показать, что $D(n) = O(\log n)$ \\

Пусть $S(k)$ - min кол-во вершин в дереве, ранг кот. равен $k$ \\
$S(0) = 1, S(1) = 2, S(k) = ?$
\[
S(k) \geq 1 + 1 + S(0) + \ldots + S(k - 2)
\]
Отсюда следует, что $S(k) \geq F_{k + 2}$, где $F_k$ - k-ое число Фиббоначи.
\[
S(k) = \Omega(\phi^{k + 2})
\]
\begin{statement}
\[
  2 + F_2 + F_3 + \ldots + F_k = F_{k + 2}
\]
\end{statement}
\begin{proof}
\[
k = 2\colon 2 + F_2 = 3 = F_3
\]
\[
k = k\colon 2 + F_2 + \ldots + F_{k + 1} = F_{k + 3}
\]
\end{proof}
\[
F_n = \frac{\phi^{n} - (-\phi)^{-n}}{\sqrt{5}}, \phi = \frac{1 + \sqrt{5}}{2} > 1
\]
\[
F_n = \Theta(\phi^{n}) \Rightarrow S(k) = \Omega(\phi^{k})
\]
Если в дереве $n$ вершин, то макс. степень корня $\leq \log_{\phi}(n)$
