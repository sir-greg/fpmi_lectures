\section{Лекция 5}
\subsection{Биномиальная куча}
\textbf{Хотим следующие операции:}
\begin{itemize}
  \item getMin()
  \item extractMin()
  \item insert(x)
  \item decreaseKey(ptr, $\triangle$)
  \item merge(heap1, heap2) - обЪединение куч.
\end{itemize}
\begin{definition}
\textbf{Биномиальное дерево} ранга $k$:
\begin{itemize}
  \item [k = 0)] $T_0$ - одна вершина
  \item [k = 1)] $T_1$ - вершина с одним ребёнком
  \item [k = 2)] $T_2$ - Дерево $T_1$, к корню кот. ещё подвешено $T_1$    
  \item [k = n)] $T_n$ - Дерево $T_{n - 1}$, к корню кот. ещё подвешено $T_{n - 1}$
\end{itemize}
Кроме того, в вершинах дерева, есть числа, удовл. усл. обыкновенной кучи (значение в родителе $\leq$ значения в сыновьях)
\end{definition}
\begin{definition}
\textbf{Биномиальная куча} - это набор биномиальных деревьев, попарно различных рангов.
\begin{example}
  ~\newline
  $T_0, T_1, T_5$ - OK \\
  $T_3, T_5, T_5$ - NOT OK
\end{example}
\end{definition}
\begin{note}
\begin{itemize}
  \item [1) ] Если в куче всего $n$ - эл-ов, то в ней не более $\log_2{n}$ - деревьев, т. к. в $T_k$ ровно $2^{k}$ вершин.
    \begin{example}
    $n = 11 = 1011_2 \Rightarrow T_0 + T_1 + T_3$
    \end{example}
  \item [2) ] Дерево ранга $k$ имеет глубину $k$
    \[
    k \leq \log_2{n}
    \]
\end{itemize}
\end{note}
Реализация:
\begin{itemize}
  \item getMin(): \\
    Храним указатель на корень с наим. значением. $ \Rightarrow O(1)$
  \item merge($H_1$, $H_2$): \\
    \begin{itemize}
      \item [1) ] Если в $H_1$ и $H_2$ не содержатся деревья одинаковых рангов, то просто объединяем.
      \item [2) ] Иначе пусть есть дерево $L_k, R_k$ - два дерева одинакового ранга. Сделаем из них $T_{k + 1}$. Повторяем процедуру, пока у нас есть деревья равных рангов. ($O(\log_2 n)$)
    \end{itemize}
  \item insert(x): \\
    Заводим биномиальную кучу из одной вершины с значением $x$, затем merge новой и старой кучи $\Rightarrow O(\log_2 n)$
  \item extractMin(): \\
    Пусть min вершина в $H_2$. На самом деле дерево $H_2$ - тоже корректная куча. Оставшуюся кучу обозначим за $H_1$. Удалим из $H_2$ min, из оставшихся деревьев составим новую кучу $H_2'$ и смёрджим его с $H_1$
  \item decreaseKey(ptr, $\triangle$): \\
    Как в бинарной. ($O(\log_2 n)$) + Проверить, не изменился ли min корень
\end{itemize}

\subsection{Амортизационный анализ}
\begin{definition}
Пусть $S$ - какая-то СД, способная обрабатывать $m$ типов запросов. Тогда ф-ции $a_1(n), a_2(n), \ldots, a_m(n)$ наз-ся учётными (амортизационными) асимптотиками ответов на запросы, если $\forall n \forall$ п-ть из $n$ запросов с типами $i_1, i_2, \ldots, i_n$ суммарное время их обработки = $O(\sum_{i = 1}^{n}a_{i_j}(n))$
\end{definition}
\begin{example}
В бинарной куче:
\begin{itemize}
  \item insert: $O(\log n)$
  \item extractMin: $O^{*}(\log n)$
  \item getMin(): $O^{*}(\log n)$
  \item erase: аморт. $O(\log n)$
\end{itemize}
Сл-но, любые $n$ запросов работают за $O(n\log n)$
\end{example}
\begin{note}
Можно даже считать так:
\begin{itemize}
  \item insert: $O^{*}(\log n)$
  \item extractMin: $O^{*}(1)$ $\leq k$
  \item getMin: $O^{*}(1)$
  \item erase: $O^{*}(1)$ $\leq k$
\end{itemize}
На $n$ запросов. \\
Из них $k$ - insert. Тогда реальное время работы: $O(k\log k + n - k)$
\end{note}
\subsubsection{Динамический массив \\ (std::vector)}
Хранит массив: $a_0, a_1, \ldots, a_{n - 1}$ \\
Отвечает на запросы:
\begin{itemize}
  \item $[]$: по $i$ вернуть $a_i$ - $O(1)$
  \item push-back x: добавить $x$ в конец массива.
  \item pop-back: удалить последний эл-т.
\end{itemize}
