\section{Лекция 2}

\subsection{Булевы функции}

Булевы значения: $\{0, 1\}$

Булева ф-ция от $k$ переменных $f: \{0, 1\}^{k} \rightarrow \{0, 1\}$

$\Rightarrow f$ принимает на вход $2^{k}$ различных кортежей. Каждому кортежу может быть сопоставлено 2 значения $\Rightarrow$. 

Общее число ф-ций - $2^{2^{k}}$

\begin{example}
$k = 1 \Rightarrow 2^{2^{k}} = 4$
\begin{center}
\begin{tabular}{ |c|c|c|c|c| } 
 \hline
 $p$ & $\perp$ & $p$ & $\neg p$ & $T$\\
 \hline
 0 & 0 & 0 & 1 & 1 \\
 1 & 0 & 1 & 0 & 1 \\
 \hline
\end{tabular}
\end{center}
\end{example}
\begin{example}
$k = 0 \Rightarrow 2^{2^{0}} = 2$ 
2 ф-ции:
\begin{equation*}
    \begin{system_or}
        f(\varepsilon) = 0 (\perp)\\
        f(\varepsilon) = 1 (T) 
    \end{system_or}
\end{equation*}
\end{example}
\begin{example}
$k = 2 \Rightarrow 2^{2^{2}} = 16$
\begin{center}
\begin{tabular}{ |c|c|c|c|c|c|c|c|c|c|c|c|c|c|c|c| } 
 \hline
 $p$ & $q$ & $\perp$ & $T$ & $p = pr_1$ & $q = pr_2$ & $\neg p$ & $\neg q$ & $\land$ & $\lor$ & $\oplus $ & $p \rightarrow q$ & $q \rightarrow p$ & $\leftrightarrow$ & $\rightarrow$ & $\leftarrow$ \\
 \hline
 0 & 0 & 0 & 1 & 0 & 0 & 1 & 1 & 0 & 0 & 0 & 1 & 1 & 1 & 0 & 0\\
 \hline
 0 & 1 & 0 & 1 & 0 & 1 & 1 & 0 & 0 & 1 & 1 & 1 & 0 & 0 & 0 & 1\\
 \hline
 1 & 0 & 0 & 1 & 1 & 0 & 0 & 1 & 0 & 1 & 1 & 0 & 1 & 0 & 1 & 0\\
 \hline
 1 & 1 & 0 & 1 & 1 & 1 & 0 & 0 & 1 & 0 & 0 & 1 & 1 & 1 & 0 & 0\\
 \hline
   &   &   &   &   &   &  &  & min & max & xor ($\neq$) & $\leq$& $\geq$ & $=$ & \\
 \hline
\end{tabular}
\end{center}
\begin{center}
\begin{tabular}{ |c|c| } 
 \hline
 $\downarrow$ & $\uparrow$ \\ 
 \hline
 1 & 1 \\
 \hline
 0 & 1 \\
 \hline
 0 & 1 \\
 \hline
 0 & 0 \\
 \hline
 Стрелка Пирса (NOR) & Штрих Шеффера (NAND) \\
 \hline
\end{tabular}
\end{center}
\end{example}

\begin{symb}
$k > 2$, $\land_k, \lor_k, \oplus_k$, ($\oplus_k$ - ф-ция чётности (PARITY))
\end{symb}
\begin{symb}
\begin{equation*}
maj(p, q, r) =
\begin{system_and}
1, p + q + r \geq 2 \\
0, p + q + r \leq 1 
\end{system_and}
\end{equation*}
Функция большинства

$maj_{2k + 1}$ - задаётся аналогичным образом
\end{symb}
Пороговые функции:
\begin{equation*}
thr_{k, n}(p_1, \ldots , p_n) = 
\begin{system_and}
1, \sum_{i = 1}^{n} p_i \geq k \\
0, \text{иначе}
\end{system_and}
\end{equation*} 

\textbf{Тернарный оператор:}
\begin{equation*}
    p ? q \colon r = 
    \begin{system_and}
   q, p = 1 \\
   r, p = 0
    \end{system_and}
\end{equation*}

\subsection{Пропозициональные ф-лы $\leftrightarrow$ Булевы ф-ции}

\begin{itemize}
    \item 
Переход $\Longrightarrow$: \underline{Вычисление} (По табл. истинности)
    \item 
Переход $\Longleftarrow$ : \underline{Представление}
\end{itemize}

\begin{example}
    $((p \land q) \lor (r \rightarrow \neg s)) \iff $ Дерево разбора

    Листья дерева $ = $ значения перменных
\end{example}

\textbf{Правила вычисления знач. ф-лы:}
\begin{symb}
~\newline

    $p_1, p_2, \ldots , p_n$ - переменные.

    $a_1, a_2, \ldots, a_n$ - значения переменных (0/1)

    $[\phi](a_1, a_2, \ldots, a_n)$ - знач. ф-лы $\phi$ на арг-тах ($a_1, a_2, \ldots, a_n$)
\end{symb}
\begin{definition}
\begin{enumerate}
    \item [1) ] $[p_i](a_1, a_2, \ldots, a_n) = a_i$
    \item [2) ] $[\neg \psi](a_1, a_2, \ldots, a_n) = neg([\psi](a_1, \ldots , a_n))$
        \begin{itemize}
            \item $\neg$ - символ из ф-лы
            \item neg - булева ф-ция
        \end{itemize}
    \item [3) ] $[(\eta \land \xi)](a_1, a_2, \ldots, a_n) = and([\eta](a_1, a_2, \ldots, a_n), [\xi](a_1, a_2, \ldots, a_n))$ 

        ($\lor$ - or, $\rightarrow$ - implies)

\end{enumerate}
\textbf{Булева ф-ция получается из пропоз. ф-лы, если провести вычисл. для всех ($a_1, a_2, \ldots, a_n$)}
\end{definition}
\begin{definition}
\textbf{Литерал} - перменная или отрицание переменной. ($p$, $\neg q$)
\end{definition}
\begin{definition}
  \textbf{Конъюнкт} - конъюнкция литералов ($p \land \neg q \land r$)
\end{definition}
\begin{definition}
    \textbf{Дизъюнкт} - дизъюнкция литералов ( $p \lor \neg q \lor r$)
\end{definition}
\begin{definition}
    \textbf{Конъюнктивная нормальная форма (КНФ)} - конъюнкция дизъюнктов ($(\neg p \lor \neg q \lor r) \land (q \lor \neg s)$)
\end{definition}
\begin{definition}
    \textbf{Дизъюнктивная нормальная форма (ДНФ)} - дизъюнкция конъюнктов ($(p \land \neg q \land r) \lor (\neg p \land s)$)
\end{definition}
\begin{theorem}
Любая булева ф-ция выразима как КНФ и как ДНФ
\end{theorem}
\begin{center}
    \begin{tabular}{ |c|c|c|c|c|c| } 
 \hline
 p & q & r & Значения ф-ции & ДНФ & КНФ \\
 \hline
 0 & 0 & 0 & 0 & & $(p \lor q \lor r) \land$ \\
 \hline
 0 & 0 & 1 & 1 & $(\neg p \land \neg q \land \neg r) \lor$ & \\
 \hline
 0 & 1 & 0 & 1 & $(\neg p \land q \land \neg r) \lor$ &\\
 \hline
 0 & 1 & 1 & 0 & & $(p \lor \neg q \lor \neg r) \land$\\
 \hline
 1 & 0 & 0 & 0 & & $(\neg p \lor q \lor r) \land$\\
 \hline
 1 & 0 & 1 & 0 & & $(\neg p \lor q \lor \neg r) \land$\\
 \hline
 1 & 1 & 0 & 1 & $(p \land q \land \neg r) \lor$ &\\
 \hline
 1 & 1 & 1 & 0 & & $(\neg p \lor \lor \neg q \lor \neg r) \land$\\
 \hline
\end{tabular}
\end{center}
\begin{example}
\[
f \equiv 0 \Rightarrow f = p \land \neg p
\] 
\end{example}
