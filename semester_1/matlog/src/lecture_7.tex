\section{Лекция 7}
\begin{theorem}[О полноте ИВ]
$\phi$ - тавтология $\Rightarrow$ $\phi$ выводима
\end{theorem}
Правило исчерп. разбора случаев:
Пусть $\Gamma$ - нек-рое мн-во ф-ул, при это $\Gamma, A \vdash B$ и $\Gamma, \neg A \vdash B$ \\
Тогда: $\Gamma \vdash B$ \\
\begin{center}
\begin{tabular}{ c c c } 
  $\Gamma, A \vdash B$ & & $\Gamma, \neg A \vdash B$ \\
 \hline
                       & $\Gamma \vdash B$ & \\
\end{tabular}
\end{center}
\begin{symb}
\[
p ^{\varepsilon} \eqcirc \begin{cases}
p, \varepsilon = 1 \\
\neg p, \varepsilon = 0
\end{cases}
\]
\end{symb}
\begin{lemma}[Основная]
Пусть $\phi$ - ф-ла от $n$ переменных ($\overline{p} = (p_1, \ldots, p_n)$).
\[
  (a_1, \ldots, a_n) \in \set{0, 1}^{n}, \phi(a_1, \ldots, a_n) = a \in \set{0, 1}
\]
\end{lemma}
Тогда:
\[
p_1^{a_1}, p_2^{a_2}, \ldots, p_n^{a_n} \vdash \phi^{a}
\]
Рассм. переход: \\
ОСНОВНАЯ ЛЕММА $\Rightarrow$ ТЕОРЕМА О ПОЛНОТЕ ИВ \\
\[
\phi \text{ - тавтология} \Rightarrow \text{ при всех } (a_1, \ldots, a_n)
\]
\[
\phi(a_1, \ldots, a_n) = 1 \underset{\text{По лемме}}{\Longrightarrow} p_1^{a_1}, \ldots, p_n^{a_n} \vdash \phi
\]
\begin{example}
$n = 3$:
le Picture
\end{example}
\begin{lemma}[Базовая]
  ~\newline
  AND-ы:
\[
A, B \vdash A \land B
\]
\[
\neg A, B \vdash \neg(A \land B)
\]
\[
A, \neg B \vdash \neg(A \land B)
\]
\[
\neg A, \neg B \vdash \neg(A \land B)
\]
  OR-ы:
  \[
  A, B \vdash A \lor B
  \]
  \[
  \neg A, B \vdash A \lor B
  \]
  \[
  A, \neg B \vdash A \lor B
  \]
  \[
  \neg A, \neg B \vdash \neg(A \lor B)
  \]
  Implication-ы:
  \[
  A, B \vdash A \rightarrow B
  \]
  \[
  \neg A, B \vdash A \rightarrow B
  \]
  \[
  A, \neg B \vdash \neg (A \vdash B)
  \]
  \[
  \neg A, \neg B \rightarrow A \rightarrow B
  \]
  И ещё: 
  \[
  \neg A \vdash \neg A
  \]
  \[
  A \vdash \neg(\neg A)
  \]
\end{lemma}
\begin{proof}[Док-во основной леммы]
  Инд-ция по построению ф-лы:
  \begin{itemize}
    \item [База)] Переменная: $p_i^{a_i} \vdash p_i^{a_i}$
    \item [Переход) ] Пусть, например:
      \[
      \phi \eqcirc (\xi \land \eta)
      \]
      \[
      \xi(a_1, \ldots, a_n) = a, \eta(a_1, \ldots, a_n) = b \Rightarrow \phi(a_1, \ldots, a_n) = a \cdot b
      \]
      По предположению индукции:
      \[
      p_1^{a_1}, p_2^{a_2}, \ldots, p_n^{a_n} \vdash \xi^{a} \text{ и } p_1^{a_1}, p_2^{a_2}, \ldots, p_n^{a_n} \vdash \eta^{b}
      \]
      По базовой лемме:
      \[
      \xi^{a}, \eta^{b} \vdash \phi^{a\cdot b}
      \]
      Запишем эти 3 вывода (\textbf{подряд}):
      \[
      p_1^{a_1}, \ldots, p_n^{a_n} \phi^{a \cdot b}
      \]
  \end{itemize}

\end{proof}
\begin{proof}[Другое док-во]
  Пусть $\Gamma$ - мн-во пропозициональных ф-л.
  \begin{definition}
  $\Gamma$ \textbf{совместно}, если при некот. значениях переменных все ф-лы из $\Gamma$ истинны.
  \end{definition}
  \begin{definition}
  $\Gamma$ - \textbf{противоречиво}, если для некот. ф-лы $\phi$ верно:
  \[
  \begin{cases}
  \Gamma \vdash \phi \\
  \Gamma \vdash \neg \phi
  \end{cases}
  \]
  \begin{theorem}
  $\Gamma$ совместна $\overset{*}{\iff} $ $\Gamma$ непротиворечива.
  \end{theorem}
  \end{definition}
  Рассмотрим связь теоремы о совм. и непрот. с теор. о корр. и полн.:
  \begin{theorem}[О корректности]
  \[
  \vdash \phi \Rightarrow \set{\neg \phi} \text{ - противор.} \overset{*}{\Longrightarrow} \set{\neg \phi} \text{ - несовм.} \Rightarrow \forall a, \neg \phi(a) = 0 \iff \phi(a) = 1 \Rightarrow
  \] 
  \[
  \Rightarrow \phi \text{ - тавтология}
  \]
  \end{theorem}
  \begin{theorem}[О полноте]
  \[
  \phi \text{ - тавтология} \Rightarrow \set{\neg \phi} \text{ - несовм.} \overset{*}{\Rightarrow} \set{\neg\phi} \text{ - противоречиво} \Rightarrow \neg\neg \phi \vdash \phi
  \] 
  \begin{center}
  \begin{tabular}{ c c c } 
    $\neg \phi \vdash B$ & & $\neg \phi \vdash \neg B$ \\
   \hline
                       & $\vdash \neg \neg \phi$ & 
  \end{tabular}
  \end{center}
  \end{theorem}
\end{proof}
\begin{proof}
\begin{itemize}
  \item [1) ] $\Gamma$ против. $\Rightarrow$  $\Gamma$ несовм.
    \begin{theorem}[Обобщённая теорема о корректности]
      Если $\Gamma \vdash A$ и все ф-лы из $\Gamma$ верны на $(a_1, \ldots, a_n)$, то и $A$ верна на том же наборе. 
    \end{theorem}
    Д-во: индукция по номеру ф-лы в выводе.
    \[
    \Gamma \text{ - совм.} \Rightarrow \text{ Все ф-лы из $\Gamma$ верны на нек-ром наборе. }
    \]
    \[
    \Gamma \vdash \phi \Rightarrow \phi \text{ верно на том же наборе}
    \]
    \[
    \Gamma \vdash \neg\phi \Rightarrow \neg\phi \Rightarrow -----||-----
    \]
    Но $\phi$ и $\neg\phi$ не м. б. верны одновременно. Противор.
  \item [2) ] $\Gamma$ непрот. $\Rightarrow$ $\Gamma$ совм. \\
    Пусть $\triangle$ непрот. Будем говорить, что $\triangle$ - полное, если для $\forall \phi$ верно $\triangle \vdash \phi$ или $\triangle \vdash \neg \phi$.
    \begin{lemma}[I]
    $\Gamma$ непрот $\Rightarrow \Gamma \subset \triangle$ для некот. полного непрот. $\triangle$
    \end{lemma}
    \begin{lemma}[II]
     $\triangle$ полное, непрот. $\Rightarrow \triangle$ - совм.
    \end{lemma}
    \begin{proof}[Док-во леммы $I$ для счётного мн-ва перемен.]
      Если переменных сч. мн-во то и ф-лы тоже. \\
      Пусть $\phi_1, \phi_2, \ldots, \phi_n$ - все ф-лы. \\
      Опр. $\Gamma_i$ по инд-ции:
      \[
      \Gamma_0 = \Gamma, \Gamma_i = \begin{cases}
      \Gamma_{i - 1} \cup \set{\phi_i}, \text{ - если это непрот.} \\
      \Gamma_{i - 1} \cup \set{\neg\phi_i} \text{ - иначе}
      \end{cases}
      \]
      \begin{statement}
      Все $\Gamma_i$ - непрот.
      \begin{proof}
      \[
        \begin{cases}
      \Gamma_{i - 1} \cup \set{\phi_i} \text{ - прот. } \Rightarrow \Gamma_{i - 1} \vdash \neg\phi_i \\
      \Gamma \cup \set{\neg\phi_i} \text{ - прот. } \Rightarrow \Gamma_{i - 1} \vdash \neg\neg \phi_i
        
        \end{cases}
      \]
      $\Rightarrow \Gamma_{i - 1}$ - прот. $\Rightarrow$ пришли к противоречию.
      
      \end{proof}
      \end{statement} 
      \[
        \Gamma_0 \subset \Gamma_1 \subset \Gamma_2 \subset \ldots
      \]
      \[
      \triangle = \bigcup_{i = 0}^{\infty} \Gamma_i \text{ - тоже непрот.}
      \]
      Если $\triangle$ прот., то прот. использ. кон. число ф-л из $\triangle$. Каждое $\delta_j$ лежит в $\Gamma_{k_j}$. Тогда прот. выв-ся из $\Gamma_{max\set{k_j}}$. Но все конечные $\Gamma_i$ непрот.
      \begin{proof}[Док-во леммые II]
      $\triangle$ - полн. $\Rightarrow$ для перем. $p_i$, $\triangle \vdash p_i \lor \triangle \vdash \neg p_i$. \\
      Набор. значений:
      \begin{equation}
        \label {sys:*}
        p_i = \begin{cases}
        1, \triangle \vdash p_i \\
        0, \triangle \vdash \neg p_i
        \end{cases}
      \end{equation}
      Д-м, что ф-лы из $\triangle$ верны на системе $(\ref{sys:*})$. Ф-ла - перем. $\Rightarrow$ согл. опр. системы $(\ref{sys:*})$:
      \[
      \phi \eqcirc \neg \psi
      \]
      \end{proof}
      Более общ утв.:
      \[
        \begin{cases}
          \triangle \vdash \phi \Rightarrow \phi \text{ верна на системе ($\ref{sys:*}$)} \\
          \triangle \not\vdash \phi \Rightarrow \phi \text{ - неверна на системе ($\ref{sys:*}$)} 
        \end{cases}
        \]
    \end{proof}
\end{itemize}
\end{proof}
