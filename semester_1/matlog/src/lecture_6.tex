\section{Лекция 6}
\begin{definition}
\textbf{Вывод} - п-ть $\phi_1, \ldots, \phi_n$, т. ч. $\forall i\colon$
\begin{itemize}
  \item $\phi_i$ - аксиома
  \item $\phi_i$ - получается по правилам MP из $\phi_i, \phi_k, j < i, k < i$. \\
    Это значит, что $\phi_k \eqcirc \phi_j \rightarrow \phi_i$
\end{itemize}
Ф-ла \textbf{выводима} $(\vdash \phi)$, если $\phi$ встреч-ся в нек-ром выводе.
\end{definition}
\begin{theorem}
$\phi$ - тавтология $ \Rightarrow (\vdash \phi)$ 
\end{theorem}
\begin{example}
\[
  (\neg A \lor B) \rightarrow (A \rightarrow B)
\]
\begin{itemize}
  \item [1)] \[
  \neg A \rightarrow (A \rightarrow B) \text{ аксиома 9}
  \]
\item [2) ] \[
  B \rightarrow (A \rightarrow B) \text{ - аксиома 9}
\]
\item [3) ] \[
  (\neg A \rightarrow (A \rightarrow B)) \rightarrow ((B \rightarrow (A \rightarrow B)) \rightarrow ((A \lor B) \rightarrow (A \rightarrow B)))
\]
\item [4) ] \[
  (B \rightarrow (A \rightarrow B)) \rightarrow ((\neg A \lor B) \rightarrow (A \rightarrow B)) \text{ - MP 1, 3}
\]
\item [5) ] \[
  (\neg A \lor B) \rightarrow (A \rightarrow B)
\]
\end{itemize}
\end{example}
\begin{definition}
Вывод из мн-ва посылок $\Gamma$ - это п-ть $\phi_1, \phi_2, \ldots, \phi_n$ при этом $\phi_i$ может быть либо аксиомой, либо эл-т $\Gamma$, либо получается по m. p.
\end{definition}

\begin{lemma}[О дедукции]
\[
\Gamma \vdash A \rightarrow B \iff \Gamma \cup \set{A} \vdash B
\]
\end{lemma}
\begin{example}[Силлогизм]
\[
 \vdash (A \rightarrow B) \rightarrow ((B \rightarrow C) \rightarrow (A \rightarrow C)) \iff
\]
\[
\iff \set{A \rightarrow B} \vdash (B \rightarrow C) \rightarrow (A \rightarrow C)
\]
\[
\iff \set{A \rightarrow B, B \rightarrow C} \vdash (A \rightarrow C)
\]
\[
\iff \set{A, A \rightarrow B, B \rightarrow C} \vdash C
\]
\begin{itemize}
  \item [1) ] $A$ - посылка
  \item [2) ] $A \rightarrow B$ - посылка
  \item [3) ] $B$ по MP 1, 2
  \item [4) ] $B \rightarrow C$ - посылка
  \item [5) ] $C$ - MP 3, 4
\end{itemize}
\end{example}
\begin{proof}
\begin{itemize}
  \item [$\Rightarrow)$] Если вывели $A \rightarrow B$, то из $\Gamma \cup \set{A}$ можно вывести $B$ по MP
  \item [$\Leftarrow)$] Пусть $\Gamma \cup \set{A} \vdash B$. Тогда сущю вывод $\phi_1, \ldots, \phi_n \eqcirc B$ из $\Gamma \cup \set{A}$ \\

    Каждый $\phi_i$ - либо акс., либо $\in \Gamma$, либо $= A$, либо вывод-ся по MP. Мы докажем по инд-ции, что $\Gamma \vdash A \rightarrow \phi_i$:
    \begin{itemize}
      \item [1) ] $\phi_i$ - акс.
        \begin{itemize}
          \item [1) ] $\phi_i$
          \item [2) ] $\phi_i \rightarrow (A \rightarrow \phi_i)$ - A1
          \item [3) ] $A \rightarrow \phi_i$, MP 1, 2.
        \end{itemize}
      \item [2) ] $\phi_i \in \Gamma$, аналогичен (1)
      \item [3) ] $\phi_i \eqcirc A$. На прошлой лекции выводили $\vdash A \rightarrow A$ без $\Gamma$
      \item [4) ] $\phi_i$ по MP: $\exists j, k, < i$:
        \[
        \phi_k \eqcirc (\phi_j \rightarrow \phi_i)
        \]
        По инд-ции: $\Gamma \vdash A \rightarrow \phi_j, \Gamma \vdash A \rightarrow \phi_k$, т. е. $\Gamma \vdash A \rightarrow (\phi_j \rightarrow \phi_i)$:
        \[
          (A \rightarrow (\phi_j \rightarrow \phi_i)) \rightarrow ((A \rightarrow \phi_j) \rightarrow (A \rightarrow \phi_i)) \text{ - A2}
        \]
        \[
          (A \rightarrow \phi_j) \rightarrow (A \rightarrow \phi_i) \text{ - MP}
        \]
        \[
          (A \rightarrow \phi_i) \text{ - MP}
        \]
    \end{itemize}
\end{itemize}
\end{proof}
\begin{example}
\[
\vdash (A \land B) \rightarrow (B \land A)
\]
\[
A \land B \vdash B \land A
\]
\begin{itemize}
  \item [1) ] $A \land B$ - посылка
  \item [2) ] $(A \land B) \rightarrow B$ - акс. 4
  \item [3) ] $B$ - MP 1, 2
  \item [4) ] $(A \land B) \rightarrow A$ - акс. 3
  \item [5) ] $A$ - MP 1, 4
  \item [6) ] $(B \rightarrow (A \rightarrow (B \land A)))$ - акс. 5
  \item [7) ] $A \rightarrow (B \land A)$ - MP 3, 6
  \item [8) ] $B \land A$ - MP 5, 7
\end{itemize}
\end{example}
\begin{lemma}[Правила введения и разбиения конъюнкции]
\[
\Gamma \cup \set{A \land B} \vdash C
\]
\[
\iff \Gamma \cup \set{A, B} \vdash C
\]
Также:
\[
\Gamma \vdash A \land B \iff \begin{cases}
\Gamma \vdash A \\
\Gamma \vdash B
\end{cases}
\]
\end{lemma}
\begin{example}
  \[
    (A \rightarrow \neg A) \rightarrow \neg A 
  \]
  Вывод: 
  \begin{itemize}
    \item [1-5)] $A \rightarrow A$
    \item [6)] $(A \rightarrow A) \rightarrow ((A \rightarrow \neg A) \rightarrow \neg A)$ - A10
    \item [7) ] $(A \rightarrow \neg A) \rightarrow \neg A$ - MP 5,6
  \end{itemize}
\end{example}
\begin{example}
\[
\vdash A \rightarrow \neg \neg A
\]
\[
\iff A \vdash \neg \neg A
\]
\[
  \vdash \neg A \rightarrow (A \rightarrow B) \iff
\]
\[
  \neg A \vdash A \rightarrow B \iff \neg A, A \vdash B \iff A \vdash \neg A \rightarrow B
\]
\[
  \vdash A \rightarrow (\neg A \rightarrow B)
\]
\[
  A \vdash \neg \neg A
\]
\begin{itemize}
  \item [1) ] $A \rightarrow (\neg A \rightarrow B)$
  \item [2) ] $A$ - посылка
  \item [3) ] $\neg A \rightarrow B$, mp 2, 1
  \item [4) ] $A \rightarrow (\neg A \rightarrow \neg B)$
  \item [5) ] $\neg A \rightarrow \neg B$, MP 2, 4
  \item [6) ] $(\neg A \rightarrow B) \rightarrow ((\neg A \rightarrow \neg B) \rightarrow \neg \neg A)$ - A10
  \item [7) ] $(\neg A \rightarrow \neg B) \rightarrow \neg \neg A$ - MP 3, 6
  \item [8) ] $\neg \neg A$ - MP 5, 7
\end{itemize}
\end{example}
\begin{lemma}[Правило рассуждения от противного]
  ~\newline
\begin{center}
\begin{tabular}{ c c c } 
  $\Gamma, A \vdash B$ & & $\Gamma, A \vdash \neg B$ \\
 \hline
                       & $\Gamma \vdash \neg A$ & 
\end{tabular}
\end{center}
\end{lemma}
\begin{proof}
\[
  \begin{cases}
\Gamma, A \vdash B \iff \Gamma \vdash A \rightarrow B \\
\Gamma, A \vdash \neg B \iff \Gamma \vdash A \rightarrow \neg B  
  \end{cases} \iff \Gamma \vdash \neg A  \text{, А10 + MP x2}
\]
\end{proof}
\begin{example}[Закон контрапозиции]
\begin{center}
\begin{tabular}{ c c c } 
  $A \rightarrow B, \neg B, A \vdash B$ & & $A \rightarrow B, \neg B, A, \vdash \neg B$\\
 \hline Рассуждение от противного
 & $A \rightarrow B, \neg B \vdash \neg A$ &  \\
 \hline ЛОД x2
 & $\vdash (A \rightarrow B) \rightarrow (\neg B \rightarrow \neg A)$ & 
\end{tabular}
\end{center}
\end{example}
\begin{example}[Закон Де Моргана]
\[
\vdash (\neg A \lor \neg B) \rightarrow \neg (A \land B)
\] 
\[
\iff (\neg A \lor \neg B) \vdash A \land B
\]
\begin{itemize}
  \item [1) ] $(A \land B) \rightarrow A$ - акс. 3
  \item [2) ] $\neg A \rightarrow \neg(A \land B)$ - закон контрапозиции.
  \item [3) ] $(A \land B) \rightarrow B$ - акс. 4
  \item [4) ] $\neg B \rightarrow \neg(A \land B)$ - контрапозиция
  \item [5) ] $(\neg A \rightarrow \neg(A \land B)) \rightarrow ((\neg B \rightarrow \neg (A \land B)) \rightarrow ((\neg A \lor \neg B) \rightarrow \neg (A \land B)))$
  \item [6)] MP 2x
\end{itemize}
\end{example}
\begin{lemma}[Правило контрапозиции]
\begin{center}
\begin{tabular}{ c} 
  $\Gamma, A \vdash B$ \\
 \hline
  $\Gamma, \neg B \vdash \neg A$
\end{tabular}
\end{center}
\end{lemma}
\begin{lemma}[Правило разбора случаев]
  \begin{center}
  \begin{tabular}{ c c c } 
    $\Gamma, A \vdash C$ & & $\Gamma, B \vdash C$ \\
   \hline
                         & $\Gamma, A \lor B \vdash C$ &\\
  \end{tabular}
  \end{center}
\end{lemma}
\begin{lemma}[Правило исчерп. разбора случаев]
\begin{center}
\begin{tabular}{ c c c } 
  $\Gamma, A \vdash B$ & & $\Gamma, \neg A \vdash B$ \\
 \hline
                       & $\Gamma \vdash B$ & \\
\end{tabular}
\end{center}
\end{lemma}
\begin{example}
\begin{center}
\begin{tabular}{ c c c } 
  $\neg \neg A, A \vdash A$ & & $\neg \neg A, \neg A \vdash A$ \\
  \hline
                            & $\neg \neg A \vdash A$ & \\
 \hline
                            & $\vdash \neg \neg A \rightarrow A$ &\\
\end{tabular}
\end{center}
\end{example}
