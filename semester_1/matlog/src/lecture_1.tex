
\section{Инфа}
\textbf{Лектор}: Мусатов \\
\textbf{Книги}: Верещагин Н. К., Шень А. "Лекции по мат. логике":
\begin{enumerate}
    \item [№ 1] Начало теории мн-в
    \item [№ 2] Языки и исчисления
    \item [№ 3] Вычислимые ф-ции
\end{enumerate}

\section{Синтаксис $\leftrightarrow$ Семантика}
\begin{definition}
\textbf{Синтаксис} - правила составления форм. выр-ий.
\end{definition}
\begin{definition}
\textbf{Семантика} - соспоставление форм выр-ия некоторого смысла.
\end{definition}
\begin{definition}
\textbf{Алфавит} - мн-во символов. (Непустое, обычно конечное)
\end{definition}
\begin{definition}
\textbf{Слово} - конечная последовательность символов алфавита. (Может быть пустым)

\textbf{Пустое слово} -  $\varepsilon$
\end{definition}
\begin{definition}
\textbf{Язык} - любое мн-во слов.

\textbf{Пустой язык} - $\emptyset$
 
\textbf{Синглетон} - $\{\varepsilon\}$
\end{definition}

\textbf{Операции} над словами:
\begin{itemize}
    \item Конкатенация: $u * v$
    \item Возведение в степень:  $u^{n} = u * u * \cdots * u$ - n раз ($u^{0} = \varepsilon$)
    \item Обращение: $u^{R} = u_n u_{n - 1} \cdots  u_1, \text{ если } u = u_1 u_2 \cdots u_n$
        \[
            (ab)^{R} = b^{R}a^{R}
        .\] 
\end{itemize}

\textbf{Отношения} над словами:
\begin{itemize}
    \item Префикс $u \sqsubset v \iff \exists w \colon uw = v$ 
    \item Суффикс $u \sqsupset v \iff \exists w \colon wu = v$
    \item Подслово $u (\text{subset}) v \iff \exists t, w \colon tuw = v$
    \item Подп-ть $u \subset v \iff$ вычеркнута часть символов $v$ и получили $u$
\end{itemize}

\textbf{Операции} над языками:
\begin{enumerate}
    \item [0) ] Теоретико-множ.
    \item [1) ] Конкатенация:
        \[
        L * M = \{u * v | u \in L, v \in M\}
        .\] 
        \[
        L * \emptyset = \emptyset
        .\] 
        \begin{example}
        \[
        L = \{a, ab\}, M = \{a, ba\}, LM = \{aa, aba, abba\}
        .\] 
        \end{example}

    \item [2) ] $L^{n} = L * L * \cdots * L$ - $n$ раз
        \[
        L^{0} = \{\varepsilon\} 
        .\] 
    \item [3) ] Итерация/Звезда Клини:
        \[
        L^{*} = L^{0} \cup L^{1} \cup L^{2} \cup \cdots = \bigcup_{k = 0}^{\infty} L^{k} 
        .\] 
        \[
        L^{+} = \bigcup_{k = 1}^{\infty} L^{k} = L^{*} * L
        .\] 
        \[
        L^{*} = L^{+} * \{\varepsilon\} 
        .\] 
        
\end{enumerate}

\section{Правильные скобочные п-ти (ПСП)}

\begin{definition}
    \textbf{ПСП} - это п-ть скобок, разбитых на пары, и в каждой паре "(" \textbf{раньше} ")".
\end{definition}
\begin{definition}
    \textbf{ПСП} - это п-ть, получ. из правил:
    \begin{enumerate}
        \item $\varepsilon$ - это ПСП;
        \item $s$ - ПСП $\Rightarrow (s)$ - ПСП;
        \item $s, t$ -  ПСП, $\Rightarrow st$ - ПСП.
    \end{enumerate}
\end{definition}
\begin{definition}
\textbf{Баланс СП} - (кол-во "(") - (кол-во ")")
\end{definition}
\begin{definition}
    \textbf{ПСП} - СП, для кот. баланс всей п-ти $ = 0$, а любого др. префикса $ \geq 0$
\end{definition}

\subsection{ОПР 1 $\Rightarrow$ ОПР 3}
Все скобки разбиты на пары $\Rightarrow$ баланс $ = 0$.

 "(" левее ")" $\Rightarrow$ в любом префиксе из каждой пары, ни одной, обе или только "(". В любом случае итоговый баланс префикса $\geq 0$.

\subsection{ОПР 2 $\Rightarrow$ ОПР 1}

Скобки, добавленные по правилу $(s)$, будут в паре.

\subsection{ОПР 3 $\Rightarrow$ ОПР 2}

Д-во: индукция по длине СП

\textbf{База:} $s = \varepsilon \Rightarrow $ подх. по опр. 2

\textbf{Осн. случ.:} $|s| > 0 \Rightarrow $ первый символ "(".

Рассм. кратчайший непустой префикс с балансом  $ = 0$:
\begin{itemize}
    \item [Случай 1: ] Это вся п-ть: $ s = (s') \Rightarrow $ для $s'$ верно ОПР 3 (т. к. любой другой баланс по случаю $ \geq 1$) $\Rightarrow$ и ОПР 2.
    \item [Случай 2: ] Это собств. префикс ($\neq \text{всей строке}$): $s = (s')t$. И для $s'$, и для $t$ - выполнено ОПР 3 $ \Rightarrow $ ОПР 2.
\end{itemize}

