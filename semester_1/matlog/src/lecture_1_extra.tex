\section{Лекция 1++}

\subsection{Синтаксис $\leftrightarrow$ Семантика}
\begin{center}
\begin{tabular}{ |c|c| }
 \hline
 Синтакис & Семантика \\
 \hline
 Пропозициональные формулы & Булевы ф-ции \\
 Пропозициональные переменные & \\
 Знаки логических действий ($\land, \lor, \rightarrow, \neg$)& \\
 Скобки & \\
 \hline
\end{tabular}
\end{center}

\subsection{Формулы с 1-ой бинарной связкой * \\ (Правильные алгебраические выр-я)}

Рекурсивное правила:
\begin{enumerate}
    \item [1) ] $p$ - переменная $\Rightarrow$ $p$ - ПАВ ( правильное алг. выр-е).
    \item [2) ] $\phi, \psi$ - ПАВ $\Rightarrow (\phi * \psi)$ - ПАВ.
        \begin{example}
            ((a * b) * (c * (d * e)))
        \end{example}
\end{enumerate}

\begin{theorem}
Между ПАВ и деревьями синт. разбора $\exists $ взаимно однозначное соотв. (биекция)
\end{theorem}

Мы докажем: для любого ПАВ $\eta$, не являющегося перменной, $\exists! $ пара ($\phi, \psi$), т. ч. $\eta \eqcirc (\phi * \psi)$

\begin{lemma}[О балансе скобок]
Баланс любого префикса ПАВ $\geq 0$, при этом $= 0$ только для $\varepsilon$ и всего ПАВ.
\end{lemma}
\begin{proof}
    Индукция по построению.
    \begin{itemize}
        \item [База: ] $p$ - переменная $\Rightarrow$ 2 префикса: $\varepsilon$ и $p$, баланс $= 0$
        \item [Переход: ] Пусть для $\phi$ и $\psi$ лемма верна. Докажем для ($\phi * \psi$)

            \begin{center}
            \begin{tabular}{ |c|c| } 
             \hline
             Префиксы & Баланс \\
             \hline
             $\varepsilon$ & 0 \\
             \hline
             $(\phi', \phi' \sqsubset \phi$ & $1 + \text{bal}(\phi') > 0$ \\
             \hline
             $(\phi * \psi', \psi' \sqsubset \psi$ & $1 + 0 + \text{bal}(\psi')> 0$ \\
             \hline
             $(\phi * \psi)$ & 0 \\
             \hline
            \end{tabular}
            \end{center}
    \end{itemize}
\end{proof}
\begin{lemma}
$\phi$ и $\psi$ восстанавливаются однозначно.
\end{lemma}
\begin{proof}
От противного: пусть $(\phi * \psi) \eqcirc (\zeta * \xi)$
\begin{itemize}
    \item [Случай 1) ] $\phi $ - собств. префикс $\zeta$, $\phi \neq \varepsilon$. Тогда в конце $\phi$ баланс $= 0$ (т. к. $\phi$ - ПАВ), и $> 0$ (т. к. $\zeta$ - ПАВ, которое на момент конца $\phi$ не кончилось) $\Rightarrow !!!$ (противоречие)
    \item [Случай 2) ]  $\phi \eqcirc \zeta$. Однако тогда и $\psi \eqcirc \xi$ (сократили одинаковые символы)
\end{itemize}
\end{proof}

Для пропозициональных формул (ПФ):

\textbf{Рекурс. опр.:}
\begin{itemize}
    \item [1) ] $p$ - переменная $\Rightarrow$ $p$ - ПФ 
    \item [2) ] $\phi, \psi$ - ПФ $\Rightarrow$ $(\phi \land \psi), (\phi \lor \psi), (\phi \rightarrow \psi)$ - ПФ.
    \item [3) ] $\phi$ - ПФ $\Rightarrow$ $\neg\phi$ - ПФ
\end{itemize}

\begin{lemma}[О балансе]
Баланс префикса ПФ $\geq 0$, при этом $ = 0$ только для $\varepsilon$, всей ПФ или $\neg\neg\ldots\neg$.
\end{lemma}
\begin{note}
\textbf{Однозначность разбора:} для любой ПФ сущ. единств. правило из (1-3) и единств. сост., из кот. она получ.
\end{note}

