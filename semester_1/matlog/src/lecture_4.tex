\section{Лекция 4}
\subsection{Системы связок}
Бывают двух типов:
\begin{itemize}
  \item Полные (все ф-ции выразимы)
    \begin{example}
    \begin{itemize}
      \item $\set{\neg, \land, \lor, \rightarrow}$
      \item $\set{\neg, \land}$
      \item $\set{\neg, \lor}$
      \item $\set{1, \oplus, \land}$
      \item $\set{\rightarrow, 0}$
        \begin{proof}
          $\neg p = p \rightarrow 0$ \\
          $p \lor q = \neg p \rightarrow q$
        \end{proof}
    \end{itemize}
    \end{example}
  \item Неполные
    \begin{itemize}
      \item $\set{\rightarrow, \land, \lor}$ - сохраняют 1
      \item $\set{\land, \oplus}$ - сохраняют 0
      \item $\set{\land, \lor, 0, 1}$ - монотонность
      \item $\set{\neg, maj_3}$ - самодвойственные ($f(\neg \overline{p}) = \neg f(\overline{p}); \overline{p} = (p_1, p_2, \ldots, p_n)$)
        Иными словами, $f = f^{*} \colon f^{*}(p_1, p_2, \ldots, p_n) = \neg f(\neg p_1, \neg p_2, \ldots, \neg p_n)$
        \begin{example}
        \[
        \land^{*} = \lor, \lor^{*} = \land
        \]
        \[
        \neg(\neg p \land \neg q) = p \lor q
        \]
        \[
        \oplus^{*} = \leftrightarrow
        \]
        \[
        \neg (\neg p \oplus \neg q) = \neg(p \oplus q) = (p \leftrightarrow q)
        \]
        \[
        h(p_1, p_2, \ldots, p_n) = f(g_1(p_1, \ldots, p_n), g_2(p_1, \ldots, p_n), \ldots, g_n(p_1, \ldots, p_n))
        \]
        \[
        h(\neg p_1, \ldots, \neg p_n) = f(g_1(\neg p_1, \ldots, \neg p_2), \ldots, g_n(\neg p_1, \ldots, \neg p_n))
        \]
        \[
        h(\neg p_1, \ldots, \neg p_n) = f(\neg g_1(p_1, \ldots, p_n), \ldots, \neg g_n(p_1, \ldots, p_n))
        \]
        \[
        h(\neg p_1, \ldots, \neg p_n) = \neg f(g_1(p_1, \ldots, p_n), \ldots, g_n(p_1, \ldots, p_n)) = \neg h(p_1, \ldots, p_n)
        \]
        \end{example}
  \item $\set{\oplus, 1}$ - Линейные (Афинные) - ф-ции, задающиеся линейными мн-нами Жегалкина
    \end{itemize}
\end{itemize}
\begin{theorem}[Критерий Поста]
Система связок полна $\iff$ она не является подмн-вом ни одного из 5-ти классов:
\begin{itemize}
  \item $P_0$ - сохр. 0
  \item $P_1$ - сохр. 1
  \item $M$ - монотонные
  \item $D(S)$ - самодвойственные
  \item $L$ - линейные
\end{itemize}
$\iff$ система содержит некот. ф-цию (ф-ции):
\[
  f_0 \not\in P_0, f_1 \not\in P_1, g \not\in M, h \not\in D, R\not\in L
\]
\end{theorem}
\begin{proof}
  ~\newline
  \begin{itemize}
    \item [\underline{Шаг 1} ]
\[
f_0 (0, 0, \ldots, 0) = 1, (\text{т. к. $f_0$ не сохр. 0})
\]
\begin{equation*}
f_0(1, 1, \ldots, 1) = \begin{system_or}
0 \Rightarrow f_0(p, p, p, \ldots, p) = \neg p \\
1 \Rightarrow f_0(p, p, p, \ldots, p) = 1
\end{system_or}
\end{equation*}
\item [\underline{Шаг 2} ] \[
f_1(1, \ldots, 1) = 0
\]
\begin{equation*}
f_1(0, \ldots, 0) = \begin{system_or}
0 \Rightarrow f_1(p, \ldots, p) = 0 \\
1 \Rightarrow f_1(p, \ldots, p) = \neg p
\end{system_or}
\end{equation*}
  \begin{center}
  \begin{tabular}{ |c|c|c| } 
   \hline
   $f_1\backslash f_0$ & $\neg$ & $1$ \\
   \hline
   $\neg$ & шаг 4 & $0 = \neg 1$ \\
   \hline
   0 & $1 = \neg 0$ & шаг 3 \\
   \hline
  \end{tabular}
  \end{center}
  0, 1, $\neg$ $\rightarrow$ шаг 5
  \item [\underline{Шаг 3}] $0, 1, g \not\in M \mapsto \neg$
    \begin{example}
      \[
        \neg p = (p \rightarrow 0)
      \]
      \[
        \neg p = (p \oplus 1)
      \]
      \[
        \neg p = exact_{1, 3}(0, 1, p)
      \]
    \end{example}
\begin{definition}
    \textbf{Монотонная} ф-ция - ф-ция, т. ч.:
    \[
    \forall p_1, q_1, \ldots, p_n, q_n \colon (\forall i \colon (p_i \leq q_i) \rightarrow f(p_1, \ldots, p_n) \leq f(q_1, \ldots, q_n))
    \]
    $\Rightarrow$ ф-ция \textbf{немонот.} $\iff$:
    \[
    \exists p_1, q_1, \ldots, p_n, q_n (\forall i \colon (p_i \leq q_i) \rightarrow g(p_1, \ldots, p_n) = 1 \land g(q_1, \ldots, q_n) = 0)
    \]
\begin{lemma}
$g$ немонотонна $ \Rightarrow$
\[
  \exists i, \exists (a_1, a_2, \ldots, a_{i - 1}, a_{i + 1}, \ldots, a_n) \colon 
\]
\[
g(a_1, \ldots, a_{i - 1}, 0, a_{i + 1}, \ldots, a_n) = 1 \land g(a_1, \ldots, a_{i - 1}, 1, a_{i + 1}, \ldots, a_n) = 0
\]
\end{lemma}
\end{definition}
    Тогда $\neg p = g(a_1, \ldots, a_{i - 1}, p, a_{i + 1}, \ldots, a_n)$

  \item [\underline{Шаг 4}] $\neg, h \not\in D \mapsto 0, 1$
    \[
    h \not\in D \Rightarrow \exists (a_1, \ldots, a_n)
    \]
    \[
    h(a_1, \ldots, a_n) = h(\neg a_1, \ldots, \neg a_n)
    \]
    \begin{example}
    \[
    \neg, \oplus \Rightarrow p \oplus \neg p = 1
    \]
    \[
    \neg, \land \Rightarrow p \land \neg p = 0
    \]
    \end{example}
    \textbf{Общий подход}:
    \[
    h(0, 1, 1, 0, 1, 0) = h(1, 0, 0, 1, 0, 1) = 1
    \]
    \[
    \Rightarrow h(\neg p, p, p, \neg p, p, \neg p) = 1, p = \overline{0, 1}
    \]
  \item [\underline{Шаг 5}]
    \[
    \neg, 0, 1, k \not\in L
    \]
    Б. О. О. в мн-не Жегалкина ф-ции $k$ есть слагаемое с $x_1x_2$
    \[
    k(x_1, \ldots, x_n) = x_1x_2 \cdot A(x_3, \ldots, x_n) \oplus x_1 \cdot B(x_3, \ldots, x_n) \oplus x_2 \cdot C(x_3, \ldots, x_n) + D(x_3, \ldots, x_n)
    \]
    Мн-н $A$ непустой $\Rightarrow \exists (a_3, \ldots, a_n) \colon A(a_3, \ldots, a_n) = 1$ \\
    Тогда $k(x_1, x_2, a_3, \ldots, a_n) = x_1x_2 \oplus x_1 \cdot B \oplus x_2 \cdot C \oplus D$

    Использование орицания позволяет менять $1$
    \begin{itemize}
      \item $B = C = 0 \Rightarrow $ выразили $x_1, x_2$, т. е. $\land$. $\land, \neg \mapsto $ ВСЁ
      \item $B = C = 1 \Rightarrow x_1 \oplus x_2 \oplus x_1x_2$, т. е. $\lor$. $\lor, \neg \mapsto$ ВСЁ
      \item $B = 0, C = 1 \Rightarrow 1 \oplus x_1 \oplus x_1x_2$, т. е. $\rightarrow$. $\rightarrow, \neg \mapsto$ ВСЁ
    \end{itemize}
\end{itemize}
\end{proof}
