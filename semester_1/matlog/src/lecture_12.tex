\section{Лекция 12}
\[
<\Z, S, = >
\]
\[
x = 0 \iff x + x = x
\]
\[
<\N, \cdot, = >
\]
\subsection{Метод автоморфизма}
Аддитивная ф-ция:
\[
f(x + y) = f(x) + f(y)
\]
Лин. ф-ция:
\[
f(\alpha \cdot x + \beta \cdot y) = \alpha f(x) + \beta f(y)
\]
Мультипликативная ф-ция:
\[
  \phi(x \cdot y) = \phi(x) \cdot \phi(y)
\]
Монотонная ф-ция:
\[
x \leq y \iff f(x) \leq f(y)
\]
Задана сигнатура $(P, \ldots, f, \ldots)$. Интерпретации с носит. $A$ и $B$:
\[
[P]_A, \ldots, [f]_A \text{ и } [P]_B, \ldots, [f]_B
\]
\[
\gamma \colon A \rightarrow B \text{ --- гомоморфизм, если}
\]
\begin{itemize}
  \item [1) ] При всех $x_1, \ldots, x_k \in A$.
    \[
    [P]_A (x_1, \ldots, x_k) \iff [P]_B (\gamma(x_1), \ldots, \gamma(x_k))
    \]
    "Предикаты сохраняются"
  \item [2) ] При всех $x_1, \ldots x_k \in A$:
    \[
    \gamma([f]_A(x_1, \ldots, x_k)) = [f]_B(\gamma(x_1), \ldots, \gamma(x_k))
    \]
     Для конст. симв.:
     \[
     \gamma([c]_A) = [c]_B
     \]
\end{itemize}
\begin{definition}
Автоморфизм:
\begin{itemize}
  \item [1) ] $A = B$
  \item [2) ] $\gamma$ - биекция
\end{itemize}
\end{definition}
\begin{theorem}[Об автоморфизмах]
\label{th:automorphism}
Пусть $A$ --- интерпр. сигнатуры $(P, \ldots, f, \ldots)$, $\alpha$ --- автоморфизм, $Q$ --- выразимый предикат. Тогда при всех $x_1, \ldots, x_k \in A$:
\begin{equation}
  \label{auth:eq}
Q(x_1, \ldots, x_k) \iff Q(\alpha(x_1), \ldots, \alpha(x_k))
\end{equation}
Сл-ие, если при некот-ром автоморфизме $\alpha$ эквиваленция ($\ref{auth:eq}$) неверна, то $Q$ невыразим:
\end{theorem}
\begin{example}
$(\Z, S, =)$
\[
\alpha(x = x + C)
\]
\[
Q(x) \iff x \divby 2
\]
\end{example}
\begin{example}
$(\Z, +, =)$
\[
\alpha(x) = -x
\]
\[
Q(x, y) \iff x > y
\]
\end{example}
\begin{example}
\[
n = 2^{a} \cdot 3^{b} \cdot k, k \not\divby 2, k \not\divby 3
\]
\[
\alpha(2^{a} \cdot 3^{b} \cdot k) = 2^{b} \cdot 3^{a} \cdot k
\]
\[
\alpha(0) = 0
\]
\[
Q(x, y) \iff x > y
\]
\end{example}
\begin{proof}[Доказательство теоремы:]
Докажем индукцией по построению:
\begin{itemize}
  \item [1) ] $t$ --- терм $\Rightarrow$ при всех $x_1, \ldots, x_k \in A$:
    \[
    [t](\alpha(x_1), \ldots, \alpha(x_k)) = \alpha([t](x_1,\ldots, x_k))
    \]
  \item[2) ] $\phi$ --- ф-ла $\Rightarrow$ При всех $x_1, \ldots, x_k \in A$:
    \[
    [\phi](\alpha(x_1), \ldots, \alpha(x_k)) \iff [\phi](x_1, \ldots, x_k)
    \]
  \item [3) ] Переменная $\alpha(x) = \alpha(x)$, конст. символ $[c] = \alpha([c])$ \\
    Конст. символ: $[c] = \alpha([c])$ \\
    Сост. терм:
    \[
    [f(t_1, \ldots, t_m)](\alpha(x_1), \ldots \alpha(x_k)) = [f]([t_1](\alpha(x_1), \ldots, \alpha(x_k)), [t_m]) = 
    \]
    \[
     = [f](\alpha([t_1](x_1, \ldots, x_k)), \ldots, \alpha([t_m](x_1, \ldots, x_k))) = 
    \]
    \[
     = \alpha([f]([t_1](x_1, \ldots, x_k), [t_m](x_1, \ldots, x_k))) = 
    \]
    \[
     = \alpha([f(t_1, \ldots, t_m)](x_1, \ldots, x_k))
    \]
    Атом. формулы --- аналогично термам
    \[
    \bigwedge_{y}^{} [\phi](\alpha(x_1), \ldots, \alpha(x_k), y) = \bigwedge_{y}^{}[\phi](\alpha(x_1), \ldots, \alpha(x_k), \alpha(y)) = 
    \]
    \[
     = \bigwedge_{y}^{} (x_1, \ldots, x_k, y) = [\forall y, \phi] (x_1, \ldots, x_k)
    \]
\end{itemize}
\end{proof}
\begin{example}
\[
<\N, S, = > \text{ --- нет автоморфизма, $\leq$ --- невыраз.}
\]
$0$ --- выразим: $x = 0 \iff \neg \exists y \colon x = S(y)$
\begin{consequence}
Выразим в $<\N, S, = > \iff $ выразим в $< \N, S, 0, = >$
\end{consequence} 
\end{example}
\begin{theorem}[Об элиминации кванторов]
\label{quant_elim}
Любая ф-ла в $< \N, S, 0, = >$ равна некот. бесквант. ф-ле
\end{theorem}
\begin{consequence}
$x \leq y$ не выраз. в $< \N, S, = >$
\end{consequence}
\begin{proof}
  $x \leq y$ выразима в $< \N, S, = > \Rightarrow x \leq y$ выразима в $< \N, S, 0, = >$ бескванторной ф-лой, т. е. пропозиц. формулой, в к-рую, вместо переменных подставл. атомарн. формулы. \\
  Ат. формулы:
  \[
  S(S(\ldots S(U))) = S(S(\ldots S(v)))
  \]
  \[
  u \text{ --- переменная или $0$}, v \text{ --- тоже}
  \]
  Значит $u = v + d, d \in \Z$ (ф-ла-комбинация кон. числа усл-ий)
  \[
  d_1, \ldots, d_n \text{ --- все числа из усл.}
  \]
  \[
  M  = max\set{d_1, \ldots, d_n} + 1
  \]
  Рассм $x = m, y = 2M$ и $x = 2M, y = M$ \\
  Все атом. ф-лы, кроме тожд. истины, будут ложны $\Rightarrow$ комбинация приним. одинаковые значения: \\
  Но $x \leq y$ верно для $x = M, y = 2M$ и неверно для $x = 2M, y = M \Rightarrow$ наша ф-ла не выр-ет $x \leq y$
\end{proof}
\begin{proof}[Доказательство теоремы об элиминации]
\begin{itemize}
  \item [1) ] Ат. ф-лы бескв.
  \item [2) ] $\phi \land \phi' \Rightarrow \neg \phi \land \neg \phi'$, аналог. для $\land, \lor, \rightarrow$
  \item [3) ] $\forall x \phi \sim \neg \exists x \neg \phi$
  \item [4) ] $\exists x \phi \sim \exists x \underset{\text{бескванторный}}{\phi'}$ \\
    Атомарные ф-лы, зависящие от $x$: $T, \perp, x = t_i$
\end{itemize}
\end{proof}
