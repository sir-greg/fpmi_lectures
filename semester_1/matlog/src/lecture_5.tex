\section{Лекция 5}
\textbf{Пропозициональные ф-лы}:
\begin{itemize}
  \item Всегда = 1 - Тавтологии - Выполнимые 
  \item М. Б. = 0 и = 1 - Опровержимые - Выполнимые 
  \item Всегда = 0 - Опровержимые - Противоречия
\end{itemize}
"Важные" тавтологии (Логические законы):
\begin{itemize}
  \item [1) ] Закон непротиворечия: 
    \[
    \neg(A \land \neg A)
    \]
  \item [2) ] Закон двойного отрицания:
    \[
    \neg\neg A \leftrightarrow A
    \]
  \item [3) ] Закон исключённого третьего: \[
    A \lor \neg A
  \]
  \begin{example}
  Неконструктивное док-во с использованием закона исключённого третьего:
  \begin{theorem}
  $\exists x, y \colon x \not\in Q, y \not\in Q, x^{y} \in Q$
  \end{theorem}
  \begin{proof}
  Рассм. выр-е: $(\sqrt{2})^{\sqrt{2}}$:
  \begin{itemize}
    \item [1) ] Оно $\in Q \Rightarrow$ нашли пример
    \item [2) ] Оно $\not\in Q \Rightarrow x = (\sqrt{2})^{\sqrt{2}}, y = \sqrt{2}$:
      \[
      x^{y} = (\sqrt{2}^{\sqrt{2}})^{\sqrt{2}} = (\sqrt{2})^{2} = 2
      \]
  \end{itemize}
  \end{proof}
  \end{example}
\item [4) ] Контрапозиция:
  \[
    (A \rightarrow B) \leftrightarrow (\neg B \rightarrow \neg A)
  \]
\item [5) ] Законы Де Моргана:
  \[
  \neg(A \land B) \leftrightarrow (\neg A \lor \neg B)
  \]
  \[
  \neg(A \lor B) \leftrightarrow (\neg A \land \neg B)
  \]
\end{itemize}
\textbf{Задача о выполнимости условий:} даны ф-лы $\phi_1, \phi_2, \ldots, \phi_n$ \\

Вопрос: могут ли они все быть одновременно истинны? \\

Это эквив. вопросу о выполнимости:
\[
  \phi_1 \land \phi_2 \land \ldots \land \phi_n
\]
\begin{example}
Превращение мат. задачи в задачу выполнимости: \\
1976г. - з-ча 4 красок решена комп. перебором. \\ 
Вершина графа $v \mapsto $ 2 бита. ($p_v, q_v$) - (область на карте) \\
$u, v$ - соседний области $\Rightarrow$ условие на отличие цветов:
\[
  (p_u \neq p_v) \lor (q_u \neq q_v)
\]
\end{example}
\subsection{Логический вывод}
\begin{definition}
\textbf{Логический вывод} - п-ть формул, в кот. каждая ф-ла либо является аксиомой, либо получается из более ранних по одному из правилу вывода.
\end{definition}

\begin{note}
  \[
    (A \rightarrow (B \rightarrow C)) \text{ - сл-ие из 2 посылок}
  \]
\end{note}
\textbf{Схемы аскиом} (Аксиомы - рез-т подстановки конкретных ф-л вместо $A, B, C$)
\begin{itemize}
  \item [1) ] $A \rightarrow (B \rightarrow A)$
  \item [2) ] $(A \rightarrow (B \rightarrow C)) \rightarrow ((A \rightarrow B) \rightarrow (A \rightarrow C))$
  \item [3) ] $(A \land B) \rightarrow A$
  \item [4) ] $(A \land B) \rightarrow B$
  \item [5) ] $A \rightarrow (B \rightarrow (A \land B))$
  \item [6) ] $A \rightarrow (A \lor B)$
  \item [7) ] $B \rightarrow (A \lor B)$
  \item [8) ] $(A \rightarrow C) \rightarrow ((B \rightarrow C) \rightarrow (A \lor B) \rightarrow C)$ - "Разбор случаев"
  \item [9) ] $\neg A \rightarrow (A \rightarrow B)$
  \item [10) ] $(A \rightarrow B) \rightarrow ((A \rightarrow \neg B) \rightarrow \neg A)$ - "Рассуждение от противного"
  \item [11) ] $A \lor \neg A$
\end{itemize}
  \textbf{Правило вывода:} \underline{modus ponens}:
  \begin{center}
  \begin{tabular}{ c c c } 
    $A$ & & $A \rightarrow B$ \\ 
   \hline
      & $B$ &
  \end{tabular}
  \end{center}

\begin{theorem}[О корректности]
  $A$ - выводима $\Rightarrow$ $A$ - тавтология
\end{theorem}
\begin{proof}
  Акс. 1-11 - тавтологии. 
  \[
  \begin{cases}
  A \text{ - тавтология } \\
  A \rightarrow B \text{ - тавтология}
  \end{cases} \Rightarrow B \text{ - тавтология}
  \]
\end{proof}
\begin{theorem}[О полноте]
$A$ - тавтология $\Rightarrow$ $A$ - выводима
\end{theorem}
\begin{symb}
  ~\newline
  $\vdash A$ - $A$ выводима \\
  $\models A$ - $A$ тавтология
\end{symb}
\begin{example}
$\vdash (A \lor B) \rightarrow (B \lor A)$
\begin{itemize}
  \item [1) ] $A \rightarrow (B \lor A)$ - акс. 7
  \item [2) ] $B \rightarrow (B \lor A)$ - акс. 6
  \item [3) ] $(A \rightarrow (B \lor A)) \rightarrow ((B \rightarrow (B \lor A)) \rightarrow ((A \lor B) \rightarrow (B \lor A)))$ - акс. 8
  \item [4) ] $(B \rightarrow (B \lor A)) \rightarrow ((A \lor B) \rightarrow (B \lor A))$ - modus ponens 1, 3
  \item [5) ] $(A \lor B) \rightarrow (B \lor A)$ - modus ponens 2, 4
\end{itemize}
\end{example}
\begin{example}
$\vdash (A \rightarrow A)$ - Закон тождества.
\begin{itemize}
  \item [1) ] $A \rightarrow ((A \rightarrow A) \rightarrow A)$ - акс. 1
  \item [2) ] $(A \rightarrow ((A \rightarrow A) \rightarrow A)) \rightarrow ((A \rightarrow (A \rightarrow A)) \rightarrow (A \rightarrow A))$ - акс. 2
  \item [3) ] $(A \rightarrow (A \rightarrow A)) \rightarrow (A \rightarrow A)$ - modus ponens 1, 2
  \item [4) ] $A \rightarrow (A \rightarrow A)$ - акс. 1
  \item [5) ] $A \rightarrow A$ - modus ponens 4, 3
\end{itemize}
\end{example}
