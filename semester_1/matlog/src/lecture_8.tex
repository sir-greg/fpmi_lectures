\section{Лекция 8}
\begin{center}
  Ф-лы \\
\begin{tabular}{ |c|c| } 
 \hline
  Выполнимые & Невыполнимые \\ 
 \hline
\end{tabular}
\end{center}
\subsection{Выр-ем задачи через вып-ть ф-л}
\begin{itemize}
  \item [1) ] \textbf{Раскраски:} \\
    Дан граф $G = (V, E)$. Цель, построить 3-раскраску
    \[
      V \rightarrow \set{1, 2, 3} \colon (v, u) \in E \Rightarrow col(u) \neq col(v)
    \]
    \begin{center}
    Вершина $u \mapsto (p_u, q_u)$ \\
    \begin{tabular}{ c c } 
      цвет & знач перем \\
      не сущ & 00 \\
      1 & 01 \\
      2 & 10 \\
      3 & 11
    \end{tabular}
    \end{center}
    Усл-ие на ребро: \\
    \[
      (v, u) \mapsto (p_u \neq p_v) \lor (q_u \neq q_v)
    \]
    Итоговая ф-ла:
    \[
    \bigcap_{(v, u) \in E}^{} (p_u \neq p_v) \lor (q_u \neq q_v)
    \]
    Вып-ма т. и т. т., когда граф раскрашен в 3 цвета.
  \item [2) ] \textbf{Расстановка ферзей}:
    \[
    n \times n \colon p_{ij} = \begin{cases}
    1, \text{ на клетке $(i, j)$ стоит ферзь} \\
    0, \text{ иначе}
    \end{cases}
    \]
    \[
      (p_{i 1} \lor \ldots \lor p_{i n}) \text{ - в i-ой строке > 1 Ф.}
    \]
    \[
      (\neg p_{ij} \lor \neg p_{ik}) \text{ - в i-ой строке $\leq 1$ Ф}
    \]
    \[
      (\neg p_{i k} \lor \neg p_{j k}) \text{ - в i-ой вертикали $\leq 1$ Ф}
    \]
    \[
      (\neg p_{i j} \lor \neg p_{i - k, j - k}) \text{ на диагонали $\leq 1$ Ф}
    \]
    \[
      (\neg p_{i j} \lor \neg p_{i - k, j - k}) \text{ на побочной диагонали $\leq 1$ Ф}
    \]
  Вся ф-ла - конкатенация всех условий.
\item [3) ] \textbf{З-ча о клике}: \\
  Дан граф $G, q_{u v} = 1 \iff (u, v) \in E$ \\
  Вопрос: $\exists?$ клика из $k$ вершин. \\
  \[
    (v_1, v_2, \ldots, v_k) \colon \forall i \neq j, (v_i, v_j) \in E
  \]
  \[
  \bigvee_{(v_1, v_2, \ldots, v_k)} \bigwedge_{i \neq j} q_{v_i, v_j} \text{ - длина $\sim C_{n}^{k}$} =
  \]
  \[
  = \frac{n!}{k! (n - k)!} > \frac{(n - k)^{k}}{k!} > \left(\frac{n - k}{k}\right)^{k} \underset{k = \frac{n}{10}}{=} 9^{\frac{n}{10}}
  \]
\end{itemize}

Можно ли понимать $v_1, v_2, \ldots, v_k$ как перемен. и написать ф-лу:
\[
\bigwedge_{i \neq j} (v_i \neq v_j \land q_{v_i, v_j}) ?
\]
Это не булева ф-ла, т. к. перем. встреч. в индексе.

\[
p_u = \begin{cases}
1, \text{ u в клике} \\
0, \text{ иначе}
\end{cases}
\]
\[
  (p_u \land p_v) \rightarrow q_{u v}
\]
\[
p_1 + p_2 + \ldots + p_n \geq k
\]
Или: $(u, v) \not\in E \Rightarrow (\neg p_u \land \neg p_v)$.

Будем делать так:
\[
  p_{i u} \text{ - вершина $u$ - $i$-ая в клике}
\]
\[
  (p_{i 1} \lor \ldots \lor p_{i n}) \text{ - под каждым номером есть вершина, $i \in \set{1, \ldots, k}$}
\]
\[
i \neq j \Rightarrow (\neg p_{i v} \lor \neg p_{j v}) \text{ - у одной верш. не м. б. 2 номеров.}
\]
\[
  (u, v) \not\in E \Rightarrow (\neg p_{i u} \lor \neg p_{j v}) \text{ - антиребро не м. б. внутри клики.}
\]
\subsubsection{Обобщаем. Метод резолюций}
Ф-ла - конъюнкция всех усл. - КНФ. \\

Пусть дана КНФ, будем рассм. её как набор дизъюнктов. \\

Правило Res: \begin{center}
\begin{tabular}{ c c c } 
  $A \lor x$ & & $B \lor \neg x$ \\
 \hline
             & $A \lor B$ - \textbf{резольвента} & 
\end{tabular}
\end{center}

\begin{statement}
Если на данном наборе вып. $A \lor x$ и $B \lor \neg x$, то вып-мо и $A \lor B$
\end{statement}
\begin{consequence}
Если исх. ф-ла вып-ма, то и все резольвенты тоже.
\end{consequence}
Пустой дизъюинкт: $\perp$
\begin{center}
\begin{tabular}{ c c c } 
  $x$ & & $\neg x$ \\ 
 \hline
      & $\perp$ &
\end{tabular}
    
\begin{tabular}{c c c}
  $x \lor y$ & & $\neg x \lor \neg y$ \\
  \hline
             & $y \lor \neg y$ &
\end{tabular}

\begin{tabular}{c c c}
  $p \lor x$ & & $p \lor r \lor \neg x$ \\
  \hline
             & $p \lor r$ & 
\end{tabular}
\end{center}
Метод резолюций: строим всё новые резольвенты, пока либо не будет выведен $\perp$, либо не прекратится появление новых дизъюнктов. \\
\begin{theorem}[О корректности метода резол.]
Если исх. ф-ла вып., то $\perp$ нельзя вывести.
\end{theorem}
\begin{proof}
Если можно вывести, то $\perp$ будет ист., но он $\equiv 0$
\end{proof}
\begin{example}
Ферзи 2 x 2
\begin{center}
\begin{tabular}{ |c|c| } 
 \hline
 $p$ & $q$ \\ 
 \hline
 $r$ & $s$ \\
 \hline
\end{tabular}
\end{center}
Усл-ие:
\[
p \lor q
\]
\[
r \lor s
\]
\[
\neg p \lor \neg q
\]
\[
\neg r \lor \neg q
\]
\[
\neg p \lor \neg s
\]
\[
\neg q \lor \neg r
\]
\begin{center}
\begin{tabular}{ c c c }
  $ p \lor q $ & & $ \neg p \lor \neg s $ \\
\hline
             & $ q \lor \neg s $ & 
\end{tabular}
\end{center}
\begin{center}
\begin{tabular}{ c c c }
$  $ & & $  $ \\
\hline
& $  $ &
\end{tabular}
\end{center}
Picture
\end{example}
\begin{theorem}(О полноте)
Если $\perp$ нельзя вывести, то ф-ла выполнима.
\end{theorem}
\begin{proof}
Все выводимые дизъюнкты разобъём на классы.
\[
C_0 \subset C_1 \subset C_2 \subset \ldots \subset C_k
\]
$C_i$ - дизъюнкты, зависящ. только от переменных $p_1, \ldots, p_i$ ($C_0 = \emptyset$, т. к. $\perp$ - невыводим). \\

Будем док-ть по инд-ции, что одновр. вып. все дизъюнкты из $C_i$. \\
ММИ:
\begin{itemize}
  \item База: $C_0 = \emptyset \Rightarrow $  очев.
  \item Переход: пусть все ф-лы из $C_{i - 1}$ вып-ны на знач. $a_1, \ldots a_{i - 1}$. Рассм. ф-лы из $C_i$, кот. ещё не выполнены за счёт этих значений. \\
    Предположим, что среди них есть ф-ла с $p_i$ и ф-ла с $\neg p_i$:
    \[
    p_i \lor D_0 \text{ и } \neg p_i \lor D_1
    \]
    Раз эти ф-лы остались, то $D_0(a_1, \ldots, a_{i - 1}) = 0$ и $D_1(a_1, \ldots, a_{i - 1}) = 0$. Но $D_0 \lor D_1$ явл-ся резольвентой: $(p_i \lor D_0), (\neg p_i \lor D_1)$. Тогда $D_0 \lor D_1 \in C_{i - 1}$, и тогда должно быть: $D_0 \lor D_1 = 1$!!! \\
    Следовательно, все оставшиеся ф-лы либо с $p_i \Rightarrow p_i = 1$, либо с $\neg p_{i} \Rightarrow p_i = 0$
\end{itemize}
\end{proof}
Как это связано с тафтологиями? А это уже совсем другая история.
