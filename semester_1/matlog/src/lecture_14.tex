\section{Лекция 14}
Расшир. ИВ на ф-лы $1$-ого порядка. \\
\begin{center}
\begin{tabular}{ |c|c|c| } 
 \hline
 $\set{\text{Вывод. ф-лы}}$ & $=$ & $\set{\text{Общезнач. ф-лы}}$ \\
 \hline
                            & $\subset$ & Теор. о корр. \\
 \hline
                            & $\supset$ & Теор. о полноте \\
 \hline

\end{tabular}
\end{center}
Новый список аксиом:
\begin{itemize}
  \item Аксиомы 1-11 \\
  \item Аксиомы 12: $\forall x \phi \rightarrow \phi(\sfrac{t}{x}), t$ --- терм, подстановка $\sfrac{t}{x}$ --- корректна.
  \item $\phi(\sfrac{t}{x}) \rightarrow \exists x \phi$ \\ 
$\phi(\sfrac{t}{x})$ --- результат замены своб. вхожд. $x$ на $t$, при этом своб. переменные из $t$ не попадают под д-ие кванторов $\phi$ 
\end{itemize}
Подстановка точно корректна, если:
\begin{itemize}
  \item [1) ] $t$ --- замкн. терм (сост. только из констант)
  \item [2) ] $t \eqcirc x$
\end{itemize}
Примеры вывод ф-л:
\begin{itemize}
  \item [0) ] Все тавтологии (с подст. формул вместо переменных)
  \item [1) ] $\forall x \phi \rightarrow \exists x \phi$ \\
Вывод:
\begin{itemize}
  \item [1. ] $\forall x \phi \rightarrow \phi$ --- $A12$ \\
  \item [2. ] $\phi \rightarrow \exists x\phi$ $A13$ \\
  \item [3. ] $\forall x \phi \rightarrow \exists x \phi$ --- силлогизм.
\end{itemize}
\textbf{Правила Бёрнайса:}
\begin{itemize}
  \item $\Sigma$-правило:
    \begin{center}
    \begin{tabular}{ c} 
      $\phi \rightarrow \psi$ \\
     \hline
     $\exists x \phi \rightarrow \psi$
    \end{tabular}
    \end{center}
    \begin{note}
    \textbf{Условие применимости:} $x$ не параметр $\psi$!
    \end{note}
  \item $\prod$-правило:
    \begin{center}
    \begin{tabular}{ c } 
      $\psi \rightarrow \phi$ \\
     \hline
      $\psi \rightarrow \forall x \phi$
    \end{tabular}
    \end{center}
    Опять же: $x$ не параметр  $\psi$
\end{itemize}
\item [2) ] $\exists x \forall y \phi \rightarrow \forall y \exists x \phi$ \\
   Вывод:
   \begin{itemize}
     \item [1.] $\forall y \phi \rightarrow \phi$ $A12$ \\
     \item [2. ] $\phi \rightarrow \exists x \phi$ $A13$
     \item [3. ] $\forall y \phi \rightarrow \exists x \phi$ --- силлогизм
     \item [4.] $\exists x \forall y \phi \rightarrow \exists x\phi$ --- $\Sigma$ --- Бёрн., $3$ 
     \item [5. ] $\exists x \forall y \phi \rightarrow \forall y \exists x \phi$ --- $\prod$--- Бёрн., $4$
   \end{itemize}
  \textbf{Правило обобщения (Gen):}
  \[
  \frac{\phi}{\forall x \phi}
  \]
  \[
  \phi \text{ общезн. } \Rightarrow \forall x \phi \text{ общезн.}
  \]
  При этом $(\phi \rightarrow \forall x \phi)$ --- необщезн.
\item [3) ] Вывод \textbf{Gen}:
  \[
    \vdash \phi \Rightarrow \vdash \forall x \phi
  \]
  \begin{itemize}
    \item [1. ] $\phi$ --- выводима \\
    \item [2. ] $\psi$ (любая замкн. акс.)
    \item [3. ] $\phi \rightarrow (\psi \rightarrow \phi)$ A1
    \item [4. ] $\psi \rightarrow \phi$ MP 1, 3
    \item [5. ] $\psi \rightarrow \forall x \phi$ --- $\prod$-Бёрн., 4
    \item [6. ] $\forall x \phi$ MP 2, 5
  \end{itemize}
\item [4) ] $\neg \exists x \phi \leftrightarrow \forall x \neg \phi$
  \[
  \neg \forall x \phi \leftrightarrow \exists x \neg \phi
  \]
  \[
  \forall x \neg \phi \rightarrow \neg \phi
  \]
  \[
  \phi \rightarrow \neg \forall x \neg \phi
  \]
  \[
  \exists x \phi \rightarrow \neg  \forall x \neg \phi
  \]
  \[
  \forall x \neg \phi \rightarrow \neg \exists x \phi \text{ --- контрапоз.}
  \]
\item [5) ] Лемма о дедукции для ИП: \\
  В кач-ве посылок используются только замкн. ф-лы (посылки также наз-ют аксиомами) \\
  Теория --- любое мн-во замкн. ф-л \\
  Модель теории --- интерпретация, в кот-рой все ф-лы теории истины. \\
  Лемма о дедукции: Пусть $\Gamma$ --- теория, $A$ --- замкн. ф-ла, $B$ --- произв. ф-ла. \\ Тогда: $\Gamma \vdash (A \rightarrow B) \iff \Gamma \cup \set{A} \vdash B$
  \begin{proof}
    \begin{itemize}
      \item [$\Rightarrow$)]
        \begin{itemize}
          \item [1. ] $A \rightarrow B$ (вывод)
          \item [2. ] $A$ --- посылка
          \item [3. ] $B$ (MP 1, 2)
        \end{itemize}
      \item [$\Leftarrow$)] Инд-ция $C_1, \ldots, C_n$ --- вывод $B$ из $\Gamma \cup \set{A}$. По инд-ции докажем $\Gamma \vdash A \rightarrow C_i$:
        \[
          C_i \text{ --- акс., эл-т $\Gamma$, ф-ла $A$ или получ. по MP --- аналог. д-ву для ИВ.}
        \]
        \[
        C_i \text{ --- получ. по $\Sigma$-прав.}
        \]
        \[
        C_i \eqcirc (\exists x \phi \rightarrow \psi), C_j \eqcirc (\phi \rightarrow \psi),j < i
        \]
        По предположению инд-ции: $\Gamma \vdash (A \rightarrow (\phi \rightarrow \psi))$ \\
        Тавтология: $(A \rightarrow (\phi \rightarrow \psi)) \leftrightarrow (\phi \rightarrow (A \rightarrow \psi))$
        \[
        \Rightarrow \Gamma \vdash (\phi \rightarrow (A \rightarrow \psi)) \Rightarrow
        \]
        \[
        \Rightarrow \Gamma \vdash (\exists x \phi \rightarrow (A \rightarrow \psi)) \text{ --- $\Sigma$-Бёрн}
        \]
        \[
        \Rightarrow \Gamma \vdash (A \rightarrow (\exists x \phi \rightarrow \psi)) \Rightarrow \Gamma \vdash (A \rightarrow C_i)
        \]
        $C_i$ --- получ. по $\prod$-правилу:
        \[
        \Gamma \vdash (A \rightarrow (\psi \rightarrow \phi)) \Rightarrow \Gamma \vdash (A \rightarrow (\psi \rightarrow \forall x \phi))
        \]
        \[
        \Rightarrow \Gamma \vdash ((A \land \psi) \rightarrow \psi) \Rightarrow \Gamma \vdash ((A \land \psi) \rightarrow \forall x \phi)
        \]
    \end{itemize}
  \end{proof}
\end{itemize}
\textbf{Слабая форма.:} \\
Теперь перейдём к теоремам: \\
Теор. о корр. ИП: $\vdash \phi \Rightarrow \phi$ --- общезначима. \\
Теор. о полн. ИП: $\phi$ --- общезнач. $\Rightarrow \vdash \phi$ \\
\textbf{Сильная форма:} \\
У любой непротиворечивой теории существует модель \\
\textbf{Сильная форма} $\Rightarrow $ \textbf{слабая форма}.
\[
\phi \text{ --- общ. } \Rightarrow \forall x \phi \text{ --- общ.} \Rightarrow \set{\neg \forall x \phi} \text{ --- не имеет модели} \Rightarrow
\]
\[
 \Rightarrow \set{\neg \forall x \phi} \text{ --- против.} \Rightarrow 
\]
\[
\Rightarrow \begin{cases}
\set{\neg \forall x \phi} \vdash A \\
\set{\neg \forall x \phi} \vdash \neg A
\end{cases} =
\begin{cases}
\vdash \neg \forall x \phi \rightarrow A \\
\vdash \neg \forall x \phi \rightarrow \neg A
\end{cases}
\]
\[
\Gamma \text{ --- непротив. теория} \Rightarrow \text{ Есть модель.}
\]
Строим модель из замкн. термов. \\
Проблема: может не быть конст. символов или функц. симв. Неясно, как опред. пред. симв.
\begin{definition}
$\Gamma$ --- полная теория, если для любой замкн. $\phi$ верно $\Gamma \vdash \phi$ или $\Gamma \vdash \neg \phi$
\end{definition}
\begin{lemma}
Любая непрот. теория вложена в нек-рую полную.
\end{lemma}
Проблема: если $\Gamma \vdash \exists x\phi$, то $\psi$ должна быть ист. для нек-рого эл-та модели.
\begin{definition}
Теория $\Gamma$ наз-ся экзистенциально полной, если из $\Gamma \vdash \exists x \phi$ следует $\Gamma \vdash \phi(\sfrac{t}{x})$ для некоторого замкн. терма $t$
\end{definition}
\begin{lemma}
Если $\Gamma$ --- непрот. теория в сигнатуре $\sigma \Rightarrow \exists \tau \supset \sigma, \triangle \supset \Gamma \colon \triangle$ --- непрот. теория в сигн. $\tau$ и $\triangle$ --- экзистенциально полная.
\end{lemma}
