\section{Лекция 12}
\begin{equation}
\label{eq:char}
a_k y_{n + k} + a_{k - 1} y_{n + k - 1} + \ldots + a_0 y_n = 0
\end{equation}
Хар-ое ур-ие:
\[
a_k x^{k} + \ldots + a_0 = 0
\]
\begin{theorem}[Основная теорема алгебры]
\label{th:base_algebra_th}
Мн-н степени $k$ имеет $k$ комплексных корней, т. е.:
\[
P(x) = a_k x^{k} + \ldots + a_0 = a_k \prod_{i = 0}^{k} (x - \lambda_i)
\]
где,
\[
P(\lambda_i) = 0, \lambda_i \in \C
\]
\end{theorem}
Возьмём эти корни из хар-ого ур-ия. Пусть:
\[
\mu_1, \ldots, \mu_r
\]
Этот же список корней, без \textbf{дубликатов}. Также:
\[
m_1, \ldots, m_r
\]
Это кратности этих корней.
\[
1 \leq r \leq k
\]
\[
\sum_{i = 1}^{r} m_i = k
\]
Обозначим за:
\[
P_l(n) = c_l n^{l} + \ldots + c_0 
\]
--- произвольный мн-н степени $l$
\begin{theorem}
\label{th:linear_recurrense_general_solution}
\begin{itemize}
  \item [1) ] \[
  \forall P_{m_1 - 1}^{1}(n), \ldots, P_{m_r - 1}^{r}(n)
  \]
  \[
 \hookrightarrow y_n = P_{m_1 - 1}^{1}(n) \mu_1^{n} + \ldots + P_{m_r}^{r}(n)\mu_r^{n}
  \]
  --- удовлетворяет $(\ref{eq:char})$
\item [2) ] Если $\set{y_n}_{n = 1}^{\infty}$ удовл. $(\ref{eq:char})$, то $\exists P_{m_1 - 1}^{1}, \ldots, P_{m_r - 1}^{r} \colon$
  \[
  y_n = < \text{ запись из п. 1} >
  \]
\end{itemize}
\end{theorem}
\subsection{Разбиение чисел на слагаемые}
\[
n \in \N \colon n = x_1 + \ldots + x_t
\]
\[
\forall i \colon x_i \in \set{n_1, \ldots, n_k}
\]
(***Офигенные примеры с помидором и попойкой***)
\begin{theorem}[Решение попойки (упор. разбиений)]
\label{solve:popoika}
\[
F(n; n_1, \ldots, n_k) = F(n - n_1; n_1, \ldots, n_k) + \ldots + F(n - n_k; n_1, \ldots, n_k)
\]
\[
F(0; n_1, \ldots, n_k) = 1; F(-n; n_1, \ldots, n_k) = 0
\]
\end{theorem}
\begin{consequence}
\[
F(n;, 1, 2, \ldots, n) = 2^{n - 1}
\]
\end{consequence}
\begin{proof}
 ММИ:
 \begin{itemize}
   \item [1) ] $n = 1 \colon F(1; 1) = 1 = 2^{1 - 1}$
   \item [2) ] \[
   F(n; 1, \ldots, n) = F(n - 1; 1, \ldots, n - 1) + 
   \]
   \[
    + F(n - 2; 1, 2, \ldots, n - 2) + \ldots + F(1; 1) + F(0; 0) = 
   \]
   \[
    = 2^{n - 2} + 2^{n - 1} + \ldots + 1 + 1 = 2^{n - 1}
   \]
 \end{itemize}
\end{proof} 
\begin{theorem}[Решение помидора (неупор. разбиения)]
\label{solve:tomato}
\[
f(n; n_1, \ldots, n_k) = f(n - n_1; n_1, \ldots, n_k) + f(n; n_2, \ldots, n_k)
\]
\[
f(0; \ldots) = 1
\]
\[
f(-n; \ldots) = 0
\]
\end{theorem}
