\section{Лекция 9}
\begin{itemize}
  \item [7) ] \[
  C_{n}^{0} - C_{n}^{1} + \ldots + (-1)^{n} C_{n}^{n} = \begin{cases}
  1, n = 0 \\
  0, n > 0
  \end{cases}
  \]
\item [8) ] Рассм. $A = \set{a_1, \ldots, a_n}, m < n, m > 0$. Выберем из $A$ все возм. $m$-разм. с повтор. Их $n^{m}$. $N = n^{m}, \alpha_1, \ldots \alpha_m$. Размещ. обладает св-вом $\alpha_i \iff \alpha_i \not\in$ размещению. 
  \[
  N(\alpha_i) = (n - 1)^{m}, N(\alpha_i, \alpha_j) = (n - 2)^{m}
  \]
  \[
  N(\alpha_1', \ldots, \alpha_n') = 0
  \]
  По формуле вкл.-искл.:
  \[
  N(\alpha_1', \ldots, \alpha_n') = n^{m} - n(n - 1)^{m} + C_{n}^{2}(n - 2)^{n} + \ldots
  \]
  \[
  \Rightarrow 0 = \sum_{k = 0}^{n} (-1)^{k} C_{n}^{k} (n - k)^{m}
  \]
\end{itemize}
\begin{task}
Задача о беспорядках: \\
\begin{definition}
Беспорядок - перестановка, при кот. $\sigma_i \neq i, \forall i = \overline{1,n}$
\end{definition}
Найдём кол-во беспорядков для $n = 100\colon$. Пусть $\alpha_i$ - св-во, при кот. $\sigma_i = i$, посчитаем:
\[
N = 100!
\]
\[
N(\alpha_1', \alpha_2', \ldots, \alpha_n') = 100! - 99! * 100 + C_{100}^{2} \cdot 98! - \ldots = \sum_{k = 0}^{n} C_{n}^{k} (n - k)! = !n
\]
При раскрытии $C$-шек, получаем:
\[
N(\alpha_1', \ldots, \alpha_n') = \frac{1}{0!} - \frac{1}{1!} + \frac{1}{2!} - \frac{1}{3!} + \ldots + \frac{1}{100!}\approx \frac{1}{e}
\]
"Физической интуицией" получаем:
\[
\left(1 - \frac{1}{100}\right)^{100} \approx \frac{1}{e}
\]
\end{task}
\subsection{Циклические п-ти}
Алфавит: $X = \set{b_1, \ldots, b_r}$. Из b-шек составляем слова длины $n$.
\[
a_1a_2\ldots a_n, \forall i, a_i \in X
\]
\[
a_1 \rightarrow a_2 \rightarrow a_3 \rightarrow \ldots \rightarrow a_{n - 1} \rightarrow a_n \rightarrow a_1
\]
Т. е. $a_1\ldots a_n$ отождествляется с $a_2\ldots a_n a_1$, $a_3\ldots a_n a_1 a_2, \ldots$.\\
Однако $r^{n}$ - кол-во обычных слов, т. е. $n$ не всегда делит $r^{n}$. 
\begin{example}
\[
r = 3, X = \set{C, O, H}, n = 4\colon
\]
Следующие слова нужно поделить на 4:
\[
COCH
\]
\[
OCHC
\]
\[
CHCO
\]
\[
HCOC
\]
Эти на 2:
\[
COCO
\]
\[
OCOC
\]
А это на 1:
\[
CCCC
\]
Для решение этой задачи, для начала, изучим следующий мощный инструмент:
\end{example}
\subsection{Формула обращения Мёбиуса}
\begin{theorem}[ОТА]
\[
\forall n \geq 2, \exists! \set{p_1, \ldots p_s}, \set{a_1, \ldots a_s} \colon
\]
\[
n = p_1^{\alpha_1} \cdot \ldots \cdot p_s^{\alpha_s}
\]
(Это наз-ся \textbf{каноническим разложением} $n$)
\end{theorem}
\begin{definition}
\textbf{Функция Мёбиуса} $\mu(n)\colon$
\[
\mu(n) = \begin{cases}
1, n = 1 \\
(-1)^{s}, n = p_1 \cdot \ldots \cdot p_s \\
0, \text{ иначе}
\end{cases}
\]
\end{definition}
\begin{lemma}
\[
\sum_{d | n}^{} \mu(d) = \begin{cases}
1, n = 1 \\
0, \text{ иначе}
\end{cases}
\]
\end{lemma}
\begin{proof}
Пусть $n \geq 2 \Rightarrow n = p_1^{\alpha_1} \cdot \ldots \cdot p_k^{\alpha_k}$ \\
\[
  (d | n) \Rightarrow (d = p_1^{\beta_1} \cdot \ldots \cdot p_k^{\beta_k}, \forall i, 0 \leq \beta_i \leq \alpha_i)
\]
\[
  \sum_{d | n}^{} \mu(d) = \sum_{\beta_1 = 0}^{\alpha_1} \ldots \sum_{\beta_k = 0}^{\alpha_k} \mu(p_1^{\beta_1} \cdot \ldots \cdot p_k^{\beta_k}) =\sum_{\beta_1 = 0}^{1} \ldots \sum_{\beta_k = 0}^{1} \mu(p_1^{\beta_1} \cdot \ldots \cdot p_k^{\beta_k})
\]
\[
 = \mu(1) + C_{k}^{1} (-1) + C_{k}^{2} (-1)^{2} + \ldots + C_{k}^{k}(-1)^{k} = \sum_{i = 0}^{k} (-1)^{k} C_{k}^{i} = 0
\]
\end{proof}
\begin{theorem}[Формула обращения Мёбиуса]
Пусть $f = f(n)$ - ф-ция $n \in \N$. Пусть $g(n) = \sum_{d | n}^{} f(d)$. Тогда:
\[
f(n) = \sum_{d | n}^{} \mu(d) g\left(\frac{n}{d}\right) = \sum_{d | n}^{} \mu\left(\frac{n}{d}\right) g(d)
\]
\end{theorem}
\begin{proof}
\[
\sum_{d | n}^{} \mu(d) \sum_{d' | \frac{n}{d}}^{} f(d') = \sum_{(d, d') \colon d \cdot d' | n}^{} \mu(d)f(d') = \sum_{d | n}^{} f(d) \sum_{d' | \frac{n}{d}}^{} \mu(d') = 
\]
\[
= f(n) \cdot \sum_{d' | 1}^{} \mu(d') + \sum_{d | n, d < n}^{} f(d) \sum_{d' | \frac{n}{d}}^{} \mu(d') = f(n) \cdot 1 + \sum_{d | n, d < n}^{} f(d) \cdot 0 = f(n)
\]
\end{proof}
