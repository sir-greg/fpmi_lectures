\section{Лекция 13}
\subsection{Диаграммы Юнга}
$n \in \N$
\[
n = x_1 + \ldots + x_t
\]
\[
x_1 \leq x_2 \leq \ldots \leq x_t
\]
\begin{symb}
Канонический вид диаграммы юнга:
\[
x_1 \colon \underset{x_1 \text{ раз}}{\circ \circ \circ \ldots \circ}
\]
\[
\vdots
\]
\[
x_k \colon \underset{x_k \text{ раз}}{\circ \circ \circ \ldots \circ \circ \circ}
\]
\end{symb}

\begin{theorem}
\label{th:diag_ung_1}
  Кол-во разб. (помидорных --- неупор.) числа $n$ на не более чем $k$ слагаемых равно кол-ву разб. числа $n + k$ на ровно $k$ слагаемых.
\end{theorem}
\begin{proof}
Добавляем слева от диаграммы юнга столбец размера $k$. Получаем биекцию.
\end{proof}
\begin{theorem}
\label{th:diag_ung_2}
  Кол-во разб. (помидорных --- неупор.) числа $n$ на не более чем $k$ слагаемых равно кол-ву разб. числа $n + \frac{k(k + 1)}{2}$ на ровно $k$ различных слагаемых.
\end{theorem}
\begin{proof}
  К $i$-ой строке слева добавляем $i$ единиц. Если числа были равными, то теперь нет. Нер-ва сохранились. Получили биекцию.
\end{proof}
\begin{theorem}
\label{th:diag_ung_3}
Кол-во разб. (помидорных --- неупор.) числа $n$ на не более чем $k$ слагаемых равно кол-ву разбиений числа $n$ на слагаемые величины $\leq k$.
\end{theorem}
\begin{proof}
Инвертируем таблицу, превращая строки в столбцы.
\end{proof}

\subsection{Эйлер}
\[
  (1 - x)(1 - x^{2}) \ldots (1 - x^{n}) \ldots = 1 - x - x^{2} + x^{5} + x^{7} - x^{12} - x^{15} + \ldots
\]
\begin{theorem}
\label{th:euler_1}
Пусть $n = \frac{3k^{2} \pm k}{2}$, $k \in \N \cup \set{0}$. Тогда коэфф. при $x^{n}$ равен $(-1)^{k}$, если же $n \neq \frac{3k^{2} \pm k}{2}$, то коэфф. равен $0$.
\end{theorem}
Посмотрим, причём здесь разбиения? А вот причём: \\
Коэффициент при $x^{n}$:
\[
  (-x^{n_1})(-x^{n_2})\ldots(-x^{n_t}) = (-1)^{t} x^{n}
\]
\[
n_1 + n_2 + \ldots + n_t = n
\]
$n_{\text{чёт}}$ --- кол-во разбиений $n$ на различные слагаемые, число кот-ых \textbf{чётно} \\
$n_{ \text{нечёт}}$ --- кол-во разбиений $n$ на различные слагаемые, число кот-ых \textbf{нечётно} \\
Тогда коэф-т при $x^{n} \rightarrow n_{\text{чёт}} - n_{\text{нечёт.}}$
\subsection{Формальные степенные ряды}
На мн-ве объектов вида:
\[
A = (a_0, a_1, \ldots, a_n, \ldots), a_i \in \R
\]
(т. е. бесконечные п-ть чисел) \\
Введём операции:
\begin{itemize}
  \item [1) ] \underline{Сложение}:
    \[
    B = (b_0, b_1, \ldots), C = A + B \Rightarrow \forall i \colon c_i = a_i + b_i
    \]
  \item [2) ] \underline{Умножение на число}: очев.
  \item [3) ] \underline{Умножение ФСР:}
    \[
    A, B, C = A\cdot B
    \]
    \[
    \forall i \colon c_i = \sum_{k = 0}^{i} a_k \cdot b_{i - k}
    \]
  \item [4) ] \underline{Взятие обратного}:
    \[
    A, C = \frac{1}{A} \iff AC = 1
    \]
    Это такое $C$, что:
    \[
    \begin{cases}
    a_0c_0 = 1 \\
    a_0c_1 + a_1c_0 = 0 \\
    a_0c_2 + a_1c_1 + a_2c_0 = 0 \\
    \vdots \\
    \end{cases}
    \]
    Система разрешима $\iff$ $a_0 \neq 0$
\end{itemize}

\begin{example}
  \[
  \frac{1}{(1 - x^{2})^{2}} = \left(\frac{1}{1 - x^{2}}\right)^{2} = \left(\frac{1}{1 - x}\right)^{2}\left(\frac{1}{1 + x}\right)^{2}
  \]
  Деля в столбик $1$ на $1 - x$, получаем:
  \[
  \frac{1}{1 - x} = 1 + x + x^{2} + x^{3} + \ldots
  \]
  \[
  \frac{1}{1 - x^{2}} = 1 + x^{2} + x^{4} + x^{6} + \ldots + x^{2n} + \ldots
  \]
  \[
  \left(\frac{1}{1 - x^{2}}\right)^{2} = 1 + 2x^{2} + 3x^{4} + \ldots + (n + 1)x^{2n}
  \]
  \[
  \left(\frac{1}{1-x}\right)^{2} = 1 + 2x + 3x^{2} + \ldots + (n + 1)x^{n}
  \]
  \[
  \left(\frac{1}{1+x}\right)^{2} = 1 - 2x + 3x^{2} + \ldots + (-1)^{n}(n + 1)x^{n}
  \]
  \[
  \left(\frac{1}{1 - x}\right)^{2}\left(\frac{1}{1 +x}\right)^{2} = \ldots + (1 \cdot (-1)^{n} \cdot (n + 1) + 2 \cdot (-1)^{n - 1} \cdot n + \ldots + (n + 1) \cdot 1) x^{n} + \ldots
  \]
  Т. е. коэф. при $x^{n}$:
  \[
  \sum_{k = 0}^{n} (k + 1) (-1)^{n - k} (n + 1 - k) = \begin{cases}
  0, n = 2l + 1, l \in \N \\
  l + 1, n = 2l, l \in \N
  \end{cases}
  \]
\end{example}
\subsection{Производящие ф-ции}
\[
a_0, \ldots, a_n, \ldots
\]
\[
f(x) = \sum_{k = 0}^{n} a_k x^{k}
\]
\[
S_n(x_0) = \sum_{k = 0}^{n} a_k x_0^{k}
\]
Ряд сходится, если:
\[
\exists \lim_{n\to \infty} S_n(x_0) \in \R
\]
\begin{theorem}[Коши-Адамар]
\label{th:gen_functions_1}
Пусть
\[
  p = \frac{1}{\overline{\lim_{n\to\infty}} \sqrt[n]{\left|a_n\right|}}
\]
Если $\left|x_0\right| < p$, то ряд с коэф. $\set{a_n}$ сх-ся. Если $\left|x_0\right| > p$, то расх-ся.
\begin{note}
Если $\left|x_0\right| < p$, то $f$ можно дифференцировать почленно:
\[
  f(x) = \sum_{k = 0}^{\infty} a_k x^{k} \Rightarrow f'(x) = \sum_{k = 1}^{\infty} ka_k x^{k - 1}
\]
\end{note}
\end{theorem}
\begin{example}
\begin{itemize}
  \item [1) ] \[
  \sum_{k = 0}^{\infty} x^{k}, \left|x_0\right| < 1 \text{ --- сх-ся}, \left|x_0\right| \geq 1 \text{ --- расх-ся}
  \]
\item [2) ] \[
  \sum_{k = 0}^{\infty} 2^{k} x^{k}, \left|x_0\right| < \frac{1}{2} \text{ --- сх-ся}, \text{иначе расх-ся}
\] 
\item [3) ] \[
a_k = \begin{cases}
2^{k}, k \text{ --- чёт} \\
-3^{k}, k \text{ --- нечет}
\end{cases}
\]
\[
\Rightarrow \left|x_0\right| < \frac{1}{3} \text{ --- сх-ся, иначе расх-ся}
\]
\item [4) ] \[
  \sum_{k = 1}^{\infty} \frac{x^{k}}{k^{2}}
\]
\[
  \left|x_0\right| < 1 \text{ --- сх-ся}
\]
\[
  \left|x_0\right| > 1 \text{ --- расх-ся}
\]
\[
  \left|x_0\right| = 1 \text{ --- СХ-СЯ и равен $\frac{\pi^{2}}{6}$}
\]
\item [5) ] \[
  \sum_{k = 1}^{\infty} \frac{x^{k}}{k}, \left|x_0\right| = 1 \text{ --- расх-ся}
\]
\end{itemize}
\end{example} 
