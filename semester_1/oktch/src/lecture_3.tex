\section{Лекция 3}

\subsection{Мощности мн-в}

\subsubsection{Парадоксы}

\textbf{Парадокс Галилея}:
\begin{center}
\begin{tabular}{ |c|c|c| } 
 \hline
 Все нат. числа & $\not\supseteq$ & полные квадаты \\
 \hline
 $n$ & $\longleftrightarrow$ & $n^{2}$ \\
 \hline
\end{tabular}
\end{center}
\textbf{Гранд-отель Гильберта}:
\begin{center}
\begin{tabular}{ |c|c|c|c|c|c|c|c|c|c|c| } 
 \hline
 1 & 2 & 3 & 4 & 5 & 6 & 7 & \ldots\\ 
 \hline
\end{tabular}
\end{center}
\begin{itemize}
  \item [1) ] Все места заняты, нужно подселить постояльца:
    \begin{solution}
      \begin{center}
      \begin{tabular}{ |c|c|c| } 
       \hline
       Новый & $\rightarrow$ & 0 \\
       \hline
       $i$ & $\rightarrow$ & $\left(i + 1\right)$ \\
       \hline
      \end{tabular}
      \end{center}
    \end{solution}
  \item [2) ] Есть своб. места, \textbf{хотим занять все комнаты имеющимися постояльцами}:
    \begin{solution}
    Если мн-во занятых комнат бесконечно, то:
    \begin{center}
    \begin{tabular}{ |c|c|c|c|c|c|c|c|c|c|c } 
     \hline
     0 & 1 & 0 & 0 & 1 & 1 & 0 & 1 & 0 & \ldots \\ 
     \hline
    \end{tabular}
    (1 - занято, 0 - свободно) $\Rightarrow$ Переносим 1 в самый ранний 0 для всех 1
\begin{comment}
    (Как делать стрелки между ячейками таблиц)???
\end{comment}
    \end{center}
    \end{solution}
  \item [3) ] 2 гранд-отеля, полностью заняты. Один закрылся, как всех заселить?
    \begin{solution}
  ~\newline
    \begin{center}
    \begin{tabular}{ |c|c|c|c|c|c|c| } 
     \hline
     0 & 1 & 2 & 3 & 4 & 5 & \ldots \\ 
     \hline
    \end{tabular}
    \end{center}
    \begin{center}
    \begin{tabular}{ |c|c|c|c|c|c|c| } 
     \hline
     a & b & c & d & e & f & \ldots \\ 
     \hline
    \end{tabular}
    \end{center}
    $\Rightarrow$
    \begin{center}
    \begin{tabular}{ |c|c|c|c|c|c|c| } 
     \hline
     0 & a & 1 & b & 2 & c & \ldots \\ 
     \hline
    \end{tabular}
    \end{center}
    \end{solution}
  \item [4) ]  Гранд-авенью, гранд-отелей. Цель: переселить всех в один отель:
    \begin{solution}
  ~\newline
    \begin{itemize}
      \item [Отель 0:] $\mapsto$ неч. номера
      \item [Отель 1: ] $\mapsto$ номера, кот. $\vdots 2, \not\vdots 4$
      \item [Отель 2:]  $\mapsto$ номера, кот. $\vdots 4, \not\vdots 8$
      \item [Отель $k$:] $\mapsto$ номера, кот $\vdots 2^{k}, \not\vdots 2^{k + 1}$
    \end{itemize}
    \end{solution}
\end{itemize}

\subsubsection{Счётных мн-в}

\begin{definition}
$A$ и $B$ \textbf{равномощны} ($A \cong B$), если $\exists$ биекция $f: A \rightarrow B$
\end{definition}
\begin{definition}
$A$ наз-ся \textbf{счётным}, если $A \cong \N$
\end{definition}
\begin{statement}
\begin{itemize}
  \item [1) ] $A$ счётно $\Rightarrow A \cup {x}$ счётно
  \item [2) ] Любое подмн-во счётного мн-ва конечно или счётно
  \item [3) ] $A, B$ счётны $\Rightarrow$ $A \cup B$ счётно
  \item [4) ] $A_0, A_1, \ldots$ - сч. $\Rightarrow \bigcup_{i = 0}^{\infty} A_i$ - сч.
    
    или: $A, B$ - сч. $\Rightarrow A \times B$ - сч.

\end{itemize}
\end{statement}
\begin{proof}
\begin{itemize}
  \item [1) ] $f: A \rightarrow \N$ - биекция 
    \[
    g: A\cup\set{x} \rightarrow \N \colon 
    \]
    \begin{equation*}
    \begin{system_and}
   g\left(x\right) = 0 \\
   g\left(y\right) = f\left(y\right) + 1, y \in A
    \end{system_and}
    \end{equation*}
  \item [2) ] \[
  f: A \rightarrow \N, \text{- биекция}; B \subset A
  \]
  \[
  g: B \rightarrow \N; g\left(x\right) = \#\set{y \in B | f\left(y\right) < f\left(x\right)}
  \]
\item[3) ] \[
f: A \rightarrow \N; g: B \rightarrow \N
\]
\begin{equation*}
h: A\cup B \rightarrow \N; h\left(x\right) = 
\begin{system_and}
2f\left(x\right), x \in A \\
2g\left(x\right) + 1, x \in B
\end{system_and}
\end{equation*}
  \item [4) ] \[
  f: A \rightarrow \N; 
  \]
  \[
  g: B \rightarrow \N; 
  \]
  \[
  h: A \times B \rightarrow \N; h\left(x, y\right) = 2^{f\left(x\right)} * \left(2g\left(y\right) + 1\right) - 1
  \]
\end{itemize}
\end{proof}

\subsubsection{Отношение равномощности}

\begin{statement}
Общие св-ва равномощности:

\begin{enumerate}
  \item [1) ] \textbf{Рефлексивность}: $A \cong A$ 

    (т. к. $id_A$ - биекция)
  \item [2)] \textbf{Симметричность}: $A \cong B \iff B \cong A$ 
   
    ($f$ - биекция $\iff$ $f^{-1}$ - биекция)
  \item [3) ] \textbf{Транзитивность}: $A \cong B, B \cong C \Rightarrow A \cong C$ 

    (т. к. композиция биекций - биекция)
\end{enumerate}
\end{statement}

\subsubsection{Сравнимость по мощности}
\begin{symb}
  \begin{itemize}
    \item
Нестрогая: $A \simlE B$, если $\exists B' \subset B, A \cong B'$ 

($A$ не более мозно чем $B$) 
\item  Строгая: $A \simless B$, если $A \simlE B, A \not\cong B$

  ($A$ менее мощно чем $B$)
  \end{itemize}
\end{symb}
\begin{statement}
Св-ва сравнимости по мощ-ти:
\begin{itemize}
  \item [1) ] Рефлексивность: $A \simlE A$; Антирефлексивность: $A \not\simless A$
  \item [2) ] Транзитивность: $A \simlE B, B \simlE C \Rightarrow A \simlE C$

    Для строгой сравнимости:
    \begin{proof}
    \[
    A \simless B, B \simless C \Rightarrow A \simless C
    \] 
    \[
    A \simlE C \text{ - из предыдущего}
    \]
    Нужно: $A \cong C$
    \end{proof}
\end{itemize}
\end{statement}

\begin{theorem} [Теорема Кантора-Бернштейна]
\[
A \simlE B, B \simlE A \Rightarrow A \cong B 
\]
\end{theorem}
\begin{proof}
\begin{itemize}
  \item [1) ] Пусть $f: A_0 \rightarrow B_1 \subset B_0$ - биекция \\ $g: B_0 \rightarrow A_1 \subset A_0$ - биекция
  \item [2) ] $B_{i + 1} = f\left(A_i\right); A_{i + 1} = g\left(B_i\right)$
  \item [3) ] $C_i = A_i \backslash A_{i + 1}; D_i = B_i \backslash B_{i + 1}$
  \item [4) ] $C = \bigcap_{i = 0}^{\infty} A_i; D = \bigcap_{i = 0}^{\infty} B_i$
\end{itemize}
\begin{statement}
  $C_i \cong D_{i + 1}$, т. е. $f: C_i \rightarrow D_{i + 1}$ - биекция

  Почему? Потому что:
  \[
  C_i = A_i \backslash A_{i + 1}; f\left(A_i\right) = B_{i + 1}, f\left(A_{i + 1}\right) = B_{i + 2}
  \]
  \[
  f\left(A_i \backslash A_{i + 1}\right) =\left(\text{т. к. $f$ - биекция}\right) f\left(A_i\right) \backslash f\left(A_{i + 1}\right) = B_{i + 1} \backslash B_{i + 2} = D_{i + 1} = f\left(C_i\right)
  \]
\end{statement}
\begin{statement}
\[
D_i \cong C_{i + 1} \left(\text{симметричо}\right)
\]
\end{statement}
\begin{consequence}
\[
C_0 \cong C_2 \cong C_4 \cong C_6 \cong \ldots 
\]
\[
C_0 \cong D_1 \cong D_3 \cong D_5 \ldots 
\]
\end{consequence}
\begin{statement}
\[
C \cong D
\]
\begin{proof}
$f$ - биекция \\ Пусть $x \in \bigcap_{i = 0}^{\infty} A_i \Rightarrow \forall i, x \in A_i \Rightarrow \forall i, f\left(x\right) \in B_{i + 1} \Rightarrow f\left(x\right) \in \bigcap_{i = 0}^{\infty} B_i$ \\
\textbf{Т. е. $f\left(C\right) \subset D$:}

Инъекция - наследуется

Сюрьекция: $y \in \bigcap_{i = 0}^{\infty}B_i \Rightarrow \forall i, y \in B_{i + 1} \Rightarrow \forall i, f^{-1}\left(y\right) \in A_i \Rightarrow f^{-1}\left(y\right) \in C$
\end{proof}

\[
  A = C \cap C_0 \cap C_1 \cap C_2 \cap C_3 \cap \ldots
\]
\[
  B = D \cap D_1 \cap D_2 \cap D_3 \cap D_4 \ldots
\]
При этом:
\begin{equation*}
\begin{system_and}
  C \cong D \\
\begin{system_and}
C_0 \cong D_1 \\
C_1 \cong D_0 
\end{system_and} \\
\begin{system_and}
C_2 \cong D_3 \\
C_3 \cong D_2
\end{system_and} \\
\vdots
\end{system_and} \Rightarrow A \cong B
\end{equation*} 
\end{statement}
\end{proof}
