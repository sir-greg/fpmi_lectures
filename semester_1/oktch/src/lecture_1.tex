\section{Инфа}
\textbf{Лектор:} Даниил Владимирович Мусатов (ему писать по поводу логики) и Райгородский Андрей Михайлович.
\begin{verbatim}
telega: @musatych
\end{verbatim}
\textbf{Об отсутствии на КР писать в тг заранее}

Задачи на КР:
\begin{itemize}
    \item Тестовые 0.8 (Или отдельно написано)
    \item Обычные (можно пересдать (1(на КР)/0.8(На след. КР)/0.5(В ДЗ)))
\end{itemize}

Задачи на ДЗ:
\begin{itemize}
    \item Перешедшие с КР (0.5)
    \item Обычные (1)
    \item Дополн. (1.5)
\end{itemize}
\begin{note}
Чтобы получить доп. задачу, нужно решить \textbf{все} обычные задачи по какой-то теме.
\end{note}

\section{Основные понятия теор. множеств}
\begin{symb}
    $x \in A \iff$ \textbf{элемент } $x$ принадлежит \textbf{мн-ву} $A$.
\end{symb}
\begin{definition}
    \textbf{Пустое мн-во} $\emptyset$ - мн-во, не содержащее ни одного эл-та.
\end{definition}
\begin{definition}
    $A$ \textbf{подмн-во} $B$ ($A \subset B$) $\iff$ 
    \[
    \forall x (x \in A \Rightarrow x \in B)
    .\] 
\end{definition}
\begin{note}
    $\forall A \colon \emptyset \subset A$
\end{note}
\begin{note}
\[
\forall x (x \in \emptyset \Rightarrow x \in A)
.\] 
\end{note}
\textbf{Свойство отношения подмножества:}
\begin{itemize}
    \item \textbf{Рефлексивность}: $A \subset A$
    \item\textbf{Транзитивность}: $A \subset B, B \subset C \Rightarrow A \subset C$ - 
    \item \textbf{Антисимметричность}: $A \subset B, B \subset A \Rightarrow A = B$
\end{itemize}
\begin{definition}[Равенство мн-в]
$A = B \iff $, если $A$ и $B$ содержат одни и те же эл-ты.
\end{definition}
\textbf{Запись конечного мн-ва:} $\{a, b, c\}$
 \begin{note}
 Из опр. рав-ва следует, что \textbf{кратность и порядок записи не важен}:
 \begin{example}
    $\{a, b, c\} = \{a, c, b, b, b, a\}$
 \end{example}
 
\end{note}
 
\begin{note}
Отличие $\in$ и $\subset$ :
\[
A = \{a, \{b\}, c, \{c\}\}
.\] 
\[
\{a\} \subset A, \{a\} \not\in A
.\]  
\[
\{b\} \not\subset A, \{b\} \in A, \{\{b\}\} \subset A
.\] 
\[
\{c\} \subset A, \{c\}\in A
.\] 
\[
\{d\} \not\in A, \{ d\}\not\in A 
.\] 
\end{note}

\textbf{Конструкция нат. чисел на основе мн-в} 

$0 = \emptyset$

$1 = \{\emptyset\}$

$2 = \{\emptyset, \{\emptyset\}\}$

$3 = \{\emptyset, \{\emptyset\}, \{\emptyset, \{\emptyset\}\}\}$

$n + 1 = \{0, 1, 2, \cdots, n\}$
~\newline

\textbf{Операции над мн-вами}
\begin{enumerate}
    \item Объединение: $A \cup B = \{ x \colon x \in A \lor x \in B\}$
    \item Пересечение: $A \cap B = \{x \colon x \in A \land x \in B\}$ 
    \item Разность: $A \backslash B = \{x \colon x \in A \land x \not\in B\}$
    \item Дополнение: $\overline{A} = \{x \colon x \not\in A\}$
    \item Симметрическая разность: $A \triangle B = \{x \colon (x \in A \lor x \in B) \land (x \not\in A \cap B)\}$
\end{enumerate}

\begin{statement}
\[
A \cup (B \cap C) = (A \cup B) \cap (A \cup C)
.\] 
\end{statement}
\begin{proof}
    В одну сторону:
\[
x \in A \cup(B \cap C) \Rightarrow 
.\] 
\begin{enumerate}
    \item $x \in A \Rightarrow x \in A \cup B \text{ и } A \cup C \Rightarrow x \in A \cup B \land x \in A \cup C$
    \item $x \in B \cap C \Rightarrow x \in B \land x \in C \Rightarrow x \in A \cup B \land  x \in A \cup C \Rightarrow x \in (A \cup B) \cap (A \cup C)$
\end{enumerate}
\end{proof}

\section{Упорядоченные пары и кортежи}
\[
    (a, b), a - \text{ 1-ый эл-т}, b - \text{ 2-ой эл-т}
\]
\textbf{Требование: } $(a, b) = (c, d) \iff a = c \land b = d$

\begin{definition}[Упрощенное определение Куратовского]
    \[
        (a, b) = \{\{a, b\}, a\}
.\] 
        \[
            (a, a) = \{\{a\}, a\}
        .\] 


\end{definition}
\section{Парадокс Рассела}
\textbf{Определим $I$:}
\[
\{\{\{\cdots a\cdots \}\}\} = I \Rightarrow I \in I, \text{(беск. кол-во скобок)}
.\] 
\[
    (I, I) = \{\{I\}, I\} = I
.\] 

\textbf{Рассмотрим:} $M = \{x \colon x \not\in x\}$
\[
    M \overset{\text{?}}{\in } M
.\] 
\begin{itemize}
    \item Пусть $M \in M$. Тогда $x \not\in x$ верно для $x = M$. Тогда $M \not\in M$. Но тогда $x \not\in x$ неверно для $x = M$. Противоречие.
    \item Аналогично  $M \not\in M \Rightarrow $ получаем парадокс.
\end{itemize}

\begin{axiom} [Аксиома фундированности]
    Не сущ. беск цепочки:
    \[
        A_1 \ni A_2 \ni A_3 \ni \cdots
    \]
\end{axiom}
\begin{note}
Это запрещает мн-во $I \text{ и } M \in M$, а также даёт однозначную интерпретацию $(a, b)$ 

Если $\{a, b\} \in a$, то возникает беск. цепочка:
\[
\{a, b\} \ni a \ni \{a, b\} \ni a \cdots 
.\] 
\end{note}

\begin{definition}
\textbf{Кортежи} - расширение пары на много эл-ов.
\begin{example}
$(a, b, c, d) = (a, (b, (c, d)))$ - кортеж
\end{example}

\begin{definition}
\textbf{Декартово произведение} мн-в $A, B$:
 \[
A\times B = \{(a, b) \colon a \in A, b \in B\}
.\] 
\end{definition}


\end{definition}

