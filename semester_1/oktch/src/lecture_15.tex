\section{Лекция 15}
\begin{task}
\[
  (a_1, b_1), \ldots, (a_f, b_f) \in \Z^{2}
\]
\[
m \in \N
\]
Вопрос: при каком наим. $f$ можно гарантировать, что сумма каких-то $m$ пар по обоим коор-там делится на $m$.
\end{task}
\begin{note}
$f \geq 4m - 3$. Пример: $m - 1$ раз повторяем $(1, 1)$, затем $m - 1$ раз $(0, 1)$, потом $(0, 0)$ и $(1, 0)$.
\end{note}
Что думали люди:
\begin{itemize}
  \item Гипотеза Кемница: $f = 4m - 3$
  \item $90$-е --- Алон и Дубинер: $f \leq 6m - 5, m >$
  \item $2000$ год: Роньяи $f \leq 4m - 2$
  \item $2005$ год: Райер: $f = 4m - 3$
\end{itemize}
\begin{proof}
Докажем это для $m = p$ --- простое.
\[
F(x_1, \ldots, x_n) \text{ --- многочлен от $n$ переменных}
\]
\[
F(x) = a_n x^{n} + a_{n - 1}x^{n - 1} + \ldots + a_0
\]
\[
F(x, y) = \sum_{}^{} c(a, b) x^{a}y^{b}
\]
Причём члены суммы --- одночлены/мономы.
\begin{definition}
\textbf{Степень монома} --- сумма степеней входящих в него переменных
\end{definition}
\begin{definition}
  \textbf{Степень полинома} --- наиб. степень мономов.
\end{definition}
\begin{theorem}[Шевалле-Варнинга]
\label{th:shevalle-warning}
Пусть $F_1, \ldots, F_k \in \Z_p[x_1, \ldots, x_n]$  \\
Пусть $\deg F_1 + \ldots + \deg F_k < n$. Рассм. систему сравнений:
\[
\begin{cases}
F_1(x_1, \ldots, x_n) \equiv 0 \pmod p \\
\vdots \\
F_k(x_1, \ldots, x_n) \equiv 0 \pmod p
\end{cases}
\]
\textbf{Утверждение теоремы}: Если $(0, \ldots, 0)$ --- решение системы, то $\exists (x_1, \ldots, x_n)$ --- нетривиальный набор, кот. тоже явл-ся решением системы.
\end{theorem}
\begin{lemma}
Пусть $(a_1, b_1), \ldots, (a_{3p}, b_{3p}) \in \Z^{2}$ и $\sum_{i = 1}^{3p} a_i \equiv \sum_{i = 1}^{3p} \equiv 0 \pmod p$. Тогда
\[
  \exists I \subset \set{1, 2, \ldots, 3p}, \left|I\right| = p, \sum_{i \in I}^{} a_i \equiv \sum_{i \in I}^{} b_i \equiv 0 \pmod p
\]
\end{lemma}
\begin{proof}
 Сделаем 3 многочлена:
 \[
 F_1(x_1, \ldots, x_{3p - 1}) = \sum_{i = 1}^{3p - 1} a_i x_{i}^{p - 1}
 \]
 \[
 F_2(\ldots) = \sum_{i = 1}^{3p - 1} b_i x_i^{p - 1}
 \]
 \[
 F_3(\ldots) = \sum_{ i = 1}^{3p - 1} x_i^{p - 1}
 \]
 \[
 \deg F_1 + \deg F_2 + \deg F_3 = 3p - 3 < 3p - 1
 \]
 Заметим, что $(0, \ldots, 0)$ --- удовл. трём мн-нам $\Rightarrow$ по т. Шевалле-Варнинга $\exists (x_1, \ldots, x_n)$, удовл. трём мн-нам, в кот. не все равны $0$ \\
 Обозначим $J$ --- мн-во номеров этих $x_i$, кот. не равны $0$. \\
 Мы знаем:
 \[
   \sum_{i = 1}^{3p - 1} a_i x_i^{p - 1} \equiv 0 \pmod p
 \]
 Заметим, что мы можем взять чисто ненулевые $x_i$, т. е. из $J$:
 \[
  \sum_{i \in J}^{} a_i x_i^{p - 1} \equiv 0 \pmod p \overset{\text{МТФ}}{\equiv} a_i \equiv 0 \pmod p
 \]
 Аналогично:
 \[
 \sum_{i \in J}^{} b_i \equiv 0 \pmod p
 \]
 \[
 \sum_{i \in J}^{} 1 \equiv 0 \pmod p
 \]
 \[
 \Rightarrow \left|J\right| \in \set{p, 2p}
 \]
 \[
 \left|J\right| = p \Rightarrow I := J
 \]
 \[
 \left|J\right| = 2p \Rightarrow I := \set{1, 2, 3, \ldots 3p} \backslash J
 \]
 Лемма доказана.
\end{proof}
Пусть $n = 4p - 2$. Предположим, что $\forall I \subset \set{1, \ldots, n}, \left|I\right| = p$, либо $\sum_{i \in I}^{} a_i \not\equiv 0 \pmod p$, либо $\sum_{i \in I}^{} b_i \not\equiv 0 \pmod p$ (Заметим, что отсюда следует то же самое и для $\left|I\right| = 3p$ по доказанной лемме)\\
Введём многочлен-КРОКОДИЛ:
\[
F(x_1, \ldots, x_n) = \left(\left(\sum_{ i = 1}^{n} a_i x_i\right)^{p - 1} - 1\right) \cdot \left(\left(\sum_{i = 1}^{n} b_i x_i\right)^{p - 1} - 1\right) \cdot
\]
\[
 \cdot \left(\left(\sum_{i = 1}^{n} x_i\right)^{p - 1} - 1\right) \cdot \left(\sigma_p(x_1, \ldots, x_n) - 2\right)
\]
Где $\sigma_p(x_1, \ldots, x_n)$ --- симметрический мн-н:
\[
\sigma_1(x_1,\ldots, x_n) = x_1 + \ldots + x_n
\]
\[
\sigma_2(x_1, \ldots, x_n) = x_1x_2 + \ldots + x_{n - 1}x_n
\]
\[
  \sigma(3)(x_1, \ldots, x_n) = x_1x_2x_3 + \ldots + x_{n - 2}x_{n - 1}x_n
\]
\[
\vdots
\]
Разберём КРОКОДИЛА по косточкам (битикам):
\[
  (x_1, \ldots, x_n) \in \set{0, 1}^{n}
\]
\begin{itemize}
  \item [1) ] Пусть число ненулевых коор-т равно $p$ или $3p$, $I$ --- мн-во ненулвых коор-т:
    \[
    \sum_{i = 1}^{n} a_i x_i = \sum_{i \in I}^{} a_i
    \]
    \[
    \sum_{i = 1}^{n} b_i x_i = \sum_{i \in I}^{} b_i
    \]
    Тогда $F(x_1, \ldots, x_n) \equiv 0 \pmod p$
  \item [2) ] Пусть число ненулевых коор-т равно $2p$, тогда:
    \[
    \sigma_p(x_1, \ldots, x_n) = C_{2p}^{p} \equiv 2 \pmod p
    \]
    \[
    \Rightarrow F \equiv 0 \pmod p
    \]
  \item [3) ] Пусть мн-во ненулевых коор-т имеет мощность, \underline{не делящуюся на $p$}.
    \[
    \Rightarrow \left(\left(\sum_{i = 1}^{n} x_i\right)^{p - 1} - 1\right) \equiv 0 \pmod p
    \]
    \[
    \Rightarrow F \equiv 0 \pmod p
    \]
  \item [4) ] Остался единственный случай, когда $(x_1, \ldots, x_n) = (0, 0, \ldots, 0)$, тогда:
    \[
    F(0, 0,\ldots, 0) = 2
    \]
\end{itemize}
Раскроем скобки, получим что слагаемое имеет соотв. вид, изменим его так:
\[
  c x_1^{\alpha_1} \ldots x_n^{\alpha_n} \rightarrow cx_1^{\beta_1}\ldots x_n^{\beta_n}
\]
\[
  \alpha_i = 0 \Rightarrow \beta_i = 0
\]
\[
  \alpha_i \geq 1 \Rightarrow \beta_i = 1
\]
Получаем полином $\widetilde{F}(x_1, \ldots, x_n)$. \\
УЛЬТРА МЕГА КАТАРСИС: \\
на $(x_1, \ldots, x_n) \in \set{0, 1}^{n}\Rightarrow F = \widetilde{F}$. Получаем, что:
\[
\widetilde{F} = 2(1 - x_1)\ldots(1 - x_n)
\]
\[
\deg \widetilde{F} = n = 4p - 2
\]
\end{proof}
