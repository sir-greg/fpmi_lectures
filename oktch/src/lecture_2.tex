\section{Отображения и соответствия}
\begin{definition}
\textbf{Соответствие} (или \textbf{многозначная ф-ция}, или \textbf{точечно-множ. отображение}) - подмн-во декартова произведения мн-в $A$ и $B$.

$F \subset A \times B$ - соответствие между $A$ и $B$
\end{definition}
\begin{note}
\textbf{Непустозначное соответствие}: $\forall x, \exists y \colon  (x, y) \in F$
\end{note}

***Картинки графика и двудольного графа***

\begin{definition}
\textbf{Отображение} - однозначное соотв.
\[
\forall x, \exists! y\colon (x, y) \in f
\] 
\[
\forall \text{- для любого}, \exists! - \text{существует единственный}
\] 
\end{definition}
\begin{definition}
\textbf{Частично определённая ф-ция}:
\[
  \forall x \colon (\neg \exists y \colon (x, y) \in F) \lor (\exists ! y \colon  (x, y) \in F)
\]
\end{definition}
\begin{definition}
\textbf{Инъекция} - отображение, т. ч. $\forall x, y (x \neq y \rightarrow f(x) \neq f(y))$
\end{definition}
\begin{definition}
    \textbf{$f(x)$} - тот элемент $z \colon (x, z) \in f$
\end{definition}
\begin{definition}
    $F(x)$ - образ $x \iff F(x) = \{z \colon (x, z) \in F\}$
\end{definition}
\begin{definition}
    \textbf{Инъективные соответствия:}
    \[
    \forall x, y (x \neq y \rightarrow F(x) \cap F(y) = \emptyset)
    \] 
\end{definition}
\begin{definition}
    \textbf{Сюрьекция} - отображение, т. ч. $\forall y, \exists x (y = f(x))$
\end{definition}
\begin{definition}
    \textbf{Сюрьективное соответствие}:
    \[
    \forall y, \exists x \colon  (x, y) \in F
    \] 
    Или по другому: $\forall y, \exists x \colon y \in F(x) $
\end{definition}
\begin{definition}
    \textbf{Биекция} - отображение, которое одновременно сюрьекция и инъекция.

    \textbf{Биекция = отображение + сюрьекция + инъекция}
\end{definition}
\begin{note}
    Отдельного понятия биективного соответствия \underline{нет}.
\end{note}
\begin{definition}
    \textbf{Обратное соответствие} $F \subset A \times B$ - $F^{-1} \subset B \times A$:
    \[
        (x, y) \in F \iff (y, x) \in F^{-1}
    \] 
\end{definition}
\begin{theorem}
    \textbf{F - Биекция} $\iff$ \textbf{F - взаимнооднозначное соответствие} (т. е. $F$ и $F^{-1}$ - отображения)
\end{theorem}
\begin{note}
        \textbf{Частично опред. ф-ция + непустознач. соотв = отображение}
\end{note}
\begin{proof}
    ~\newline
        \begin{itemize}
            \item $F$ явл. инъективным соответствием $\iff$ $F^{-1} - $ частично опред. ф-ция.
            \item $F$ явл. сюрьективным соответствием $\iff F^{-1} - $ непустозначное соотв.
        \end{itemize}
\end{proof}

\subsection{Образ и прообраз}
\begin{definition}
Пусть $S \subset A$. Тогда образ $S$: 
\begin{itemize}
    \item Для отображения: $f(S) = \{f(x) | x \in S\}$ 
    \item Для соотв.: $F(S) = \bigcup_{x \in S}^{} F(x)$
\end{itemize}
\end{definition}

\begin{definition}
Пусть $T \subset B$. Тогда прообраз $T$:
\begin{itemize}
    \item Для отображения: $f^{-1}(T) = \{x | f(x) \in T\}$
    \item Для соотв.: $F^{-1} = \{x | F(x) \cap T \neq \emptyset\}$
\end{itemize}
\end{definition}

\begin{statement}
$F(S \cap Q) \subset F(S) \cap F(Q)$
\end{statement}
\begin{proof}
Пусть $y \in F(S\cap Q) \Rightarrow \exists x \in S \cap  Q \colon  y \in F(x)$:
\begin{equation*}
\begin{system_and}
\exists x \in S \colon  y \in F(x) \\
\exists x \in Q \colon  y \in F(x)
\end{system_and}
\Rightarrow
\begin{system_and}
y \in F(S) \\
y \in F(Q) 
\end{system_and}
\Rightarrow y \in F(S) \cap F(Q)
\end{equation*}
\end{proof}
\begin{statement}[\underline{Обратное.}]
Если $F$ - инъективно, то
\[
    F(S) \cap F(Q) \subset F(S \cap  Q)
\]
\end{statement}
\begin{proof}
    \[
    y \in F(S) \cap  F(Q) \Rightarrow
    \] 
    \begin{equation*}
    \begin{system_and}
    y \in F(S) \\
    y \in F(Q)
    \end{system_and}
    \Rightarrow
    \begin{system_and}
    \exists x_1 \in S \colon y \in F(x_1) \\
    \exists x_2 \in Q \colon y \in F(x_2)
    \end{system_and}
    \Rightarrow x_1 \neq x_2 \Rightarrow \text{Нарушает инъективность} 
    \end{equation*}
    \[
     \Rightarrow x_1 = x_2 = x \Rightarrow \exists x \in S \cap Q \colon y \in F(x)
    \] 
\end{proof}
\subsection{Композиция}
\begin{definition}
 \textbf{Композиция отображений} $f \circ g$, опр. так:
 \[
 f \circ g(x) = f(g(x))
 \] 
\end{definition}
\begin{definition}
    \textbf{Композиция соотв.} $F \circ G$
    \begin{equation*}
    \begin{system_and}
    F: B \rightarrow C \\
    G: A \rightarrow B
    \end{system_and}
    \Rightarrow
    F \circ G(x) = F(G(x))
    \end{equation*}
    Причём $G(x)$ - это мн-во значений $\Rightarrow$ $F(G(x))$ - образ $G(x)$
    
    Или, эквив.: $(x, z) \in F \circ G \iff \exists y ((x, y) \in G \land (y, z) \in F)$
\end{definition}
\textbf{Свойства композиции:}
\begin{enumerate}
    \item [1) ] \textbf{Ассоциативность: } $F \circ (G \circ H) = (F \circ G) \circ H$
    \item [2) ] \textbf{Отсутствие} коммутативности (в общем случае): $F \circ G \neq G \circ F$
\end{enumerate}

\begin{symb}
\textbf{Тождественное отображение: } \[
    id_A: A \rightarrow A
\] 
\[
    id_A(x) = x
\] 
\[
G: A \rightarrow B \Rightarrow G \circ id_A(x) = id_B \circ G(x) = G(x)
\] 
\end{symb}
\begin{statement}
    Если $F: A \rightarrow A$ - биекция, то:
    \[
    F \circ F^{-1} = id_A = F^{-1} \circ F
    \] 
\end{statement}
\begin{symb}
\textbf{Мн-во всех отображений из $A$ в $B$} будем называть $B^{A}$
\end{symb}
\begin{statement}
Если $|A| = n$ и $|B| = k$, то $|B^{A}| = k^{n}$
\end{statement}
\begin{theorem}
Пусть $A, B, C$ - мн-ва. Тогда:
\begin{enumerate}
    \item [1) ] $A^{C} \times B^{C} \sim (A \times B)^{C}$
    \item [2) ] $A^{B \cup C} \sim A^{B} \times A^{C}, B \cap C \neq \emptyset$
    \item [3) ] $A^{B\times C} \sim (A^{B})^{C}$ 
\end{enumerate}
\end{theorem}
\begin{proof}
\begin{enumerate}
    \item [1) ] \begin{equation*}
            \begin{system_and}
                f: C \rightarrow A \\
                g: C \rightarrow B
            \end{system_and} \longleftrightarrow h: C \rightarrow A \times B, h(x) = (f(x), g(x))
    \end{equation*} 
\item [2) ] \begin{equation*}
    \begin{system_and}
        f: B \rightarrow A \\
        g: C \rightarrow A
    \end{system_and} \longleftrightarrow h: B \cup C \rightarrow A \Rightarrow
    h(x) = 
    \begin{system_and}
    f(x), x \in B \\
    g(x), x \in C
    \end{system_and}
\end{equation*}
\item [3) ] \begin{equation*}
        \begin{system_and}
f: B \times C \rightarrow A \\
g: C \rightarrow A^{B}
        \end{system_and} \Rightarrow g(x): B \rightarrow A \Rightarrow g(x)(z) = f(z, x) 
\end{equation*} 
\end{enumerate}
\end{proof}
