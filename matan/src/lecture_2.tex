%\documentclass[12pt]{article}
%\usepackage[T1, T2A]{fontenc}
%\usepackage[utf8]{inputenc}
%\usepackage[russian]{babel}
%\usepackage{amsmath}
%\usepackage{amsthm}
%\usepackage{amssymb}
%\usepackage{esvect}
%\usepackage{listings}
%\usepackage{xcolor}
%\usepackage{mathrsfs}
%
%% for large comments
%\usepackage{blindtext, xcolor}
%\usepackage{comment}
%
%% for inkscape pictures
%\usepackage{import}
%\usepackage{pdfpages}
%\usepackage{transparent}
%\usepackage{xcolor}
%
%\newcommand{\incfig}[2][1]{%
%    \def\svgwidth{#1\columnwidth}
%    \import{./figures/}{#2.pdf_tex}
%}
%
%\pdfsuppresswarningpagegroup=1
%
%% for systems of equations
%\newenvironment{system_and}%
%{\left\lbrace\begin{array}{@{}l@{}}}%
%{\end{array}\right.}
%% for unions of equations
%\newenvironment{system_or}%
%{\left\lbrack\begin{array}{@{}l@{}}}%
%{\end{array}\right.}
%
%\renewcommand{\C}{\mathbb{C}}
%\newcommand{\R}{\mathbb{R}}
%\newcommand{\Q}{\mathbb{Q}}
%\newcommand{\Z}{\mathbb{Z}}
%\newcommand{\N}{\mathbb{N}}
%
%\newcommand{\floor}[1]{\left\lfloor #1 \right\rfloor}
%\newcommand{\ceil}[1]{\left\lceil #1 \right\rceil}
%
%% style of code listings
%%\definecolor{codegreen}{rgb}{0,0.6,0}
%%\definecolor{codegray}{rgb}{0.5,0.5,0.5}
%%\definecolor{codepurple}{rgb}{0.58,0,0.82}
%%\definecolor{backcolour}{rgb}{0.95,0.95,0.92}
%%
%%\lstdefinestyle{mystyle}{
%%    backgroundcolor=\color{backcolour},
%%    commentstyle=\color{codegreen},
%%    keywordstyle=\color{magenta},
%%    numberstyle=\tiny\color{codegray},
%%    stringstyle=\color{codepurple},
%%    basicstyle=\ttfamily,
%%    breakatwhitespace=false,
%%    breaklines=true,
%%    captionpos=b,
%%    keepspaces=true,
%%    numbers=left,
%%    numbersep=5pt,
%%    showspaces=false,
%%    showstringspaces=false,
%%    showtabs=false,
%%    tabsize=4
%%}
%
%\newtheorem{theorem}{\underline{Теорема}}[section]
%\newtheorem{lemma}[theorem]{\underline{Лемма}}
%\newtheorem{statement}{\underline{Утверждение}}[section]
%\newtheorem{axiom}{\underline{Аксиома}}[section]
%\newtheorem*{note}{\underline{Замечание}}
%\newtheorem*{symb}{\underline{Обозначение}}
%\newtheorem*{example}{\underline{Пример}}
%\newtheorem*{consequence}{\underline{Следствие}}
%\newtheorem*{solution}{\underline{Решение}}
%
%\theoremstyle{definition}
%\newtheorem{definition}{\underline{Определение}}[section]
%
%\theoremstyle{definition}
%\newtheorem{task}{\underline{Задача}}[section]
%
%\title{Матан. Лекция 2}
%\author{Сергей Григорян}

\section{Некот. обозначения}
\begin{itemize}
    \item $a < b \iff a \leq b  \land  a \neq b$
    \item $a \geq b \iff b \leq a$
    \item $a > b \iff b < a$
    \item $ a - b = a + (-b) $
    \item $\frac{a}{b} = a * b^{-1} (b\neq0)$
\end{itemize}

\section{Чем занимаемся дальше}
Все дальнейшее сводим к аксиомам:
\begin{example}
\begin{enumerate}
    \item $\forall a  \in R \colon a * 0 = 0$
        \begin{proof}
            \[
        a \cdot  0 = a \cdot (0 + 0) = a \cdot  0 + a \cdot  0     | -a\cdot  0
            .\] 
            \[
            a * 0 + (-a * 0) = a * 0 + (a * 0 + (-a * 0))
            .\] 
            \[
            0 = a * 0 + 0 = a * 0
            .\] 
        \end{proof}
    \item $(-1) * a + 1 * a = ((-1) + 1) * a = 0 * a = 0$
\end{enumerate}
\end{example}
\begin{example}
\begin{enumerate}
    \item $\forall a, b  \in R (a \leq b \Rightarrow -b \leq -a)$ 
        \[
        -b = a - a - b \leq b - a - b = -a
        .\] 
    \item $\forall a  \in  R \backslash \{0\} \colon  (a^{2} > 0)$ 
        \begin{proof}
        \begin{itemize}
            \item [a) ] $a > 0 \Rightarrow a^{2} > 0$
            \item [b) ] $a < 0 \Rightarrow -a > 0 \Rightarrow (-a)(-a) > 0 \Rightarrow -(-a^{2}) = a^{2}$
        \end{itemize}
        \end{proof}
        
\end{enumerate}
\end{example}

\begin{task}
$P = \{x  \in  R \colon  0 < x\}$

Док-те, что  :
\begin{enumerate}
    \item [1) ] $x, y  \in P \Rightarrow x + y, x * y  \in P$
    \item [2) ] $\forall x  \in R \backslash \{0\} (x  \in P  \lor -x \in P)$
\end{enumerate}
\end{task}

\begin{definition}
    \begin{equation*}
        |x| = \begin{system_and}
            x, x \geq 0 \\
            -x, x < 0
        \end{system_and}
    \end{equation*}
\end{definition}
\begin{example}
\begin{enumerate}
    \item Если $a  \in \R $ и $M \geq 0$, то ($|a| \leq M \iff -M \leq a \leq M$)
        \begin{proof}
        $|a| \leq M \Rightarrow -|a| \geq -M$
        \begin{itemize}
            \item [a) ] $a \geq 0, -M \leq 0 \leq a = |a| \leq M$ 
            \item [b) ] $a < 0, -M \leq -|a| = a < 0 \leq M$
        \end{itemize}
        \end{proof}
    \item $\forall a, b \in R (|a + b| \leq |a| + |b|)$ 
        \begin{proof}
       \[
            \pm a \leq |a|, \pm b \leq |b|
       .\]  
       \[
       \Rightarrow \pm (a + b) \leq |a| + |b| \Rightarrow |a + b| \leq |a| + |b|
       \] 
        \end{proof}
        
\end{enumerate}
\end{example}
\section{Множество $\N$}

\begin{definition}
Мн-во $S \subset \R$ наз-ся \textbf{индуктивным}, если $1  \in S$ и ($x \in S \Rightarrow x + 1 \in S$)
\end{definition}
\begin{note}
$\N$ - пересечение всех индуктивных мн-в.
\end{note}
На определении $\N$ основан \textbf{принцип мат. индукции.}

Пусть $P(n), n  \in \N$. Если $P(1) $ - истина и $(\forall n (P(n) \text{ - ист. }  \Rightarrow P(n + 1) \text{ - ист.}))$. То $P(n)$ - истина для $\forall n  \in N$

$S = \{n  \in \N \colon P(n) \text{ - истина}\} \subset \N $ - индуктивно. $\Rightarrow S = \N$ 

\begin{note}
Если $x, y  \in \N, x < y \text{, то } y - x = n  \in N$, в частности, $y = x + n \geq x + 1$
\end{note}
\begin{theorem}
Пусть $A \subset N$ - непустое, тогда $\exists m = min(A) (m \in A \colon \forall n  \in A (m \leq n))$
\end{theorem}
\begin{proof} 
~\newline

Предположим, что в $A$ нет мин. эл-та. 

Рассм. $M = \{x  \in \N \colon \forall  n  \in A (x < n)\}$

$1  \in M \text{ }(1  \not \in A)$ 

Пусть $x  \in M$. Предпл., что $x + 1  \not \in M$:

$x + 1  \not \in M \iff \exists m \in A \colon  (x + 1 \geq m)$

По опр-ю $x \in M \Rightarrow x < m \Rightarrow x + 1 \leq m \Rightarrow m = min(A) !!!$

Итак $1 \in M (x  \in M \Rightarrow x + 1  \in  M) \Rightarrow M \subset \N \Rightarrow M = \N \Rightarrow A = \emptyset !!!$
\end{proof}

\section{Множества  $\Z$ и $\Q$}
\[
    \Z = -\N \cup \{0\} \cup \N
\]
\[
    \Q = \{\frac{m}{n} \colon m \in \Z \land n \in N\}
\]

\begin{example}[Применение аксиомы непрерывности]
\[
A = \{a \in \R \colon  a > 0 \land a^{2} < 2\} \ni 1
.\] 
\[
B = \{b \in \R \colon  b > 0 \land b^{2} > 2\} \ni 2
.\] 
Пусть $a \in A, b \in B$

\[
0 < b^{2} - a^{2} = (b - a)(b + a) \Rightarrow 0 < b - a \Rightarrow a < b
.\] 
По аксиоме непрерывности $\exists c \in \R \colon \forall  a \in A, b \in B \colon (a \leq c \leq b)$

В част-ти $1 < c < 2$. Покажем, что  $c^{2} = 2$

Предпл. что $c^{2} < 2 \iff c \in A$. Пусть $\varepsilon \in (0;1)$ ; тогда:
\[
    (c + \varepsilon)^{2} = c^{2} + \varepsilon(2c + \varepsilon) < c^{2} + 5\varepsilon
.\] 
\[
    \varepsilon \leq \frac{2 - c^{2}}{5} 
.\] 
\[
    (c + \varepsilon)^{2} < c^{2} + 5\varepsilon \leq c^{2} + 2 - c^{2} = 2 \Rightarrow c + \varepsilon \in A!!!
.\] 

Аналогичным образом, доказываем, что $c^{2} > 2$ не выполн-ся.
\[
    \Rightarrow c^{2} = 2
\]
\end{example}

\section{Точные грани числовых мн-в}
\begin{definition}
Пусть $E \subset \R$ - непусто.

Число $M$ наз-ся \textbf{верхней гранью} мн-ва $E$, если $\forall x \in E (x \leq M)$

Мн-во $E$ наз-ся \textbf{ограниченным сверху}, если $\exists $ хотя бы одна верхняя грань для $E$.

Число $M$ наз-ся \textbf{нижней гранью} мн-ва $E$, если $\forall x \in E (x \geq M)$

Мн-во $E$ наз-ся \textbf{ограниченным снизу}, если $\exists $ хотя бы одна нижняя грань для $E$.

Мн-во $E$ \textbf{ограничено}, если $E$ ограничено сверху и снизу.
\end{definition}
\begin{task}
Док-ть: $E \text{ - огранич. } \iff \exists C > 0 \colon  \forall x \in E (|x| \leq C)$
\end{task}

\begin{definition}
Пусть $E \subset \R$ - непустое числовое мн-во. Наименьшая из верхних граней мн-ва $E$ наз-ся \textbf{точной верхней гранью (супремумом)} мн-ва $E$ ($\sup E$)

Наибольшая из нижних граней мн-ва $E$ наз-ся \textbf{точной нижней гранью (инфимумом)} мн-ва $E$ ($\inf E$)
\end{definition}
\begin{note}
Определение точных граней можно записать на языке нер-ств:

\begin{equation}
    c = \sup E \iff
    .
\end{equation}
\begin{enumerate}
    \item [1) ] $\forall x \in E (x \leq c);$ 
    \item [2) ] $\forall  \varepsilon > 0  \exists  x \in E (x > c - \varepsilon)$
\end{enumerate}


\begin{equation}
b = \inf E \iff
.\end{equation} 
\begin{enumerate}
    \item [1) ] $\forall x \in E (x \geq b);$ 
    \item [2) ] $\forall \varepsilon > 0, \exists x' \in E (x' < b + \varepsilon)$
\end{enumerate}

Действ-но, $1)$ в $(1)$ означает, что $c$ - верх. грань $E$. $2)$ в $(1)$ означ, что любое $c' < c$ не явл. верх. гр. $E$. Сл-но, $c$ - точная верхняя грань $E$. Аналогично для  $(2)$.
\end{note}

\begin{theorem}[Принцип полноты Вейерштрасса]
Всякое непустое огр. сверху (снизу) мн-во имеет точную верхнюю (нижнюю) грань.
\end{theorem}
\begin{proof}
Пусть $A \subset \R$ и ограничено сверху. 

Рассм. $B = \{b \in \R\colon b \text{ - верх. грань $A$}\}$. Тогда $B \neq \emptyset$ и $\forall a \in A \forall b \in B (a \leq b)$. По аксиоме непр-ти $\exists c \in \R \colon \forall a \in A, \forall b \in B(a \leq c \leq b)$. 

Из нер-ва $a \leq c \Rightarrow c $ верх. грань $A$

Из правого нер-ва любое $c' < c \colon c' \not\in B$, т.е. $c'$ не явл. верх. гранью $A$. Сл-но, $c = \sup A$.
\end{proof} 
