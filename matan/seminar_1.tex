\documentclass[a4, 12pt]{article}
\usepackage[T1, T2A]{fontenc}
\usepackage[utf8]{inputenc}
\usepackage[russian]{babel}
\usepackage{amsmath}
\usepackage{amsthm}
\usepackage{amssymb}
\usepackage{esvect}
\usepackage{listings}
\usepackage{xcolor}


% for large comments
\usepackage{blindtext, xcolor}
\usepackage{comment}

% for inkscape pictures
\usepackage{import}
\usepackage{xifthen}
\usepackage{pdfpages}
\usepackage{transparent}

\newcommand{\incfig}[1]{%
    \def\svgwidth{\columnwidth}
    \import{./figures/}{#1.pdf_tex}
}

\renewcommand{\C}{\mathbb{C}}
\newcommand{\R}{\mathbb{R}}
\newcommand{\Q}{\mathbb{Q}}
\newcommand{\Z}{\mathbb{Z}}
\newcommand{\N}{\mathbb{N}}

\newcommand{\floor}[1]{\left\lfloor #1 \right\rfloor}
\newcommand{\ceil}[1]{\left\lceil #1 \right\rceil}

% style of code listings
\definecolor{codegreen}{rgb}{0,0.6,0}
\definecolor{codegray}{rgb}{0.5,0.5,0.5}
\definecolor{codepurple}{rgb}{0.58,0,0.82}
\definecolor{backcolour}{rgb}{0.95,0.95,0.92}

\lstdefinestyle{mystyle}{
    backgroundcolor=\color{backcolour},
    commentstyle=\color{codegreen},
    keywordstyle=\color{magenta},
    numberstyle=\tiny\color{codegray},
    stringstyle=\color{codepurple},
    basicstyle=\ttfamily\footnotesize,
    breakatwhitespace=false,
    breaklines=true,
    captionpos=b,
    keepspaces=true,
    numbers=left,
    numbersep=5pt,
    showspaces=false,
    showstringspaces=false,
    showtabs=false,
    tabsize=4
}

\newtheorem{theorem}{\underline{Теорема}}[section]
\newtheorem{lemma}[theorem]{\underlind{Лемма}}
\newtheorem{statement}{\underline{Утверждение}}[section]
\newtheorem*{note}{\underline{Замечание}}
\newtheorem*{symb}{\underline{Обозначение}}
\newtheorem*{example}{\underline{Пример}}
\newtheorem*{consequence}{\underline{Следствие}}
\newtheorem*{solution}{\underline{Решение}}

\theoremstyle{definition}
\newtheorem{definition}{\underline{Определение}}[section]

\title{Введение в матан. \\ Семинар 1}
\author{Сергей Григорян}

\begin{document}
\maketitle
\newpage
\section{Инфа}
~\newline

\textbf{Семинарист:} Останин Павел Антонович

\textbf{Телега:} 
\begin{verbatim}
@paulostanin
\end{verbatim}

\textbf{БРС} = 30 (сем) + 30 (письм экз.) + 60 (уст. ответ)

\textbf{30 cем:} коллоквиум после 1-ого задания (7), семестровая КР после 2-ого зад (7), семинарская КР по 3-ему заданию (4), 3 тетради - 2б, активность на семинаре (3), проверка теор. знаний (3 очка)

\textbf{Подготовка к КР} - решение КР прошлых лет (прошлый, позапрошлый, ..)

Задания:
\begin{enumerate}
    \item 7-12 окт.
    \item 11-16 нояб.
    \item 9-14 дек.
\end{enumerate}

\section{Производные}

\begin{definition}
Пусть  $f$, определена в $(x_0 - \delta, x_0 + \delta)$. Если:
\[
\exists \lim_{h\to_0} \frac{f(x_0 + h) - f(x_0)}{h} 
,\] 
то он, называется производной $f'(x_0)$ функции $f$ в $x_0$.
\end{definition}

\begin{definition}
\textbf{Промежуток} - это подмн-во $\R \colon \forall a, b \in I$, все точки между $a$ и $b$ лежат между ними и тоже в $I$.
\end{definition}
\begin{definition}
\textbf{Дифференцирование} - это процесс вычисления производной.
\end{definition}

Пусть $f$ и $g$ - дифференцируемы в $x_0$, $\alpha$ и $\beta$ - константы. Тогда:
\begin{enumerate}
    \item $(\alpha f + \beta g) = \alpha f' + \beta g'$  
    \item $(fg)' = f'g + g'f$
    \item $(\frac{f}{g}) = \frac{f'g - fg'}{g^2}$
\end{enumerate}

\begin{note}
Часто удобнее делать вот так:
\[
    (\frac{f}{g}) = \frac{f'}{g} + f * (\frac{1}{g})' = \frac{f'}{g} - \frac{f}{g^{2}} * g'
.\] 
\end{note}
 


\section{Техника нахождения первообр. I}
\begin{definition}
\textbf{F} - это первообр. $f$ на пром-ке $I$, если $\forall x \in I \colon F'(x) = f(x)$
\end{definition}

\begin{definition}
\textbf{$\int f(x)dx$} - это мн-во всех первообразных $f$
\end{definition}

\begin{statement}
$\forall$ две первообразные $f$ отличаются на константу
\end{statement}

\begin{example}
~\newline
    
$\int \sin xdx = -\cos x + C$ 

$dx$ - по какой переменной интегрируем
\end{example}

\begin{example}
$\int \frac{dx}{x} = ln|x| + C $

\end{example}

\begin{example}
$\int x^\alpha dx = x^{\alpha+1} *\frac{1}{\alpha + 1} + C$
\end{example}

\begin{example}
$\int \frac{dx}{\sqrt{a^2 - x^2}} dx = \arcsin \frac{x}{a} + C   $
\end{example}

\begin{example}
$\int \frac{dx}{a^2 + x^2} dx = \frac{1}{a}\arcctg\frac{x}{a} + C $
\end{example}

\begin{example}
$\int \frac{dx}{\sqrt{a^2 + x^2}} = ln(x + \sqrt{a^2 + x^2}) + C$
\end{example}

\begin{example}
$\int \frac{dx}{a^2 - x^2} = \frac{1}{2a}ln|\frac{a + x}{a - x}| + C$
\end{example}

\begin{example}
\[
    \int f(\phi(x))\phi'(x) dx = F(\phi'(x)) + C 
.\] 
\[
    F' = f
.\] 
\end{example}

$\arctg$

\section{Дифференциа-}




\end{document}
