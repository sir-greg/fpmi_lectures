\documentclass[12pt]{article}
\usepackage[T1, T2A]{fontenc}
\usepackage[utf8]{inputenc}
\usepackage[russian]{babel}
\usepackage{amsmath}
\usepackage{amsthm}
\usepackage{amssymb}
\usepackage{esvect}
\usepackage{listings}
\usepackage{xcolor}
\usepackage{mathrsfs}


% for large comments
\usepackage{blindtext, xcolor}
\usepackage{comment}

% for inkscape pictures
\usepackage{import}
\usepackage{xifthen}
\usepackage{pdfpages}
\usepackage{transparent}

\newcommand{\incfig}[1]{%
    \def\svgwidth{\columnwidth}
    \import{./figures/}{#1.pdf_tex}
}

\renewcommand{\C}{\mathbb{C}}
\newcommand{\R}{\mathbb{R}}
\newcommand{\Q}{\mathbb{Q}}
\newcommand{\Z}{\mathbb{Z}}
\newcommand{\N}{\mathbb{N}}

\newcommand{\floor}[1]{\left\lfloor #1 \right\rfloor}
\newcommand{\ceil}[1]{\left\lceil #1 \right\rceil}

% style of code listings
%\definecolor{codegreen}{rgb}{0,0.6,0}
%\definecolor{codegray}{rgb}{0.5,0.5,0.5}
%\definecolor{codepurple}{rgb}{0.58,0,0.82}
%\definecolor{backcolour}{rgb}{0.95,0.95,0.92}
%
%\lstdefinestyle{mystyle}{
%    backgroundcolor=\color{backcolour},
%    commentstyle=\color{codegreen},
%    keywordstyle=\color{magenta},
%    numberstyle=\tiny\color{codegray},
%    stringstyle=\color{codepurple},
%    basicstyle=\ttfamily\footnotesize,
%    breakatwhitespace=false,
%    breaklines=true,
%    captionpos=b,
%    keepspaces=true,
%    numbers=left,
%    numbersep=5pt,
%    showspaces=false,
%    showstringspaces=false,
%    showtabs=false,
%    tabsize=4
%}

\newtheorem{theorem}{\underline{Теорема}}[section]
\newtheorem{lemma}[theorem]{\underlind{Лемма}}
\newtheorem{statement}{\underline{Утверждение}}[section]
\newtheorem*{note}{\underline{Замечание}}
\newtheorem*{symb}{\underline{Обозначение}}
\newtheorem*{example}{\underline{Пример}}
\newtheorem*{consequence}{\underline{Следствие}}
\newtheorem*{solution}{\underline{Решение}}

\theoremstyle{definition}
\newtheorem{definition}{\underline{Определение}}[section]
\theoremstyle{definition}
\newtheorem{task}{\underline{Задача}}[section]

\title{Введение в матан. \\ Лекция 1}
\author{Сергей Григорян}

\begin{document}
\maketitle
\newpage
\section{Инфа}
\textbf{Лектор:} Редкозубов Вадим Витальевич

\section{Учебники}
 \begin{itemize}
    \item Зорич. В. А. " Мат. анализ";
    \item Виноградов О. Л. "Мат. анализ".
 \end{itemize}

\section{П. 1. Действительные числа}
\subsection{Вспомогательные конструкции}
~\newline

$x \in \{a, b\} \Rightarrow x = a \text{ или } x = b$ - неуп. пара 

$(a, b) - \text{ уп. пара}$

 $(a, b) = (c, d) \iff a = c \text{ и } b = d$

$A, B - \text{ мн-ва}$, $A \cdot B = \{(a, b) \colon a \in A \lor b \in B\}$

\begin{definition}
Пусть $X, Y$ - мн-ва
Ф-цией $f\colon X \rightarrow Y$ наз-ся ф-ла $P(x, y)$, т. ч. $\forall x \in X$ сущ-ет утв. $y \in Y$,что $P(x, y)$ - истина. Пишут $y = f(x)$ или $f\colon x \Rightarrow y$.
\end{definition}
\begin{definition}
Ф-ции $f, g\colon X \rightarrow Y$ называются равными, если $\forall x \in X \colon (f(x) = g(x))$. Пишут $f = g$.
\end{definition}

\begin{symb}
$f: X \rightarrow Y$, $X$ - область опред. ф-ции

\begin{enumerate}
    \item $A \subset X$ 

    $f(A) = \{f(x) \colon x \in A\}$, образ $A$.

    $f(X)$ - мн-во значений $f$.

    \item $B \subset Y$

    $f^{-1}(B) = \{x \in X \colon f(x) \in B\}$ - прообраз $B$.

    \item $f: X \rightarrow Y, g: Z \rightarrow X \Rightarrow f \circ g: Z \rightarrow Y, f \circ g(z) = f(g(z))$ - композиция ф-ций $f$ и $g$.
\end{enumerate}
\end{symb}

\begin{statement}
$f \circ (g \circ h) = (f \circ g) \circ h$
\end{statement}

\begin{definition}
Ф-ция $f: X \rightarrow Y$ наз-ся  \textbf{инъекцией}, если $\forall x_1, x_2 \in X (f(x_1) = f(x_2) \Rightarrow x_1 = x_2), x_1 \neq x_2 \Rightarrow f(x_1) \neq f(x_2)$ 

\textbf{сюрьекцией}, если $f(X) = Y$

\textbf{биекцкией} = \textbf{сюрьекцией} + \textbf{инъекция}
\end{definition}
\begin{example}
\begin{enumerate}
    \item 
        
$f: \{0, 1, 2\} \rightarrow \{1, 2\}$ 

$f(0) = 1, f(1) = f(2) = 2$

Это \textbf{сюрьекция}
    \item
$f: \{1, 2\} \rightarrow \{0, 1, 2\}$ 

$f(1) = 2, f(2) = 1$ 

Это \textbf{инъекция}
\end{enumerate}

\end{example}

\begin{example}
$id: X \rightarrow X, \forall x \in X (id(x) = x)$ - это \textbf{тождественная ф-ция}.
\end{example}

\begin{example}
Пусть $f: X \rightarrow Y$ - биекция

$\Rightarrow y = f(x)$ - имеет \textbf{1 решение}. Тогда:

$f^{-1}: Y \rightarrow X, x = f^{-1}(y) \iff y = f(x)$ - обратная к $f$ ф-ция.

$f^{-1} \circ f = id_X, f \circ f^{-1} = id_Y$
\end{example}

\begin{task}
\begin{enumerate}
    \item Композиция инъекций (сюрьекций, биекция) яв-ся инъекцией (сюрьекцией, биекцией).
    \item Обр-я ф-ция к биек. $f: X \rightarrow Y$ - явл. биекцией.
\end{enumerate}
\end{task}

\begin{definition}
Пусть $A, \Lambda \neq \emptyset$

Говорят, что $A$ - \textbf{семейство, индексированное эл-ми } $ \Lambda$, если $\exists \phi: \Lambda \rightarrow A$ - сюрьекция.

Пишут $A = \{a_\lambda\}_{\lambda \in \Lambda}$, где $a_\lambda = \phi(\lambda)$
 
$\mathscr{A} = \{A_\lambda\}_{\lambda \in \Lambda}$ 

\[
    \bigcup_{\lambda \in \Lambda} A_\lambda = \{x \colon \exists \lambda \in \Lambda (x \in A_\Lambda)\}
\]

\[
    \bigcap_{\lambda \in \Lambda} A_\lambda = \{x \colon \forall \lambda \in \Lambda (x \in A_\lambda)\}
\]

\end{definition}

\begin{example}
~\newline
    
$A_1 = \{n \in \N \colon n > 1 \text{ и } n \neq 2m \colon \forall m > 1\}$

$A_2 = \{n \in A_1\colon n \neq 3m \colon \forall m > 1\}$

$\bigcap_{n \in \N} A_n - $ мн-во простых чисел.
\end{example}

\begin{theorem}[Закон Де Моргана]
Для любого мн-ва $E$ верно:
\begin{enumerate}
    \item
        \[
            E \backslash \bigcup_{\lambda \in \Lambda} A_\lambda = \bigcap_{\lambda \in \Lambda} (E \backslash A_\lambda)
        \]

    \item
        \[
            E \backslash \bigcap_{\lambda \in \Lambda} A_\lambda = \bigcup_{\lambda \in \Lambda} (E \backslash A_\lambda)
        \]
\end{enumerate}
\end{theorem}

\begin{proof}
~\newline

\begin{enumerate}
    \item 
        \[
            x \in E \backslash \bigcup_{\lambda \in \Lambda} A_\lambda  \iff x \in E \land x \not\in \bigcup_{\lambda \in \Lambda} A_\lambda \iff x \in E \land (\forall \lambda \in \Lambda (x \not\in A_\lambda))
        \]

\[
    \iff \forall \lambda \in \Lambda (x \in E \land x \not\in A_\lambda) \iff \forall \lambda \in \Lambda (x \in E \backslash A_\lambda) \iff x \in \bigcap (E \backslash A_\lambda)
\]
    \item
        \[
            x \in E \backslash \bigcap_{\lambda \in \Lambda}^{}A_\lambda \iff x \in E \land x \not\in \bigcap_{\lambda\in\Lambda}^{}A_\lambda \iff x \in E \land \exists \lambda \in \Lambda (x \not\in A_\lambda)
        .\] 
        \[
        \iff \exists \lambda \in \Lambda (x \in E \land x \not\in A_\lambda) \iff \exists \lambda \in \Lambda (x \in E \backslash A_\lambda) \iff \bigcup_{\lambda \in \Lambda}^{} (E \backslash A_\lambda)
        .\] 
\end{enumerate}

\end{proof}

\subsection{Аксиомат. опр-е мн-ва действ. чисел}

На мн-ве $\R$ опр-ны операции "+": ($\R \cdot \R \rightarrow \R$), "*": ($\R \cdot \R \rightarrow \R$),удовл. аксиомам.

\begin{itemize}
    \item [A1: ] $\forall a, b \in \R (a + b) + c = a + (b + c)$;
    \item [A2: ] $\forall a, b \in \R a + b = b + a$ ;
    \item [A3: ] $\exists 0 \in \R \forall a \in \R a + 0 = a$ ;
    \item [A4: ] $\forall a \in \R \exists (-a) \in \R a + (-a) = 0$.
    \item [M1: ] $\forall a, b, c \in \R (a * b) * c = a * (b * c),$ 
    \item [M2: ] $\forall a, b \in \R a * b = b * a$,
    \item [M3: ] $\exists 1 \in \R, 1 \neq 0, \forall a \in \R, a * 1 = a$,
    \item [M4: ] $\forall a \in \R, a \neq 0, \exists a^{-1} \in \R a * a^{-1} = 1$,
    \item [AM: ] $\forall a, b, c \in R a * (b + c) = ab + ac$
\end{itemize}

На мн-ве $\R$ введено отношение порядка "$\leq$", удовл. след. аксиомам:

\begin{itemize}
    \item [O1: ] $\forall a, b, c \in \R$ 
        \begin{itemize}
            \item [(i): ] $a \leq a$;
            \item [(ii): ] $a \leq b, b \leq a \iff a = b$ ;
            \item [(iii): ]$ a \leq b \land b \leq c \Rightarrow a \leq c$
        \end{itemize}

    \item [O2: ] $\forall a, b \in \R \colon a \leq b \lor b \leq a$
    \item [O3: ] Если $a, b, c \in \R$ и $a \leq b$, то $a + c \leq b + c$ ;
    \item [O4: ] Если $a, b, c \in \R, a \leq b$ и $0 \leq c$, то $ac \leq bc$ ;
\end{itemize}

\textbf{Аксиома непрерывности: } Для любых непустых $A, B \subset \R$, т. ч. $\forall a \in A, b \in B, a \leq b; \exists c \in \R  \colon \forall a \in A, b \in B (a \leq c \leq b)$
\end{document}
