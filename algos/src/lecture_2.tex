\section{Сортировки}

\begin{task}
$a_1, a_2, \ldots, a_n$ - дано
Найти $b = $sort($a$)
\end{task}
\begin{task}
    $a_1, a_2, \ldots, a_n$ - дано. Найти перестановку $G: \{1, 2, \ldots , n\} \rightarrow \{1, 2, \ldots, n\} \colon a_{G(1)} \leq a_{G(2)} \leq \ldots \leq a_{G(n)}$ 
\end{task}

\begin{statement}
Пусть $T(n) \geq n$. Тогда, если одна задача решается за $O(T(n))$, то и вторая тоже.
\end{statement}
\begin{proof}
    $2 \Rightarrow 1 \colon b_1 = a_{G(1)}, \ldots , b_n = a_{G(n)}$ 

    $1 \Rightarrow 2\colon$ Отсортируем массив пар: $  (a_1, 1), (a_2, 2), \ldots, (a_n, n)$
    \begin{equation*}
        (x_1, y_1) \leq (x_2, y_2) \iff
        \begin{system_or}
        x_1 < x_2 \\
        x_1 = x_2, y_1 < y_2
        \end{system_or}
    \end{equation*}
\end{proof}
\begin{theorem}[Оценка кол-во сравнений в сортировке сравнениями]
Если алгоритм может только сравнивать эл-ты, то время сортировки массива $\in \Omega(n\log n)$
\end{theorem}
\begin{proof}
$a_1, a_2, \ldots, a_n$ - все эл-ты попарно различны.

Вопрос алгоса: $a_i ? a_j$ (Дерево сравнений)

Рез-т работы алгоритма зависит от $n$ и п-ти получаемых ответов ($<, >$)

Пусть алгоритм всегда совершает $\leq q$ запросов ($a_i ? a_j$). Тогда есть не более $2^{0} + 2^{1} + \ldots + 2^{q} = 2^{q + 1} - 1 \leq 2^{q + 1}$ возм. протоколов работы алгоритма. Должно быть хотя бы $n!$ разл. протоколов. $\Rightarrow$
\[
n! \leq 2^{q + 1} \Rightarrow q = \Omega(n \log n)?
\] 
\end{proof}
\begin{lemma}
\[
\log (n!) = \theta (n \log n)
\] 
\end{lemma}
\begin{proof}
\begin{enumerate}
    \item [1) ]
        \[
            n! = 1 * 2 * \ldots * n \leq n^{n}
        \]
         \[
         \log_2(n!) \leq n\log_2(n) \Rightarrow \log(n!) = O(n\log n)
         \]    
     \item [2) ] \[
     n = 2k \Rightarrow n! \geq (k + 1) * \ldots * (2k) \geq k^{k + 1}
     \] 
     \[
     \log_2(n!) \geq \log_2(k^{k + 1}) = (k + 1)\log_2 k \geq \frac{n}{2} \log_2 (\frac{n}{2}) = \frac{1}{2}n (\log n - 1) = \frac{1}{2}n \log n - \frac{1}{2} \geq \frac{1}{2} n \log n
     \] 
     \[
     \Rightarrow \log_2(n!) = \Omega(n \log_2 n)
     \] 
     Аналогично, при $n = 2k + 1$
\end{enumerate}
\end{proof}

\subsection{Merge Sort (Сортировка слиянием)}
Алгоритм:
\begin{enumerate}
    \item [1) ] Если массив длины 1, то выходим, иначе делим его на 2 половины;
    \item [2) ] Рекурсивно сортируем половины
    \item [3) ] "Мёрджим" две половины.
\end{enumerate}
Операция merge:
\lstset{style=mystyle}
\begin{lstlisting}[language=Python, caption=Merge]
merge(a[0..n - 1], b[0..m - 1], to[0..n + m - 1]):
    i = 0, j = 0
    while (i < n || j < m):
        if (j == m || (i < n && a[i] <= b[j])):
            to[i + j] = a[i]
            ++i
        else
            to[i + j] = b[j]
            ++j
\end{lstlisting}
\lstset{style=mystyle}
\begin{lstlisting}[language=Python, caption=MergeSort]
MergeSort(a[0..n - 1]):
    if (n <= 1) return
    k = n / 2
    l = a[0..k]
    r = a[k + 1..n - 1]
    MergeSort(l)
    MergeSort(r)
    Merge(l, r, a)
\end{lstlisting}
\textbf{Асимптотика:} $T(n) = 2T(\frac{n}{2}) + O(n) \Rightarrow T(n) = O(n\log n)$

\begin{task}
    \begin{enumerate}
        \item [1) ] Сделать потребление памяти $O(n)$
        \item [2) ] Сделать нерекурсивный MergeSort
    \end{enumerate}
\end{task}

\begin{task}[О числе инверсий в массиве]
~\newline

$a_1, a_2, \ldots , a_n$ - дано

Инверсия ($i, j$) := $ i < j \land a_i > a_j$
\end{task}
\begin{solution}
Для решения просто модифицируем merge:

\lstset{style=mystyle}
\begin{lstlisting}[language=Python, caption=Merge for inversions]
// ans - global variable
merge(a[0..n - 1], b[0..m - 1], to[0..n + m - 1]):
    i = 0, j = 0
    while (i < n || j < m):
        if (j == m || (i < n && a[i] <= b[j])):
            to[i + j] = a[i]
            ++i
            ans += j
        else
            to[i + j] = b[j]
            ++j
            ans += i
\end{lstlisting}
\end{solution}

\subsection{Quick Sort \\ (Быстрая сортировка; Сортировка Хоара)}
\lstset{style=mystyle}
\begin{lstlisting}[language=Python, caption=Quick Sort]
Partition(a[0..n - 1], x):
    B =>  < x
    C =>  = x
    D =>  > x
    return (|B|, |C|, |D|)

QuickSort(a[0..n - 1]):
    if (n <= 1) return;
    x = a[random(0, n - 1)]
    l, m, r = Partition(a, x)
    QuickSort(a[1..l - 1])
    QuickSort(a[l + m..n - 1])
\end{lstlisting}
\begin{theorem}
В среднем асимпт. = $O(n \log n)$
\end{theorem}
\subsection{Quick Select}
\begin{definition}
$a_1, a_2, \ldots, a_n$ - массив
\textbf{k-ая порядковая статистика:} - эл-т на $k$-ом эл-те после сортировки.
\end{definition}
\begin{task}
Найти k-ую порядковую статистику в массиве $a$.
\end{task}
\begin{solution}
    ~\newline
    
\lstset{style=mystyle}
\begin{lstlisting}[language=Python, caption=QuickSelect]
QuickSelect(a[1..n], k):
    if (n == 1) return a[1];
    l, m, r = Partition(a, a[random(1, n)])
    if (k <= l) return QuickSelect(a[1..l], k)
    if (k <= l + m) return x
    return QuickSelect(a[l + m..n], k - l - m)
\end{lstlisting}
\end{solution}
