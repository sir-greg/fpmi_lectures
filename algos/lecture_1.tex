\documentclass[12pt]{article}
\usepackage[T1, T2A]{fontenc}
\usepackage[utf8]{inputenc}
\usepackage[russian]{babel}
\usepackage{amsmath}
\usepackage{amsthm}
\usepackage{amssymb}
\usepackage{esvect}
\usepackage{listings}
\usepackage{xcolor}


% for large comments
\usepackage{blindtext, xcolor}
\usepackage{comment}

% for inkscape pictures
\usepackage{import}
\usepackage{xifthen}
\usepackage{pdfpages}
\usepackage{transparent}

\newcommand{\incfig}[1]{%
    \def\svgwidth{\columnwidth}
    \import{./figures/}{#1.pdf_tex}
}

\renewcommand{\C}{\mathbb{C}}
\newcommand{\R}{\mathbb{R}}
\newcommand{\Q}{\mathbb{Q}}
\newcommand{\Z}{\mathbb{Z}}
\newcommand{\N}{\mathbb{N}}

\newcommand{\floor}[1]{\left\lfloor #1 \right\rfloor}
\newcommand{\ceil}[1]{\left\lceil #1 \right\rceil}

% style of code listings
\definecolor{codegreen}{rgb}{0,0.6,0}
\definecolor{codegray}{rgb}{0.5,0.5,0.5}
\definecolor{codepurple}{rgb}{0.58,0,0.82}
\definecolor{backcolour}{rgb}{0.95,0.95,0.92}

\lstdefinestyle{mystyle}{
    backgroundcolor=\color{backcolour},
    commentstyle=\color{codegreen},
    keywordstyle=\color{magenta},
    numberstyle=\tiny\color{codegray},
    stringstyle=\color{codepurple},
    basicstyle=\ttfamily,
    breakatwhitespace=false,
    breaklines=true,
    captionpos=b,
    keepspaces=true,
    numbers=left,
    numbersep=5pt,
    showspaces=false,
    showstringspaces=false,
    showtabs=false,
    tabsize=4
}

\newtheorem{theorem}{\underline{Теорема}}[section]
\newtheorem{lemma}[theorem]{\underlind{Лемма}}
\newtheorem{statement}{\underline{Утверждение}}[section]
\newtheorem{axiom}{\underline{Аксиома}}[section]
\newtheorem*{note}{\underline{Замечание}}
\newtheorem*{symb}{\underline{Обозначение}}
\newtheorem*{example}{\underline{Пример}}
\newtheorem*{consequence}{\underline{Следствие}}
\newtheorem*{solution}{\underline{Решение}}

\theoremstyle{definition}
\newtheorem{definition}{\underline{Определение}}[section]

\theoremstyle{definition}
\newtheorem{task}{\underline{Задача}}[section]

\title{Алгоритмы и структуры данных. \\ Лекция 1}
\author{Сергей Григорян}

\begin{document}
\maketitle
\newpage
\section{Инфа}
\textbf{Лектор:} Степанов Илья Данилович.

\begin{verbatim}
telegram: @irkstepanov
\end{verbatim}

\section{Основные понятия}
~\newline
\textbf{Фабула решения задачи}
\begin{itemize}
    \item Условие
    \item Алгоритм (реализация)
    \item Корректность
    \item Асимптотика/Время работы
\end{itemize}

\textbf{Элементарные действия}
\begin{itemize}
    \item Сложение, умножения, сравнение чисел;
    \item Условные конструкции;
    \item Обращение по индексу (! Большое кол-во = иногда вредно);
\end{itemize}

\textbf{Модель вычислений:} RAM модель (Random Access Memory)

\begin{note}
Бывают и др. модели:
\begin{itemize}
    \item Параллельные вычисления
    \item Внешняя память
\end{itemize}

\end{note}

\section{Асимптотика, время работы}

\begin{example}
Найти минимум в массиве.
\lstset{style=mystyle}
\begin{lstlisting}[language=Python, caption=Нахождение минимума]
int n;
read(n);
int a[n];
read(a);
int x = +inf;
for i = 0..n-1:
    if (x < a[i]):
        x = a[i]
print(x)
\end{lstlisting}
Асимптотика: $O(n)$
\end{example}

\begin{definition}
Пусть $f, g: \N \rightarrow \N$. Тогда:
\[
    f = O(g) \iff \exists N, C \colon \forall n \geq N \colon f(n) \leq C * g(n)
\]
\begin{example}
\[
6n + 4 = O(n), 6n + 4 \leq 7n, \text{ при } n \geq 4
.\] 
\end{example}

\end{definition}

\begin{statement}
$f = O(g) \iff \exists D \colon \forall n \colon f(n) \leq D * g(n)$
\end{statement}
\begin{proof}
    ~\newline
\begin{itemize}
    \item [$\Leftarrow$)] $N = 1 \colon n \geq N, C = D, f(n) \leq c * g(n)$
    \item [$\Rightarrow$)] Надо обеспечить:
        \[
        f(1) \leq D g(1)
        \] 
        \[
        f(2) \leq D g(2)
        \] 
        \[
            \vdots
        \]
         \[
        f(N) \leq D g(N)
        .\] 
        \[
            \Rightarrow D = max(C, \frac{f(1)}{g(1)}, \frac{f(2)}{g(2)}, \cdots, \frac{f(N)}{g(N)})
        \]
\end{itemize}
\end{proof}

\begin{definition}
Пусть $f, g: \N \rightarrow \N$. Тогда $f = \Omega(g) \text{, если } \exists C > 0, N \colon \forall n \geq N$: 
\[
f(n) \geq C * g(n)
\] 
\end{definition}
\begin{definition}
$f, g: \N \rightarrow \N$. Тогда $f = \Theta(g) \iff \exists c_1, c_2 > 0, N \colon \forall n \geq N$:
\[
c_1 * g(n) \leq f(n) \leq c_2 * g(n)
.\] 
\end{definition}
\begin{example}
\begin{enumerate}
    \item $n^{a}, n^{b}$, $a, b = const$ 

        \[
        n^{a} = O(n^{b}), a \leq b
        .\] 
        \[
        n^{a} = \Omega(n^{b}), a \geq b
        .\] 
        \[
        n^{a} = \Theta(n^{b}), a = b
        .\] 
    \item $\log_{a}{n} = \Theta(\log_{b}{n}); a, b = const$ 
    \item $n^{n} = O(2^{2^{n}}), 2^{2^{n}} = \omega(n^{n})$
\end{enumerate}
\end{example}

\begin{statement}
\[
\log_{}{n}^{a} < n^{b} < c^{n}, \forall a > 0, b > 0, c > 1
.\] 
\end{statement}
\begin{statement}
Пусть $T(n)$ - время работы влгоритма на входных данных. Пусть:
\[
T(n) = T(\floor{\frac{n}{2}}) + T(\ceil{\frac{n}{2}}) + O(n)
.\] 
Тогда $T(n) = O(n\log_{}{n})$
\end{statement}
\begin{proof}
$T(n) \leq T(\floor{\frac{n}{2}}) + T(\ceil{\frac{n}{2}}) + Dn$ 

Докажем по индукции, что:
\[
    T(n) \leq C * n\log_{}{n}, \text{ при } n \geq 2
\]

\textbf{База индукции}: $T(2), T(3), \cdots, T(10)$ - Взяли $C$, чтоб было верно.

\textbf{Переход}: Пусть $T(k) \leq Ck\log_{2}{k}, k \leq n - 1$ 

Докажем для $k = n$:

\[
T(n) \leq 2 * T(\ceil{\frac{n}{2}}) + Dn
\] 
\[
\ceil{\frac{n}{2}} \leq \frac{n + 1}{2} \Rightarrow
\] 
\[
\Rightarrow T(n) \leq 2 * T(\ceil{\frac{n}{2}}) + Dn \leq
\] 
\[
\leq 2 * C * \frac{n + 1}{2} \log_{2}{\frac{n + 1}{2}} + Dn = C(n + 1)(\log_{2}{(n + 1)} - 1) + Dn ==
\] 
\[
n + 1 \leq n\sqrt{2}
\] 
\[
\Rightarrow \log_{2}{n + 1} \leq \log_{2}{n\sqrt{2}} = \log_{2}{n} + \frac{1}{2}
\] 
\[
== C(n + 1)(\log_{2}{n} - \frac{1}{2}) + Dn = Cn\log_{2}{n} - \frac{1}{2}Cn + C\log_{2}{n} - \frac{1}{2}C + Dn \leq Cn\log_{2}{n}
.\] 
Достаточно д-ть, что $Dn + C\log_{2}{n} \leq \frac{1}{2}Cn$

Для этого дост. положить $C \geq 6D$ :

 \[
 C = 6D
 .\] 
 \[
 Dn + 6D\log_{2}{n} \leq 3Dn
 .\] 
 \[
 6\log_{2}{n} \leq 2n
 .\] 
\end{proof}

\section{Бинарный поиск}
\begin{task}
$a_0 \leq a_1 \leq a_2 \leq \cdots \leq a_{n - 1}$
 
Узнать, есть ли $x$ в $a$.

\textbf{Наивное решение}: q запросов $\Rightarrow O(nq)$
\end{task}

\begin{solution}
Используем бинпоиск:
\lstset{style=mystyle}
\begin{lstlisting}[language=C++, caption=Binary Search]
int left = 0, right = n;
while (right - left > 1) {
    mid = (left + right) / 2 
    if (a[mid] > x) right = mid
    else left = mid
}
if (a[left] == x) print("Yes");
else print("No");
\end{lstlisting}
Асимптотика: $O(\log_{2}{n})$
\end{solution}



\end{document}
