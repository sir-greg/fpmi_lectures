\documentclass[a4, 12pt]{article}
\usepackage[T1, T2A]{fontenc}
\usepackage[utf8]{inputenc}
\usepackage[russian]{babel}
\usepackage{amsmath}
\usepackage{amsthm}
\usepackage{amssymb}
\usepackage{esvect}

% for large comments
\usepackage{blindtext, xcolor}
\usepackage{comment}

% for inkscape pictures
\usepackage{import}
\usepackage{xifthen}
\usepackage{pdfpages}
\usepackage{transparent}

\newcommand{\incfig}[1]{%
    \def\svgwidth{\columnwidth}
    \import{./figures/}{#1.pdf_tex}
}

\renewcommand{\C}{\mathbb{C}}
\newcommand{\R}{\mathbb{R}}
\newcommand{\Q}{\mathbb{Q}}
\newcommand{\Z}{\mathbb{Z}}
\newcommand{\N}{\mathbb{N}}

\newcommand{\floor}[1]{\left\lfloor #1 \right\rfloor}
\newcommand{\ceil}[1]{\left\lceil #1 \right\rceil}

\newtheorem{theorem}{\underline{Теорема}}[section]
\newtheorem{lemma}[theorem]{\underlind{Лемма}}
\newtheorem{statement}{\underline{Утверждение}}[section]
\newtheorem*{note}{\underline{Замечание}}
\newtheorem*{symb}{\underline{Обозначение}}
\newtheorem*{example}{\underline{Пример}}
\newtheorem*{consequence}{\underline{Следствие}}
\newtheorem*{solution}{\underline{Решение}}

\theoremstyle{definition}
\newtheorem{definition}{\underline{Определение}}[section]

\title{Алгоритмы и структуры данных. \\ Лекция 1}
\author{Сергей Григорян}

\begin{document}
\maketitle
\newpage
\section{Телега препода}
telegram - @EeeDA (Долгов Даниил)

\section{Пояснение за O-нотацию}
$O(n) \iff T(n) \geq c * n$, n - размер входных данных, c - const

В общем:
\[
    f, g \colon \N \rightarrow \N
.\] 
\[
    f \in O(g) \iff \exists c \colon f(n) \leq c * g(n)
.\] 
\begin{example}

\[
    f(n) = 5n
\]
\[
    g(n) = 20n
\]
\[
    f \in O(g), g \in O(f)
\]
\end{example}
\begin{example}
\[
f(n) = 5n,
\] 
\[
g(n) = n^2,
\]     
\[
f \in O(g), 5n \leq 5n^2
.\] 
\[
g \not\in O(f), \not\exists c \colon g(n) \leq c * f(n), \forall n 
.\] 
\end{example}

\section{Задачи}
\subsection{Задача 1}
$a_1, a_2, \cdots, a_n$ - числа, все кроме одного встречаются дважды
За O(n) найти число, кот. встреч. 1 раз

\begin{solution}
xor-sum
\end{solution}
\subsection{Задача 2}
--- Кроме двух ----
\begin{solution}
Пусть $a, b$ - искомые числа. Считаем $a \oplus b = c$. Пусть i-ый бит $c$ = 1 (Он существует, т. к. иначе $ a = b $). Разбиваем числа на 2 группы (с единицей в i-ом бите и с нулем). Считаем xor этих групп. Эти xor-ы наш ответ.
\end{solution}

\subsection{Задача 3}
~\newline

$a_1, a_2, \cdots, a_{n - 2}, a_i \in \{1, 2, \cdots, n\}$ 

Найти $b, c \in \{1, \cdots, n\} \colon $ их нет в ${a}$

Время: $O(n)$ 

Память: $O(1)$
\begin{solution}
\begin{itemize}
    \item[I) ] "Добавить" числа 1..n, и сводим к задаче (2)
    \item[II) ] $S = \sum_{i = 1}^{n - 2} a_i$ 

          $ b + c = S - \frac{n(n+1)}{2}$

          Пусть $b < c $

          $a_i \leq \frac{b+c}{2}$

          $q = \floor{\frac{b+c}{2}} \Rightarrow b \in \{1, 2, \cdots, q\}$

          Только одно число отсутсвует в этом мн-ве $\Rightarrow$
          \[
              b = \floor{\frac{q(q+1))}{2}} - \sum_{i = 1}{q} a_i
          \]
          \[
              c = S - \frac{n(n+1)}{2} - b
          .\] 
\end{itemize}
\end{solution}

\subsection{Задача 4}
Дана матрица ($n * m$). По горизонталям и вертикалям числа возрастают. Найти данное число $x$.

Время: $O(n + m)$ 

Память: $O(1)$

\begin{solution}
Идем по контуру. solved.
\end{solution}

\subsection{Задача 5}
Есть $n$ бин строк длины $n$. Найти строку $t \colon t \neq s_i, \forall i=1..n$

Время: $O(n)$

\begin{solution}
Напишем строки в столбец. Возьмём строку $t '\colon  t'_i = s_{ii}$. Тогда $t = inverse(t')$
\end{solution}

\subsection{Задача 6}
$a_1, a_2, \cdots, a_n$, есть число, кот. встреч. $ > \floor{\frac{n}{2}}$ раз. Найти его.

\begin{solution}
Поддерживаем тек. моду и счётчик, сколько раз тек. мода встретилась в массиве. +1, если встрет. она, иначе -1. Если счётчик $ = 0$, то меняем тек. моду на встреч. элем.

Почему работает:
\begin{proof}
    Рассм. первый момент обнуления. Увидим, что кол-во истинной моды на ост. суффиксе $> \frac{1}{2}$ размеры, суффикса. После последнего такого открезка, получаем, что, т.к. счётчик не может обнулиться, то полученная мода - истинная.
\end{proof}

\end{solution}

\subsection{Задача 7}
Слишком изи, чтобы писать.

\end{document}

