\documentclass[a4, 12pt]{article}
\usepackage[T1, T2A]{fontenc}
\usepackage[utf8]{inputenc}
\usepackage[russian]{babel}
\usepackage{amsmath}
\usepackage{amsthm}
\usepackage{amssymb}
\usepackage{esvect}
\usepackage{listings}
\usepackage{xcolor}


% for large comments
\usepackage{blindtext, xcolor}
\usepackage{comment}

% for inkscape pictures
\usepackage{import}
\usepackage{xifthen}
\usepackage{pdfpages}
\usepackage{transparent}

\newcommand{\incfig}[1]{%
    \def\svgwidth{\columnwidth}
    \import{./figures/}{#1.pdf_tex}
}

\renewcommand{\C}{\mathbb{C}}
\newcommand{\R}{\mathbb{R}}
\newcommand{\Q}{\mathbb{Q}}
\newcommand{\Z}{\mathbb{Z}}
\newcommand{\N}{\mathbb{N}}

\newcommand{\floor}[1]{\left\lfloor #1 \right\rfloor}
\newcommand{\ceil}[1]{\left\lceil #1 \right\rceil}

% style of code listings
\definecolor{codegreen}{rgb}{0,0.6,0}
\definecolor{codegray}{rgb}{0.5,0.5,0.5}
\definecolor{codepurple}{rgb}{0.58,0,0.82}
\definecolor{backcolour}{rgb}{0.95,0.95,0.92}

\lstdefinestyle{mystyle}{
    backgroundcolor=\color{backcolour},
    commentstyle=\color{codegreen},
    keywordstyle=\color{magenta},
    numberstyle=\tiny\color{codegray},
    stringstyle=\color{codepurple},
    basicstyle=\ttfamily\footnotesize,
    breakatwhitespace=false,
    breaklines=true,
    captionpos=b,
    keepspaces=true,
    numbers=left,
    numbersep=5pt,
    showspaces=false,
    showstringspaces=false,
    showtabs=false,
    tabsize=4
}

\newtheorem{theorem}{\underline{Теорема}}[section]
\newtheorem{lemma}[theorem]{\underlind{Лемма}}
\newtheorem{statement}{\underline{Утверждение}}[section]
\newtheorem*{note}{\underline{Замечание}}
\newtheorem*{symb}{\underline{Обозначение}}
\newtheorem*{example}{\underline{Пример}}
\newtheorem*{consequence}{\underline{Следствие}}
\newtheorem*{solution}{\underline{Решение}}

\theoremstyle{definition}
\newtheorem{definition}{\underline{Определение}}[section]

\title{ОКТЧ. Семинар 1}
\author{Сергей Григорян}

\begin{document}
\maketitle
\newpage
\section{Контакты}
\begin{verbatim}
telegram = @ax_equals_b
\end{verbatim}

\section{Основные понятия}
\begin{definition}
Мн-во - первичное понятие
\end{definition}

\begin{symb}
    \[
        \{1, 2, 3\}
    \]
    \[
    \{n \in \N \colon 5 | n\}
    .\] 
    \[
    \{x^2 \colon x \in \{1, 2, \cdots, 5\}\}
    .\] 
\end{symb}

\begin{symb}
~\newline

Принадлежность: $a \in A$ 

Все эл-ты из $A$ содерж. в $B$ : $A \subset B \iff \forall a \in A \colon a \in B$

\end{symb}

Факты:
\begin{itemize}
    \item[a) ] $A \subset A$ - \textbf{рефлексивность} 
    \item[b) ] $A \subset B$, $B \subset A$ $\iff A = b$ - \textbf{антисимметричность}
    \item[c) ] $A \subset B, B \subset C \Rightarrow A \subset C$ - \textbf{транзитивность}\\
        $\forall a \in A \Rightarrow a \in B \Rightarrow a \in C$
    \item[d) ] $\emptyset \subset A$ 
\end{itemize}

\begin{definition}
~\newline
\textbf{Объединение мн-в}  $A$ и $B$ = $A \cup B = \{x | x \in A \lor  x \in B\}$
\end{definition}
\begin{definition}
~\newline
\textbf{Пересечение мн-в} $A$ и $B$ = $A \cap B = \{x | x \in A \land x \in B\}$
\end{definition}
\begin{definition}
~\newline
\textbf{Разностью мн-в} $A$ и $B$ = $A \backslash B =  \{x | x \in A \land x \not\in B\}$
\end{definition}
\begin{definition}
~\newline
\textbf{Симм. разн-ю мн-в} $A$ и $B$ = $A \triangle B = \{x | x \in A \backslash B \lor x \in B \backslash A\}$
\end{definition}

\begin{statement}
$A \cup (B \cap C) = (A \cup B) \cap (A \cup C)$
\end{statement}
\begin{proof}
\[
A \cup (B \cap C) \iff  x \in A \land x \in B \cap C \iff 
.\] 
\[
x \in A \lor (x \in B \land x \in C) \iff x \in A \lor x \in B \land x \in A \lor x \in C
.\] 
\[
x \in A \cup B \land x \in A \cup C \iff (A \cup B) \cap (A \cup C)
.\] 
\end{proof}

\begin{symb}
\textbf{Универсум} $U$ - мн-во, кот. принадлежат все рассм. эл-ты.
\end{symb}
$\Rightarrow \overline{A} = U \backslash A$

\begin{definition}
    \textbf{Кортеж} - упоряд. набор эл-ов:
    \begin{itemize}
        \item Кортеж длины 0 $ = \emptyset$ 
        \item Если $T = (a_1, \cdots, a_n)$, то $(a, a_1, \cdots, a_n) = {a, {a, T}}$ - кортеж длины $n + 1$ 
        \item Кортеж длины 2 - \textbf{упорядоченная пара}.
    \end{itemize}
\end{definition}

\begin{definition}
    \textbf{Декартово произ-е} $A \times B = \{(a, b) | a \in A, b \in B\}$
\end{definition}
\begin{definition}
    \textbf{Декартова степень} $A^n = A \times A \times \cdots \times A \underset{trust me}{\longleftrightarrow}  (a_1, a_2, \cdots, a_n) $
\end{definition}


\end{document}
